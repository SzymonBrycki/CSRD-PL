\documentclass[10pt, a4paper, twocolumn, openright]{book}
\usepackage[pdftex, breaklinks=true]{hyperref}
\usepackage{polski}
\usepackage[utf8]{inputenc}
\usepackage{hyperref}
\usepackage{makeidx}
\usepackage{tabularx}
\usepackage{xcolor }

\definecolor{purple}{HTML}{92268F}

\makeindex

\newcommand{\mytext }[1] {{\color{purple}  \texttt {#1}}}

% \title{Cypher System Reference Document 2024-07-02 (Edycja Polska)}
% \author{Zespół Monte Cook Games\thanks{Strona projektu: \url{https://www.montecookgames.com/cypher-system-open-license/}} \and Szymon ``Kaworu'' Brycki\thanks{\href{mailto:szymon.brycki@gmail.com}{\tt szymon.brycki@gmail.com}}}

\begin{document}

\begin{titlepage}
	\centering
	{\Huge\bfseries\title  CCypher System Reference Document \par}
	\vspace{1cm}
	{\large\itshape 2024-07-02 \par}
	{\large\itshape Edycja polska \par}
	\vspace{1cm}
	{\normalsize Zespół Monte Cook Games,   Szymon ``Kaworu'' Brycki \par}
	\vspace{1cm}
	{\normalsize Licencja: \bfseries Cypher System Open License\par}
	\vspace{1cm}
	{\normalsize Stworzono w technologii \LaTeX \par}
	\vspace{1cm}
	{\large \today \par}
\end{titlepage}

% \maketitle

\tableofcontents

% here go all the chapters

\chapter {Jak grać w Cypher System}

Zasady Cypher System są całkiem proste i cała rozgrywka bazuje na ledwie kilku podstawowych konceptach.

Ten rozdział zapewnia krótkie wyjaśnienie jak grać w tę grę, i jest przydatny dla dopiero uczących się rozgrywki. Kiedy zrozumiesz już podstawowe koncepty, będziesz pewnie chcieć przeczytać \mytext{Zasady Gry} po więcej szczegółów. 
Cypher System korzysta z kości dwudziestościennej (k20) by określić wynik większości akcji. Za każdym razem, gdy wymagany jest rzut, a nie podano kości, rzuć k20.

Mistrz Gry określa stopień trudności danego zadania. Istnieje 10 stopni trudności. Tak więc, trudność można określić na skali od 1 do 10.

Każda trudność ma minimalny wynik powiązany z sobą. Minimalny wynik (inaczej zwany stopniem trudności) to zawsze 3x poziom trudności, więc stopień trudności 1 ma minimalny wynik 3, a stopień trudności 4 ma minimalny wynik 12. By odnieść sukces, należy wyrzucić minimalny wynik lub więcej danego ST. Patrz Tabela Stopnie Trudności po więcej danych.
Umiejętności postaci, przydatne okoliczności lub doskonały ekwipunek mogą zmniejszyć trudność zadania. Dla przykładu, postać wytrenowana we wspinaczce może zamienić trudność 6 testu wspinaczki na trudność 5. Nazywa się to  Ułatwianiem albo Obniżaniem trudności o jeden stopień (albo po prostu Obniżaniem trudności, gdzie przyjmuje się domyślnie, że dotyczy ona jednego stopnia). Jeśli postać jest wyspecjalizowana we wspinaczce, zamienia ona trudność 6 na trudność 4. Nazywa się to Obniżaniem trudności o dwa stopnie. Obniżanie poziomu trudności może także być nazywane ułatwieniem zadania. Niektóre sytuacje zwiększają, lub Utrudniają, trudność zadania. Jeśli zadanie jest utrudnione, należy zwiększyć jego trudność o jeden poziom.

Umiejętność to kategoria wiedzy, zdolności lub aktywności powiązania z zadaniem, np.: wspinaczka, geografia lub perswazja. Postać, która posiada umiejętność, jest lepsza w powiązanych z nią zadaniach niż postać, która nie posiada danej umiejętności. Posta posiada albo wytrenowaną (do pewnego stopnia) umiejętność, albo wyspecjalizowaną (bardzo dużą).
Jeśli jesteś wytrenowany w umiejętności powiązanej z danym zadaniem, ułatwiasz rzut o stopień. Jeśli jesteś wyspecjalizowany, obniżasz poziom trudności o dwa stopnie. Umiejętność nigdy nie może obniżyć trudności testu o więcej niż dwa stopnie. 

Wszystko inne co obniża trudność danego zadania nazywa się Wysiłkiem. (Wysiłek opisano szczegółowo w rozdziale Zasady Gry).

Podsumowując, trzy rzeczy mogą obniżyć trudność zadania: umiejętności, atuty i Wysiłek. 

Jeśli ułatwisz rzut tak mocno, że jego trudność wynosi 0, wtedy automatycznie uzyskujesz sukces i nie musisz rzucać kośćmi. 

\section {Kiedy rzucać kośćmi?}

Za każdym razem, gdy Twoja postać chce wykonać jakieś zadanie, MG daje mu Poziom Trudności i rzucasz k20 przeciwko Stopniowi Trudności powiązanemu z danym Poziomem Trudności.

Kiedy wyskakujesz z płonącego pojazdu, zamachujesz się toporem na zmutowanąbestię, płyniesz poprzez rwącą rzekę, identyfikujesz dziwne urządzenie, przekonujesz handlarza, by dał Ci niższą cenę, tworzysz obiekt, korzystasz z mocy, by kontrolować umysł przeciwnika lub korzystasz z laserowego działka, by zrobić dziurę w ścianie, wykonujesz rzut k20.
Jednakże, jeśli twój Poziom Trudności ma wartość 0, rzut nie jest konieczny – automatycznie uzyskujesz sukces. Wiele akcji ma trudność 0. Przykłady to przejście przez pokój i otworzenie drzwi, skorzystanie ze specjalnej zdolności lotu, korzystanie z mocy, by ochronić swojego przyjaciela przed promieniowaniem, lub aktywowanie urządzenia (które się rozumie) by stworzyć pole siłowe. To wszystko to są rutynowe akcje i nie wymagają one rzutów.

Korzystając z umiejętności, atutów i Wysiłku, można teoretycznie obniżyć trudność dowolnej akcji do 0 i zlikwidować konieczność rzucania kostką. Przejście po wąskim drewnie jest trudne dla większości ludzi, ale nie dla doświadczonego gimnastyka. Możesz nawet obniżyć trudność ataku na swojego wroga do 0 i odnieść sukces bez rzucania.

Jeśli nie ma rzutu, nie ma szansy, by odnieść porażkę. Jednakże, nie ma także szansy na wyjątkowy sukces (w Cypher System zazwyczaj oznacza to wyrzucenie 19 lub 20, co jest znane jako specjalne rzuty; rozdział Zasady Gry omawia je w szczegółach).

\begin{table*}[t]
 \centering
 \begin{tabular}{l l l l}
   Poziom trudności & Opis & Stopień trudności & Szczegóły  \\ \hline
    0 & Rutyna & 0 & Każdy może to zrobić zawsze \\ \hline
    1 & Proste & 3 & Większość ludzi może to zrobić przez większość czasu \\ \hline
 \end{tabular}
  \caption {Tabela: Trudność zadań}
  \label {Tabela: Trudność zadań}
  \end{table*}

\section {Walka}\index{Walka!Wstęp}
Wykonywanie ataków w walce działa tak samo jak inne rzuty – MG określa trudność zadania, a następnie należy rzucić k20 przeciwko Stopniowi Trudności.

Trudność Twojego testu ataku zależy od tego, jak bardzo potężny jest przeciwnik. Istoty mają poziomy od 0 do 10, tak jak i zadania, które może wykonać postać. Zazwyczaj trudność rzutu to ST powiązanie z poziomem istoty. Dla przykładu, atak na bandytę 2 poziomu to zadanie o Poziomie Trudności 2, więc Stopień Trudności wynosi 6. 

Trzeba zaznaczyć, że gracze wykonują wszystkie rzuty w Cypher System. Jeśli gracz atakuje istotę, ten gracz wykonuje rzut na atak. Jeśli istota atakuje gracza, to on wykonuje rzut obronny. 

Obrażenia zadawane przez atak nie są definiowane przez rzut kością – jest to stała wartość bazująca na broni lub ataku. Dla przykładu, włócznia zawsze zadaje 4 punkty obrażeń.

Twój Pancerz redukuje obrażenia które otrzymujesz. Otrzymujesz Pancerz za noszenie fizycznej zbroi (takiej jak skórzana kurtka e współczesnym świecie lub pancerz w świecie fantasy) lub ze specjalnych zdolności. Tak jak wartość obrażeń, Pancerz to stała wartość, nie wynik rzutu. Jeśli jesteś zaatakowany, odejmij swój Pancerz od otrzymanych obrażeń. Dla przykładu, skórzana kurtka daje Ci +1 do Pancerza, co oznacza, że otrzymujesz o 1 mniejsze obrażenia z ataków. Jeśli ktoś trafi Cię atakiem nożem za 2 punkty obrażeń, kiedy ją nosisz, otrzymasz tylko 1 punkt obrażeń. Jeśli Pancerz redukuje obrażenia do 0, wtedy nie otrzymujesz w ogóle żadnych obrażeń. 

Kiedy widzisz w zasadach gry słowo „Pancerz” pisane wielką literą, odnosi się do to statystyki Pancerz – do liczby, o którą obniżasz obrażenia. Kiedy widzisz „pancerz” pisany małą literą, dotyczy to dowolnego fizycznego pancerza, który postać może nosić. 

Fizyczne bronie posiadają 3 kategorie: lekkie, średnie i ciężkie.  

Lekkie bronie zadają tylko 2 punkty obrażeń, ale ułatwiają rzuty na atak, ponieważ są szybkie i łatwe w użyciu. Lekkie bronie to ciosy pięścią, kopnięcia, maczugi, noży, toporki ręczne, rapiery, małe pistolety itp. Bronie, które są małe, są broniami lekkimi.
Średnie bronie zadają 4 punkty obrażeń. Średnie bronie to między innymi miecze, topory bojowe, większe maczugi, kusze, włócznie, pistolety, blastery itp. Większość broni to bronie średnie. Wszystko, co może być użyte w jednej dłoni (nawet, jeśli często korzysta się z dwóch, jak w przypadku kostura i włóczni) jest średnią bronią. 

Ciężkie bronie zadają 6 punktów obrażeń, i trzeba korzystać z obydwu dłoni, by z nich korzystać. Ciężkie bronie to wielkie miecze, młoty bojowe, potężne topory, halabardy, ciężkie kusze, karabinki laserowe itp. Wszystko, z czego trzeba korzystać z obydwu dłoni, to ciężkie bronie.

\section {Specjalne wyniki rzutów}\index{Specjalne wyniki rzutów!Wstęp}

Kiedy wyrzucasz naturalne 19 (k20 pokazuje „19”) i test jest sukcesem, uzyskujesz mniejszy efekt. W walce, mniejszy efekt zadaje dodatkowe 3 do obrażeń, lub, jeśli wolisz efekt specjalny, możesz odrzucić wroga do tyłu, rozproszyć jego uwagę lub coś podobnego. Kiedy nie walczysz, mniejszy efekt może oznaczać, że wykonałeś akcję ze stylem. Przykładowo, gdy przeskakujesz przez płot, lądujesz z gracją na własnych stopach, lub gdy przekonujesz kogoś, wierzy on, że jesteś mądrzejszy, niż jesteś naprawdę. W innych słowach, nie tylko osiągasz zwykły sukces, ale także uzyskujesz pomniejszy bonus. 

Kiedy wyrzucasz naturalne 20 (k20 pokazuje „20”) i rzut się powiódł, uzyskujesz dodatkowo większy efekt. Jest to podobne do mniejszego efektu, ale na większą skalę. W walce, zadajesz dodatkowe 4 punkty obrażeń, ale znowu, można zamiast tego wybrać jakiś efekt dodatkowy, taki jak przewrócenie wroga, ogłuszenie go, lub wzięcie akcji dodatkowej. Poza walką, większy efekt oznacza, że dzieje się coś korzystnego, w zależności od okoliczności. Dla przykładu, kiedy wspinasz się na ścianę, robisz to dwa razy szybciej. Kiedy rzut daje Ci większy efekt, możesz zamiast tego skorzystać z mniejszego efektu, jeśli taka jest Twoja wola.

W walce (i tylko wtedy) jeśli rzucisz naturalne 17 lub 18 na rzucie na atak, zadajesz – odpowiednio - dodatkowe 1 lub 2 punkty obrażeń. Te rzuty nie dają żadnych innych specjalnych efektów – tylko zwiększają obrażenia.

(Po więcej informacji o specjalnych wynikach rzutów i tym, jak wpływają na walkę i inne akcje, patrz Zasady Gry).

Wyrzucenie naturalnej 1 jest zawsze złe. To oznacza, że MG wprowadza nowe utrudnienie do sceny. 

\section {Słowniczek}\index{Słowniczek}

{\bfseries Mistrz Gry (MG)}: Gracz, który nie ma własnej postaci, a który zamiast tego kieruje całą fabułą i wszystkimi BN-ami.

{\bfseries Bohater Niezależny (BN)}: Postać kierowana przez MG. Myśl o niej jak o pomniejszej postaci w historii, lub jak o złoczyńcy lub oponencie. Wlicza się w to każde każda istota lub potwór.

{\bfseries Drużyna}: Grupa BG (i może jacyś BN-i sojusznicy).

{\bfseries Bohater Gracza (BG)}: Postać odgrywana przez gracza zamiast przez MG. Myśl o BG jak o głównych bohaterach historii.

{\bfseries Gracz}: Gracz, który kieruje BG.

{\bfseries Sesja}: Pojedyncza doświadczenie roleplayowe. Zazwyczaj trwa kilka godzin. Czasami jedną przygodę można ukończyć w czasie jednej sesji. Częściej, jedna przygoda zajmuje kilka sesji.

{\bfseries Przygoda}: Pojedyncza część kampanii z początkiem i końcem. Zazwyczaj zdefiniowana na początku przez wspólny cel BG i na końcu przez to, czy go osiągnęli, czy też nie. 

{\bfseries Kampania}: Seria sesji połączona wspólną historią (lub połączonymi historiami) z tymi samymi BG. Często, lecz nie zawsze, kampania to zbiór przygód.

{\bfseries Postać}: Cokolwiek, co podejmuje akcje w grze. Choć wliczają się w to BG i ludzcy BN-i, technicznie wliczają się w to potwory, kosmici, mutanci, automatony, ruchome rośliny itp. Synonimem jest „istota” bądź „potwór”.

\section {Zasięg i szybkość}\index{Zasięg i szybkość}

Zasięg dzieli się na 4 ogólnikowe kategorie: bliski, średni, daleki i bardzo daleki.

Bliski zasięg to odległość ręki lub paru kroków. Jeśli postać stoi w małym pokoju, wszystko wokół jest w jej bliskim zasięgu. Górna granica bliskiego zasięgu to 3 metry.

Średni zasięg to wszystko, co jest większe od bliskiego, ale mniejsze niż 15 metrów.

Długi dystans to wszystko większe od średniego zasięgu, ale mniejsze niż 30 metrów.

Bardzo długi zasięg to wszystko większe od długiego dystansu, ale nie większe niż 150 metrów. Poza tym zasięgiem, odległości zawsze się ściśle określone – 300 metrów, 1,5 kilometra itp.

Ogólną ideą tego systemu jest to, że nie trzeba dokładnie mierzyć i określać odległości. Bliski zasięg to tutaj, obok postaci. Średni zasięg to blisko postaci. Długi dystans jest dalej, a bardzo długi – znacznie dalej.
Wszystkie bronie i specjalne zdolności korzystają z tych terminów. Dla przykładu, wszystkie bronie do walki wręcz mają bliski zasięg – służą przecież do walki wręcz i można je użyć tylko na osobach, które stoją obok nas. Nóż do rzucania (i większość innych broni rzucanych) mają średni zasięg. Łuk ma długi zasięg. Pocisk Adepta także ma średni zasięg.

Postać może się przemieścić o bliski zasięg jako część innej akcji. Innymi słowy, może ona podejść do panelu kontrolnego i z niego skorzystać. Może przejść przez mały pokój i zaatakować wroga. Mogę otworzyć drzwi i przejść przez nie.

Postać może się przemieścić o średni zasięg jeśli poświęci na to całą akcję w turze. Może także spróbować się przemieścić o długi zasięg w jednej akcji, ale trzeba wykonać rzut, by stwierdzić, czy postać się nie poślizgnęła lub  przewróciła w efekcie tak szybkiego ruchu.

Dla przykładu, jeśli BG walczą z grupą kultystów, każda postać może zaatakować, ogólnie rzecz ujmując, dowolnego kultystę wręcz – wszyscy są w zasięgu. Dokładne pozycje nie są tak ważne. Istoty w walce zawsze się zresztą poruszają. Jednakże, jeden z kultystów został z tyłu by wystrzelić z pistoletu i BG może musieć poświęcić całą akcję, by się do niego dostać. Nie ma większego znaczenia, czy ten kultysta jest 6 metrów od postaci graczy, czy może 12 – po prostu jest w średnim zasięgu. Ma znaczenie, czy kultysta stoi o więcej niż 15 metrów od BG, ponieważ wtedy zasięg by się zwiększył do dalekiego.

(Wiele zasad w tej grze unika konieczności nadmiernej precyzji. Czy naprawdę się liczy to, czy duch jest o 13, czy o 18 stóp od Ciebie? Najpewniej nie. Taki rodzaj niepotrzebnej ścisłości tylko spowalnia rozgrywkę i odciąga uwagę od akcji i fabuły, zamiast być miłym dodatkiem do opowiadanej historii.)

\section {Punkty doświadczenia}\index{Punkty doświadczenia!Wstęp}

Punkty doświadczenia (PD-ki) są nagrodą dawaną graczom, gdy GM wtrąca się narrację (nazywamy to Wtrąceniem MG) z nowym i niespodziewanym wyzwaniem. Dla przykładu, w środku walki, MG może poinformować graczy, że upuszczają oni swoje bronie. Jednakże, aby się wtrącić w taki sposób, MG musi dać graczowi 2 PD-ki. Nagrodzony gracz, z kolei, musi natychmiast dać jednego z owych PD-ków innemu graczowi, uzasadniając to (może ten gracz miał dobry pomysł, powiedział zabawny żart, wykonał akcję, która ocaliła życie jakiegoś BN/BG itp.).

Alternatywnie, gracz może odrzucić Wtrącenie MG. Jeśli on tak uczyni, nie otrzymuje on 2 PD-ków od GM, i musi wydać i PD z posiadanych przez siebie. Jeśli gracz nie ma PD-ków, nie może odrzucić Wtrącenia MG.

MG może także dać graczom PD-ki pomiędzy sesjami, jako nagrody za dokonywanie odkryć podczas gry. Odkrycia to ciekawe fakty, cudowne sekrety, potężne artefakty, odpowiedzi na pytania lub rozwiązania problemów (np.: gdzie porywacze przetrzymują swoje ofiary lub jak gracze naprawią statek kosmiczny). Nie otrzymujesz PD-ków za zabijanie potworów lub przezwyciężanie zwykłych trudności podczas gry. Odkrycia są duszą Cypher System.

Punkty Doświadczenia głównie służą awansowaniu postaci na poziomy (po detale, patrz: rozdział Tworzenie Własnej Postaci), ale gracz może także wydać 1 PD-ek, by przerzucić kość i wybrać leszy z dwóch wyników. 

\section  {Cyphery}\index{Cyphery!Wstęp}

Cyphery to zdolności, z których można skorzystać tylko raz. W wielu kampaniach, cyphery nie są fizycznymi obiektami – mogą być zaklęciem rzuconym na postać, błogosławieństwem od boga, lub po prostu zrządzeniem losu, które daje chwilową przewagę. W pewnych kampaniach, cyphery to obiekty fizyczne które postaci mogą z sobą nosić. Niezależnie od tego, czy cyphery to przedmioty, czy też nie, są częścią postaci (tak jak ekwipunek lub specjalna zdolność). I są czymś, z czego postać może skorzystać podczas gry. Forma, którą przyjmują fizyczne cyphery, zależy od settingu. W świecie fantasy mogą być różdżkami lub eliksirami, ale w grze science fiction mogą być obcymi kryształami lub prototypowymi technologiami.

Postaci często będą znajdowały nowe cyphery, więc gracze powinni równie często z nich korzystać. Ponieważ cyphery zawsze będą odmienne od innych cypherów, postać zawsze będzie miała nowe specjalne zdolności do wypróbowania. 

\section {Inne kości}

W dodatku do k20, potrzebujesz jeszcze k6 (sześciościennej kostki). Czasami będziesz potrzebował k100 (do losowania numerów od 1 do 100), co można osiągnąć, rzucając k20 dwa razy – ostatnia liczba pierwszego rzutu to “dziesiątki” a ostatnia liczba drugiego rzutu to “jedności”. Dla przykładu, rzut 17 i 9 daje nam 79, a 3 i 18 daje nam 38, a rzucenie 20 i 10 daje nam 00 (także znane jako 100). Jeśli masz k10 (dziesięciościenną kostkę) możesz skorzystać z niej zamiast z k20 by losować liczby od 1 do 100.

(k6 jest najczęściej wykorzystywana do rzutów na odzyskiwanie zdrowia i do określania poziomu cypherów).
\chapter{Tworzenie własnej postaci}

Ta sekcja wyjaśnia jak stworzyć postać, którą się będzie odgrywało w Cypher System. Należy podjąć parę decyzji, które wpłyną na postać, tak więc im lepiej rozumiesz postać, którą pragniesz zagrać, tym łatwiejsze będzie tworzenie postaci. W ten proces wlicza się rozumienie wartości trzech statystyk w grze i wybieranie trzech aspektów, które określają zdolności postaci.

\section{Statystyki postaci}\index{Statystyki postaci}

Każda postać gracza posiada trzy statystyki. Te statystyki to Moc, Szybkość i Intelekt. Są to ogólne kategorie które dotyczą wielu różnych, lecz powiązanych aspektów postaci.

\subsection{Moc}\index{Statystyki postaci!Moc}

Moc określa jak silna i wytrzymała jest postać. Koncepty siły, wytrzymałości, kondycji, twardości i zdolności fizycznych – wszystko to zawiera się w tej statystyce. Moc nie jest zależna od rozmiaru, zamiast tego, jest to wartość bezwzględna. Słoń ma więcej Mocy niż najsilniejszy tygrys, który ma więcej mocy niż najmocniejszy szczur, który ma więcej mocy od najmocniejszego pająka.

Rzuca się na Moc, kiedy chce się wyważyć drzwi, wytrzymać wiele dni bez jedzenia lub wyzdrowieć z choroby. To także główny sposób na określenie, jak wiele obrażeń Twoja postać może wytrzymać w niebezpiecznej sytuacji. Fizyczne postaci, twarde postaci, i postaci skupione na walce powinny zainwestować w Moc.

(O Mocy można myśleć jak o Mocy/Zdrowiu, gdyż określa ona jak potężna jest postać i jak wiele obrażeń może wytrzymać).

\subsection{Szybkość}\index{Statystyki postaci!Szybkość}

Szybkość opisuje jak szybka i dobrze fizycznie skoordynowana jest postać. Ta statystyka to szybkość, zdolność ruchu, zręczność i refleks. Rzuca się na szybkość, gdy chce się uniknąć ataku, zakraść gdzieś, lub rzucić trafnie piłką. Pomaga ona określić, czy możesz się poruszyć szybciej w swojej turze. Zręczne, szybkie lub cicho poruszające się postaci będą chciały mieć wysoką Szybkość, jak i te, które głównie atakują broniami dystansowymi. 

(O Szybkości można myśleć jak o Szybkości/Zręczności, gdyż dotyczy ogólnej szybkości i refleksu).

\subsection{Intelekt}\index{Statystyki postaci!Intelekt}

Ta statystyka określa jak bardzo bystra, dobrze wykształcona i lubiana jest postać. Wlicza się w to inteligencja, mądrość, charyzma, edukacja, myślenie krytyczne, bystrość, siła woli i urok osobisty. Rzuca się na Intelekt, gdy chce się rozwiązać łamigłówkę, zapamiętać fakty, opowiedzieć przekonujące kłamstwo i użyć mocy psionicznych. Postaci zainteresowane efektywną komunikacją, byciem uczonymi lub posiadającymi moce nadnaturalne powinny zainwestować w Intelekt.

(O Intelekcie można także myśleć jak o Intelekcie/Osobowości, ponieważ odnosi się zarówno do inteligencji, jak i do charyzmy).

\section{Pule, Skupienie i Wysiłek}\index{Pule, Skupienie i Wysiłek}

Każda z trzech statystyk ma dwie części składowe: Pulę i Skupienie. Pula reprezentuje czystą, wrodzoną zdolność, a Skupienie reprezentuje wiedzę o tym, jak z niej skorzystać. Trzeci element jest powiązany z tymi konceptami: Wysiłek. Kiedy postać naprawdę chce zakończyć rzut sukcesem, stosuje ona Wysiłek.

(Twoje pula statystyk, jak i Wysiłek i Skupienie, są zależne od typu postaci, deskryptora i specjalizacji, które sam wybierasz. Masz jednak w tym zakresie bardzo wielką dowolność.)


\subsection{Pula}\index{Pule, Skupienie i Wysiłek!Pula}

Twoja Pula to najbardziej podstawowy komponent statystyki. Porównanie Pul obydwu istot da ci ogólną informacje, która z nich jest lepsza w danej statystyce. Dla przykładu, postać z Pulą Mocy 16 jest silniejsza (ogólnie mówiąc) niż postać z Pulą Mocy 12. Większość postaci zaczyna grę z Pulą od 9 do 12 w większości statystyk – jest to wartość zwykłego, szarego człowieka. 

Kiedy Twoja postać jest zraniona, chora lub zaatakowana, tymczasowo tracisz punkty z jednej ze swoich Pul. Natura ataku określa, z której Puli odejmowane są punkty. Dla przykładu, fizyczny atak mieczem redukuje Pulę Mocy, trucizna, która odbiera zręczność redukuje Szybkość, a pioniczny atak redukuje Intelekt. Możesz także wydać punkty z jednej z Pul, by obniżyć trudność zadania (patrze: Wysiłek, poniżej). Możesz odpocząć, aby \mytext{odzyskać utracone punkty Pul}, i  pewne specjalne zdolności lub cyphery mogą zezwolić na odzyskanie utraconych punktów szybciej.


\subsection{Skupienie}\index{Pule, Skupienie i Wysiłek!Skupienie}

Choć Pule są podstawowym miernikiem statystyk, Twoja Skupienie jest także ważne. Kiedy coś wymaga, abyś zapłacił punktami z Puli, Twoje Skupienie redukuje ten koszt. Redukuje ono także koszt stosowania Wysiłku z danej Puli.

Dla przykładu, powiedzmy, że masz umiejętność psionicznego ataku, której aktywacja kosztuje 1 punkt z Puli Intelektu. Odejmij swoje Skupienie w Intelekcie od kosztu aktywacji, a wynik określa ile płacisz punktów z Puli, by wykorzystać psioniczny atak. Jeśli Twoje Skupienie zredukuje koszt do 0, możesz korzystać z tej zdolności za darmo.

Twoje Skupienie może być inne dla każdej statystyki. Dla przykładu, możesz mieć Skupienie w Mocy na 1, Skupienie w Szybkości na 1 i i Skupienie w Intelekcie na 0. Zawsze będziesz miał Skupienie przynajmniej w jednej statystyce. Twoje Skupienie w niej redukuje punkty wydawane z Puli tej statystyki, ale nie z innych Pul. Twoje Skupienie w Mocy redukuje koszty związane z Pulą Mocy, ale nie wpływa na Pule Szybkości bądź Intelektu. Kiedy Skupienie w statystyce sięga 3, możesz zawsze stosować jeden poziom Wysiłku za darmo.

Postać, która ma niską Pulę Mocy, ale wysokie Skupienie w Mocy, ma potencjał, by wykonywać akcje Mocy lepiej niż postać, która ma Skupienie w Mocy na 0. Wysokie Skupienie pozwoli jej zredukować koszt punktów z Puli, co znaczy, że mają więcej punktów na stosowanie Wysiłku.

\subsection{Wysiłek}\index{Pule, Skupienie i Wysiłek!Wysiłek}

Kiedy Twoja postać naprawdę pragnie ukończyć zadanie sukcesem, może zastosować Wysiłek. Dla początkującej postaci, stosowanie Wysiłku wymaga wydania 3 punktów z Puli statystyki stosownej do akcji. Tak więc, jeśli Twoja postać pragnie uniknąć ataku (Pula Szybkości) i chcesz zwiększyć szanse na sukces, możesz zastosować Wysiłek, płacąc 3 punktami z Puli Szybkości. Wysiłek ułatwia zadanie o jeden stopień. Inaczej, nazywa się to zastosowaniem jednego poziomu Wysiłku.

Nie musisz stosować Wysiłku, jeśli tego nie chcesz. Jeśli wybierzesz zastosowanie Wysiłku do zadania, musisz to zrobić zanim zdecydujesz się na rzut – nie możesz najpierw rzucić, a potem zadecydować o Wysiłku, jeśli uzyskałeś słaby rzut. 

Stosowanie większego Wysiłku może obniżyć trudność zadania jeszcze dalej – każdy dodatkowy poziom Wysiłku ułatwia zadanie o jeden stopień. Zastosowanie jednego poziomu Wysiłku obniża trudność o jeden stopień, dwóch poziomów Wysiłku – o 2 stopnie itp. Jednakże, każdy dodatkowy poziom Wysiłku po pierwszym kosztuje 2 punkty z Puli statystyki zamiast 3. Tak więc zastosowanie dwóch poziomów Wysiłku kosztuje 5 punktów (3 za 1-szy poziom plus 2 za 2-gi), zastosowanie trzech poziomów Wysiłku kosztuje 7 punktów z Puli (3 plus 2 plus 2) itp.

Każda postać posiada statystykę zwaną Wysiłek, która określa maksymalny poziom Wysiłku, który dana postać może zastosować. Początkująca (1-szo poziomowa) postać ma Wysiłek 1, co oznacza, że może zastosować 1 poziom Wysiłku w rzucie. Bardziej doświadczone postaci mają wyższy Wysiłek i mogą stosować więcej poziomów Wysiłku. Dla przykładu, postać, której Wysiłek wynosi 3 może zastosować 3 poziomy Wysiłku, by zredukować trudność rzutu.

Kiedy stosujesz Wysiłek, odejmij swoje Skupienie w odpowiedniej statystyce od całościowego kosztu Wysiłku w punktach z Puli. Dla przykładu, wykonujesz rzut na Szybkość. By zwiększyć szansę na sukces, decydujesz się zastosować 1 poziom Wysiłku, co ułatwi zadanie. Normalnie, kosztowałoby to 3 punkty z Puli Szybkości. Jednakże, masz Skupienie w Szybkości na 2, więc odejmujesz tą liczbę od kosztu. W efekcie, Wysiłek kosztuje Cię tylko 1 punkt z Twojej Puli Szybkości.

Co, gdybyś zastosował dwa poziomy Wysiłku do rzutu, zamiast tylko jednego? To by ułatwiło zadanie o dwa stopnie. Normalnie, kosztowałoby to 5 punktów z Puli, ale po odjęciu Skupienia w Szybkości o wartości 2, finalny koszt to tylko 3 punkty.

Kiedy Skupienia w statystyce osiąga 3, możesz stosować jeden poziom wysiłku za darmo. Dla przykładu, jeśli masz Skupienia w Szybkości na 3 i stosujesz 1 poziom Wysiłku na rzucie na Szybkość, będzie Cię to kosztowało 0 punktów z Twojej Puli Szybkości. (Normalnie, jeden poziom Wysiłku kosztuje 3 punkty, ale po odjęciu Skupienia w Szybkości od tego numeru, redukujemy je do 0.).

Umiejętności i inne przewagi także ułatwiają zadania, i można z nich skorzystać razem z Wysiłkiem. Dodatkowo, Twoja postać może mieć specjalne zdolności lub ekwipunek, które mogą pozwolić Ci wykorzystać Wysiłek do specjalnych zadań, takich jak przewrócenie przeciwnika przy pomocy ataku lub wpłynięcie na wiele celów przy pomocy mocy, która zazwyczaj dotyczy tylko jednej osoby.

(Kiedy stosujesz Wysiłek w walce wręcz, masz opcję wydania punktów albo z Puli Szybkości, albo z Puli Mocy. Kiedy wykonujesz atak dystansowy, możesz wydać punkty tylko z Puli Szybkości. Ta zasada odzwierciedla fakt, że w walce wręcz czasem korzysta się z brutalnej siły, a czasami z finezji, ale w atakach dystansowych zawsze chodzi o dobre wycelowanie.)

\subsection{Wysiłek i obrażenia}\index{Pule, Skupienie i Wysiłek!Wysiłek i obrażenia}

Zamiast stosować Wysiłek, by ułatwić atak, można go zastosować, by zwiększyć obrażenia zadawane w tym ataku. Na każdy poziom Wysiłku, który się stosuje w ten sposób, zadaje się dodatkowe 3 punkty obrażeń. To działa dla każdego rodzaju ataku, który zadaje obrażenia, niezależnie od tego, czy to miecz, kusza, psioniczny atak czy coś jeszcze innego.

Kiedy korzystasz z Wysiłku, by zwiększyć obrażenia ataku obszarowego, takiego jak eksplozji wywołanej przez zdolność Adepta \mytext{Wybuch}, zadajesz dodatkowe 2 punkty obrażeń zamiast 3. Jednakże, dodatkowe punkty są zadawane wszystkim celom w obszarze działania zdolności. Dodatkowo, nawet jeśli jeden lub więcej celów nie ponosi obrażeń w wyniku tego konkretnego ataku (ze względu na nieudany rzut na atak), dalej otrzymują oni 1 punkt obrażeń.

\subsection{Wiele użyć Wysiłku i Skupienia}\index{Pule, Wysiłek i Skupienie!Wiele użyć Wysiłku i Skupienia}

Jeśli Twój Wysiłek wynosi 2 lub więcej, możesz zastosować Wysiłek na wiele sposobów w jednej akcji. Dla przykładu, jeśli wykonujesz atak, możesz zastosować Wysiłek, by ułatwić atak i by zadać więcej obrażeń.

Totalny Wysiłek, z którego korzystasz, nie może być większy od Twojej wartości Wysiłku. Dla przykładu, jeśli Twój wysiłek to 2, możesz zastosować dwa poziomy Wysiłku. Możesz wykorzystać jeden z nich, by ułatwić atak, a drugi, by zwiększyć jego obrażenia, by ułatwić atak o dwa stopnie, ale nie zwiększać obrażeń, lub by nie ułatwiać ataku, ale zwiększyć obrażenia dwukrotnie.

Możesz wykorzystać Skupienie danej statystyki tylko jeden raz na akcję. Dla przykładu, jeśli stosujesz Wysiłek na ataku Mocy i zwiększasz obrażenia oraz ułatwiasz cios, możesz skorzystać z Skupienia w Mocy, by obniżyć koszt jednego z tych zastosować Wysiłku, ale nie dwóch. Jeśli wydasz 1 punkt Intelektu na aktywowanie ataku psionicznego i jeden poziom Wysiłku na ułatwienie ataku, możesz skorzystać z Skupienia w Intelekcie do jednej z tych rzeczy, ale nie obydwu.

\section{Przykładowe Statystyki}

Początkująca postać walczy z wielkim szczurem. BG rzuca się ze swoją włócznią na tego szczura, który jest istotą 2 poziomu i w związku z tym jego Stopień Trudności wynosi 6. Postać stoi wyżej od szczura i atakuje go z góry i MG uznaje, że to dobra taktyka i przyznaje atut który ułatwia atak o jeden stopień (trudność wynosi teraz 1). To obniża Stopień Trudności do 3. Atak włócznią bazuje na Mocy – postać ma Pulę Mocy 11 i Skupienie w Mocy 0. Przed wykonaniem rzutu, decyduje się ona zastosować poziom Wysiłku, by ułatwić atak. To kosztuje 3 punkty z Puli Mocy, redukując jej obecną wartość do 8. Ale te punkty są dobrze wydane. Obniża to trudność z 1 do 0, więc nie ma potrzeby, by wykonać rzut – atak automatycznie trafia.  

Inna postać chce przekonać strażnika, by pozwolił jej wejść do prywatnego biura w celu rozmowy z ważnym szlachcicem. MG oznajmia, że jest to akcja Intelektu. Postać jest na 3 poziomie i ma Wysiłek 3, Pulę Intelektu 13 i Skupienie w Intelekcie 1. Przed wykonaniem rzutu, gracz musi zadecydować, czy stosuje Wysiłek. Może on zastosować 1, 2 lub 3 poziomy Wysiłku, lub też nie zastosować żadnego. Ta akcja jest dla niego ważna, więc decyduje on się na zastosowanie 2 poziomów Wysiłku, ułatwiając zadanie o 2 stopnie. Dzięki Skupieniu w Intelekcie, płaci on tylko 4 punkty z Puli Intelektu (3 punkty za pierwszy poziom Wysiłku, plus 2 punkty za drugi poziom, minus 1 z Skupienia). Wydanie tych punktów redukuje jego Pulę Intelektu do 9. MG uznaje, że przekonanie strażnika jest zadaniem poziomu 3 (wymagającym) z Stopniem Trudności 9; dwa poziomy Wysiłku obniżają trudność zadania do poziomu trudności 1 (łatwe) a Stopień Trudności do 3. Gracz rzuca k20 i otrzymuje 8. Ponieważ ten wynik to co najmniej Stopień Trudności zadania, postać odnosi sukces. Jednakże, gdyby nie zastosowała ona żadnego Wysiłku, odniosłaby porażkę, ponieważ jej rzut (8) byłby mniejszy niż Stopień Trudności (9).

\section{Poziomy postaci}\index{Poziomy postaci}

Każda postać zaczyna grę na 1-szym poziomie. Poziom mierzy moc, wytrzymałość i zdolności postaci. Postaci awansują do 6 poziomu. Gdy postać osiąga wyższe poziomy, uzyskuje ona więcej zdolności, zwiększa swój Wysiłek, i może zwiększyć Skupienie lub liczbę punktów w Pulach. Ogolnie mówiąc, nawet postaci na 1-szym poziomie są całkiem nieźle uzdolnione. Można spokojnie założyć, że mająjuż jakieś doświadczenie. To nie jest sytuacja “od zera do bohatera”.ale raczej przykład kompetentnych ludzi polepszających swoje zdolności i wiedzę. Awansowanie na wyższe poziomy nie jest tak naprawdę celem postaci w Cypher System, lecz raczej reprezentuje to, jak postać zmienia siew czasie przygód.

Aby awansować na następny poziom, postaci muszą zyskać Punkty Doświadczenia przez wypełnianie celów postaci, uczestniczenie w przygodach i odkrywanie nowych rzeczy – ten system traktuje o zarówno odkryciach, jak i eksploracji, a także o osiąganiu osobistych celów. Punkty doświadczenia mają wiele zastosowań, a jedno z nich to zakupywanie korzyści dla postaci. Po zakupie czterech korzyści, postać awansuje na następny poziom. Każda korzyść kosztuje 4 PD-ki i można je kupować w dowolnej kolejności, ale trzeba zakupić każdy z nich (a następnie awansować na następny poziom) zanim można zakupić tę samą korzyść ponownie. Cztery korzyści postaci to:

\begin{itemize}
    \item Zwiększenie Zdolności: Dodaj 4 do swoich Pul. Możesz wybrać dowolną ilość z tych punktów na dowolne Pule.
    \item Zbliżanie się do Doskonałości: Zwiększ o 1 Skupienie w Mocy, Szybkości bądź Intelekcie (ty decydujesz).
    \item Dodatkowy Wysiłek: Zwiększ swoją wartość Wysiłku o 1.
    \item Umiejętności: Uzyskujesz trening w jednej umiejętności swojego wyboru, innej niż atak lub obrona. 
\end{itemize}    

Jak zapisano w Zasadach Gry, postać wytrenowana w danej umiejętności ułatwia powiązane z nią zadania o jeden stopień. Możesz wybrać dowolną umiejętność, której sobie zażyczysz, taką jak wspinaczka, skakanie, perswazja lub skradanie się. Możesz także wybrać jakąś dziedzinę wiedzy, taką jak historia lub geologia. Możesz nawet wybrać umiejętność bazującą na specjalnych zdolnościach swojej postaci. Dla przykładu, jeśli Twoja postać może uderzyć w przeciwnika mocą mentalną, możesz być wytrenowany w korzystaniu z tej zdolności, ułatwiając zadanie korzystania z niej. Jeśli wybierzesz umiejętność, w której już jesteś wytrenowany, zyskujesz w niej specjalizację, ułatwiając związane z nią rzuty o dwa stopnie zamiast jednego.

(Umiejętności to szeroka kategoria rzeczy, których postać może sienauczyć i wykonać. Patrz poniżej po przykładową listę umiejętności).

\begin{itemize}
        \item Inne Opcje: Gracze mogą także wydać 4 PD-ki na zakupienie innych opcji, zamiast zyskać nową umiejętność. Zakupienie dowolnej opcji z poniższej listy liczy się jak Umiejętność celem awansowania na następny poziom. Opcje speclajne to:
        \item Obniżenie kosztu noszenia zbroi. Ta opcja obniża koszt noszenia zbroi o 1.
        \item Dodaj 2 do swoich rzutów na odzyskiwanie zdrowia. 
        \item Wybierz nową zdolność zapewnianą przez swój typ, z obecnego poziomu lub niższego.
\end{itemize}

\section{Deskryptory, Type i Specjalizacje postaci}

By stworzyć postać, tworzysz proste zdanie, które ją opisuje. To zdanie przyjmuje następującą formę: “Jestem [umieść tutaj przymiotnik] [umieść tutaj rzeczownik] który [umieść tutaj czasownik].”

Tak więc powstaje zdanie “Jestem przymiotnik opisujący rzeczownik który czasownikuje”. Dla przykładu, możesz powiedzieć “Jestem Dzikim Wojownikiem który Kontroluje Bestie” lub “Jestem Czarującym Poszukiwaczem, który Stawia Umysł Ponad Materią”. 

W tym zdaniu, przymiotnik nazywany jest \mytext{deskryptorem}.

Rzeczownik to \mytext{typ} twojej postaci.

Czasownik jest nazywany \mytext{specjalizacją}.

Pomimo tego, że typ postaci znajduje się w środku zdania, to od niego zaczniemy. (Tak jak w zdaniu, rzeczownik jest podstawą).
Twój typ postaci to jądro twojej postaci. W niektórych grach fabularnych, można nim nazwać klasę postaci. Twój typ pozwala Ci określić miejsce Twojej postaci w świecie i jej relacje z innymi ludźmi w settingu. To jest rzeczownik w zdaniu “Jestem przymiotnik opisujący rzeczownik który czasownikuje”.

Możesz wybrać z czterech typów postaci – \mytext{Wojownika, Adepta, Poszukiwacza} lub \mytext{Mówcy}.

Twój deskryptor definiuje postać – wpływa na wszystko, co robisz. Twój deskryptor umieszcza Twoją postać w pewnej sytuacji (pierwszej przygodzie na początku kampanii) i pomaga zapewnić jej motywację. To przymiotnik w zdaniu “Jestem przymiotnik opisujący rzeczownik który czasownikuje”.
Jeśli Twój MG nie powie inaczej, możesz wybrać dowolny z deskryptorów postaci.

Specjalizacja to to, co Twoja postać robi najlepiej. Specjalizacja nadaje Twojej postaci specyficzność i zapewnia interesujące nowe zdolności, które mogą się przydać. Twoja Specjalizacja także pomaga Ci zrozumieć Twoje miejsce w grupie BG. Jest to czasownik w zdaniu “Jestem przymiotnik opisujący rzeczownik który czasownikuje”.

Istnieje wiele specjalizacji postaci. Twój wybór zależeć będzie zapewne od settingu i opowiadanej historii.

(Możesz wykorzystać Smaczki z odpowiedniego rozdziału, żeby zmodyfikować typy postaci tak, by pasowały do odmiennych sesji.)

\section{Specjalne Zdolności}\index{Zdolności!Wstęp}

Typ postaci i specjalizacja zapewniają BG specjalne zdolności na każdym nowym poziomie. Korzystanie z tych zdolności zazwyczaj kosztuje punkty z Puli statystyk; koszt podano w nawiasie po nazwie zdolności. Twoje Skupienie w odpowiedniej statystyce może obniżyć jej koszt, ale pamiętaj, że możesz stosować Skupienie tylko raz na akcję. Dla przykładu, powiedzmy, że Adept z Skupieniem w Intelekcie 2 chce skorzystać z zdolności Blast, aby stworzyć ładunek energii, co kosztuje 1 punkt Intelektu. Chce on także zwiększyć obrażenia z ataku korzystając z Wysiłku, co kosztuje 3 punkty Intelektu. Całościowy koszt tej akcji wynosi 2 punkty z Puli Intelektu (1 punkt za pocisk, plus 3 punkty za skorzystanie z Wysiłku, minus 2 punkty z Skupienia).

Czasami koszt umiejętności ma znak plusa (+) po liczbie. Dla przykładu, koszt może zostać podany jako “2+ punktów Intelektu”. To oznacza, że można wydać więcej punktów lub więcej poziomów Wysiłku, by ulepszyć zdolność, co wyjaśniono w jej opisie.

Wiele specjalnych zdolności daje postaci opcję zrobienia czegoś, czego normalnie nie mogłaby wykonać, jak np.: wytwarzanie promieni zimna lub atakowanie wielu celów naraz. Używanie takich zdolności jest akcją samą w sobie, a koniec opisu zdolności zawiera słowo “Akcja” aby o tym przypomnieć. Może on także zapewnić więcej informacji o tym kiedy lub jak wykonać ową akcję. 

Pewne specjalne zdolności pozwalają Ci wykonać znaną już akcję – akcję, którą można wykonać i bez tego – w odmienny sposób. Dla przykładu, zdolność może Ci pozwolić nosić ciężką zbroję, obniżyć trudność rzutów obronnych na Szybkość, lub dodać 2 punkty ognistych obrażeń do Twoich obrażeń od broni. Te zdolności nazywa się umożliwiaczami. Korzystanie z nich nie jest uważane za akcję. Umożliwiacze albo działają ciągle (np.: możliwość noszenia ciężkiej zbroi, co nie jest akcją) lub dzieją się jako część innej akcji (np.: dodawanie obrażeń od ognia do Twoich ataków, co jest częścią akcji ataku). Jeśli specjalna zdolność jest umożliwiaczem, na końcu jej opisu znajduje się słowo “Umożliwiacz” aby o tym przypomnieć.

Pewne zdolności określają swoją długość, ale zawsze możesz zakończyć wcześniej dowolną ze swoich zdolności, jeśli tylko sobie tego życzysz.
(Ponieważ Cypher System to uniwersalny system i dotyczy wielu gatunków, nie zawsze wszystkie deksryptory, typy i specjalizacje mogą być dostępne graczom. MG decyduje co jest dostępne w tej konkretnej grze i czy coś jest zmodyfikowane – poinformuje on o tym swoich graczy.)


\section{Umiejętności}\index{Umiejętności}

Czasami Twoja postać zyskuje trening w specyficznej umiejętności lub zadaniu. Przykładowo, Twoja specjalizacja może oznaczać, że jesteś wytrenowany w skradaniu się, wspinaczce i skakaniu, lub społecznych interakcjach. Innym razem, Twoja postać może wybrać umiejętność, w której jest wytrenowana, i wybierasz taką umiejętność, o której myślisz, że może się przydać w przyszłości.

Cypher System nie ma definitywnej, danej raz na zawsze listy umiejętności. Jednakże, poniżej jest trochę sugestii:

\begin{itemize}
     \item  Astronomia
    \item Utrzymywanie równowagi
    \item Biologia
    \item Botanika
    \item Noszenie ciężarów
    \item Wspinaczka
    \item Komputery
    \item Kłamstwo
    \item Przebrania
    \item Ucieczka
    \item Geografia
    \item Geologia
    \item Leczenie
    \item Historia
    \item Identyfikacja
    \item Inicjatywa
    \item Zastraszanie
    \item Skakanie
    \item Obrabianie skóry
    \item Otwieranie zamków
    \item Maszyny
    \item Kowalstwo
    \item Percepcja
    \item Perswazja
    \item Filozofia
    \item Fizyka
    \item Kradzież kieszonkowa
    \item Pilotowanie
    \item Naprawy
    \item Jeździectwo
    \item Niszczenie
    \item Skradanie się
    \item Pływanie
    \item Prowadzenie pojazdów
    \item Obróbka drewna
 \end{itemize}  
   
Możesz wybrać umiejętność, która obejmuje parę z tych aktywności (interakcje społeczne mogą pokrywać oszustwo, zastraszanie i perswazję) lub bardziej wąskie (ukrywanie sięjako skradanie się, gdy się nie rusza). Możesz też wymyślić umiejętności-profesje, takie jak piekarz, marynarz lub drwal. Jeśli chcesz wybrać umiejętność, której nie ma na liście, najpewniej najlepiej zapytać najpierw swojego MG, ale ogólnie, najważniejsze to taki wybór umiejętności, który pasuje do konceptu postaci.

Pamiętaj, że jeśli zyskujesz umiejętność, w której już jesteś wytrenowany, zostajesz w niej wyspecjalizowany. Ponieważ opisu umiejętności może być nieco mylny, stwierdzenie ,czy jesteś wytrenowany czy wyspecjalizowany może zająć nieco czasu. Dla przykładu, możesz być wytrenowany w kłamaniu,a poźniej dostać umiejętnośc do wszystkich społecznych interakcji, co oznacza, że Twoje kłamstwa są wyspecjalizowane, a inne aktywności społeczne są tylko wytrenowane. Bycie wytrenowanym 3 razy w umiejętności nie jest lepsze niż bycie wytrenowanym tylko 2 razy (innymi słowy, wyspecjalizowany to najlepszy możliwy poziom).

Tylko umiejętności pozyskane przez typ lub inne rzadkie przypadki pozwalają Ci być wyuczonym w ataku lub obronie.

Jeśli pozyskasz specjalną zdolność przez swój typ, specjalizację lub inny aspekt Twojej postaci, możesz wybrać ką w miejsce umiejętności i być wytrenowanym lub wyspecjalizowanym w danej specjalniech zdolności. Dla przykładu, jeśli masz atak mentalny, a nadchodzi czas na wybór umiejętności, możesz wybrać umiejętność w ataku mentalnym. Ułatwiłoby to ten atak za każdym razem, gdy jest używany. Każda zdolność, którą posiadasz, może mieć swoją własną umiejętność w tym właśnie celu. Nie możesz wybrać “wszystkich mocy mentalnych” lub “wszystkich zaklęć” jako jednej umiejętności i być w niej wytrenowany lub wyspecjalizowany, gdyż ta kategoria jest stanowczo zbyt szeroka.

W większości kampanii, biegłość w języku jest uważana za umiejętność. Wiec jeśli chcesz mówić po francusku, jest to traktowane tak samo jak bycie wytrenowanym w biologii lub pływaniu.
\section{Typ}\index{Typ}

Typ postaci to najważniejsza cecha Twojej postaci. Twój typ pozwala określić miejsce postaci w świecie i jej relację z innymi ludźmi. Jest to rzeczownik w zdaniu “Jestem przymiotnik rzeczownik który czasownikuje”.

(W pewnych grach RPG, typ postaci może zostać nazwany klasą postaci.)

Możesz wybrać z 4 typów postaci: Wojownika, Adepta, Odkrywcy i Mówcy. Jednakże, możesz nie chcieć korzystać z tych ogólnikowych nazw na nie. Ten rozdział oferuje parę bardziej specyficznych nazw na każdy typ, które mogą być stosowne, w zależności od świata przedstawionego. Odkryjesz, że nazwy takie jak “Wojownik” czy też “Odkrywca” nie zawsze pasują do gier dziejących się w świecie współczesnym. Jak zawsze, możesz zrobić, co uznasz za stosowne. (Twój typ określa kim jest postać. Powinieneś korzystać z dowolnej nazwy na typ, tak długo, jak pasuje zarówno do postaci, jak i do settingu.)

Ponieważ typ to podstawa, na której się buduje postać, warto się zastanowić, jaka relacja łączy go z settingiem. Aby z tym pomóc, typy to w zasadzie ogólne archetypy. Wojownik, dla przykładu, może być wszystkim, od rycerza w lśniącej zbroi, przez gliniarza na ulicy po cybernetycznego weterana tysiąca futurystycznych wojen.
Aby lepiej dostosować cztery typy do różnych settingów, istnieją różne metody zwane posmakami,  zaprezentowane w stosownym rozdziale, by pomóc w dostosowaniu typów do konwencji fantasy, science fiction, lub innych (lub by dostosować typy do pomysłu na postać).

Dalej, bardziej fundamentalne opcje dla \mytext{dalszej customizacji} są dostępne na końcu tego rozdziału. 

\subsubsection{Wtrącenie Gracza}\index{Wtrącenie Gracza}

Wtrącenie gracza oznacza, że gracz wybiera zmianę czegoś w kampanii, czyniąc rzeczy łatwiejszymi dla jego postaci. Konceptualnie, jest to przeciwieństwo wtrącenia MG: zamiast MG dawać PD graczowi i wprowadzać niespodziewaną komplikacje dla jego postaci, gracz wydaje 1 PD i wprowadza rozwiązanie problemu lub komplikacji. To, co może zrobić wtrącenie gracza, to zmienić świat gry lub obecne okoliczności zamiast bezpośrednio zmieniać postać. Dla przykładu, wtrącenie mówiące, że cypher, z którego właśnie się skorzystało, ma dodatkowe użycie byłoby właściwe, ale wtrącenie uzdrawiające postać nie byłoby. Jeśli gracz nie ma PD-ków do wydania, nie może wprowadzić wtrącenia gracza. 

Parę wtrąceń gracza jest zasugerowanych pod każdym typem. Warto jednak zaznaczyć, że nie każde wtrącenie gracza jest stosowne w każdej sytuacji. MG może zezwolić graczom na inne sugestie wtrąceń, ale ostatecznie to on decyduje, czy dane wtrącenie jest stosowne do typu postaci i danej sytuacji. Jeśli MG odmawia wtrącenia, gracz nie wydaje 1 PD-ka i wtrącenie nie następuje.

Korzystanie z intruzji nie wymaga od postaci akcji, by je zastosować. Po prostu ono następuje.

(Wtrącenie gracza powinno być ograniczone do nie więcej niż jednego wtrącenia na gracza na jedną sesję.)

\subsubsection{Akcje obronne}\index{Akcje obronne}

Akcje obronne występują wtedy, gdy gracz rzuca, by uchronić się od czegoś nieporządanego, co mogłoby się wydarzyć jego BG. Rodzaj akcji obronnej ma znaczenie, gdy rozważamy Wysiłek.

\textbf{Obrona Mocy}: Używa się jej do odporności na trucizny, choroby i wszystko inne, co można przezwyciężyć siłą i zdrowiem.

\textbf{Obrona Szybkości}: Używa się jej do unikana ciosów i uciekana od niebezpieczeństw. To najczęściej wykorzystywany rodzaj akcji obronnej.

\textbf{Obrona Intelektu}: Używa się jej do odpieranie ataków mentalnych i wszystkiego, co może wpłynać na czyiś umysł.

\printindex

\listoftables

\end{document}