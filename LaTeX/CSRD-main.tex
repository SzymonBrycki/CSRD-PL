\documentclass[10pt, a4paper, twocolumn, openright]{book}
\usepackage[pdftex, breaklinks=true]{hyperref}
\usepackage{polski}
\usepackage[utf8]{inputenc}
\usepackage{hyperref}
\usepackage{makeidx}
\usepackage{tabularx}
\usepackage{xcolor }
\usepackage{soul}
\usepackage{afterpage}

% TABLES WIDTH !!!
% 0.10 and 0.45

\definecolor{purple}{HTML}{92268F}
% \definecolor{gray}{HTML}{D3D3D3}

% \sethlcolor{gray} 

\makeindex

\newcommand{\mytext }[1] {{\color{purple}  \texttt {#1}}}

% \title{Cypher System Reference Document 2024-07-02 (Edycja Polska)}
% \author{Zespół Monte Cook Games\thanks{Strona projektu: \url{https://www.montecookgames.com/cypher-system-open-license/}} \and Szymon ``Kaworu'' Brycki\thanks{\href{mailto:szymon.brycki@gmail.com}{\tt szymon.brycki@gmail.com}}}

\begin{document}

\begin{titlepage}
	\centering
	{\Huge\bfseries\title  CCypher System Reference Document \par}
	\vspace{1cm}
	{\large\itshape 2024-07-02 \par}
	{\large\itshape Edycja polska \par}
	\vspace{1cm}
	{\normalsize \textbf{Oryginalne zasady}: Monte Cook Games \par}
	{\normalsize \textbf{Polskie tłumaczenie}: Szymon ``Kaworu'' Brycki \par}
	\vspace{1cm}
	{\normalsize Licencja: \bfseries Cypher System Open License\par}
	\vspace{1cm}
	{\normalsize Stworzono w technologii \LaTeX \par}
	\vspace{1cm}
	{\large \today \par}
\end{titlepage}

% \maketitle

\tableofcontents

% here go all the chapters

\chapter {Jak grać w Cypher System}

Zasady Cypher System są całkiem proste i cała rozgrywka bazuje na ledwie kilku podstawowych konceptach.

Ten rozdział zapewnia krótkie wyjaśnienie jak grać w tę grę, i jest przydatny dla dopiero uczących się rozgrywki. Kiedy zrozumiesz już podstawowe koncepty, będziesz pewnie chcieć przeczytać \mytext{Zasady Gry} po więcej szczegółów. 
Cypher System korzysta z kości dwudziestościennej (k20) by określić wynik większości akcji. Za każdym razem, gdy wymagany jest rzut, a nie podano kości, rzuć k20.

Mistrz Gry określa stopień trudności danego zadania. Istnieje 10 stopni trudności. Tak więc, trudność można określić na skali od 1 do 10.

Każda trudność ma minimalny wynik powiązany z sobą. Minimalny wynik (inaczej zwany stopniem trudności) to zawsze 3x poziom trudności, więc stopień trudności 1 ma minimalny wynik 3, a stopień trudności 4 ma minimalny wynik 12. By odnieść sukces, należy wyrzucić minimalny wynik lub więcej danego ST. Patrz Tabela Stopnie Trudności po więcej danych.
Umiejętności postaci, przydatne okoliczności lub doskonały ekwipunek mogą zmniejszyć trudność zadania. Dla przykładu, postać wytrenowana we wspinaczce może zamienić trudność 6 testu wspinaczki na trudność 5. Nazywa się to  Ułatwianiem albo Obniżaniem trudności o jeden stopień (albo po prostu Obniżaniem trudności, gdzie przyjmuje się domyślnie, że dotyczy ona jednego stopnia). Jeśli postać jest wyspecjalizowana we wspinaczce, zamienia ona trudność 6 na trudność 4. Nazywa się to Obniżaniem trudności o dwa stopnie. Obniżanie poziomu trudności może także być nazywane ułatwieniem zadania. Niektóre sytuacje zwiększają, lub Utrudniają, trudność zadania. Jeśli zadanie jest utrudnione, należy zwiększyć jego trudność o jeden poziom.

Umiejętność to kategoria wiedzy, zdolności lub aktywności powiązania z zadaniem, np.: wspinaczka, geografia lub perswazja. Postać, która posiada umiejętność, jest lepsza w powiązanych z nią zadaniach niż postać, która nie posiada danej umiejętności. Posta posiada albo wytrenowaną (do pewnego stopnia) umiejętność, albo wyspecjalizowaną (bardzo dużą).
Jeśli jesteś wytrenowany w umiejętności powiązanej z danym zadaniem, ułatwiasz rzut o stopień. Jeśli jesteś wyspecjalizowany, obniżasz poziom trudności o dwa stopnie. Umiejętność nigdy nie może obniżyć trudności testu o więcej niż dwa stopnie. 

Wszystko inne co obniża trudność danego zadania nazywa się Wysiłkiem. (Wysiłek opisano szczegółowo w rozdziale Zasady Gry).

Podsumowując, trzy rzeczy mogą obniżyć trudność zadania: umiejętności, atuty i Wysiłek. 

Jeśli ułatwisz rzut tak mocno, że jego trudność wynosi 0, wtedy automatycznie uzyskujesz sukces i nie musisz rzucać kośćmi. 

\section {Kiedy rzucać kośćmi?}

Za każdym razem, gdy Twoja postać chce wykonać jakieś zadanie, MG daje mu Poziom Trudności i rzucasz k20 przeciwko Stopniowi Trudności powiązanemu z danym Poziomem Trudności.

Kiedy wyskakujesz z płonącego pojazdu, zamachujesz się toporem na zmutowanąbestię, płyniesz poprzez rwącą rzekę, identyfikujesz dziwne urządzenie, przekonujesz handlarza, by dał Ci niższą cenę, tworzysz obiekt, korzystasz z mocy, by kontrolować umysł przeciwnika lub korzystasz z laserowego działka, by zrobić dziurę w ścianie, wykonujesz rzut k20.
Jednakże, jeśli twój Poziom Trudności ma wartość 0, rzut nie jest konieczny – automatycznie uzyskujesz sukces. Wiele akcji ma trudność 0. Przykłady to przejście przez pokój i otworzenie drzwi, skorzystanie ze specjalnej zdolności lotu, korzystanie z mocy, by ochronić swojego przyjaciela przed promieniowaniem, lub aktywowanie urządzenia (które się rozumie) by stworzyć pole siłowe. To wszystko to są rutynowe akcje i nie wymagają one rzutów.

Korzystając z umiejętności, atutów i Wysiłku, można teoretycznie obniżyć trudność dowolnej akcji do 0 i zlikwidować konieczność rzucania kostką. Przejście po wąskim drewnie jest trudne dla większości ludzi, ale nie dla doświadczonego gimnastyka. Możesz nawet obniżyć trudność ataku na swojego wroga do 0 i odnieść sukces bez rzucania.

Jeśli nie ma rzutu, nie ma szansy, by odnieść porażkę. Jednakże, nie ma także szansy na wyjątkowy sukces (w Cypher System zazwyczaj oznacza to wyrzucenie 19 lub 20, co jest znane jako specjalne rzuty; rozdział Zasady Gry omawia je w szczegółach).

\begin{table*}[t]
 \centering
 \begin{tabular}{l l l l}
   Poziom trudności & Opis & Stopień trudności & Szczegóły  \\ \hline
    0 & Rutyna & 0 & Każdy może to zrobić zawsze \\ \hline
    1 & Proste & 3 & Większość ludzi może to zrobić przez większość czasu \\ \hline
 \end{tabular}
  \caption {Tabela: Trudność zadań}
  \label {Tabela: Trudność zadań}
  \end{table*}

\section {Walka}\index{Walka!Wstęp}
Wykonywanie ataków w walce działa tak samo jak inne rzuty – MG określa trudność zadania, a następnie należy rzucić k20 przeciwko Stopniowi Trudności.

Trudność Twojego testu ataku zależy od tego, jak bardzo potężny jest przeciwnik. Istoty mają poziomy od 0 do 10, tak jak i zadania, które może wykonać postać. Zazwyczaj trudność rzutu to ST powiązanie z poziomem istoty. Dla przykładu, atak na bandytę 2 poziomu to zadanie o Poziomie Trudności 2, więc Stopień Trudności wynosi 6. 

Trzeba zaznaczyć, że gracze wykonują wszystkie rzuty w Cypher System. Jeśli gracz atakuje istotę, ten gracz wykonuje rzut na atak. Jeśli istota atakuje gracza, to on wykonuje rzut obronny. 

Obrażenia zadawane przez atak nie są definiowane przez rzut kością – jest to stała wartość bazująca na broni lub ataku. Dla przykładu, włócznia zawsze zadaje 4 punkty obrażeń.

Twój Pancerz redukuje obrażenia które otrzymujesz. Otrzymujesz Pancerz za noszenie fizycznej zbroi (takiej jak skórzana kurtka e współczesnym świecie lub pancerz w świecie fantasy) lub ze specjalnych zdolności. Tak jak wartość obrażeń, Pancerz to stała wartość, nie wynik rzutu. Jeśli jesteś zaatakowany, odejmij swój Pancerz od otrzymanych obrażeń. Dla przykładu, skórzana kurtka daje Ci +1 do Pancerza, co oznacza, że otrzymujesz o 1 mniejsze obrażenia z ataków. Jeśli ktoś trafi Cię atakiem nożem za 2 punkty obrażeń, kiedy ją nosisz, otrzymasz tylko 1 punkt obrażeń. Jeśli Pancerz redukuje obrażenia do 0, wtedy nie otrzymujesz w ogóle żadnych obrażeń. 

Kiedy widzisz w zasadach gry słowo „Pancerz” pisane wielką literą, odnosi się do to statystyki Pancerz – do liczby, o którą obniżasz obrażenia. Kiedy widzisz „pancerz” pisany małą literą, dotyczy to dowolnego fizycznego pancerza, który postać może nosić. 

Fizyczne bronie posiadają 3 kategorie: lekkie, średnie i ciężkie.  

Lekkie bronie zadają tylko 2 punkty obrażeń, ale ułatwiają rzuty na atak, ponieważ są szybkie i łatwe w użyciu. Lekkie bronie to ciosy pięścią, kopnięcia, maczugi, noży, toporki ręczne, rapiery, małe pistolety itp. Bronie, które są małe, są broniami lekkimi.
Średnie bronie zadają 4 punkty obrażeń. Średnie bronie to między innymi miecze, topory bojowe, większe maczugi, kusze, włócznie, pistolety, blastery itp. Większość broni to bronie średnie. Wszystko, co może być użyte w jednej dłoni (nawet, jeśli często korzysta się z dwóch, jak w przypadku kostura i włóczni) jest średnią bronią. 

Ciężkie bronie zadają 6 punktów obrażeń, i trzeba korzystać z obydwu dłoni, by z nich korzystać. Ciężkie bronie to wielkie miecze, młoty bojowe, potężne topory, halabardy, ciężkie kusze, karabinki laserowe itp. Wszystko, z czego trzeba korzystać z obydwu dłoni, to ciężkie bronie.

\section {Specjalne wyniki rzutów}\index{Specjalne wyniki rzutów!Wstęp}

Kiedy wyrzucasz naturalne 19 (k20 pokazuje „19”) i test jest sukcesem, uzyskujesz mniejszy efekt. W walce, mniejszy efekt zadaje dodatkowe 3 do obrażeń, lub, jeśli wolisz efekt specjalny, możesz odrzucić wroga do tyłu, rozproszyć jego uwagę lub coś podobnego. Kiedy nie walczysz, mniejszy efekt może oznaczać, że wykonałeś akcję ze stylem. Przykładowo, gdy przeskakujesz przez płot, lądujesz z gracją na własnych stopach, lub gdy przekonujesz kogoś, wierzy on, że jesteś mądrzejszy, niż jesteś naprawdę. W innych słowach, nie tylko osiągasz zwykły sukces, ale także uzyskujesz pomniejszy bonus. 

Kiedy wyrzucasz naturalne 20 (k20 pokazuje „20”) i rzut się powiódł, uzyskujesz dodatkowo większy efekt. Jest to podobne do mniejszego efektu, ale na większą skalę. W walce, zadajesz dodatkowe 4 punkty obrażeń, ale znowu, można zamiast tego wybrać jakiś efekt dodatkowy, taki jak przewrócenie wroga, ogłuszenie go, lub wzięcie akcji dodatkowej. Poza walką, większy efekt oznacza, że dzieje się coś korzystnego, w zależności od okoliczności. Dla przykładu, kiedy wspinasz się na ścianę, robisz to dwa razy szybciej. Kiedy rzut daje Ci większy efekt, możesz zamiast tego skorzystać z mniejszego efektu, jeśli taka jest Twoja wola.

W walce (i tylko wtedy) jeśli rzucisz naturalne 17 lub 18 na rzucie na atak, zadajesz – odpowiednio - dodatkowe 1 lub 2 punkty obrażeń. Te rzuty nie dają żadnych innych specjalnych efektów – tylko zwiększają obrażenia.

(Po więcej informacji o specjalnych wynikach rzutów i tym, jak wpływają na walkę i inne akcje, patrz Zasady Gry).

Wyrzucenie naturalnej 1 jest zawsze złe. To oznacza, że MG wprowadza nowe utrudnienie do sceny. 

\section {Słowniczek}\index{Słowniczek}

{\bfseries Mistrz Gry (MG)}: Gracz, który nie ma własnej postaci, a który zamiast tego kieruje całą fabułą i wszystkimi BN-ami.

{\bfseries Bohater Niezależny (BN)}: Postać kierowana przez MG. Myśl o niej jak o pomniejszej postaci w historii, lub jak o złoczyńcy lub oponencie. Wlicza się w to każde każda istota lub potwór.

{\bfseries Drużyna}: Grupa BG (i może jacyś BN-i sojusznicy).

{\bfseries Bohater Gracza (BG)}: Postać odgrywana przez gracza zamiast przez MG. Myśl o BG jak o głównych bohaterach historii.

{\bfseries Gracz}: Gracz, który kieruje BG.

{\bfseries Sesja}: Pojedyncza doświadczenie roleplayowe. Zazwyczaj trwa kilka godzin. Czasami jedną przygodę można ukończyć w czasie jednej sesji. Częściej, jedna przygoda zajmuje kilka sesji.

{\bfseries Przygoda}: Pojedyncza część kampanii z początkiem i końcem. Zazwyczaj zdefiniowana na początku przez wspólny cel BG i na końcu przez to, czy go osiągnęli, czy też nie. 

{\bfseries Kampania}: Seria sesji połączona wspólną historią (lub połączonymi historiami) z tymi samymi BG. Często, lecz nie zawsze, kampania to zbiór przygód.

{\bfseries Postać}: Cokolwiek, co podejmuje akcje w grze. Choć wliczają się w to BG i ludzcy BN-i, technicznie wliczają się w to potwory, kosmici, mutanci, automatony, ruchome rośliny itp. Synonimem jest „istota” bądź „potwór”.

\section {Zasięg i szybkość}\index{Zasięg i szybkość}

Zasięg dzieli się na 4 ogólnikowe kategorie: bliski, średni, daleki i bardzo daleki.

Bliski zasięg to odległość ręki lub paru kroków. Jeśli postać stoi w małym pokoju, wszystko wokół jest w jej bliskim zasięgu. Górna granica bliskiego zasięgu to 3 metry.

Średni zasięg to wszystko, co jest większe od bliskiego, ale mniejsze niż 15 metrów.

Długi dystans to wszystko większe od średniego zasięgu, ale mniejsze niż 30 metrów.

Bardzo długi zasięg to wszystko większe od długiego dystansu, ale nie większe niż 150 metrów. Poza tym zasięgiem, odległości zawsze się ściśle określone – 300 metrów, 1,5 kilometra itp.

Ogólną ideą tego systemu jest to, że nie trzeba dokładnie mierzyć i określać odległości. Bliski zasięg to tutaj, obok postaci. Średni zasięg to blisko postaci. Długi dystans jest dalej, a bardzo długi – znacznie dalej.
Wszystkie bronie i specjalne zdolności korzystają z tych terminów. Dla przykładu, wszystkie bronie do walki wręcz mają bliski zasięg – służą przecież do walki wręcz i można je użyć tylko na osobach, które stoją obok nas. Nóż do rzucania (i większość innych broni rzucanych) mają średni zasięg. Łuk ma długi zasięg. Pocisk Adepta także ma średni zasięg.

Postać może się przemieścić o bliski zasięg jako część innej akcji. Innymi słowy, może ona podejść do panelu kontrolnego i z niego skorzystać. Może przejść przez mały pokój i zaatakować wroga. Mogę otworzyć drzwi i przejść przez nie.

Postać może się przemieścić o średni zasięg jeśli poświęci na to całą akcję w turze. Może także spróbować się przemieścić o długi zasięg w jednej akcji, ale trzeba wykonać rzut, by stwierdzić, czy postać się nie poślizgnęła lub  przewróciła w efekcie tak szybkiego ruchu.

Dla przykładu, jeśli BG walczą z grupą kultystów, każda postać może zaatakować, ogólnie rzecz ujmując, dowolnego kultystę wręcz – wszyscy są w zasięgu. Dokładne pozycje nie są tak ważne. Istoty w walce zawsze się zresztą poruszają. Jednakże, jeden z kultystów został z tyłu by wystrzelić z pistoletu i BG może musieć poświęcić całą akcję, by się do niego dostać. Nie ma większego znaczenia, czy ten kultysta jest 6 metrów od postaci graczy, czy może 12 – po prostu jest w średnim zasięgu. Ma znaczenie, czy kultysta stoi o więcej niż 15 metrów od BG, ponieważ wtedy zasięg by się zwiększył do dalekiego.

(Wiele zasad w tej grze unika konieczności nadmiernej precyzji. Czy naprawdę się liczy to, czy duch jest o 13, czy o 18 stóp od Ciebie? Najpewniej nie. Taki rodzaj niepotrzebnej ścisłości tylko spowalnia rozgrywkę i odciąga uwagę od akcji i fabuły, zamiast być miłym dodatkiem do opowiadanej historii.)

\section {Punkty doświadczenia}\index{Punkty doświadczenia!Wstęp}

Punkty doświadczenia (PD-ki) są nagrodą dawaną graczom, gdy GM wtrąca się narrację (nazywamy to Wtrąceniem MG) z nowym i niespodziewanym wyzwaniem. Dla przykładu, w środku walki, MG może poinformować graczy, że upuszczają oni swoje bronie. Jednakże, aby się wtrącić w taki sposób, MG musi dać graczowi 2 PD-ki. Nagrodzony gracz, z kolei, musi natychmiast dać jednego z owych PD-ków innemu graczowi, uzasadniając to (może ten gracz miał dobry pomysł, powiedział zabawny żart, wykonał akcję, która ocaliła życie jakiegoś BN/BG itp.).

Alternatywnie, gracz może odrzucić Wtrącenie MG. Jeśli on tak uczyni, nie otrzymuje on 2 PD-ków od GM, i musi wydać i PD z posiadanych przez siebie. Jeśli gracz nie ma PD-ków, nie może odrzucić Wtrącenia MG.

MG może także dać graczom PD-ki pomiędzy sesjami, jako nagrody za dokonywanie odkryć podczas gry. Odkrycia to ciekawe fakty, cudowne sekrety, potężne artefakty, odpowiedzi na pytania lub rozwiązania problemów (np.: gdzie porywacze przetrzymują swoje ofiary lub jak gracze naprawią statek kosmiczny). Nie otrzymujesz PD-ków za zabijanie potworów lub przezwyciężanie zwykłych trudności podczas gry. Odkrycia są duszą Cypher System.

Punkty Doświadczenia głównie służą awansowaniu postaci na poziomy (po detale, patrz: rozdział Tworzenie Własnej Postaci), ale gracz może także wydać 1 PD-ek, by przerzucić kość i wybrać leszy z dwóch wyników. 

\section  {Cyphery}\index{Cyphery!Wstęp}

Cyphery to zdolności, z których można skorzystać tylko raz. W wielu kampaniach, cyphery nie są fizycznymi obiektami – mogą być zaklęciem rzuconym na postać, błogosławieństwem od boga, lub po prostu zrządzeniem losu, które daje chwilową przewagę. W pewnych kampaniach, cyphery to obiekty fizyczne które postaci mogą z sobą nosić. Niezależnie od tego, czy cyphery to przedmioty, czy też nie, są częścią postaci (tak jak ekwipunek lub specjalna zdolność). I są czymś, z czego postać może skorzystać podczas gry. Forma, którą przyjmują fizyczne cyphery, zależy od settingu. W świecie fantasy mogą być różdżkami lub eliksirami, ale w grze science fiction mogą być obcymi kryształami lub prototypowymi technologiami.

Postaci często będą znajdowały nowe cyphery, więc gracze powinni równie często z nich korzystać. Ponieważ cyphery zawsze będą odmienne od innych cypherów, postać zawsze będzie miała nowe specjalne zdolności do wypróbowania. 

\section {Inne kości}

W dodatku do k20, potrzebujesz jeszcze k6 (sześciościennej kostki). Czasami będziesz potrzebował k100 (do losowania numerów od 1 do 100), co można osiągnąć, rzucając k20 dwa razy – ostatnia liczba pierwszego rzutu to “dziesiątki” a ostatnia liczba drugiego rzutu to “jedności”. Dla przykładu, rzut 17 i 9 daje nam 79, a 3 i 18 daje nam 38, a rzucenie 20 i 10 daje nam 00 (także znane jako 100). Jeśli masz k10 (dziesięciościenną kostkę) możesz skorzystać z niej zamiast z k20 by losować liczby od 1 do 100.

(k6 jest najczęściej wykorzystywana do rzutów na odzyskiwanie zdrowia i do określania poziomu cypherów).
\chapter{Tworzenie własnej postaci}

Ta sekcja wyjaśnia jak stworzyć postać, którą się będzie odgrywało w Cypher System. Należy podjąć parę decyzji, które wpłyną na postać, tak więc im lepiej rozumiesz postać, którą pragniesz zagrać, tym łatwiejsze będzie tworzenie postaci. W ten proces wlicza się rozumienie wartości trzech statystyk w grze i wybieranie trzech aspektów, które określają zdolności postaci.

\section{Statystyki postaci}\index{Statystyki postaci}

Każda postać gracza posiada trzy statystyki. Te statystyki to Moc, Szybkość i Intelekt. Są to ogólne kategorie które dotyczą wielu różnych, lecz powiązanych aspektów postaci.

\subsection{Moc}\index{Statystyki postaci!Moc}

Moc określa jak silna i wytrzymała jest postać. Koncepty siły, wytrzymałości, kondycji, twardości i zdolności fizycznych – wszystko to zawiera się w tej statystyce. Moc nie jest zależna od rozmiaru, zamiast tego, jest to wartość bezwzględna. Słoń ma więcej Mocy niż najsilniejszy tygrys, który ma więcej mocy niż najmocniejszy szczur, który ma więcej mocy od najmocniejszego pająka.

Rzuca się na Moc, kiedy chce się wyważyć drzwi, wytrzymać wiele dni bez jedzenia lub wyzdrowieć z choroby. To także główny sposób na określenie, jak wiele obrażeń Twoja postać może wytrzymać w niebezpiecznej sytuacji. Fizyczne postaci, twarde postaci, i postaci skupione na walce powinny zainwestować w Moc.

(O Mocy można myśleć jak o Mocy/Zdrowiu, gdyż określa ona jak potężna jest postać i jak wiele obrażeń może wytrzymać).

\subsection{Szybkość}\index{Statystyki postaci!Szybkość}

Szybkość opisuje jak szybka i dobrze fizycznie skoordynowana jest postać. Ta statystyka to szybkość, zdolność ruchu, zręczność i refleks. Rzuca się na szybkość, gdy chce się uniknąć ataku, zakraść gdzieś, lub rzucić trafnie piłką. Pomaga ona określić, czy możesz się poruszyć szybciej w swojej turze. Zręczne, szybkie lub cicho poruszające się postaci będą chciały mieć wysoką Szybkość, jak i te, które głównie atakują broniami dystansowymi. 

(O Szybkości można myśleć jak o Szybkości/Zręczności, gdyż dotyczy ogólnej szybkości i refleksu).

\subsection{Intelekt}\index{Statystyki postaci!Intelekt}

Ta statystyka określa jak bardzo bystra, dobrze wykształcona i lubiana jest postać. Wlicza się w to inteligencja, mądrość, charyzma, edukacja, myślenie krytyczne, bystrość, siła woli i urok osobisty. Rzuca się na Intelekt, gdy chce się rozwiązać łamigłówkę, zapamiętać fakty, opowiedzieć przekonujące kłamstwo i użyć mocy psionicznych. Postaci zainteresowane efektywną komunikacją, byciem uczonymi lub posiadającymi moce nadnaturalne powinny zainwestować w Intelekt.

(O Intelekcie można także myśleć jak o Intelekcie/Osobowości, ponieważ odnosi się zarówno do inteligencji, jak i do charyzmy).

\section{Pule, Skupienie i Wysiłek}\index{Pule, Skupienie i Wysiłek}

Każda z trzech statystyk ma dwie części składowe: Pulę i Skupienie. Pula reprezentuje czystą, wrodzoną zdolność, a Skupienie reprezentuje wiedzę o tym, jak z niej skorzystać. Trzeci element jest powiązany z tymi konceptami: Wysiłek. Kiedy postać naprawdę chce zakończyć rzut sukcesem, stosuje ona Wysiłek.

(Twoje pula statystyk, jak i Wysiłek i Skupienie, są zależne od typu postaci, deskryptora i specjalizacji, które sam wybierasz. Masz jednak w tym zakresie bardzo wielką dowolność.)


\subsection{Pula}\index{Pule, Skupienie i Wysiłek!Pula}

Twoja Pula to najbardziej podstawowy komponent statystyki. Porównanie Pul obydwu istot da ci ogólną informacje, która z nich jest lepsza w danej statystyce. Dla przykładu, postać z Pulą Mocy 16 jest silniejsza (ogólnie mówiąc) niż postać z Pulą Mocy 12. Większość postaci zaczyna grę z Pulą od 9 do 12 w większości statystyk – jest to wartość zwykłego, szarego człowieka. 

Kiedy Twoja postać jest zraniona, chora lub zaatakowana, tymczasowo tracisz punkty z jednej ze swoich Pul. Natura ataku określa, z której Puli odejmowane są punkty. Dla przykładu, fizyczny atak mieczem redukuje Pulę Mocy, trucizna, która odbiera zręczność redukuje Szybkość, a pioniczny atak redukuje Intelekt. Możesz także wydać punkty z jednej z Pul, by obniżyć trudność zadania (patrze: Wysiłek, poniżej). Możesz odpocząć, aby \mytext{odzyskać utracone punkty Pul}, i  pewne specjalne zdolności lub cyphery mogą zezwolić na odzyskanie utraconych punktów szybciej.


\subsection{Skupienie}\index{Pule, Skupienie i Wysiłek!Skupienie}

Choć Pule są podstawowym miernikiem statystyk, Twoja Skupienie jest także ważne. Kiedy coś wymaga, abyś zapłacił punktami z Puli, Twoje Skupienie redukuje ten koszt. Redukuje ono także koszt stosowania Wysiłku z danej Puli.

Dla przykładu, powiedzmy, że masz umiejętność psionicznego ataku, której aktywacja kosztuje 1 punkt z Puli Intelektu. Odejmij swoje Skupienie w Intelekcie od kosztu aktywacji, a wynik określa ile płacisz punktów z Puli, by wykorzystać psioniczny atak. Jeśli Twoje Skupienie zredukuje koszt do 0, możesz korzystać z tej zdolności za darmo.

Twoje Skupienie może być inne dla każdej statystyki. Dla przykładu, możesz mieć Skupienie w Mocy na 1, Skupienie w Szybkości na 1 i i Skupienie w Intelekcie na 0. Zawsze będziesz miał Skupienie przynajmniej w jednej statystyce. Twoje Skupienie w niej redukuje punkty wydawane z Puli tej statystyki, ale nie z innych Pul. Twoje Skupienie w Mocy redukuje koszty związane z Pulą Mocy, ale nie wpływa na Pule Szybkości bądź Intelektu. Kiedy Skupienie w statystyce sięga 3, możesz zawsze stosować jeden poziom Wysiłku za darmo.

Postać, która ma niską Pulę Mocy, ale wysokie Skupienie w Mocy, ma potencjał, by wykonywać akcje Mocy lepiej niż postać, która ma Skupienie w Mocy na 0. Wysokie Skupienie pozwoli jej zredukować koszt punktów z Puli, co znaczy, że mają więcej punktów na stosowanie Wysiłku.

\subsection{Wysiłek}\index{Pule, Skupienie i Wysiłek!Wysiłek}

Kiedy Twoja postać naprawdę pragnie ukończyć zadanie sukcesem, może zastosować Wysiłek. Dla początkującej postaci, stosowanie Wysiłku wymaga wydania 3 punktów z Puli statystyki stosownej do akcji. Tak więc, jeśli Twoja postać pragnie uniknąć ataku (Pula Szybkości) i chcesz zwiększyć szanse na sukces, możesz zastosować Wysiłek, płacąc 3 punktami z Puli Szybkości. Wysiłek ułatwia zadanie o jeden stopień. Inaczej, nazywa się to zastosowaniem jednego poziomu Wysiłku.

Nie musisz stosować Wysiłku, jeśli tego nie chcesz. Jeśli wybierzesz zastosowanie Wysiłku do zadania, musisz to zrobić zanim zdecydujesz się na rzut – nie możesz najpierw rzucić, a potem zadecydować o Wysiłku, jeśli uzyskałeś słaby rzut. 

Stosowanie większego Wysiłku może obniżyć trudność zadania jeszcze dalej – każdy dodatkowy poziom Wysiłku ułatwia zadanie o jeden stopień. Zastosowanie jednego poziomu Wysiłku obniża trudność o jeden stopień, dwóch poziomów Wysiłku – o 2 stopnie itp. Jednakże, każdy dodatkowy poziom Wysiłku po pierwszym kosztuje 2 punkty z Puli statystyki zamiast 3. Tak więc zastosowanie dwóch poziomów Wysiłku kosztuje 5 punktów (3 za 1-szy poziom plus 2 za 2-gi), zastosowanie trzech poziomów Wysiłku kosztuje 7 punktów z Puli (3 plus 2 plus 2) itp.

Każda postać posiada statystykę zwaną Wysiłek, która określa maksymalny poziom Wysiłku, który dana postać może zastosować. Początkująca (1-szo poziomowa) postać ma Wysiłek 1, co oznacza, że może zastosować 1 poziom Wysiłku w rzucie. Bardziej doświadczone postaci mają wyższy Wysiłek i mogą stosować więcej poziomów Wysiłku. Dla przykładu, postać, której Wysiłek wynosi 3 może zastosować 3 poziomy Wysiłku, by zredukować trudność rzutu.

Kiedy stosujesz Wysiłek, odejmij swoje Skupienie w odpowiedniej statystyce od całościowego kosztu Wysiłku w punktach z Puli. Dla przykładu, wykonujesz rzut na Szybkość. By zwiększyć szansę na sukces, decydujesz się zastosować 1 poziom Wysiłku, co ułatwi zadanie. Normalnie, kosztowałoby to 3 punkty z Puli Szybkości. Jednakże, masz Skupienie w Szybkości na 2, więc odejmujesz tą liczbę od kosztu. W efekcie, Wysiłek kosztuje Cię tylko 1 punkt z Twojej Puli Szybkości.

Co, gdybyś zastosował dwa poziomy Wysiłku do rzutu, zamiast tylko jednego? To by ułatwiło zadanie o dwa stopnie. Normalnie, kosztowałoby to 5 punktów z Puli, ale po odjęciu Skupienia w Szybkości o wartości 2, finalny koszt to tylko 3 punkty.

Kiedy Skupienia w statystyce osiąga 3, możesz stosować jeden poziom wysiłku za darmo. Dla przykładu, jeśli masz Skupienia w Szybkości na 3 i stosujesz 1 poziom Wysiłku na rzucie na Szybkość, będzie Cię to kosztowało 0 punktów z Twojej Puli Szybkości. (Normalnie, jeden poziom Wysiłku kosztuje 3 punkty, ale po odjęciu Skupienia w Szybkości od tego numeru, redukujemy je do 0.).

Umiejętności i inne przewagi także ułatwiają zadania, i można z nich skorzystać razem z Wysiłkiem. Dodatkowo, Twoja postać może mieć specjalne zdolności lub ekwipunek, które mogą pozwolić Ci wykorzystać Wysiłek do specjalnych zadań, takich jak przewrócenie przeciwnika przy pomocy ataku lub wpłynięcie na wiele celów przy pomocy mocy, która zazwyczaj dotyczy tylko jednej osoby.

(Kiedy stosujesz Wysiłek w walce wręcz, masz opcję wydania punktów albo z Puli Szybkości, albo z Puli Mocy. Kiedy wykonujesz atak dystansowy, możesz wydać punkty tylko z Puli Szybkości. Ta zasada odzwierciedla fakt, że w walce wręcz czasem korzysta się z brutalnej siły, a czasami z finezji, ale w atakach dystansowych zawsze chodzi o dobre wycelowanie.)

\subsection{Wysiłek i obrażenia}\index{Pule, Skupienie i Wysiłek!Wysiłek i obrażenia}

Zamiast stosować Wysiłek, by ułatwić atak, można go zastosować, by zwiększyć obrażenia zadawane w tym ataku. Na każdy poziom Wysiłku, który się stosuje w ten sposób, zadaje się dodatkowe 3 punkty obrażeń. To działa dla każdego rodzaju ataku, który zadaje obrażenia, niezależnie od tego, czy to miecz, kusza, psioniczny atak czy coś jeszcze innego.

Kiedy korzystasz z Wysiłku, by zwiększyć obrażenia ataku obszarowego, takiego jak eksplozji wywołanej przez zdolność Adepta \mytext{Wybuch}, zadajesz dodatkowe 2 punkty obrażeń zamiast 3. Jednakże, dodatkowe punkty są zadawane wszystkim celom w obszarze działania zdolności. Dodatkowo, nawet jeśli jeden lub więcej celów nie ponosi obrażeń w wyniku tego konkretnego ataku (ze względu na nieudany rzut na atak), dalej otrzymują oni 1 punkt obrażeń.

\subsection{Wiele użyć Wysiłku i Skupienia}\index{Pule, Wysiłek i Skupienie!Wiele użyć Wysiłku i Skupienia}

Jeśli Twój Wysiłek wynosi 2 lub więcej, możesz zastosować Wysiłek na wiele sposobów w jednej akcji. Dla przykładu, jeśli wykonujesz atak, możesz zastosować Wysiłek, by ułatwić atak i by zadać więcej obrażeń.

Totalny Wysiłek, z którego korzystasz, nie może być większy od Twojej wartości Wysiłku. Dla przykładu, jeśli Twój wysiłek to 2, możesz zastosować dwa poziomy Wysiłku. Możesz wykorzystać jeden z nich, by ułatwić atak, a drugi, by zwiększyć jego obrażenia, by ułatwić atak o dwa stopnie, ale nie zwiększać obrażeń, lub by nie ułatwiać ataku, ale zwiększyć obrażenia dwukrotnie.

Możesz wykorzystać Skupienie danej statystyki tylko jeden raz na akcję. Dla przykładu, jeśli stosujesz Wysiłek na ataku Mocy i zwiększasz obrażenia oraz ułatwiasz cios, możesz skorzystać z Skupienia w Mocy, by obniżyć koszt jednego z tych zastosować Wysiłku, ale nie dwóch. Jeśli wydasz 1 punkt Intelektu na aktywowanie ataku psionicznego i jeden poziom Wysiłku na ułatwienie ataku, możesz skorzystać z Skupienia w Intelekcie do jednej z tych rzeczy, ale nie obydwu.

\section{Przykładowe Statystyki}

Początkująca postać walczy z wielkim szczurem. BG rzuca się ze swoją włócznią na tego szczura, który jest istotą 2 poziomu i w związku z tym jego Stopień Trudności wynosi 6. Postać stoi wyżej od szczura i atakuje go z góry i MG uznaje, że to dobra taktyka i przyznaje atut który ułatwia atak o jeden stopień (trudność wynosi teraz 1). To obniża Stopień Trudności do 3. Atak włócznią bazuje na Mocy – postać ma Pulę Mocy 11 i Skupienie w Mocy 0. Przed wykonaniem rzutu, decyduje się ona zastosować poziom Wysiłku, by ułatwić atak. To kosztuje 3 punkty z Puli Mocy, redukując jej obecną wartość do 8. Ale te punkty są dobrze wydane. Obniża to trudność z 1 do 0, więc nie ma potrzeby, by wykonać rzut – atak automatycznie trafia.  

Inna postać chce przekonać strażnika, by pozwolił jej wejść do prywatnego biura w celu rozmowy z ważnym szlachcicem. MG oznajmia, że jest to akcja Intelektu. Postać jest na 3 poziomie i ma Wysiłek 3, Pulę Intelektu 13 i Skupienie w Intelekcie 1. Przed wykonaniem rzutu, gracz musi zadecydować, czy stosuje Wysiłek. Może on zastosować 1, 2 lub 3 poziomy Wysiłku, lub też nie zastosować żadnego. Ta akcja jest dla niego ważna, więc decyduje on się na zastosowanie 2 poziomów Wysiłku, ułatwiając zadanie o 2 stopnie. Dzięki Skupieniu w Intelekcie, płaci on tylko 4 punkty z Puli Intelektu (3 punkty za pierwszy poziom Wysiłku, plus 2 punkty za drugi poziom, minus 1 z Skupienia). Wydanie tych punktów redukuje jego Pulę Intelektu do 9. MG uznaje, że przekonanie strażnika jest zadaniem poziomu 3 (wymagającym) z Stopniem Trudności 9; dwa poziomy Wysiłku obniżają trudność zadania do poziomu trudności 1 (łatwe) a Stopień Trudności do 3. Gracz rzuca k20 i otrzymuje 8. Ponieważ ten wynik to co najmniej Stopień Trudności zadania, postać odnosi sukces. Jednakże, gdyby nie zastosowała ona żadnego Wysiłku, odniosłaby porażkę, ponieważ jej rzut (8) byłby mniejszy niż Stopień Trudności (9).

\section{Poziomy postaci}\index{Poziomy postaci}

Każda postać zaczyna grę na 1-szym poziomie. Poziom mierzy moc, wytrzymałość i zdolności postaci. Postaci awansują do 6 poziomu. Gdy postać osiąga wyższe poziomy, uzyskuje ona więcej zdolności, zwiększa swój Wysiłek, i może zwiększyć Skupienie lub liczbę punktów w Pulach. Ogolnie mówiąc, nawet postaci na 1-szym poziomie są całkiem nieźle uzdolnione. Można spokojnie założyć, że mająjuż jakieś doświadczenie. To nie jest sytuacja “od zera do bohatera”.ale raczej przykład kompetentnych ludzi polepszających swoje zdolności i wiedzę. Awansowanie na wyższe poziomy nie jest tak naprawdę celem postaci w Cypher System, lecz raczej reprezentuje to, jak postać zmienia siew czasie przygód.

Aby awansować na następny poziom, postaci muszą zyskać Punkty Doświadczenia przez wypełnianie celów postaci, uczestniczenie w przygodach i odkrywanie nowych rzeczy – ten system traktuje o zarówno odkryciach, jak i eksploracji, a także o osiąganiu osobistych celów. Punkty doświadczenia mają wiele zastosowań, a jedno z nich to zakupywanie korzyści dla postaci. Po zakupie czterech korzyści, postać awansuje na następny poziom. Każda korzyść kosztuje 4 PD-ki i można je kupować w dowolnej kolejności, ale trzeba zakupić każdy z nich (a następnie awansować na następny poziom) zanim można zakupić tę samą korzyść ponownie. Cztery korzyści postaci to:

\begin{itemize}
    \item Zwiększenie Zdolności: Dodaj 4 do swoich Pul. Możesz wybrać dowolną ilość z tych punktów na dowolne Pule.
    \item Zbliżanie się do Doskonałości: Zwiększ o 1 Skupienie w Mocy, Szybkości bądź Intelekcie (ty decydujesz).
    \item Dodatkowy Wysiłek: Zwiększ swoją wartość Wysiłku o 1.
    \item Umiejętności: Uzyskujesz trening w jednej umiejętności swojego wyboru, innej niż atak lub obrona. 
\end{itemize}    

Jak zapisano w Zasadach Gry, postać wytrenowana w danej umiejętności ułatwia powiązane z nią zadania o jeden stopień. Możesz wybrać dowolną umiejętność, której sobie zażyczysz, taką jak wspinaczka, skakanie, perswazja lub skradanie się. Możesz także wybrać jakąś dziedzinę wiedzy, taką jak historia lub geologia. Możesz nawet wybrać umiejętność bazującą na specjalnych zdolnościach swojej postaci. Dla przykładu, jeśli Twoja postać może uderzyć w przeciwnika mocą mentalną, możesz być wytrenowany w korzystaniu z tej zdolności, ułatwiając zadanie korzystania z niej. Jeśli wybierzesz umiejętność, w której już jesteś wytrenowany, zyskujesz w niej specjalizację, ułatwiając związane z nią rzuty o dwa stopnie zamiast jednego.

(Umiejętności to szeroka kategoria rzeczy, których postać może sienauczyć i wykonać. Patrz poniżej po przykładową listę umiejętności).

\begin{itemize}
        \item Inne Opcje: Gracze mogą także wydać 4 PD-ki na zakupienie innych opcji, zamiast zyskać nową umiejętność. Zakupienie dowolnej opcji z poniższej listy liczy się jak Umiejętność celem awansowania na następny poziom. Opcje speclajne to:
        \item Obniżenie kosztu noszenia zbroi. Ta opcja obniża koszt noszenia zbroi o 1.
        \item Dodaj 2 do swoich rzutów na odzyskiwanie zdrowia. 
        \item Wybierz nową zdolność zapewnianą przez swój typ, z obecnego poziomu lub niższego.
\end{itemize}

\section{Deskryptory, Type i Specjalizacje postaci}

By stworzyć postać, tworzysz proste zdanie, które ją opisuje. To zdanie przyjmuje następującą formę: “Jestem [umieść tutaj przymiotnik] [umieść tutaj rzeczownik] który [umieść tutaj czasownik].”

Tak więc powstaje zdanie “Jestem przymiotnik opisujący rzeczownik który czasownikuje”. Dla przykładu, możesz powiedzieć “Jestem Dzikim Wojownikiem który Kontroluje Bestie” lub “Jestem Czarującym Poszukiwaczem, który Stawia Umysł Ponad Materią”. 

W tym zdaniu, przymiotnik nazywany jest \mytext{deskryptorem}.

Rzeczownik to \mytext{typ} twojej postaci.

Czasownik jest nazywany \mytext{specjalizacją}.

Pomimo tego, że typ postaci znajduje się w środku zdania, to od niego zaczniemy. (Tak jak w zdaniu, rzeczownik jest podstawą).
Twój typ postaci to jądro twojej postaci. W niektórych grach fabularnych, można nim nazwać klasę postaci. Twój typ pozwala Ci określić miejsce Twojej postaci w świecie i jej relacje z innymi ludźmi w settingu. To jest rzeczownik w zdaniu “Jestem przymiotnik opisujący rzeczownik który czasownikuje”.

Możesz wybrać z czterech typów postaci – \mytext{Wojownika, Adepta, Poszukiwacza} lub \mytext{Mówcy}.

Twój deskryptor definiuje postać – wpływa na wszystko, co robisz. Twój deskryptor umieszcza Twoją postać w pewnej sytuacji (pierwszej przygodzie na początku kampanii) i pomaga zapewnić jej motywację. To przymiotnik w zdaniu “Jestem przymiotnik opisujący rzeczownik który czasownikuje”.
Jeśli Twój MG nie powie inaczej, możesz wybrać dowolny z deskryptorów postaci.

Specjalizacja to to, co Twoja postać robi najlepiej. Specjalizacja nadaje Twojej postaci specyficzność i zapewnia interesujące nowe zdolności, które mogą się przydać. Twoja Specjalizacja także pomaga Ci zrozumieć Twoje miejsce w grupie BG. Jest to czasownik w zdaniu “Jestem przymiotnik opisujący rzeczownik który czasownikuje”.

Istnieje wiele specjalizacji postaci. Twój wybór zależeć będzie zapewne od settingu i opowiadanej historii.

(Możesz wykorzystać Smaczki z odpowiedniego rozdziału, żeby zmodyfikować typy postaci tak, by pasowały do odmiennych sesji.)

\section{Specjalne Zdolności}\index{Zdolności!Wstęp}

Typ postaci i specjalizacja zapewniają BG specjalne zdolności na każdym nowym poziomie. Korzystanie z tych zdolności zazwyczaj kosztuje punkty z Puli statystyk; koszt podano w nawiasie po nazwie zdolności. Twoje Skupienie w odpowiedniej statystyce może obniżyć jej koszt, ale pamiętaj, że możesz stosować Skupienie tylko raz na akcję. Dla przykładu, powiedzmy, że Adept z Skupieniem w Intelekcie 2 chce skorzystać z zdolności Blast, aby stworzyć ładunek energii, co kosztuje 1 punkt Intelektu. Chce on także zwiększyć obrażenia z ataku korzystając z Wysiłku, co kosztuje 3 punkty Intelektu. Całościowy koszt tej akcji wynosi 2 punkty z Puli Intelektu (1 punkt za pocisk, plus 3 punkty za skorzystanie z Wysiłku, minus 2 punkty z Skupienia).

Czasami koszt umiejętności ma znak plusa (+) po liczbie. Dla przykładu, koszt może zostać podany jako “2+ punktów Intelektu”. To oznacza, że można wydać więcej punktów lub więcej poziomów Wysiłku, by ulepszyć zdolność, co wyjaśniono w jej opisie.

Wiele specjalnych zdolności daje postaci opcję zrobienia czegoś, czego normalnie nie mogłaby wykonać, jak np.: wytwarzanie promieni zimna lub atakowanie wielu celów naraz. Używanie takich zdolności jest akcją samą w sobie, a koniec opisu zdolności zawiera słowo “Akcja” aby o tym przypomnieć. Może on także zapewnić więcej informacji o tym kiedy lub jak wykonać ową akcję. 

Pewne specjalne zdolności pozwalają Ci wykonać znaną już akcję – akcję, którą można wykonać i bez tego – w odmienny sposób. Dla przykładu, zdolność może Ci pozwolić nosić ciężką zbroję, obniżyć trudność rzutów obronnych na Szybkość, lub dodać 2 punkty ognistych obrażeń do Twoich obrażeń od broni. Te zdolności nazywa się umożliwiaczami. Korzystanie z nich nie jest uważane za akcję. Umożliwiacze albo działają ciągle (np.: możliwość noszenia ciężkiej zbroi, co nie jest akcją) lub dzieją się jako część innej akcji (np.: dodawanie obrażeń od ognia do Twoich ataków, co jest częścią akcji ataku). Jeśli specjalna zdolność jest umożliwiaczem, na końcu jej opisu znajduje się słowo “Umożliwiacz” aby o tym przypomnieć.

Pewne zdolności określają swoją długość, ale zawsze możesz zakończyć wcześniej dowolną ze swoich zdolności, jeśli tylko sobie tego życzysz.
(Ponieważ Cypher System to uniwersalny system i dotyczy wielu gatunków, nie zawsze wszystkie deksryptory, typy i specjalizacje mogą być dostępne graczom. MG decyduje co jest dostępne w tej konkretnej grze i czy coś jest zmodyfikowane – poinformuje on o tym swoich graczy.)


\section{Umiejętności}\index{Umiejętności}

Czasami Twoja postać zyskuje trening w specyficznej umiejętności lub zadaniu. Przykładowo, Twoja specjalizacja może oznaczać, że jesteś wytrenowany w skradaniu się, wspinaczce i skakaniu, lub społecznych interakcjach. Innym razem, Twoja postać może wybrać umiejętność, w której jest wytrenowana, i wybierasz taką umiejętność, o której myślisz, że może się przydać w przyszłości.

Cypher System nie ma definitywnej, danej raz na zawsze listy umiejętności. Jednakże, poniżej jest trochę sugestii:

\begin{itemize}
     \item  Astronomia
    \item Utrzymywanie równowagi
    \item Biologia
    \item Botanika
    \item Noszenie ciężarów
    \item Wspinaczka
    \item Komputery
    \item Kłamstwo
    \item Przebrania
    \item Ucieczka
    \item Geografia
    \item Geologia
    \item Leczenie
    \item Historia
    \item Identyfikacja
    \item Inicjatywa
    \item Zastraszanie
    \item Skakanie
    \item Obrabianie skóry
    \item Otwieranie zamków
    \item Maszyny
    \item Kowalstwo
    \item Percepcja
    \item Perswazja
    \item Filozofia
    \item Fizyka
    \item Kradzież kieszonkowa
    \item Pilotowanie
    \item Naprawy
    \item Jeździectwo
    \item Niszczenie
    \item Skradanie się
    \item Pływanie
    \item Prowadzenie pojazdów
    \item Obróbka drewna
 \end{itemize}  
   
Możesz wybrać umiejętność, która obejmuje parę z tych aktywności (interakcje społeczne mogą pokrywać oszustwo, zastraszanie i perswazję) lub bardziej wąskie (ukrywanie sięjako skradanie się, gdy się nie rusza). Możesz też wymyślić umiejętności-profesje, takie jak piekarz, marynarz lub drwal. Jeśli chcesz wybrać umiejętność, której nie ma na liście, najpewniej najlepiej zapytać najpierw swojego MG, ale ogólnie, najważniejsze to taki wybór umiejętności, który pasuje do konceptu postaci.

Pamiętaj, że jeśli zyskujesz umiejętność, w której już jesteś wytrenowany, zostajesz w niej wyspecjalizowany. Ponieważ opisu umiejętności może być nieco mylny, stwierdzenie ,czy jesteś wytrenowany czy wyspecjalizowany może zająć nieco czasu. Dla przykładu, możesz być wytrenowany w kłamaniu,a poźniej dostać umiejętnośc do wszystkich społecznych interakcji, co oznacza, że Twoje kłamstwa są wyspecjalizowane, a inne aktywności społeczne są tylko wytrenowane. Bycie wytrenowanym 3 razy w umiejętności nie jest lepsze niż bycie wytrenowanym tylko 2 razy (innymi słowy, wyspecjalizowany to najlepszy możliwy poziom).

Tylko umiejętności pozyskane przez typ lub inne rzadkie przypadki pozwalają Ci być wyuczonym w ataku lub obronie.

Jeśli pozyskasz specjalną zdolność przez swój typ, specjalizację lub inny aspekt Twojej postaci, możesz wybrać ką w miejsce umiejętności i być wytrenowanym lub wyspecjalizowanym w danej specjalniech zdolności. Dla przykładu, jeśli masz atak mentalny, a nadchodzi czas na wybór umiejętności, możesz wybrać umiejętność w ataku mentalnym. Ułatwiłoby to ten atak za każdym razem, gdy jest używany. Każda zdolność, którą posiadasz, może mieć swoją własną umiejętność w tym właśnie celu. Nie możesz wybrać “wszystkich mocy mentalnych” lub “wszystkich zaklęć” jako jednej umiejętności i być w niej wytrenowany lub wyspecjalizowany, gdyż ta kategoria jest stanowczo zbyt szeroka.

W większości kampanii, biegłość w języku jest uważana za umiejętność. Wiec jeśli chcesz mówić po francusku, jest to traktowane tak samo jak bycie wytrenowanym w biologii lub pływaniu.
\section{Typ}\index{Typ}

Typ postaci to najważniejsza cecha Twojej postaci. Twój typ pozwala określić miejsce postaci w świecie i jej relację z innymi ludźmi. Jest to rzeczownik w zdaniu “Jestem przymiotnik rzeczownik który czasownikuje”.

(W pewnych grach RPG, typ postaci może zostać nazwany klasą postaci.)

Możesz wybrać z 4 typów postaci: Wojownika, Adepta, Odkrywcy i Mówcy. Jednakże, możesz nie chcieć korzystać z tych ogólnikowych nazw na nie. Ten rozdział oferuje parę bardziej specyficznych nazw na każdy typ, które mogą być stosowne, w zależności od świata przedstawionego. Odkryjesz, że nazwy takie jak “Wojownik” czy też “Odkrywca” nie zawsze pasują do gier dziejących się w świecie współczesnym. Jak zawsze, możesz zrobić, co uznasz za stosowne. (Twój typ określa kim jest postać. Powinieneś korzystać z dowolnej nazwy na typ, tak długo, jak pasuje zarówno do postaci, jak i do settingu.)

Ponieważ typ to podstawa, na której się buduje postać, warto się zastanowić, jaka relacja łączy go z settingiem. Aby z tym pomóc, typy to w zasadzie ogólne archetypy. Wojownik, dla przykładu, może być wszystkim, od rycerza w lśniącej zbroi, przez gliniarza na ulicy po cybernetycznego weterana tysiąca futurystycznych wojen.
Aby lepiej dostosować cztery typy do różnych settingów, istnieją różne metody zwane posmakami,  zaprezentowane w stosownym rozdziale, by pomóc w dostosowaniu typów do konwencji fantasy, science fiction, lub innych (lub by dostosować typy do pomysłu na postać).

Dalej, bardziej fundamentalne opcje dla \mytext{dalszej customizacji} są dostępne na końcu tego rozdziału. 

\subsubsection{Wtrącenie Gracza}\index{Wtrącenie Gracza}

Wtrącenie gracza oznacza, że gracz wybiera zmianę czegoś w kampanii, czyniąc rzeczy łatwiejszymi dla jego postaci. Konceptualnie, jest to przeciwieństwo wtrącenia MG: zamiast MG dawać PD graczowi i wprowadzać niespodziewaną komplikacje dla jego postaci, gracz wydaje 1 PD i wprowadza rozwiązanie problemu lub komplikacji. To, co może zrobić wtrącenie gracza, to zmienić świat gry lub obecne okoliczności zamiast bezpośrednio zmieniać postać. Dla przykładu, wtrącenie mówiące, że cypher, z którego właśnie się skorzystało, ma dodatkowe użycie byłoby właściwe, ale wtrącenie uzdrawiające postać nie byłoby. Jeśli gracz nie ma PD-ków do wydania, nie może wprowadzić wtrącenia gracza. 

Parę wtrąceń gracza jest zasugerowanych pod każdym typem. Warto jednak zaznaczyć, że nie każde wtrącenie gracza jest stosowne w każdej sytuacji. MG może zezwolić graczom na inne sugestie wtrąceń, ale ostatecznie to on decyduje, czy dane wtrącenie jest stosowne do typu postaci i danej sytuacji. Jeśli MG odmawia wtrącenia, gracz nie wydaje 1 PD-ka i wtrącenie nie następuje.

Korzystanie z intruzji nie wymaga od postaci akcji, by je zastosować. Po prostu ono następuje.

(Wtrącenie gracza powinno być ograniczone do nie więcej niż jednego wtrącenia na gracza na jedną sesję.)

\subsubsection{Akcje obronne}\index{Akcje obronne}

Akcje obronne występują wtedy, gdy gracz rzuca, by uchronić się od czegoś nieporządanego, co mogłoby się wydarzyć jego BG. Rodzaj akcji obronnej ma znaczenie, gdy rozważamy Wysiłek.

\textbf{Obrona Mocy}: Używa się jej do odporności na trucizny, choroby i wszystko inne, co można przezwyciężyć siłą i zdrowiem.

\textbf{Obrona Szybkości}: Używa się jej do unikana ciosów i uciekana od niebezpieczeństw. To najczęściej wykorzystywany rodzaj akcji obronnej.

\textbf{Obrona Intelektu}: Używa się jej do odpieranie ataków mentalnych i wszystkiego, co może wpłynać na czyiś umysł.
\cleardoublepage

\subsection{Wojownik}\index{Typ!Wojownik}

Fantasy/Baśń: Wojownik, rycerz, barbarzyńca, żołnierz, walkyria.

Współczesność/Horror/Romans: policjant, żołnierz, strażnik, detektyw, ochroniarz, atleta.

Science fiction: oficer bezpieczeństwa, wojownik, żołnierz, najemnik.

Superbohaterowie/Post-apokalipsa: bohater. 

Jesteś dobrym sprzymierzeńcem w potyczce. Wiesz, jak korzystać z broni i chronić siebie. W zależności od konwencji i settingu, może to znaczyć, że nosisz miecz i tarczę na arenie gladiatorów, posiadasz karabin maszynowy i zestaw granatów przydatne w wymianie ognia, lub posiadasz blastera i zasilany pancerz, z których korzystasz na obcej planeci

Rola w grze: Wojownicy są fizyczni i zorientowani na akcję. Cześciej rozwiązują problemy, korzystając z siły, niż na inne sposoby, i często wybierają najprostszą drogę do osiągnięcia swoich celów. 

Rola w drużynie: Wojownicy najczęściej zadają i biorą na klatę najwięcej obrażeń w bitwie. To od nich zależy obrona reszty członków drużyny przed atakami. To czasami oznacza, że wojownicy bywają liderami, przynajmniej w walce i w obliczu innych niebezpieczeństw.

Rola społeczna: Wojownicy nie zawsze są żołnierzami lub najemnikami. Każdy, kto jest gotów do odrobinę przemocy w swoim życiu, lub choćby jej potencjał, może być Wojownikiem, mówiąc ogólnikowo. Wliczają się w to strażnicy, policjanci, marynarze lub ludzie innych profesji, którzy wiedzą, jak się bronić.

Zaawansowani Wojownicy: W miarę, jak wojownicy awansują na poziomy, ich umiejętności bitewne – zarówno obrony, jak i ataku – zwiększają się do niemożliwych poziomów. Na wyższych poziomach, mogą oni często przeciwstawić się grupom wrogów lub stanąć 1-na-1 przeciwko dowolnemu przeciwnikowi.

\subsubsection{Historia Wojownika}

Twój typ pomaga Ci określić połączenie Twojej postaci z settingiem. Rzuć k20 lub wybierz z poniższej listy, by określić pewien fakt odnośnie Twojej historii, który łączy Twoją postać ze światem. Możesz także stworzyć swój własny fakt historyczny.

\begin{table*}[t]
 \centering
 \begin{tabularx}{\textwidth}{| p{0.10\textwidth} | X |}
  \hline
  \textbf{k20} & \textbf{Historia Wojownika}  \\ \hline
    1 & Byłeś w armii i dalej masz przyjaciół, którzy tam są. Twój były dowódca dobrze Cię pamięta. \\ \hline
    2 & Byłeś ochroniarzem bogatej kobiety, która oskarżyła Cię o kradzież. Opuściłeś jej służbę w cieniu podejrzeń. \\ \hline
    3 & Byłeś ochroniarzem lokalnego baru, i właściciele pamiętają Cię.  \\ \hline
    4 & Trenowałeś z szanowanym mentorem. Trzyma on Cię w estymie, ale ma wielu wrogów. \\ \hline
    5 & Trenowałeś w odosobnionym zakonie. Mnisi myślą o Tobie jak o bracie, ale dla wszystkich innych jesteś obcym. \\ \hline
    6 & Nie masz formalnego wyszkolenia. Twoje zdolności po prostu są (naturalnie bądź nie). \\ \hline
    7 & Spędziłeś czas na ulicach i byłeś przez pewien czas w więzieniu. \\ \hline
    8 & Zapisano Cię do służby wojskowej, ale uciekłeś po niedługim czasie. \\ \hline
    9 & Służyłeś jako ochroniarz dla potężnego kryminalisty, który teraz jest Ci winny życie. \\ \hline
    10 & Pracowałeś jako oficer policji lub detektyw. Każdy Cię zna, ale opinie o Tobie są różne. \\ \hline
    11 & Twoje starsze rodzeństwo to niesłynna postać, która żyje w hańbie.  \\ \hline
    12 & Służyłeś jako strażnik komuś, kto dużo podróżował. Znasz ludzi w wielu miejscach. \\ \hline
    13 & Twój najlepszy przyjaciel to nauczyciel lub badacz. Jest on świetnym źródłem wiedzy. \\ \hline
    14 & Ty i Twój przyjaciel palicie ten sam rodzaj rzadkiego, drogiego tytoniu. Spotykacie się co tydzień, by porozmawiać i zapalić. \\ \hline
    15 & Twój wuj prowadzi teatr w mieście. Znasz wszystkich aktorów i masz wolny wstęp na występy. \\ \hline
    16 & Twój przyjaciel-rzemieślnik czasami prosi Cię o pomoc. Jednakże, płaci on dobrze. \\ \hline
    17 & Twój mentor napisał książkę o sztukach walki. Czasami ludzie pragną Cię odszukać i zapytać o jej dziwne zapisy. \\ \hline
    18 & Ktoś, z kim walczyłeś ramię w ramię w armii, teraz jest burmistrzem lokalnego miasteczka. \\ \hline
    19 & Ocaliłeś życie rodziny, gdy jej dom płonął. Mają oni u Ciebie dług, a ich sąsiedzi traktują Ciebie jak bohatera. \\ \hline
    20 & Twój stary trener dalej spodziewa się, że wrócisz i sprzątniesz po jego zajęciach; gdy to robisz, dzieli się on z Tobą okazjonalnie ciekawymi plotkami. \\ \hline
 \end{tabularx}
  \caption {Historia Wojownika}
  \label {Historia Wojownika}
 \end{table*}
 
\subsubsection{Wojownik - Wtrącenia Gracza}

Możesz wydać 1 PD by skorzystać z poniższych wtrąceń gracza, jeśli jest to stosowne do sytuacji, a MG się zgodzi.

Perfekcyjna pozycja: Walczysz przynajmniej z trzema wrogami i każdy z nich stoi w odpowiednim miejscu, możesz więc wykorzystać ruch, który ćwiczyłeś dawno temu, co pozwala Ci zaatakować wszystkich trzech w jednej akcji. Wykonaj odrębne rzuty na atak dla kazego z wrogów. Jesteś ograniczony Wysiłkiem, który możesz wykorzystać w jednej akcji.

Stary Przyjaciel: Towarzysz borni z przeszłości pojawia sięnagle i pomaga w tym, co teraz robisz. Jest on na własnej misji i nie może zostać dłużej niż czas potrzebny na udzielenie pomocy, porozmawianie przez chwilę i być może na wspólne zjedzenie szybkiego posiłku.

Słabość Broni: Broń Twojego przeciwnika ma słaby punkt. Podczas walki, szybko się ona psuje i spada o dwa stopnie w dół na \mytext{liczniku obrażeń przedmiotu}.

\begin{table*}[t]
 \centering
 \begin{tabularx}{\textwidth}{ | X | X |}
  \hline
   \textbf{Statystyka} & \textbf{Początkowa Wartość Puli}  \\ \hline
    Moc & 10  \\ \hline
    Szybkość & 10  \\ \hline
    Intelekt & 8  \\ \hline
 \end{tabularx}
  \caption {Pule Statystyk Wojownika}
  \label {Pule Statystyk Wojownika}
 \end{table*}
 
 Otrzymujesz dodatkowe 6 punktów do podziału pomiędzy Pule, jakkolwiek sobie życzysz.
 
\subsubsection{Wojownik pierwszego poziomu}

Pierwszo-poziomowi wojownicy mają następujące zdolności:

Wysiłek: Twój Wysiłek to 1.

Fizyczna Natura: Masz Skupienie w Mocy 1 i Skupienie w Szybkości 0 lub Skupienie w Mocy 0 i Skupienie w Szybkości 1. Niezależnie od tego, Twoje Skupienie w Intelekcie to 0.

Korzystanie z Cypherów: Możesz nosić dwa Cyphery w danym czasie.

Bronie: Jesteś wytrenowany w lekkich, średnich i ciężkich broniach i nie stosuje siędo Ciebie kara za używanie jakiegokolwiek rodzaju broni. Umożliwienie.

Początkowy Ekwipunek: Odpowiednie ubranie i dwie bronie Twojego wyboru, plus jeden drogi przedmiot, dwa przedmioty średniej ceny i cztery niedrogie.

Specialne Zdolności: Wybierz cztery zdolności z poniższej listy. Nie możesz wybrać tej samej zdolności więcej niż raz, chyba, że jej opis mówi inaczej. Pełny opis wszystkich zdolności znajduje się w rozdziale \mytext{Zdolności}, który także zawiera opis Posmaków i zdolności Specjalizacji w pojedynczym, sporym katalogu.

\begin{itemize}
\item Broń Niepotrzebna
\item Kontrola Bitewna
\item Na Straży
\item Ogłuszenie
\item Szybki Rzut
\item Ulepszone Skupienie
\item Umiejętności Fizyczne
\item Wyszkolony Bez Zbroi
\item Wyszkolony w Zbroi
\item Zamach
\item Zdolności Bojowe
\end{itemize}

\subsubsection{Wojownik Drugiego Poziomu}

Wybierz dwie zdolności z poniższej lisy (lub z niższego poziomu) i dodaj je do swoich zdolności. Dodatkowo, możesz zamienić jedną ze zdolności niższego poziomu na inną z niższego poziomu.

\begin{itemize}
\item Krwawienie
\item Miażdżący Cios
\item Następny Atak
\item Przeładowanie
\item Umiejętny Atak
\item Umiejętna Obrona
\end{itemize}

\subsubsection{Wojownik Trzeciego Poziomu}

Wybierz trzy zdolności z poniższej listy (lub z niższego poziomu) i dodaj je do swoich zdolności. Dodatkowo, możesz zamienić jedną ze zdolności niższego poziomu na inną z niższego poziomu.

\begin{itemize}
\item Chwytaj Moment
\item Cięcie
\item Cios z Wyciągnięciem
\item Czujność
\item Ekspercki Użytkownik Cypherów
\item Furia
\item Odporność na Energię
\item Ostrzał Ciągły
\item Podwójny Strzał
\item Przywykły do Noszenia Zbroi
\item Reakcja
\item Śmiertelna Salwa
\end{itemize}

\subsubsection{Wojownik Czwartego Poziomu}

Wybierz dwie z poniższych zdolności (lub z niższego poziomu) i dodaj je do swoich zdolności. Dodatkowo, możesz zamienić jedną z zdolności niższego poziomu na inną z niższego poziomu.

\begin{itemize}
\item Dodatkowy Wysiłek
\item Doświadczony Obrońca
\item Finta
\item Pęd
\item Przełamanie Obrony
\item Wycelowanie
\item Wyjątkowo Wytrzymały
\item Zręczny Wojownik
\item Zwiększony Efekt
\end{itemize}

\subsubsection{Wojownik Piątego Poziomu}

Wybierz trzy zdolności z poniższej listy (lub z niższego poziomu) i dodaj je do swoich zdolności. Dodatkowo, możesz zamienić jedną ze zdolności niższego poziomu na inną zdolność niższego poziomu.

\begin{itemize}
\item Atak z Wyskoku
\item Blok
\item Mistrzostwo Ataków
\item Mistrzostwo Obrony
\item Mistrzowska Biegłość w Pancerzach
\item Ulepszony Sukces
\item Potrójny Wystrzał
\item Zaawansowany Użytkownik Cypherów
\end{itemize}

(Pamiętaj, że na wyższych poziomach, można wybrać zdolności z niższych poziomów. Czasami jest to najlepszy sposób, by uzyskać dokładnie taką postać, jakiej pragniesz. Jest to zwłaszcza prawdziwe odnośnie zdolności, które zapewniają umiejętności, które zazwyczaj można wybrać więcej niż jeden raz.)

\subsubsection{Wojownik Szóstego Poziomu}

Wybierz dwie ze zdolności z poniższej listy (lub z niższego poziomu) i dodaj je do swoich zdolności. Dodatkowo, możesz zamienić jedną ze zdolności niższego poziomu na inną z niższego poziomu.

\begin{itemize}
\item Broń i Cios
\item Chwila Wspaniałości
\item Morderca
\item Ostateczny Cios
\item Wielokrotny Atak
\item Znowu i Znowu
\end{itemize}

\subsubsection{Przykładowy Wojownik}

Ray chce stworzyć Wojownika do współczesnej kampanii. Decyduje się on na byłego członka armii, który jest silny i szybki. 3 z wolnych punktów idą do Puli Mocy, a pozostałe 3 do Puli Szybkości. Jego Statystyki to teraz Moc 13, Szybkość 13 i Intelekt 8. Jako, że postać jest na 1-szym  poziomie, jej Wysiłek to 1, jej Skupienie w Mocy to 1, a Skupienie w Szybkości i Intelekcie to 0. Jego postać nie jest szczególnie mądra lub charyzmatyczna.

Chce on korzystać z dużego noża bojowego (średnia broń, która zadaje 4 punkty obrażeń) i .357 Magnum (ciężki pistolet, który zadaje 6 punktów obrażeń, ale wymaga dwóch rąk do korzystania). Ray decyduje się na nie noszenie żadnej zbroi, gdyż nie pasuje to do settingu, tak więc, jako swoją pierwszą zdolność wybiera \mytext{Wyszkolony Bez Zbroi}, co ułatwia jego Obronę Szybkości. Jako drugą zdolność wybiera \mytext{Zdolności Bojowe}, by zadawać większe obrażenia swoim wielkim nożem. 

Ray chce być zarówno szybki, jak i wytrzymały, wybiera więc \mytext{Ulepszone Skupienie}. Daje mu to Skupienie w Szybkości na 1. Jako ostatnią zdolność, wybiera \mytext{Umiejętności Fizyczne} i wybiera pływanie i bieganie. 

Wojownik może mieć przy sobie maksymalnie 2 cyphery. GM decyduje, że pierwszy cypher Raya to pigułka, która regeneruje 6 punków Mocy po połknęciu, a jego drugi cypher to mały, łatwy do ukrycia granat, który eksploduje jak ognista bomba, gdy go się rzuci, zadając 3 punkty obrażeń wszystkim w bliskim zasięgu. 

Ray dalej musi wybrać deskryptor i specjalizację. Przeglądając deskryptory, Ray wybiera \mytext{Silnego}, co zwiększa jego Pulę Mocy do 17. Jest także wytrenowany w skakaniu i niszczeniu przedmiotów. (Jeśli Ray wybrałby skakanie jako jedną ze swoich umiejętności fizycznych, teraz dzięki deskryptorowi byłby wyspecjalizowany w skakaniu, zamiast być wytrenowanym). Bycie Islnym daje też Ray’owi dodatkową średnią bądź ciężką broń. Wybiera kij baseballowy, który przechowuje w bagażniku swojego auta. 

Jako swoją specjalizację, Ray wybiera \mytext{Mistrzowsko Posługuje się Bronią}. Daje mu to kolejną broń wysokiej jakości. Wybiera dodatkowy nóż bojowy i pyta MG, czy może z niego korzystać w lewej ręce – nie do wykonywania ataków, lecz jako tarczę. To ułatwi jego rzuty na Obronę Szybkości, jeśli ma obydwie bronie w rękach (“tarcza” liczy się jako atut). MG się zgadza. Podczas gry, ciężko będzie trafić Wojownika Ray’a – kjest wytrenowany w rzutach na Obronę Szybkości, a jego dodatkowy nóż obniża rzuty o kolejny stopień. 

Dzięki jego specjalizacji, zadaje także dodatkowy 1 punkt obrażeń w walce swoją wybranąbronią. Teraz zadaje 6 punktów obrażeń swoim ostrzem. Postać Raya to śmiercionośny wojownik, zapewne rozpoczynający grę z reputacją jako walczący nożami. 

Jako swój motyw fabularny, Ray wybiera \mytext{Pokonać Wroga}. Ten wróg, Ray decyduje, to nikt inny jak jego stary przyjaciel z armii, który wszedł na ścieżkę zła.
\subsection{Adept}\index{Typ!Adept}

Fantasy/baśń: mag, czarodziej, czarnoksiężnik, kleryk, druid, jasnowidz, diabolista, dotknięty przez Fae.

Współczesność/Horror/Romans: psionik, okultysta, wiedźma, praktykujący magię, medium, szalony naukowiec.

Science fiction: psionik, telepata, jasnowidz, skanujący, ESP-er, abominacja.

Superbohaterowie/Post-apokalipsa: mag, czarownik, dzierżący moc, psionik, telepata.

Władasz mocami i zdolnościami poza ludzkim doświadczeniem, zrozumienie i czasami wiarą. Może to być magia, psionika, zdolności mutanta, lub po prostu skomplikowane urządzenia, w zależności od settingu. (“Magia” to termin, który stosujemy tutaj bardzo luźno. To termin na wszystkie wspaniałe, możliwie nadnaturalne rzeczy, które może zrobić Twoja postać, a inne nie mogą. Może to być skutek posługiwania się odpowiednim sprzętem, kontaktu z duchami, mutacji, psioniki, nanotechnologii lub innych źródeł.)

Rola w grze:Adepci to zazwyczaj inteligentni, myślący ludzie. Bardzo często myślą ostrożnie, zanim podejmą akcję i polegają na swoich nadnaturalnych zdolnościach.

Rola w drużynie: Adepci nie są potężni w bezpośredniej walce, choć często posiadają zdolności, które są wspaniałym uzupełnieniem zdolności bojowych ich towarzyszy, zarówno defensywnie, jak i ofensywnie. Czasami posiadają zdolności, które pomagają im przezwyciężać trudności i wyzwania. Dla przykładu, jeśli grupa musi się przedostać przez zamknięte drzwi, Adept może być w stanie je zniszczyć lub przeteleportować wszystkich na ich drugą stronę.

Rola społeczna: W settingach w których moce nadnaturalne są rzadkie, tajemnicze lub wywołują strach, Adepci są zazwyczaj także rzadcy i wywołujący strach. Pozostają wtedy w ukryciu. Kiedy jest inaczej, Adepci są częstsi i bardziej bezpośredni. Mogą nawet zostać liderami swoich społeczności.

Zaawansowani Adepci: Nawet na niższych poziomach, moce Adeptów zapierają dech w piersiach. Na wyższych poziomach, Adepci mogą dokonać prawdziwie wielkich czynów, które mogą przekształcić materię i środowisko wokół nich.
(Adepci prawie zawsze są paranormalni lub nadludzcy w jakimś sensie – czarodzieje, psionicy itp. Jeśli gra, w którą gracie, nie posiada takich postaci, Adept mógłby być szarlatanem, który udaje magiczne zdolności przy pomocy trików i ukrytych urządzeń, lub gadżeciarzem z “przydatnym paskiem” pełnym dziwnych narzędzi. Lub w Twoim świecie może nie być Adeptów. To także jest ok.)

\subsubsection{Adept - Wtrącenia Gracza}

Kiedy grasz Adeptem, możesz wydać 1 PD na jedne z poniższych wtrąceń gracza, jeśli sytuacja jest stosowna i MG się zgodzi.
 
Przydatna Awaria: Urządzenie, z którego korzysta się przeciwko Tobie, ulega awarii. Może ono zranić użytkownika lub jednego z jego sprzymierzeńców w ciągu jednej tury, lub aktywować dramatyczny i rozpraszający efekt uboczny, trwający parę tur.

Nagłe Olśnienie: Doświadczasz nagłego olśnienia, które zapewnia jasną odpowiedz lub sugeruje następne kroki w temacie ważnego pytanie, problemu lub przeszkody na Twojej drodze. 

Cudowna Aktywacja: Nieaktywne, zrujnowane lub najwyraźniej-zniszczone urządzenia chwilowo się aktywuje i wykonuje przydatną akcję w kontekście obecnej sytuacji. Może to kupić Ci trochę czasu na znalezienie lepszego rozwiązania, przezwyciężyć komplikację która wpływa na Twoje moce, lub po prostu umożliwić skorzystanie z zużytego cyphera lub artefaktu jeszcze raz. 

\begin{table*}[t]
 \centering
 \begin{tabularx}{\textwidth}{ | X | X |}
  \hline
  \textbf{ Statystyka} & \textbf{Początkowa Wartość Puli}  \\ \hline
    Moc & 7 \\ \hline
    Szybkość & 9 \\ \hline
    Intelekt & 12 \\ \hline
 \end{tabularx}
  \caption {Pule Statystyk Adepta}
  \label {Pule Statystyk Adepta}
 \end{table*}
 
 Otrzymujesz 6 dodatkowych punktów do podziału pomiędzy Pule statystyk, zgodnie z własną wolą.
 
\subsubsection{Historia Adepta}

Twój typ pomaga Ci określić Twoje miejsce w settingu. Rzuć k20 lub wybierz z poniższej listy, by określić konkretny fakt odnośnie Twojej historii, która łączy Cię z resztą świata. Możesz także stworzyć swój własny fakt. 

 \begin{table*}[t]
 \centering
 \begin{tabularx}{\textwidth}{| p{0.10\textwidth} | X |}
  \hline
  \textbf{d20} & \textbf{Historia Adepta}  \\ \hline
    1 & Służyłeś jako uczeń u Adepta, którego respektowało i bało się wielu ludzi. Teraz nosisz jego brzemię. \\ \hline
    2 & Studiowałeś w szkole słynącej z jej mrocznych nauczycieli i absolwentów. \\ \hline
    3 & Nauczyłeś się swoich zdolności w świątyni mało znanego boga. Jego kapłani i wierni, choć niezbyt liczni, respektują i adorują Twoje talenty i potencjał. \\ \hline
    4 & Kiedy podróżowałeś samotnie, ocaliłeś życie potężnej osoby. Ma ona względem Ciebie dług wieczności. \\ \hline
    5 & Twoja matka była potężnym Adeptem za życia, pomagała też ludziom w okolicy. Patrzą oni na Ciebie ciepło, ale także spodziewają się wiele po Tobie. \\ \hline
    6 & Wisisz pieniądze wielu ludziom i nie masz pieniędzy, by spłacić swój dług. \\ \hline
    7 & Zaliczyłeś gigantyczną klęskę w swoich początkowych studiach z nauczycielem i teraz uczysz się na własną rękę. \\ \hline
    8 & Nauczyłeś się swoich zdolności szybciej, niż Twoi nauczyciele widzieli u któregokolwiek ze swoich uczniów. Potężni tego świata zwrócili na Ciebie swoją uwagę i obserwują Cię intensywnie.  \\ \hline
    9 & Zabiłeś dobrze znanego kryminalistę w samoobronie, zyskując respekt wielu i nieprzyjaźń paru niebezpiecznych ludzi. \\ \hline
    10 & Uczyłeś się na Wojownika, ale Twoje uzdolnienia w kierunku Adepta ostatecznie skierowały Cię na odmienną ścieżkę. Twoi dawni kompani nie rozumieją Cię, ale mimo to Cię szanują. \\ \hline
    11 & Kiedy studiowałeś na Adepta, pracowałeś jako asystant w banku, zaprzyjaźniajac się z właścicielem i klientami. \\ \hline
    12 & Twoja rodzina posiada wielką winnicę niedaleko, znaną ze swojego dobrego wina i uczciwości biznesowej. \\ \hline
    13 & Trenowałeś przez pewien czas z grupą wpływowych Adeptów, którzy dalej darzą Cię przyjaźnią. \\ \hline
    14 & Pracowałeś w ogrodach pałacowych wpływowego szlachcica lub bogatej osoby. Nie pamięta ona Cię, ale zaprzyjaźniłeś się z jej młodą córką. \\ \hline
    15 & Eksperyment, który przeprowadziłeś w przeszłości, kompletnie nie wypalił. Ludzie z tamtej okolicy zapamiętali Cię jako niebezpiecznego i bezmyślnego typka. \\ \hline
    16 & Pochodzisz z dalekiego miejsca, gdzie byłeś dobrze znany i traktowany, ale ludzie tutaj traktują Cię z dużą podejrzliwością. \\ \hline
    17 & Ludzie, których spotykasz, wydają się trzymać na dystans ze względu na dziwne piętna na Twojej twarzy. \\ \hline
    18 & Twój najlepszy przyjaciel to także Adept. Ty i Twój przyjaciel dzielicie się odkryciami i sekretami. \\ \hline
    19 & Znasz lokalnego kupca bardzo dobrze. Ponieważ zapewniłeś mu dużo przychodu, oferuje Ci on zniżki i specjalne traktowanie.  \\ \hline
    20 & Należysz to sekretnego klubu, który spotyka sieco miesiąc, by wypić i porozmawiać. \\ \hline
 \end{tabularx}
  \caption {Historia Adepta}
  \label {Historia Adepta}
 \end{table*}
 
 \subsubsection{Adept Pierwszego Poziomu}
 
 Pierwszo-poziomowi Adepci posiadają następujące zdolności:
 
Wysiłek: Twój Wysiłek to 1.

Geniusz: Masz Skupienie w Intelekcie 1 oraz Skupienie w Mocy i Szybkości 0.

Eksperckie Korzystanie z Cypherów: Możesz nosić 3 cyphery w danym czasie.

Początkowy Ekwipunek: Stosowne ubranie, plus 2 drogie przedmioty, dwa przedmioty średniej ceny i do 4 niedrogich przedmiotów Twojego wyboru.

Bronie: Możesz korzystać z lekkich broni bez żadnej kary. Posiadasz nieumiejętność w średnich i ciężkich broniach – Twoje ataki z średnimi i ciężkimi broniami są utrudnione.

Specjalne zdolności: Wybierz 4 zdolności z poniższej listy. Nie możesz wybrać danej zdolności więcej niż 1 raz, chyba, że jej opis stanowi inaczej. Pełny opis każdej z dostępnych zdolności znajduje się w rozdziale \mytext{Zdolności}, który zawiera także zdolności Posmaków i specjalizacje w jednym, rozbudowanym katalogu. (Zdolności Adepta wymagają przynajmniej jednej wolnej ręki, chyba, że MG mówi inaczej.)

\begin{itemize}
\item Zamglenie
\item Usunięcie Wspomnień
\item Daleki Krok
\item Sztuczki Magiczne
\item Trening Magiczny
\item Pocisk
\item Pchnięcie
\item Pole Renozansowe
\item Skan
\item Strzaskanie
\item Magia Obronna
\end{itemize}

\subsubsection{Adept Drugiego Poziomu}

Wybierz jedną ze zdolności z poniższej listy (lub z niższego poziomu) i dodaj do swoich zdolności. Dodatkowo, możesz zamienić jedną ze zdolności z niższego poziomu na inną zdolność z niższego poziomu.

\begin{itemize}
\item Adaptacja
\item Czytanie Myśli
\item Odzyskanie Wspomnień
\item Ujawnienie
\item Unoszenie Się
\item Tnące Światło
\item Zastój
\end{itemize}

\subsubsection{Adept Trzeciego Poziomu}

Wybierz dwie zdolności z poniższej listy (lub z niższego poziomu) i dodaj do swoich zdolności. Dodatkowo, możesz wybrać jedną ze zdolności niższego poziomu i zamienić na inną z niższego poziomu. 

\cleardoublepage

\subsection{Odkrywca}\index{Typ!Odkrywca}

Fantasy/Baśń: Odkrywca, poszukiwacz przygód, badacz tajemnic.

Współczesność/Horror/Romans: atleta, odkrywca, poszukiwacz przygód, detektyw, badacz, pionier, reporter śledczy.

Science fiction: odkrywca, poszukiwacz przygód, podróżnik, planetolog, ksenobiolog.

Superbohaterowie/Post-apokalipsa: poszukiwacz przygód, stróż prawa.

Jesteś osobą akcji i fizycznych zdolności, bez lęku patrzącą ku nieodkrytemu. Podróżujesz do dziwnych, egzotycznych i niebezpiecznych miejsc, i odkrywasz nowe rzeczy. Oznacza to, że masz duże zdolności fizyczne, ale zapewne także jesteś dobrze wykształcony. 

Rola w grze: Choć Odkrywcy mogą być uczonymi i dobrze wykształconymi, są przede wszystkim zainteresowani akcją. Mierzą się ze śmiertelnymi niebezpieczeństwami i okropnymi przeszkodami praktycznie codziennie.

Rola w drużynie: Odkrywcy czasami pracują sami, ale częściej są częścią zespołu z innymi postaciami. Odkrywca często przoduje i przeciera szlak. Jednakże, często zatrzymują się i badają to, co ich zaintrygowało po drodze. 

Rola społeczna: Nie wszyscy Odkrywcy przedzierają się przez dzicz lub badają stare ruiny. Czasami, Odkrywca to nauczyciel, naukowiec, detektyw lub reporter śledczy. W każdym wypadku, Odkrywca z odwagą zmaga się z nowymi wyzwaniami i zbiera wiedzę, którą może się dzielić z innymi.

Zaawansowani Odkrywcy: Wysokopoziomowi Odkrywcy zyskują więcej umiejętności, trochę zdolności bojowych i dużo zdolności, które pomagają im poradzić sobie z niebezpieczeństwem. W skrócie, stają się uniwersalni, zdolni dać sobie radę z każdym wyzwaniem. 

\subsubsection{Odkrywca - Wtrącenia Gracza}

Kiedy grasz Odkrywcą, możesz wydać 1 PD by skorzystać z poniższych \mytext{wtrąceń gracza}, jeśli sytuacja jest odpowiednia i MG się zgadza.

Szczęśliwa Awaria: Pułapka lub niebezpieczne urządzenie doświadcza awarii, zanim może Ciebie zranić.

Nieoczekiwana Wskazówka: W momencie, gdy myślisz, że kompletnie zgubiłeś drogę, element krajobrazu, drogowskaz, lub po prostu ułożenie terenu sprawia, że odkrywasz najlepszą drogę naprzód, przynajmniej w tym momencie.

Słaba Trucizna: Trucizna lub choroba okazuje się nie być tak poważna, jak na początku wyglądała, i zadaje tylko połowę obrażeń, które zadałaby normalnie. 

\begin{table*}[t]
 \centering
 \begin{tabularx}{\textwidth}{ | X | X  |}
  \hline
   \textbf{Statystyka} & \textbf{Początkowa Wartość Puli} \\ \hline
    Moc & 10  \\ \hline
    Szybkość & 9  \\ \hline
    Intelekt & 9  \\ \hline
 \end{tabularx}
  \caption {Pula Statystyk Odkrywcy}
  \label {Pula Statystyk Odkrywcy}
 \end{table*}
 
 Otrzymujesz dodatkowe 6 punktów, które możesz rozdzielić pomiędzy swoje Pule zgodnie ze swoim życzeniem.
 
 \subsubsection{Historia Odkrywcy}
 
Twój typ pomaga Ci określić połączenie Twojej postaci z settingiem. Rzuć k20 lub wybierz z poniższej listy, by określić konkretny fakt o Twojej historii, który łączy Cię zresztą świata. Możesz także stworzyć swój własny fakt.

 \begin{table*}[t]
 \centering
 \begin{tabularx}{\textwidth}{| p{0.10\textwidth} | X |}
  \hline
  \textbf{k20} & \textbf{Historia Odkrywcy}  \\ \hline
    1 & Byłeś gwiazdą sportu w swoim liceum. Dalej jesteś w dobrej kondycji, ale człowieku, co to było wtedy! \\ \hline
    2 & Twój brat jest głównym śpiewakiem w naprawdę popularnym zespole. \\ \hline
    3 & Dokonałeś szeregu odkryć podczas swoich podróży, ale nie wszystkie okoliczności, by na nich zarobić, jeszcze pojawiły się przed Tobą. \\ \hline
    4 & Byłeś policjantem, ale zrezygnowałeś z pracy po doświadczeniu korupcji w siłach porządkowych. \\ \hline
    5 & Twoi rodzice byli misjonarzami, więc spędziłeś dużą część swojego młodego życia, podróżując do egzotycznych miejsc.  \\ \hline
    6 & Służyłeś w armii z honorem. \\ \hline
    7 & Otrzymałeś pomoc od sekretnej organizacji, która opłaciła Twoją edukację. Teraz ona pragnie znacznie więcej od Ciebie.  \\ \hline
    8 & Uczęszczałeś na prestiżowy uniwersytet dzięki stypendium dla sportowców, ale lśniłeś zarówno na boisku, jak i podczas zajęć.  \\ \hline
    9 & Twój najlepszy przyjaciel z dzieciństwa jest teraz wplywowym członkiem rządu. \\ \hline
    10 & Byłeś nauczycielem. Twoi studenci wspominają Cię miło. \\ \hline
    11 & Przez krótki czas byłeś kryminalistą, który został złapany i poszedł do więzenia – potem próbowałeś wyjść na prostą. \\ \hline
    12 & Twoje największe jak dotąd odkrycie zostało ukradzione przez Twojego rywala.  \\ \hline
    13 & Należysz do ekskluzywnej organizacji Odkrywców, której istnienie nie jest szeroko znane. \\ \hline
    14 & Zostałeś porwany jako dziecko w tajemniczych okolicznościach, are wróciłeś do domu bezpieczny. Media dalej czasami wspominają ową sytuację.  \\ \hline
    15 & Kiedy byłeś młody, byłeś uzależniony od narkotyków, a teraz powoli wstajesz na nogi. \\ \hline
    16 & Kiedy badałeś odległą lokację, dostrzegłeś coś, czego nigdy nie byłeś w stanie wyjaśnić. \\ \hline
    17 & Posiadasz mały bar lub restaurację. \\ \hline
    18 & Opublikowałeś książkę o swoich odkryciach i poczynaniach, która zyskała pewne uznanie. \\ \hline
    19 & Twoja siostra posiada sklep i daje Tobie pokaźną zniżkę. \\ \hline
    20 & Twój ojciec to wysoki rangą oficer w armii i posiada wiele koneksji. \\ \hline
 \end{tabularx}
  \caption {Historia Odkrywcy}
  \label {Historia Odkrywcy}
 \end{table*}
 
 \subsubsection{Odkrywca Pierszego Poziomu}
 
Odkrywca pierwszego poziomu ma poniższe zdolności:

Wysiłek: Twój Wysiłek to 1.

Fizyczna Natura: Masz Skupienie w Mocy 1, w Szybkości i Intelekcie zaś – 0.

Korzystanie z Cypherów: Możesz mieć przy sobie 2 cyphery naraz.

Początkujący Ekwipunek: Odpowiednie ubranie i broń Twojego wyboru, plus 2 drogie przedmioty, 2 przedmioty średniej ceny i do 4 niedrogich przedmiotów.

Bronie: Możesz korzystać z lekkich i średnich broni bez kary. Posiadasz nieumiejętność z ciężkimi brońmi – Twoje ataki nimi są utrudnione.

Specjalne Zdolności: Wybierz cztery z poniższych zdolności. Nie możesz wybrać tej samej zdolności więcej niż raz, chyba, że jej opis stanowi inaczej. Pełen opis wszystkich zdolności znajduje się w rozdziale \mytext{Zdolności}, który także zawiera zdolności Posmaków i specjalizacji w pojedynczym, rozległym katalogu. 

\begin{itemize}
\item Blok
\item Broń Niepotrzebna
\item Deszyfracja
\item Mięśnie z Żelaza
\item Przypływ Pewności Siebie
\item Szybkostopy
\item Ulepszone Skupienie
\item Umiejętności Fizyczne
\item Umiejętności Wiedzy
\item Wyszkolony Bez Zbroi
\item Wyszkolony we Wszystkich Broniach
\item Wyszkolony w Zbroi
\item Wytrzymałość
\item Zmysł Niebezpieczeństwa
\item Znajdowanie Drogi
\end{itemize}

\subsubsection{Odkrywca Drugiego Poziomu}

Wybierz cztery z poniższych zdolności (lub z niższego poziomu) i dodaj do swoich zdolności. Dodatkowo, możesz zamienić jedną ze zdolności niższego poziomu na inną z niższego poziomu.

\begin{itemize}
\item Ciekawy
\item Instynkt Niebezpieczeństwa
\item Koordynacja Ręka-Oko
\item Na Straży
\item Negacja Zagrożenia
\item Oko do Szczegółów
\item Pomoc Bez Akcji
\item Szybkie Odzyskanie Zdrowia
\item Ucieczka
\item Umiejętna Obrona
\item Umiejętności Podróżnicze
\item Umiejętności Śledcze
\item Zwiększenie Zasięgu
\item Zniszczenie
\end{itemize}

\subsubsection{Odkrywca Trzeciego Poziomu}

Wybierz trzy zdolności z poniższej listy (lub niższego poziomu) i dodaj do swoich zdolności. Dodatkowo, możesz zamienić jedną ze zdolności niższego poziomu na inną zdolność z niższego poziomu.

\begin{itemize}
\item Bieg i Walka
\item Bieg Przez Przeszkody
\item Chwytaj Moment
\item Ekspercki Użytkownik Cypherów
\item Kontrolowany Upadek
\item Łamiący Kamienie
\item Odporność
\item Przemyślenie Problemów
\item Przywykły do Noszenia Zbroi
\item Ucieczka od Złego Losu
\item Umiejętny Atak
\item Zignorowanie Bólu
\item Znajdywacz Pułapek
\end{itemize}

\subsubsection{Odkrywca Czwartego Poziomu}

Wybierz dwie z poniższych zdolności (lub z niższego poziomu) i dodaj do swoich zdolności. Dodatkowo, możesz zamienić jedną ze zdolności niższego poziomu na inną niższego poziomu.

\begin{itemize}
\item Biegacz
\item Ciche Kroki
\item Czytając Znaki
\item Umiejętność Eksperta
\item Wyjątkowo Wytrzymały
\item Zręczny Wojownik
\item Zwiększony Efekt
\end{itemize}

\subsubsection{Odkrywca Piątego Poziomu}

Wybierz trzy z poniższych zdolności (lub z niższego poziomu) i dodaj do swoich zdolności. Dodatkowo, możesz zamienić jedną ze zdolności niższego poziomu na inną zdolność niższego poziomu.

\begin{itemize}
\item Atak z Wyskoku
\item Blokowanie
\item Czujny
\item Mistrzostwo Obrony
\item Mistrz Ruchu
\item Obdarzenie Fizyczne
\item Paczka Przyjaciół
\item Trudny do Zamordowania
\item Wydanie Rozkazu
\item Zaawansowany Użytkownik Cypherów
\end{itemize}

\subsubsection{Odkrywca Szóstego Poziomu}

Wybierz trzy z poniższych zdolności (lub z niższego poziomu) i dodaj do swoich zdolności. Dodatkowo, możesz zamienić jedną z zdolności niższego poziomu na inną z niższego poziomu.

\begin{itemize}
\item Dzielona Obrona
\item Dzikie Zdrowie
\item Mistrzostwo Ataków
\item Mistrzowska Biegłość w Pancerzach
\item Wielokrotny Atak
\item Większa Negacja Zagrożenia
\item Zainspirowanie Skoordynowanych Akcji
\item Znowu i Znowu
\end{itemize}

\subsubsection{Przykładowy Odkrywca}

Sam decyduje się na stworzenie Odkrywcy do kampanii science fiction. Ta postać będzie wytrzymałym odkrywcą badającym obce światy. Wydaje onu 3 wolne punkty na Pulę Mocy, 2 na Pulę Szybkości i 1 na Pulę Intelektu. Teraz jenu Pule to: Moc 13, Szybkość 11 i Intelekt 10. Jako postać pierwszego poziomu, ma on Wysiłek na 1, Skupienie w Mocy 1, a w Szybkości i Intelekcie 0. Póki co postać jest dosyć uniwersalna.

Sam zaczyna wybierać zdolności. Wybiera onu \mytext{Zmysł Niebezpieczeństwa} i \mytext{Przypływ Pewności Siebie}, uważając, że będą przydatne w wielu sytuacjach. Wybiera onu także \mytext{Wyszkolony w Zbroi}, uważając, że postać będzie nosić high-techową średnią zbroję w czasie eksploracji. Jako ostatnią zdolność, onu wybiera \mytext{Umiejętności Wiedzy}, wybierając biologię i geologię, by pomóc podczas misji planetarnych.

Odkrywca Sama może mieć przy sobie dwa cyphery, które w tym settingu mają naturę nanotechnologiczną. MG decyduje, że jeden z nich to zastrzyk nanitów, dający +1 do Skupienia w Mocy po użyciu, a drugi to urządzenie pozwalające na stworzenie jednego dowolnego prostego przedmiotu trzymanego w dłoni, zgodnie z życzeniem użytkownika. 

Odkrywca Sama nie jest naprawdę zorientowany na walkę, ale czasami Wszechświat to niebezpieczne miejsce, więc nosi on przy sobie średni blaster.

Sam dalej potrzebuje deskryptora i specjalizacji. Patrząc na rozdział o Deskryptorach, wybiera onu \mytext{Wytrzymały}, co zwiększa jej Pulę Mocy do 17. Postać ta będzie się też szybciej leczyć i lepiej działać, gdy jest zraniona. Jest ona wyszkolona w Obronie Mocy, ale ma \mytext{nieumiejętność} w inicjatywie – jednakże, anuluje to się ze Zmysłem Niebezpieczeństwa. Sam mogłoby się cofnąć i wybrać coś innego zamiast Zmysłu Niebezpieczeństwa, ale lubi onu to i tak to zostawia. Ogólnie rzecz ujmując, deskryptor uczynił postać bardzo wytrzymałą, nawet jeśli nieco powolną. 
Na swoją specjalizację, Sam wybiera \mytext{Bada Ciemnie Miejsca} (w tym przypadku ruiny obcych cywilizacji). Daje to postaci nieco dodatkowych umiejętności: szukanie, słuchanie, wspinaczkę, balansowanie i skakanie. Ten Badacz jest całkiem-całkiem.

Na swój motyw fabularny, Sam wybiera \mytext{Biznes}. Badanie obcych ruin czasem powoduje odkrycie dziwnych reliktów, i Sam uznału, że może je przetransportować do osób trzecich, zamiast pozwolić im wpaść w ręce piratów lub bogatych kolekcjonerów. Za małą opłatą, oczywiście. 


\cleardoublepage

\subsection{Mówca}\index{Typ!Mówca}

Fantasy/Baśń: bard, mówca, skald, emisariusz, kapłan, rzecznik.

Współczesność/Horror/Romans: dyplomata, lider, manipulator, minister, mediator, prawnik.

Science fiction: dyplomata, empata, konsul, legat.

Superbohaterowie/Post-Apokalipsa: władca marionetek, mesmerysta.

Jesteś dobry, jeśli chodzi o słowa i ludzi. Wyplątujesz się z niebezpieczeństw przy pomocy języka, i sprawiasz, że ludzie robią to, czego pragniesz.

Rola w grze: Mówcy są bystrzy i charyzmatyczni. Lubią ludzi i, co ważniejsze, rozumieją ich. To pomaga Mówcą sprawić, by inni zrobili to, co musi być zrobione.

Rola w drużynie: Mówca to bardzo często “twarz” drużyny – jest osobą, którą mówi za wszystkich i negocjuje z innymi. Walka i akcja nie są silną stroną Mówcy, więc inne postaci muszą czasami chronić Mówcę w chwili kryzysu.
Rola społeczna: Mówcy to często liderzy polityczni lub religijni. Równie często, jednakże, są oszustami lub kryminalistami.
Zaawansowani Mówcy: Wysokopoziomowi Mówcy korzystają ze swoich zdolności, by kontrolować i manipulować ludźmi, a także wspierać swoich przyjaciół. Mogą oni dzięki rozmowie uniknąć niebezpieczeństwa, a nawet użyć słów jak broni.

\subsubsection{Mówca - Wtrącenia Gracza}

Kiedy grasz Mówcą, możesz wydać 1 PD, by skorzystać z poniższych \mytext{wtrąceń gracza}, jeśli sytuacja jest odpowiednia i MG się zgadza.

Przyjazny BN: BN którego nie znasz, ktoś, kogo znasz słabo, lub ktoś, kogo znasz, a kto nie był szczególnie przyjazny w przeszłości postanawia Ci pomóc, lecz nie musi on wyjaśnić czemu. Może poproszi Cię potem o zwrot przysługi, w zależności od tego, w jak wielki kłopoty się wpakuje. 

Perfekcyjna Sugestia: Kompan lub inny przyjazny BN sugeruje, co zrobić w kontekście ważkiego pytania, problemu lub przeszkody na Twojej drodze.

Niespodziewany Prezent: BN wręcza Ci fizyczny dar, którego sięnie spodziewałeś, który ułatwia Twoje problemy, lub zapewnia nowy wzgląd i oświeca w kontekście sytuacji, której nie pojmujesz jak należy.

\begin{table*}[t]
 \centering
 \begin{tabularx}{\textwidth}{ | X | X  |}
  \hline
    \textbf{Statystyka} & \textbf{Początkowa Wartość Puli} \\ \hline
    Moc & 8  \\ \hline
    Szybkość & 9  \\ \hline
    Intelekt & 11  \\ \hline
 \end{tabularx}
  \caption {Pula Statystyk Mówcy}
  \label {Pula Statystyk Mówcy}
 \end{table*}
 
Otrzymujesz dodatkowe 6 punktów do rozdzielenia pomiędzy Twoje Pule statystyk, zgodnie z własnym życzeniem.
 
\subsubsection{Historia Mówcy}

Twój typ pomaga określić Twoje połączenie z settingiem. Rzuć k20 lub wybierz z poniższej listy konkretny fakt o swojej historii, który zapewnia połączenie zresztą świata. Możesz także stworzyć swój własny fakt.


\subsubsection{Opcje Tworzenia Postaci - Fantasy}\index{Opcje Tworzenia Postaci!Fantasy}

W pewnych przypadkach, poniższe pomysły wymagają pewnych zmian zgodnie z Posmakiem, co opisano w opcjach postaci; powinieneś pracować ze swoim MG w celu aplikacji owych zmian, zgodnie z duchem kampanii. Większość specjalizacji w tej sekcji występuje w Cypher System – specjalizacje z gwiazdką (*) można znaleźć dalej w tym dokumencie. Niektóre z tych opcji sugerują zamianę zdolności z typu na zdolność z Posmaku takiego jak walka, magia lub skradanie się.

Alchemik: W rozumieniu tego, że alchemik to ktoś, kto robi magiczne przedmioty i tym podobne, Adept i Odkrywca to odpowiednie typy dla alchemika-naukowca. Aby stworzyć ogólnego alchemika, który robi mikstury z magicznymi właściwościami, wybierz specjalizację Włada Zaklęciami (zamiast zaklęć, masz eliksiry). Aby stworzyć alchemika, który zamienia się w potężną i niebezpieczną istotę, wybierz Wyje do Księżyca. Dla alchemika, który kocha rzucać bombami, wybierz Nosi Halo Ognia. Aby stworzyć uzdrowiciela, wybierz Uzdrawia.

Barbarzyńca: Barbarzyńca to najpewniej Wojownik lub (jeśli wolisz się skupić nie tylko na walce) Odkrywca. Dobre specjalizacje ,które można wybrać, to: Żyje w Dziczy, Mistrzowsko Posługuje się Bronią, Nie Potrzebuje Broni, Nigdy się Nie Poddaje, Jest Bardzo Silny i Wpada w Furię. 

Bard: Bardowie w fikcji fantasy i grach są trubadurami, minstrelami i opowiadaczami historii, być może z magicznymi zdolnościami. Bardowie to zazwyczaj Odkrywcy lub Mówcy. Odpowiednie specjalizacje to: Zabawia, Pomaga Swoim Przyjaciołom, Infiltruje i Włada Zaklęciami.

Kleryk lub Kapłan: Kapłani z dobrym wykształceniem to zazwyczaj Adepci lub Mówcy, ale wojowniczy są zazwyczaj Wojownikami (możliwe, że z Posmakiem magia). Aby stworzyć typowego kleryka z szerokim wachlarzem zdolności, wybierz specjalizację Otrzymuje Boskie Błogosławieństwo.

\begin{itemize}
\item Kleryk (burza): Ujeżdża Błyskawicę, Grzmi
\item Kleryk (oszustwo): Przyjmuje Zwierzęcy Kształt (patrz także opcje dla łotrzyków) 
\item Kleryk (śmierć): Zadaje się z Martwymi, Mówi z Duchami
\item Kleryk (światło): Jaśnieje Światłem, Otrzymuje Boskie Błogosławieństwo
\item Kleryk (wiedza): Szybko się Uczy, Jest Jasnowidzem, Wolałby Czytać
\item Kleryk (wojna): Mistrzowsko Posługuje się Bronią (patrz także opcje dla wojowników)
\item Kleryk (życie): Chroni Słabszych, Wspiera Społeczność, Uzdrawia
\end{itemize}

Zabójca/Szpieg: Odkrywca i Wojownik są dobrymi typami dla takiej postaci. Stosowne specjalizacje to Mistrzowsko Posługuje się Bronią, Porusza się jak Kot, Morduje i Pracuje w Ciemnych Uliczkach.

Druid: Jako bardzo specyficzny rodzaj kapłana natury, druid to zazwyczaj Adept lub Odkryca (obydwie opcje być może z Posmakiem magii). Typowy druid to ma najpewniej specjalność Otrzymuje Boskie Błogosławieństwo lub Żyje w Dziczy, a;e po bardziej specyficzne opcje, patrz niżej:

\begin{itemize}
\item Druid (transformacja): Jest Stworzony z Kamienia,  Przyjmuje Zwierzęcy Kształt*, Spaceruje w Dzikich Lasach*
\item Druid (więź z naturą): Mówi Głosem Ziemi
\item Druid (zwierzęcy towarzysz): Kontroluje Bestie, Włada Rojem
\item Druid (żywiołak): Jest Stworzony z Kamienia, Nosi Halo Ognia, Porusza się jak Wiatr, Ujeżdza Błyskawicę,  Rides the Lightning, Przywdziewa Połyskliwy Lód
\end{itemize}

Wojownik: Jak sama nazwa wskazuje, wojownik prawie zawsze będzie Wojownikiem, ale niektórzy to Badacze. Typowy wojownik najpewniej posiada bezpośrednią specjalność, taką jak Mistrzowski Posługuje się Bronią lub Dzierży Magiczną Broń*. Po dodatkowe opcje w zależności od specjalizacji, patrz poniżej:

\begin{itemize}
\item Wojownik (strażnik): Nosi Egzotyczną Tarczę, Chroni Wrót, Masters Defense, Nigdy się Nie Poddaje, Jest Jednoosobowym Bastionem.
\item Wojownik (walka na dystans): Ma Licencję na Broń, Rzuca ze Śmiertelną Dokładnością
\item Wojownik (wręcz): Walczy Nieczysto, Walcząc, Porywa Tłum, Szuka Kłopotów, Nie Potrzebuje Broni, Dzierży Dwie Bronie Naraz
\end{itemize}

Rewolwerowiec: Rewolwerowiec to najpewniej Wojownik lub Eksplorer, ale niektórzy są Mówcami z Posmakiem walki. Stosowne specjalności to Ma Licencję na Broń, Mistrzowsko Posługuje się Bronią, Pływał z Piratami i Dzierży Magiczną Broń*.

Inkwizytor: Inkwizytorzy to zazwyczaj Odkrywcy, Mówcy lub Wojownicy, w zależności od tego, czy gracz chce mieć wiele umiejętności, być dobrym w interakcji społecznej, lub w walce. Stosowne specjalności to Infiltruje, Zaprowadza Sprawiedliwość lub Działa pod Przykrywką.

Kupiec: Odkrywca ze specjalizacją skupioną na interakcjach społecznych, taką jak Zabawia lub Przewodzi, mógłby być dobrym kupcem, ale bardziej oczywistym wyborem byłby Mówca.

Mnich lub Mistrz Sztuk Walki: Jako mistrzowie walki bez broni, są to zazwyczaj Wojownicy lub Odkrywcy (możliwe, że z Posmakiem walki). Odpowiednie specjalizacje to Walcząc, Porywa Tłum, Nie Potrzebuje Broni i Rzuca ze Śmiertelną Dokładnością. 

Paladyn/Święty Rycerz: Jako święci wojownicy, którzy mają do dyspozycji magię i moce walki, paladyni to zazwyczaj Wojownicy lub Odkrywcy (w obydwu przypadkach zmodyfikowani Posmakiem magii). Dobre specjalności dla takiej postaci to Chroni Wrót, Chroni Słabszych, Zaprowadza Sprawiedliwość, Zabije Potwory i Dzierży Magiczną Broń. 

Łowca: Łowcy mają mieszankę moc walki i magicznych, i z tego względu są zazwyczaj Odkrywcami (możliwe, że z Posmakiem walki) lub Wojownikiem (możliwe, że z Posmakiem umiejętność i wiedza). Odpowiednie specjalności dla łowcy to: Kontroluje Bestie, Poluje, Żyje w Dziczy, Zabija Potwory, Rzuca ze Śmiertelną Dokładnością i Dzierży Dwie Bronie Naraz.

Łotrzyk lub Złodziej: Większość łotrzyków to Odkrywcy, ale postać skupiona na interakcjach społecznych mogłaby być Mówcą (możliwe, że z Posmakiem skradanie się). Specjalności dobre dla łotrzyka to Bada Ciemne Miejsca, Walczy Nieczysto, Poluje, Infiltruje, Jest Poszukiwany Przez Prawo, Porusza się jak Kot, Pływał z Piratami i Pracuje w Ciemnych Uliczkach.

Zaklinacz: Zaklinacza, to dla naszych potrzeb, magowie, którzy posiadają wrodzoną moc magiczną (w przeciwieństwie do czarodziejów, którzy muszą się tego nauczyć). Większość Zaklinaczy to Adepci, ale niektórzy są Odkrywcami lub Mówcami. Specjalność Włada Zaklęciami daje typowemu zaklinaczowi różne zdolności, a większość specjalności zapewnia zaklęcia tematyczne. Po zaklinaczy z poszczególnych linii krwi, patrz poniżej:

\begin{itemize}
\item Zaklinacz (anioł): Jaśnieje Światłem, Otrzymuje Boskie Błogosławieństwo, Posiada Magicznego Sprzymierzeńca
\item Zaklinacz (przeznaczenie): Ma Szlachetną Krew, Został Przepowiedziany
\item Zaklinacz (smok): Nosi Halo Ognia, Ujeżdża Błyskawicę, Przywdziewa Połyskliwy Lód
\item Zaklinacz (żywiołak): Jest Stworzony z Kamienia, Nosi Halo Ognia, Włada Magnetyzmem, Porusza się jak Wiatr, Ujeżdża Błyskawicę, Przywdziewa Połyskliwy Lód
\item Zaklinacz (fae): Przyjmuje Zwierzęcy Kształt
\item Zaklinacz (demon): Nosi Halo Ognia, Posiada Magicznego Sprzymierzeńca
\item Zaklinacz (nieumarły): Zadaje się z Martwymi, Mówi do Duchów
\end{itemize}

Trikster lub Oszust: Te bystrzaki to zazwyczaj Mówcy, ale niekiedy są Adeptami, jeśli są bardzo magiczni (lub Odkrywcami, jeśli nie są magiczni w ogóle). Wybór specjalności to między innymi Walczy Nieczysto, Pracuje w Ciemnych Uliczkach lub Zabawia.

Czarodziej wojenny: Te nietypowe postaci mieszają korzystanie z broni z magią – wybierz Wojownika z Posmakiem magii lub Odkrywcę z Posmakami magii lub walki. Specjalności, które mogę Cię zainteresować, to Walczą,c Porywa Tłum, Mistrzowsko Włada Broniami, lub Dzierży Magiczną Broń.  

Czarownik lub Wiedźma: Dla celów tej listy, czarownik i wiedźma to magowie, którzy uzyskali moc magiczną z paktu, który zawarli z bytami spoza rzeczywistości. Większość czarowników to Adepci, ale Odkrywcy lub Mówcy (możliwe, że z Posmakiem magia) mogą być ciekawymi opcjami. Odpowiednie specjalności to Tańczy z Czarną Materią, Posiada Magicznego Sprzymierzeńca, Włada Rojem, Izoluje Umysł od Ciała i Został Przepowiedziany. W zależności od patrona i paktu, większość specjalności zaklinacza i czarodzieja będzie ok.

Dziki mag: Ci, którzy korzystają z chaotycznej magii, to zazwyczaj Adepci, ale może t obyć także Odkrywca lub Mówca z Posmakiem magii. Najlepszą specjalizacją byłoby Włada Dziką Magią.

Czarodziej: Dla celów tej listy, czarodzieje uczą się magicznej wiedzy przez wiele lat, aby zdobyć zdolność rzucania zaklęć (w przeciwieństwie do zaklinaczy, czarnoksiężników itp.). Czarodzieje to zazwyczaj Adepci, ale czarodziej zorientowany na ludzi może być Mówcą (być może z Posmakiem magii). Aby stworzyć ogólnego czarodzieja, wybierz specjalizację Włada Zaklęciami. Po bardziej wyspecjalizowanych czarodziejów, patrz niżej

\begin{itemize}
\item Czarodziej (znawca odrzuceń): Absorbuje Energię, Stawia Umysł Ponad Materią, Przywdziewa Połyskliwy Lód
\item Czarodziej (znawca przywołań): Kontroluje Bestie, Posiada Magicznego Sprzymierzeńca
\item Czarodziej (znawca poznań): Szybko się Uczy, Jest Jasnowidzem, Izoluje Umysł od Ciała, Rozwiązuje Zagadki
\item Czarodziej (znawca zauroczeń): Włada Mocami Mentalnymi, Przewodzi
\item Czarodziej (znawca wywołań): Nosi Halo Ognia, Jaśnieje Światłem, Ujeżdza Błyskawicę, Grzmi, Przewdziewa Połyskliwy Lód
\item Czarodziej (iluzjonista): Przebudza Sny, Tworzy Iluzje
\item Czarodziej (nekromanta): Zadaje sięz Umarłymi, Mówi do Duchów
\item Czarodziej (znawca transmutacji): Kontroluje Grawitację, Stawia Umysł Ponad Materią, Przyjmuje Zwierzęcy Kształt
\end{itemize}

\paragraph{Zaklęcia Przygotowane i Spontaniczne}\index{Zaklęcia Przygotowane i Spontaniczne}

Magiczne postaci otrzymują swoje zdolności (które mogą być zaklęciami, rytuałami lub czymś innym) ze swojego typu i specjalności, i mogą korzystać z owych zdolności jak uzn ają za stosowne tak długo, jak wydadzą punkty z Puli. To technicznie czyni ich bardziej jak spontaniczny czarujący. Jeśli wolisz zagrać czymś bardziej jak czarodziej przygotowujący zaklęcia, z większą ilością czarów, z których wybierasz małą ilość każdego dnia, rozważ specjalność skupioną na zaklęciach, taką jak Otrzymuje Boskie Błogosławieństwo, Włada Zaklęciami lub Mówi Głosem Ziemi i rozważ dalszą customizację opcjonalną zasadą rzucania zaklęć. 
\subsubsection{Dalsza Customizacja}\index{Dalsza Customizacja}

Zasady w tej sekcji są bardziej zaawansowane i zawsze zależą od MG. Mogą być użyte, by MG dostosował typ do konwencji lub settingu, lub przez gracza i MG, by dostosować koncept postaci.

\paragraph{Modyfikacja Aspektów Typu}

Poniższe aspekty czterech typów postaci mogą zostać zmodyfikowane podczas tworzenia postaci. Inne zdolności nie powinny być zmienione.

Pule Statystyk: Każda Pula postaci ma wartość startową. Gracz może zamieniać punkty w Pulach kosztem 1-na-1. Dla przykładu, może on przesunąć 2 punkty z Mocy na 2 punkty w Szybkości. Jednakże, żadna początkowa Statystyka nie może być wyższa niż 20.

Skupienie: Gracz może zacząć grę ze Skupieniem w dowolnej Statystyce na 1.

Korzystanie z Cypherów: Jeśli gracz odda zdolność noszenie jednego cyphera, uzyskuje on dodatkową umiejętność swojego wyboru.

Bronie: Pewne typy mają statyczne zdolności pierwszego poziomu które pozwalają im korzystać z lekkich, średnich i/lub ciężkich broni bez kary. Wojownicy mogą korzystać z wszystkich broni, Odkrywcy z lekkich i średnich, a Adepci i Mówcy mogą korzystać tylko z lekkich broni. Każda z tych zdolności może zostać poświęcona, by zyskać trening w odmiennej umiejętności, którą wybierze gracz.

\paragraph{Wady i Kary}

W dodatku do innych opcji customizacji, gracz może wybrać wzięcie kar lub wad, by zyskać dalsze bonusy.

Słabość: Słabość to, esencjalnie, przeciwieństwo Skupienia. Jeśli masz Słabość 1 w Szybkości, wszystkie akcje Szybkości wymagają od Ciebie dodatkowego 1 punktu z Twojej Puli. W każdym momencie, gracz może dać swojej postaci słabość w jednej statystycei otrzymać +1 do Skupienia w jednej z pozostałych dwóch. Tak więc gracz może wziąć słabość 1 w Szybkości i otrzymać +1 do swojego Skupienia w Mocy.

Normalnie, możesz mieć słabość tylko w statystyce, w której Twoje Skupienie wynosi 0. Co więcej, nie możesz mieć więcej niż jednej słabości, i niem ożesz mieć słabości większej niż 1, chyba, że dodatkowa słabość pochodzi z innego źródła (takiego jak zaraza lub niepełnosprawność wynikająca z akcji lub kondycji w grze).

Nieumiejętność: Nieumiejętności są jak negatywne umiejętności. Czynią jeden rodzaj akcji trudniejszym. Jeśli postać wybiera nieumiejętność, zyskuje ona umiejętność swojego wyboru. Normalnie, postać może mieć tylko jedną nieumiejętność, chyba, że pozostałe pochodzą z innego źródła (takiego jak deskryptor, choroba lub niepełnosprawność wynikająca z akcji lub kondycji w grze).

\paragraph{Posmaki}\index{Posmaki}

Posmaki to grupa specjalnych zdolności które MG i gracze mogą wykorzystać, aby zmienić typ postaci – np.: w zgodzie z settingiem lub konwencją. Dla przykładu, jeśli gracz chce stworzyć czarodzieja, który jest też złodziejem, może zagrać Adeptem z Posmakiem w skradaniu się. W settingu science fiction, Wojownik może mieć także wiedzę o maszynach, więc postać może mieć posmak w technologii. 

Na danym poziomie, zdolności z standardowego typu są wymieniane na zdolności z posmaku. Tak więc, aby dodać Zmysł Niebezpieczeństwa z posmaku skradanie się do Wojownika, trzeba poświęcić coś innego – być może Ogłuszenie. Teraz postać może wybrać Zmysł Niebezpieczeństwa, tak jak każdą inną zdolność pierwszego poziomu, ale nigdy nie może wziąć Ogłuszenia.

MG zawsze powinien brać udział w modyfikacji typu za pośrednictwem posmaku. Dla przykładu, może on określić, że w grze science fiction chce stworzyć type zwany “Glam”, czyli Mówcę z pewnymi zdolnościami technologicznymi – konkretniej tymi, które czynią z niego ekstrawaganckiego pilota statków kosmicznych. Tak więc, zamienia on pierwszopoziomowe zdolności Fałszywa Tożsamość i Zainspirowanie Agresji na Implant Wizualnej Identyfikacji i Umiejętności Technologiczne, tak, że postać może połączyć siębezpośrednio ze statkiem i mieć umiejętności komputerowe i pilotażu.

Ostatecznie, posmak to głównie narzędzia dla MG do łatwego tworzenia typów zrobionych pod kampanie, poprzez parę lekkich zmian tu i ówdzie. Choć gracze mogą chcieć skorzystać z posmaków, by stworzyć postać, jakiej pragną, pamiętaj, że mogą oni także dookreślić swojego BG przy pomocy deskryptorów i specjalizacji. 

Dostępne posmaki to: skradanie się, technologia, magia, walka oraz umiejętności i wiedza.
Pełen opis wszystkich zdolności można znaleźć w rozdziale Zdolności, który zawiera także opisy zdolności typów i specjalności w jednym pokaźnym katalogu.

\subparagraph{Posmak - Skradanie się}\index{Posmaki!Skradanie się}

Postaci z posmakiem skradanie się są dobre w skradaniu, infiltracji rożnych miejsc, gdzie nie powinny być i oszukiwaniu innych. Korzystają z tych zdolności na rożne sposoby, wliczając walkę. Odkrywca z posmakiem skradanie się może być złodziejem, a Wojownik zabójcą. Odkrywca z posmakiem skradanie się w settingu superbohaterskim może być pogromcą przestępców, który chodzi po ulicach nocami.

\textbf{Poziom pierwszy}

\begin{itemize}
\item Zmysł Niebezpieczeństwa
\item Prowokacja
\item Zwinne Dłonie
\item Oportunista
\item Umiejętności Złodzieja
\end{itemize}

\textbf{Poziom drugi}

\begin{itemize}
\item Człowiek-Guma
\item Wyczulenie na Okazję
\item Ucieknij
\item Niemożliwy do Zaskoczenia
\item Atak z Zaskoczenia
\end{itemize}

\textbf{Poziom trzeci}
\begin{itemize}
\item Zniknięcie
\item Z Cieni
\item Ryzykant
\item Wewnętrzna Obrona
\item Przekierowanie Ataku
\item Bieg i Walka
\item Chwytaj Moment
\end{itemize}

\textbf{Poziom czwarty}
\begin{itemize}
\item Czatownik
\item Bolesne Uderzenie
\item Przechytrzenie
\item Wyczulone Zmysły
\item Błyskotliwe Ruchy
\end{itemize}

\textbf{Poziom piąty}
\begin{itemize}
\item Cios Strytobójcy
\item Maska
\item Kontratak
\item Niezwykłe Szczęście
\end{itemize}

\textbf{Poziom szósty}
\begin{itemize}
\item Wykorzystanie Przewagi
\item Odsunięcie się
\item Szczęście Złodzieja
\item Zmiana Przeznaczenia
\end{itemize}

\subparagraph{Posmak - Technologia}\index{Posmaki!Technologia}

Postaci z posmakiem technologii zazwyczaj występują w settingach sci-fi lub przynajmniej współczesnych (choć wszystko jest możliwe). Dobrze im idzie używanie, radzenie sobie z i tworzenia maszyn. Odkrywca z posmakiem technologii może być pilotem statku kosmicznego, a Mówca nano-kapłanem.

Pewnie z mniej zorientowanych na komputery zdolności tego posmaku mogą pasować do steampunka, a postać żyjąca współcześnie może wykorzystać te zdolności, które nie są powiązane ze statkami kosmicznymi i ultra-technologią.

\textbf{Poziom pierwszy}
\begin{itemize}
\item Implant Wizualnej Identyfikacji
\item Haker
\item Interfejs Maszyn
\item Popsucie Maszyny
\item Umiejętności Technologiczne
\item Majsterkowanie
\end{itemize}

\textbf{Poziom drugi}
\begin{itemize}
\item Interfejs Zasięgowy
\item Wydajność Maszyny
\item Przeciążenie Maszyny
\item Serv-0
\item Serv-0 Obrońca
\item Naprawa Serv-0
\item Mistrzostwo Narzędzi
\end{itemize}

\textbf{Poziom trzeci}
\begin{itemize}
\item Mechaniczna Telepatia
\item Skany Serv-0
\item Obeznanie ze Statkiem Kosmicznym
\item Mowa Statku Kosmicznego
\item Ostrzał Ciągły
\end{itemize}

\textbf{Poziom czwarty}
\begin{itemize}
\item Więź z Maszyną
\item Walczący z Robotami
\item Celowanie Serv-0
\item Serv-0 Wojownik
\item Serv-0 Szpieg 
\end{itemize}

\textbf{Poziom piąty}
\begin{itemize}
\item Kontrola Maszyny
\item Naprawa na Oko
\item Kompan-Maszyna
\end{itemize}

\textbf{Poziom szósty}
\begin{itemize}
\item Zbieranie Informacji
\item Mistrz Maszyn 
\end{itemize}

\subparagraph{Posmak - Magia}\index{Posmaki!Magia}

Znasz się trochę na magii. Możesz nie być czarodziejem, ale znasz podstawy – jak to działa, i jak zrobić parę wspaniałych rzeczy. Oczywiście, w Twoim settingu, “magia” może oznaczać moce psioniczne, moce mutantów, dziwną technologięobcych lub cokolwiek innego. Odkrywca z posmakiem magii może być czarodziejem-łowcą, a Mówca z posmakiem magii może być zaklinaczem-bardem. Choć Adept z posmakiem magii jest dalej Adeptem, możesz odkryć, że zamiana paru jego zdolności na poniższe dopieszcza Twoją postać tak, jak sobie tego wymarzyłeś. 

\textbf{Poziom pierwszy}
\begin{itemize}
\item Błogosławieństwo Bóstw
\item Umysł-Twierdza
\item Macki Mocy
\item Sztuczki Magiczne
\item Trening Magiczny
\item Link Mentalny
\end{itemize}

\textbf{Poziom drugi}
\begin{itemize}
\item Promień Odrzucający
\item Przywołanie
\item Pole Siłowe
\item Kłódka
\item Naprawa Ciała
\end{itemize}

\textbf{Poziom trzeci}
\begin{itemize}
\item Dalekie Spojrzenie
\item Kwiat Ognia
\item Rzut
\item Moc na Odległość
\item Przywołanie Wielkiego Pająka
\end{itemize}

\textbf{Poziom czwarty}
\begin{itemize}
\item Obrona Przed Żywiołami
\item Zapłon
\item Przełamanie Obrony
\end{itemize}

\textbf{Poziom piąty}
\begin{itemize}
\item Stworzenie
\item Boska Interwencja
\item Szczęki Smoka
\item Szybka Podróż
\item Prawdziwe Zmysły
\end{itemize}

\textbf{Poziom szósty}
\begin{itemize}
\item Relokacja
\item Przywołanie Demona
\item Podróż Między Światami
\item Słowo Śmierci
\end{itemize}

\subparagraph{Posmak - Walka}\index{Posmaki!Walka}

Posmak walka czyni postać bardziej śmiercionośną. Mówca z tym posmakiem w settingu fantasy może być wojennym bardem. Badacz z posmakiem walka w historycznym świecie może być piratem. Adept z tym posmakiem może w świecie science fiction być weteranem tysiąca psionicznych wojen.


\textbf{Poziom pierwszy}
\begin{itemize}
\item Zmysł Niebezpieczeństwa
\item Przywykły do Noszenia Zbroi
\item Wyszkolony w Średnich Broniach
\end{itemize}

\textbf{Poziom drugi}
\begin{itemize}
\item Zew Krwi
\item Zdolności Bojowe
\item Wyszkolony Bez Zbroi
\end{itemize}

\textbf{Poziom trzeci}
\begin{itemize}
\item Wyszkolony we Wszystkich Broniach
\item Umiejętny Atak
\item Umiejętna Obrona
\item Następny Atak
\end{itemize}

\textbf{Poziom czwarty}
\begin{itemize}
\item Zręczny Wojownik
\item Śmiertelna Salwa
\item Furia
\item Przekierowanie Ataku
\item Ostrzał Ciągły
\end{itemize}

\textbf{Poziom piąty}
\begin{itemize}
\item Doświadczony Obrońca
\item Trudny Cel
\item Blokowanie
\end{itemize}

\textbf{Poziom szósty}
\begin{itemize}
\item Większa Umiejętność Ataku
\item Mistrzowska Biegłość w Pancerzach
\item Mistrzostwo Obrony
\end{itemize}

\subparagraph{Posmak - Umiejętności i Wiedza}\index{Posmaki!Umiejętności i Wiedza}

Ten posmak jest dla postaci, które posiadają wiedzę i bardziej realistyczne aplikacje swoich talentów. Jest mniej kinematyczny i dramatyczny niż nadnaturalne zdolności lub moc zaatakowania naraz kilku wrogów, ale czasami doświadczenie lub know-how jest dobrym rozwiązaniem problemów. Wojownik z posmakiem umiejętności i wiedza może być wojskowym inżynierem. Badacz może być polowym naukowcem. Mówca z tym posmakiem może być nauczycielem.

\textbf{Poziom pierwszy}
\begin{itemize}
\item Umiejętności Międzyludzkie
\item Umiejętności Śledcze
\item Umiejętności Wiedzy
\item Umiejętności Fizyczne
\item Umiejętności Podróżnicze
\end{itemize}


\textbf{Poziom drugi}
\begin{itemize}
\item Dodatkowa Umiejętność
\item Mistrzostwo Narzędzi
\item Zrozumienie
\end{itemize}

\textbf{Poziom trzeci}
\begin{itemize}
\item Skupienie na Umiejętności
\item Improwizacja
\end{itemize}

\textbf{Poziom czwarty}
\begin{itemize}
\item Wiele Umiejętności
\item Szybki Umysł
\item Specjalizacja w Zadaniu
\end{itemize}

\textbf{Poziom piąty}
\begin{itemize}
\item Wyszkolony w Średnich Broniach
\item Czytając Znaki
\end{itemize}

\textbf{Poziom szósty}
\begin{itemize}
\item Umiejętny Atak
\item Umiejętna Obrona
\end{itemize}



\printindex

\listoftables

\end{document}