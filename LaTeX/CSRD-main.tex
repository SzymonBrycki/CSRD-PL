\documentclass[10pt, a4paper, twocolumn, openright]{book}

\usepackage[pdftex, breaklinks=true]{hyperref}
\usepackage{polski}
\usepackage[utf8]{inputenc}
\usepackage{hyperref}
\usepackage{makeidx}
\usepackage{tabularx}
\usepackage{xcolor }
\usepackage{soul}
\usepackage{afterpage}
\usepackage[T1]{fontenc}
\usepackage{csquotes}
\usepackage{longtable}
\usepackage{supertabular,booktabs}
\usepackage{pdfpages}

% TABLES WIDTH !!!
% 0.10 and 0.45

\definecolor{purple}{HTML}{92268F}
\definecolor{red}{HTML}{FF0000}
% \definecolor{gray}{HTML}{D3D3D3}

% \sethlcolor{gray} 

\newcommand{\mytext }[1] {{\color{purple} \textbf{ \texttt {#1}}}}

% \title{Cypher System Reference Document 2024-07-02 (Edycja Polska)}
% \author{Zespół Monte Cook Games\thanks{Strona projektu: \url{https://www.montecookgames.com/cypher-system-open-license/}} \and Szymon ``Kaworu'' Brycki\thanks{\href{mailto:szymon.brycki@gmail.com}{\tt szymon.brycki@gmail.com}}}

\makeindex

\begin{document}

\includepdf[pages=1, noautoscale=true, width=\paperwidth]{graphics/okładka 3.pdf}

\begin{titlepage}
	\centering
	{\Huge \bfseries \title  _Dokument Referencyjny  \break Cypher System \par}
	\vspace{1cm}
	{\large \itshape 2024-07-02 \par}
	{\large \itshape  Edycja polska \par}
	\vspace{1cm}
	{\normalsize \textbf{Oryginalne zasady}: Monte Cook Games \par}
	{\normalsize \textbf{Polskie tłumaczenie}: Szymon ``Kaworu'' Brycki \par}
	\vspace{1cm}
	{\normalsize Licencja: \bfseries Cypher System Open License\par}
	\vspace{1cm}
	{\large \textbf{Copyright © 2024 Monte Cook Games. Some rights reserved.} \par}
	\vspace{1cm}
	{\large Stworzono w technologii \LaTeX \par}
	\vspace{1cm}
	{\large \today \par}
\end{titlepage}

% \maketitle

\tableofcontents

% here go all the chapters

\input{src/Jak grać w Cypher System.tex}
\input{src/Tworzenie własnej postaci.tex}
\input{src/Typ.tex}
\input{src/Wojownik.tex}
\input{src/Adept.tex}
\input{src/Odkrywca.tex}
\input{src/Mówca.tex}
\input{src/Opcje Tworzenia Postaci - Fantasy.tex}
\input{src/Dalsza Customizacja.tex}
\input{src/Deskryptory.tex}
\section{Specjalizacje}\index{Specjalizacje}

Specjalizacja czyni postać wyjątkową. Specjalziacje postaci w grupie nie powinny się powtarzać. Specjalizacja daje postaci korzyści podczas tworzenia postaci i z każdym następnym poziomem. Jest to czasownik w zdaniu ``Jestem przymiotnikiem rzeczownikiem który czasownikuje''.

Ten rozdział zawiera około 100 przykładowych specjalizacji, takich jak Nosi Halo Ognia, Wolałby Czytać i Pilotuje Statki Kosmiczne. Te specjalizacje mogą być wybrane i użyte przez gracza, lub przez MG, który dodaje je do listy dostępnych specjalizacji dla swoich graczy w następnej kampanii. 

Dodatkowo, dalsza część rozdziału zapewnia narzędzia dla MG, który chciałby stworzyć swoją własną specjalizację, tak, by pasowała do wymogów danej gry lub kampanii.

\subsection{Wybieranie specjalizacji}\index{Specjalizacje!Wybieranie specjalizacji}

Nie wszystkie specjalizacje pasują do każdej konwencji. Rozdział Konwencja zapewnia porady i pomoc, ale tutaj pójdą pewne uogólnienia. Oczywiście, MG może zezwolić na każdą specjalizację w swoim settingu. Specjalizacje są ważną jego częścią, gdyż np.: obecność na liście dozwolonych specjalizacji Włada Mocami Mentalnymi oznacza, że w tym świecie istnieją moce psioniczne, Wyje do Księżyca oznacza istnienie w nim likantropów, a Pilotuje Statki Kosmiczne sprawia, że w settingu, no cóż, istnieją statki kosmiczne. 

Kiedy wybiera się specjalizację dla postaci, otrzymuje ona specjalne połączenie z jedną lub więcej innych postaci graczy, zdolność pierwszego poziomu i być może dodatkowy ekwipunek, niezbędny, by korzystać z mocy zapewnianych przez specjalizację, lub który dobrze z nią współgra. Dla przykładu, postać będąca rzemieślnikiem może potrzebować wielu narzędzi. Postać, która ciągle płonie, potrzebuje ubrań odpornych na ogień. Postać rysująca magiczne runy może potrzebować pędzelka i farm. Postać zabijająca potwory mieczem potrzebuje miecza. I tak dalej. Jednakże, wiele specjalizacji nie wymaga dodatkowego ekwipunku. Każda specjalizacja oferuje także jedne lub dwie sugestie na Wtrącenia MG z listą możliwych konsekwencji naprawdę dobrych i złych rzutów kością.  
 
Parę specjalizacji w tym rozdziale zapewnia “zamianę z typem”, która pozwala zamienić zdolność typu na zdolność specjalizacji. Gracz nie musi poczynić tej zamiany, ale ma taką możliwość. Dla przykładu, specjalizacja Kocha Pustkę zapewnia opcję zyskania zdolności Mam Kombinezon Kosmiczny, Będę Podróżnikiem zamiast zdolności typu.

W miarę, jak postać zyskuje nowe poziomy, specjalizacja daje więcej zdolności. Każdy poziom jest zazwyczaj oznaczony jako Akcja lub Umożliwienie, czyni inne akcje lepszymi lub zapewnia jakieś inne korzyści, ale nie akcję. Zdolność, która pozwala bohaterowi razić wrogów laserami jest Akcją. Zdolność, która daje więcej dodatkowych obrażeń, kiedy się wykonuje akcję, jest Umożliwieniem. Umożliwienie jest wykorzystywane w tej samej turze co inna akcja, i często jest częścią innej akcji. Korzyści każdego poziomu są niezależne i kumulują się z korzyściami z innych poziomów (chyba, że zaznaczono inaczej). Tak więc jeśli zdolność pierwszego poziomu daje +1 do Pancerza, a zdolność czwartego poziomu także daje +1 do Pancerza, to postać na czwartym poziomie ma w sumie pancerz na +2.
Na poziomach trzecim i szóstym, postać może wybrać zdolność z dwóch opcji.

Możesz także wybrać, czy chcesz rozwinąć historie będącą opisem danej specjalizacji (choć nie jest to wymagane).

\subsection{Połączenia z innymi BG}\index{Specjalizacje!Połączenia z innymi BG}

Wybierz połączenie, które pasuje do specjalności. Jeśli jesteś MG wybierającym (lub tworzącym) jeden lub kilka specjalizacji dla swoich postaci, wybierz do 4 z poniższych połączeń.

\begin{itemize}
\item Wybierz innego BG. Z nieznanych Tobie powodów, ta postać jest kompletnie odporna na Twoje zdolności specjalizacji, niezależnie od tego, czy pragniesz jej nimi pomóc, czy zagrozić.
\item Wybierz innego BG. Wiedziałeś o jego istnieniu przez lata, ale nie sądziłeś, że on Ciebie znał.
\item Wybierz innego BG. Zawsze chcesz mu zaimponować, ale sam nie wiesz czemu.
\item Wybierz innego BG. Ta postać ma nawyk, który Cię wkurza, ale poza tym jesteś pod wrażeniem jej zdolności.
\item Wybierz innego BG. Ta postać posiada potencjał w okiełznaniu Twojego stylu walki, paradygmatu lub innej zdolności zapewnianej przez Twoją specjalizację. Chciałbyś ją potrenować, ale nie jesteś dobrze przygotowany do nauczania (być może) a ona może nie być zainteresowana (znowu: być może).
\item Wybierz innego BG. Jeśli znajduje się on w bliski zasięgu kiedy walczysz, czasami zapewnia on atut, a czasami utrudnia Twój test na atak (szansa 50% na jedno lub drugie, rzucane raz na walkę).
\item Wybierz innego BG. Kiedyś ocaliłeś mu życie i teraz ma dług wdzięczności. Nie jesteś z tego jakoś bardzo zachwycony – zrobiłeś, co należało zrobić i tyle.
\item Wybierz innego BG. Ta postać ostatnio Cię wyśmiała, co naprawdę Cię zraniło. Jak zamierzasz sobie z tym poradzić (jeśli w ogóle) zależy od Ciebie.
\item Wybierz innego BG. Ta postać wie, że cierpiałeś z powodu robotów w przeszłości. To, czy nienawidzisz robotów, zależy od Ciebie, co może wpłynąć na Twoją relację z tym BG, jeśli jest on przyjacielem robotów lub posiada robotyczne protezy.
\item Wybierz innego BG. Ta postać pochodzi z tego samego miejsca co ty i znaliście się jako dzieci. 
\item Wybierz innego BG. W przeszłości, nauczył on Cię paru trików dy wykorzystania w walce.
\item Wybierz innego BG. Ta postać nie pochwala Twoich metod.
\item Wybierz innego BG. Dawno temu, byliście po przeciwnych stronach barykady w starciu. Wygrałeś, choć w jego oczach “oszukiwałeś” (ale z Twojej perspektywy wszystko jest ok). Może on chcieć ponownego starcia, choć to zależy od niego.
\item Wybierz innego BG. Zawsze próbujesz zachwycić tą postać swoimi umiejętnościami, sprytem, wyglądem lub odwagą. Może jest ona Twoim rywalem, może pragniesz jej szacunku, a może jest ona Twoim obiektem westchnień.
\item Wybierz innego BG. Boisz się, że jest on zazdrosny o Twoje zdolności i martwisz się, że może to doprowadzić do problemów. 
\item Wybierz innego BG. Przypadkowo został złapany w pułapkę, którą założyłeś i musiał się wydostać z niej o własnych siłach.
\item Wybierz innego BG. Kiedyś zostałeś zatrudniony, by wyśledzić kogoś, kto był blisko tej postaci.
\item Wybierz dwóch BG (najlepiej takich, którzy mogą znaleźć się na trajektorii Twoich ataków). Kiedy nie trafiasz atakiem i MG decyduje, że atak uderzył w kogoś innego niż w Twój cel, trafia on w jedną tych dwóch postaci.
\item Wybierz jednego BG. Nie jesteś pewien jak ani skąd, ale ta postać posiada butelki rzadkiego alkoholu i może go dla Ciebie sprowadzić za pół ceny.
\item Wybierz jednego BG. Ostatnio straciłeś posiadany przedmiot i przekonałeś samego siebie, że to ten BG go skradł. Czy tak jest w istocie zależy od niego.
\item Wybierz jednego BG. On zawsze zdaje się wiedzieć, gdzie jesteś, lub przynajmniej w którym kierunku się znajdujesz w stosunku do niego.
\item Wybierz jednego BG. Patrzenie jak używasz swoich zdolności specjalizacyjnych wydaje się budzić w nim nieprzyjemne wspomnienia. To wspomnienie leży w gestii tego BG, choć może on nie być w stanie przywołać je na poziomie świadomości.
\item Wybierz jednego BG. Coś sprawia, że jego obecność przeszkadza w Twoich zdolnościach. Kiedy stoi on obok Ciebie, Twoje zdolności specjalizacyjne kosztują 1 dodatkowy punkt więcej.
\item Wybierz jednego BG. Coś sprawia, że uzupełnia on Twoje zdolności. Kiedy stoi on obok Ciebie, pierwsza zdolność specjalizacji, z której korzystasz w ciągu danej doby, kosztuje 2 punkty mniej.
\item Wybierz jednego BG. Znasz tą postać już jakiś czas, i pomogła ona Ci zyskać kontrolę nad Twoimi zdolnościami specjalizacyjnymi.
\item Wybierz jednego BG. Kiedyś w przeszłości tej postaci, miała ona dewastujące doświadczenie, kiedy próbowała zrobić coś, co Tobie przychodzi łatwo dzięki Twojej specjalizacji. To od niej zależy, czy powiedziała Ci o tym.
\item Wybierz jednego BG. Jego okazjonalna niezdarność i głośne zachowania irytują Cię.
\item Wybierz jednego BG. W niedalekiej przeszłości, gdy trenowaliście, przypadkowo trafiłeś go swoim atakiem, poważnie go raniąc. To od niego zależy, czy żywi urazę, czy też może wybaczył tobie.
\item Wybierz jednego BG. Wisi Ci on dużą sumę pieniędzy.
\item Wybierz jednego BG. W niedalekiej przeszłości, kiedy uciekaliście od jakiegoś zagrożenia, przypadkowo zostawiłeś tą postać z tyłu, by dała sobie radę sama. Przetrwała ona, ale ledwie. Od gracza tamtego BG zależy, czy jego postać dalej jest zła, czy może Ci wybaczyła.
\item Wybierz jednego BG. Niedawno, przypadkowo (lub celowo) sprawił on, że znalazłeś się w niebezpieczeństwie. Wszystko z Tobą teraz w porządku, ale masz się na baczności w jego obecności.
\item Wybierz jednego BG. Z Twojego punktu widzenia, wydaje się on nerwowy w związku z pewną ideą, osobą lub sytuacją. Chciałbyś nauczyć go jak być bardziej wyluzowanym (jeśli tylko Ci na to pozwoli).
\item Wybierz jednego BG. Kiedyś nazwał Cię on tchórzem.
\item Wybierz jednego BG. Ta postać zawsze rozpoznaje Cię i Twoje ślady, nawet kiedy jesteś w przebraniu lub uciekłeś z danego miejsca dawno temu.
\item Wybierz jednego BG. Niechcący spowodowałeś wypadek, który sprawił, że zapadł on w sen tak głęboki, że nie obudził się przez trzy dni. To, czy Ci wybaczył, czy też nie, zależy od niego.
\item Wybierz jednego BG. Jesteś przekonany, że jesteście w jakiś sposób spokrewnieni.
\item Wybierz jednego BG. Przypadkowo dowiedziałeś się czegoś, co on próbuje utrzymać w tajemnicy. 
\item Wybierz jednego BG. Jest on szczególnie wrażliwy na co bardziej widocznie zdolności Twojej specjalizacji, i czasami doznaje szoku trwającego parę rund, co utrudnia jego akcje. 
\item Wybierz jednego BG. Najwyraźniej posiada on cenny przedmiot, który kiedyś był Twój, a który przegrałeś w grach hazardowych lata temu.
\item Wybierz jednego BG. Gdyby nie Ty, ta postać oblałaby w przeszłości test zdolności umysłowych.
\item Wybierz jednego BG. Bazując na paru komentarzach, które podsłuchałeś, podejrzewasz, że nie darzy on Twojej strefy kompetencji lub ulubionego hobby wielką estymą. 
\item Wybierz jednego BG, którego specjalizacja jest powiązana z Twoją. Połączenie wpływa na nie w pewien sposób. Dla przykładu, jeśli postać korzysta z broni, Twoja zdolność specjalizacyjna czasami ulepsza ten atak w pewien sposób.
\item Wybierz jednego BG. Panicznie boi się on wysokości. Chciałbyś nauczyć go, jak być bardziej wyluzowanym na wysokościach. To od niego zależy decyzja, czy się zgodzi zaakceptować Twoją pomoc.
\item Wybierz jednego BG. Jest on skeptyczny odnośnie Twoich twierdzeń o czymś ważnym, co się przytrafiło Tobie w przeszłości. Może on nawet chcieć cię zdyskredytować lub odkryć “tajemnicę” Twojej historii, choć to zależy od niego.
\item Wybierz jednego BG. Ma on talent do dostrzegania, gdzie Twoje plany mają słabe punkty.
\item Wybierz jednego BG. Twarz tej postaci jest tak intrygująca z powodów, których nie rozumiesz, że czasami ją szkicujesz w piasku lub innym medium, do którego masz dostęp.
\item Wybierz jednego BG. Ta postać ma dodatkowy zwykły przedmiot od Ciebie – może to być coś, co zrobiłeś lub po prostu coś, co jej ofiarowałeś. (Dany gracz wybiera przedmiot.)
\item Wybierz jednego BG. Wynajął on Cię, byś wykonał dla niego pewną robotę. Otrzymałeś zapłatę, ale jeszcze nie wykonałeś tej pracy.
\item Wybierz jednego BG. Pracowaliście razem kiedyś, i skończyło się to źle.
\item Wybierz jednego BG. Kiedy stoi on obok Ciebie i poświęca swoją akcję na skoncentrowanie się, by ci pomóc,  jedna z Twoich zdolności specjalizacji ma podwojony zasięg.
\end{itemize}

\subsection{O specjalizacji}

Specjalizacje w tej książce celowo mają opis w ledwie kilku zdaniach, by można było je zastosować w wielu konwencjach. Zdanie lub dwa podsumowuje każdy z nich. Po wyborze przez Ciebie specjalizacji, masz opcję rozszerzenia jej opisu, tak, by pasowała do settingu lub postaci.

Dla przykładu, jeśli wybierasz Działa pod Przykrywką, opis tej specjalizacji to “Udając kogoś innego, poszukujesz odpowiedzi, których potężni tego świata nie chcą wyjawić”. Jeśli wybierasz Tworzy Dziwną Naukę, opis brzmi “Twoje nadnaturalne wejrzenie i zdolności tworzą z Ciebie naukowca zdolnego tworzyć cuda”. Te opisy zapewniają czego potrzebujesz, by korzystać z Specjalności.

Jednakże, jeśli sobie życzysz (i tylko, jeśli sobie życzysz – nie ma takiego obowiązku) możesz dodać więcej do tych opisów w sposób, który pasuje do Twojej gry. Dla przykładu, jeśli wybierasz Działa pod Przykrywką i Tworzy Dziwną Naukę dla sesji współczesnej, takiej jak horror, urban fantasy, sesja szpiegowska lub coś podobnego, możesz rozbudować opisy, jak pokazano w poniższych przykładach.

\textbf{Działa pod Przykrywką}: Szpiegostwo nie jest czymś, o czym masz jakąkolwiek wiedzę. Przynajmniej chcesz, by wszyscy wokół w to wierzyli, ponieważ naprawdę, zostałeś wytrenowany jako szpieg lub tajny agent. Możesz pracować dla rządu lub dla siebie. Możesz być funkcjonariuszem policji lub przestępcą. Możesz nawet być dziennikarzem śledczym.

Niezależnie od okoliczności, pozyskujesz informacje, które inni chcieliby zachować w tajemnicy. Zbierasz szeptane plotki, historie i dowody, i wykorzystujesz tę wiedzę w swoich własnych misjach oraz, jeśli to stosowne, zapewniasz swoim mocodawcom informacje, których pożądają. Alternatywnie, możesz sprzedać wiedzę, którą pozyskałeś, tym, którzy płacą najwięcej.

Najpewniej nosisz ciemne kolory – czarny, szarości lub ciemny błękit – by pomóc Ci wmieszać się w cienie, chyba, że przebrałeś się akurat za kogoś innego.

\textbf{Tworzy Dziwną Naukę}: Możesz być szanowanym naukowcem, publikującym w naukowych czasopismach. Lub możesz być uznawany za szaleńca przez innych, podążając za dziwnymi teoriami, którzy inni uznają za niedostatecznie dowiedzione. Prawdą jest jednak, że masz szczególny dar do przesuwania granic tego, co możliwe. Możesz pozyskać nową perspektywę i odblokować dziwne zjawiska dzięki swoim eksperymentom. Tam, gdzie inni widzą masę bzdur, Ty przeczesujesz teorie spiskowe dla olśnienia. Możesz robić swoje badania jako badacz rządowy, uniwersytecki, naukowiec korporacyjny, lub z wnętrza swojego własnego garażu. Zawsze jednak przesuwasz granice tego, co możliwe. 

Najpewniej dbasz o swoją pracę bardziej niż o trywialności pokroju własnego wyglądu, miłe zachowanie, lub społeczne normy, jednakże, ekscentryk Twojego pokroju nawet tutaj może się wyłamywać stereotypom. 

Jeśli chcesz pójść dalej, możesz także określić skąd zdolności Twojej specjalizacji się biorą. W zależności od konwencji, mogą ona brać się z treningu, magicznych run, poprzez zdolności cybernetyczne, dziedzictwo genetyczne lub ponieważ masz dostęp do zaawansowanej technologii. Dla przykładu, postać może być w stanie atakować błyskawicami ponieważ dostała się pod wpływ dziwnego promieniowania lub ponieważ posiada blaster elektryczny. Z drugiej strony, może tak się dziać, ponieważ intensywny trening odblokował dla niej dostęp do magii błyskawic. Możliwości są prawie nieskończone, i od Ciebie zależy, czy je wylistujesz, czy też nie.  Niezależnie od tego, jak zdolności zostały pozyskane, wystarczy, że działają.

\subsection{Specializacje}

Pełen opis wszystkich zdolności można znaleźć w odpowiednim rozdziale, który ma opisy typów, posmaków i zdolności w jednym bogatym katalogu.

\subsubsection{Absorbuje Energię}\index{Specjalizacje!Lista!Absorbuje Energię}

Władasz energią i zamieniasz ją na inne jej rodzaje.

Poziom 1: Absorpcja Energii Kinetycznej

Poziom 1: Wyzwolenie Energii

Poziom 2: Zasilenie Przedmiotu

Poziom 3: Absorpcja Czystej Energii lub Ulepszona Absorpcja Energii Kinetycznej

Poziom 4: Przeładowanie Energii

Poziom 5: Zasilenie istoty

Poziom 6: Zasilenie Tłumu lub Przeładowanie Urządzenia

Wtrącenia MG: Energia wyładowuje się w destruktywny sposób. Pewni drapieżcy żywią się czystą energią. Przypadkowy przedmiot zostaje wyssany z energii.

\subsubsection{Bada Ciemne Miejsca}\index{Specjalizacje!Lista!Bada Ciemne Miejsca}

Jesteś archetypowym łowcą skarbów i znalazcą zgubionych rzeczy.

Poziom 1: Wspaniały Odkrywca

Poziom 2: Wspaniały Infiltrator

Poziom 3: Dostosowanie Oczu

Poziom 3: Nocne Uderzenie lub Śliski Klient

Poziom 4: Ciężko Zapracowana Odporność

Poziom 5: Eksplorator Ciemności

Poziom 6: Oślepiający Atak lub W Objęciach Mroku

Wtrącenia MG: Przedmioty wypadają Ci z kieszeni lub torby w mroku, mapy Ci się gubią, pozyskane informacje nie zawierają istotnego szczegółu.  

\subsubsection{Buduje Roboty}\index{Specjalizacje!Lista!Buduje Roboty}

Twoje robotyczne twory robią to, czego od nich zażądasz. 

(Słowo “robot” użyte w tej specializacji jest używane, nawet jeśli roboty tworzone przed Ciebie mogą być odmienne od tych tworzonych przez kogoś innego, w zależności on konwencji. Roboty steampunkowe, organiczne lub nawet magiczne golemy – do 
tego wszystkiego odnosi się tutja słowo “robot”.)

Poziom 1: Robot-Asystent

Poziom 1: Twórco Robotów

Poziom 2: Kontrola Robotów

Poziom 3: Kompan-Ekspert lub Umiejętna Obrona

Poziom 4: Unowocześnienie Robota

Poziom 5: Armia Robotów

Poziom 6: Robotyczna Ewolucja lub Unowocześnienie Robota

Wtrącenia MG: Robot zostaje zhackowany, działa randomowo lub niespodziewanie wybucha.

\subsubsection{Chroni Słabszych}\index{Specjalizacje!Lista!Chroni Słabszych}

Pomagasz słabszym, pragnącym pomocy i bezsilnym.

Poziom 1: Odwaga

Poziom 1: Tarcza Obronna

Poziom 2: Wierny Obrońca

Poziom 2: Empatia

Poziom 3: Podwójni Bronieni lub Prawdziwy Strażnik

Poziom 4: Wyzwanie Bojowe

Poziom 5: Chętna Ofiara

Poziom 6: Resuscytacja lub Prawdziwy Obrońca

Wtrącenia MG: Postać skupiona na ochronie innych może czasami wystawić samą siebie do ataku.

\subsubsection{Chroni Wrota}\index{Specjalizacje!Lista!Chroni Wrota}

Każdy chce mieć Ciebie po swojej stronie w walce, ponieważ nic Cię nie omija.

Poziom 1: Ufortyfikowana Pozycja

Poziom 1: Do Mnie!

Poziom 2: Moc i Umysł

Poziom 3: Budujący Umocnienia lub Odbicie Ataków

Poziom 4: Większa Ulepszona Moc

Poziom 5: Pole Wzmacniające

Poziom 6: Generacja Pola Siłowego lub Atak Oszałamiający

Wtrącenia MG: Strategicznie ważna struktura się zapada. Wróg atakuje z niespodziewanej strony.

\subsubsection{Dotyka Nieba}\index{Specjalizacje!Lista!Dotyka Nieba}

Kontrolujesz pogodę.

Poziom 1: Unoszenie się

Poziom 2: Zbroja Wiatru

Poziom 3: Promienie Mocy lub Przywołanie Burzy

Poziom 4: Jeździec Wiatru

Poziom 5: Emisja Zimna

Poziom 6: Kontrola Pogody lub Rydwan Wiatru

Wtrącenia MG: Sojusznik jest przypadkowo trafiony przez błyskawicę. Niespodziewane uziemienie zadaje obrażenia. Pogoda jest zmieniona w niewłaściwy sposób i burza wyrywa się spod kontroli. 

\subsubsection{Działa pod Przykrywką}\index{Specjalizacje!Lista!Działa pod Przykrywką}

Pod przebraniem kogoś innego, szukasz odpowiedzi, które potężni tego świata pragną zachować dla siebie. 

(Ktoś kto Działa pod Przykrywką może mieć zestaw do przebierania się jako dodatkowy ekwipunek).

Poziom 1: Śledztwo

Poziom 2: Przebranie

Poziom 3: Agent-Prowokator lub Bieg i Walka

Poziom 4: Niezłe Oszustwo

Poziom 5: Korzystanie z Dostępnych Opcji

Poziom 6: Zaufaj Swemu Szcześciu lub Śmiertelny Cios

Wtrącenia MG: Pech może zepsuć najlepszy plan. Przebrenie zawodzi. Sprzymierzeńcy okazują się również być agentami.

\subsubsection{Dzierży Dwie Bronie Naraz}\index{Specjalizacje!Lista!Dzierży Dwie Bronie Naraz}

Dzierżysz stal w obydwu rękach, gotowy stanąć naprzeciwko każdego wroga. 

Poziom 1: Podwójne Władanie Lekkimi Broniami 

Poziom 2: Podwójny Cios

Poziom 2: Infiltrator

Poziom 3: Podwójne Władanie Średnią Bronią lub Precyzyjne Cięcie

Poziom 4: Podwójna Obrona

Poziom 5: Podwójne Rozproszenie Uwagi

Poziom 6: Rozbrojenie lub Wielokrotny Atak

Wtrącenia MG: Ostrze łamie się w połowie lub broń wypada z dłoni swego nosiciela. 

\subsubsection{Dzierży Magiczną Broń}\index{Specjalizacje!Lista!Dzierży Magiczną Broń}

Posiadasz broń o dziwnych właściwościach i Twoja wiedza o jej mocy pozwoliła Ci stworzyć unikalny styl walki.

Poziom 1: Zaczarowana Broń

Poziom 1: Wrodzona Moc

Poziom 1: Naładowanie Broni

Poziom 2: Uderzenie Mocy

Poziom 3: Szybki Atak lub Rzut Zaczarowaną Bronią

Poziom 4: Broń Defensywna

Poziom 5: Zaczarowany Ruch

Poziom 6: Smiertelny Cios lub Wielokrotny Atak

Wtrącenia MG: Broń się psuje lub zostaje upuszczona. Postać traci połąćzenie ze swoją bronią aż do czasu, gdy wykorzysta swoj akcję, by odnowić połączenie. Energia broni rozładowuje się w niespodziewany sposób. 

\subsubsection{Fruwa Szybciej Niż Pocisk}\index{Specjalizacje!Lista!Fruwa Szybciej Niż Pocisk}

Możesz latać i jesteś supersilny, ciężki w uszkodzeniu, a także szybki. Czy jest coś, czego nie możesz zrobić?

Poziom 1: Unoszenie się

Poziom 2: Większy Ulepszony Potencjał

Poziom 3: Ukryta Siła lub Rentgen w Oczach

Poziom 4: W Mgnieniu Oka

Poziom 4: Rozpęd

Poziom 5: Jeszcze Żywy

Poziom 6: palące światło lub Zignorowanie Przeszkody

Wtrącenia MG: Nemezis Cię odnajduje. Odnaleziono dziwny materiał, któy niweluje moce postacu. 

\subsubsection{Gra w Zbyt Wiele Gier}\index{Specjalizacje!Lista!Gra w Zbyt Wiele Gier}

Lekcje, refleks i strategie, których się nauczyłeś, grając w zbyt wiele gier, mają zastosowanie w prawdziwym życiu, gdzie ludzie, którzy nie grają dostatecznie dużo muszą się szczególnie męczyć. 

Poziom 1: Lekcje z Gier

Poziom 1: Gamer

Poziom 2: Oczy Przyzwyczajone do Ciemności

Poziom 2: Odporność na Sztuczki

Poziom 3: Cel Snipera lub Ulepszone Skupienie w Szybkości

Poziom 4: Gierki Umysłowe

Poziom 4: Ulepszony Intelekt

Poziom 5: Wytrzymałość Gracza

Poziom 6: Regeneracja Umysłu lub Bóg Gier

Wtrącenia MG: Chybiony atak trafie nie ten cel. Ekwipunek siępsuje. Czasami ludzie reaguję nagatywnie na kogoś, kto przeżył większość swego życia w wyimaginowanych światach gier.

\subsubsection{Grzmi}\index{Specjalizacje!Lista!Grzmi}

Emitujesz destruktywne dźwięki i manipulujesz nimi.

Poziom 1: Promień Grzmotu

Poziom 2: Bariera Konwersji Dźwięku

Poziom 3: Tłumienie Dźwięków lub Echolokacja

Poziom 4: Okrzyk Roztrzaskania

Poziom 5: Subsoniczny Hałas

Poziom 5: Wzmocnienie Dźwięku

Poziom 6: Trzęsienie Ziemi lub Śmiertelna Wibracja

Wtrącenia MG: Głośne hałasy przyciągają uwagę.

\subsubsection{Ignoruje Fizyczny Dystans}\index{Specjalizacje!Lista!Ignoruje Fizyczny Dystans}

Możesz się teleportować w jedno miejsce z drugiego poprzez krótki pobyt w równoległym wymiarze.

Poziom 1: Wymiarowy Ścisk

Poziom 2: Oportunista

Poziom 3: Obronna Teleportacja lub Skoki Teleportacyjne

Poziom 4: Krótka Teleportacja

Poziom 5: Średnia Teleportacja

Poziom 6: Teleportacja lub Rana Teleportacyjna

Wtrącenia MG: Teleportacja kończy się źle, umieszczając postać w niebezpiecznym miejscu. Bezwładność (np.: wskutek spadania) trwa podczas teleportacji, raniąc postać. 

\subsubsection{Infiltruje}\index{Specjalizacje!Lista!Infiltruje}

Subtelność, chytrość i ukradkowość pozwalają Ci na dostęp tam, gdzie inni nie mogą.

Poziom 1: Umiejętności Złodzieja

Poziom 1: Wyczucie Pobudek

Poziom 2: Impersonacja

Poziom 2: Ucieczka, nie Walka

Poziom 3: Świadomość lub Umiejętny Atak

Poziom 4: Niewidzialność

Poziom 5: Unik

Poziom 6: Pranie Mózgu lub Odsunięcie się

Wtrącenia MG: Szpiegów traktuje się okrutnie, gdy się ich złapie. Ich sprzymierzeńcy się ich wypierają. Pewnych sekretów lepiej nigdy nie poznać.

\subsubsection{Interpretuje Prawo}\index{Specjalizacje!Lista!Interpretuje Prawo}

Jest twoją rzeczą naginanie innych do swoich poglądów.

Poziom 1: Dyplomata

Poziom 1: Wiedza Prawnicza

Poziom 2: Debata

Poziom 3: Przydatna Pomoc lub Ulepszone Skupienie w Inteligencji

Poziom 4: Przerażenie

Poziom 5: Nikt nie Wie Lepiej

Poziom 6: Większy Ulepszony Potencjał lub Prawnik-Stażysta

Wtrącenia MG: Ludzie nie lubią wszystkowiedzących. Rozproszenie lub przeszkodzenie przeszkadza w argumencie prawniczym.

\subsubsection{Istnieje Częściowo Poza Fazą}\index{Specjalizacje!Lista!Istnieje Częściowo Poza Fazą}

Częściowo przezroczysty, jesteś w części poza fazą i możesz się przemieszczać przez ciała stałe. 

Poziom 1: Przechodzenie Przez Ściany

Poziom 2: Defensywne Znikanie

Poziom 3: Atak Fazowy lub Drzwi Fazowe

Poziom 4: Duch

Poziom 5: Nietykalny

Poziom 6: Ulepszony Atak Fazowy lub Wyfazowanie Wroga

Wtrącenia MG: Postać wyfazowuje się w nieznany wymiar. Postać gubi się w dużym ciele stałym. 

\subsubsection{Istnieje w Dwóch Miejscach Naraz}\index{Specjalizacje!Lista!Istnieje w Dwóch Miejscach Naraz}

Istniejesz w dwóch miejscach w tym samym czasie.

Poziom 1: Kopia

Poziom 2: Dzielone Zmysły

Poziom 3: Ulepszona Kopia lub Odporna Kopia

Poziom 4: Przekaz Obrażeń

Poziom 5: Skoordynowany Wysiłek

Poziom 6: Wielość lub Odporna Kopia

Wtrącenia MG: Obserwacja świata z dwóch odmiennych perspektyw dezorientuje postać, powodując zawroty głowy, wymioty lub konfuzję.

\subsubsection{Izoluje Umysł od Ciała}\index{Specjalizacje!Lista!Izoluje Umysł od Ciała}

Twój umysł opuszcza Twoje ciało, by widzieć odległe miejsca i poznawać sekrety, których nie da się poznać inaczej.

Poziom 1: Trzecie Oko

Poziom 2: Otwarty Umysł

Poziom 2: Wyostrzone Zmysły

Poziom 3: Zdalne Trzecie Oko lub Odnalezienie Ukrytych

Poziom 4: Sensor

Poziom 5: Psioniczny Pasażer

Poziom 6: Projekcja Mentalna lub Ulepszony Sensor

Wtrącenia MG: Ponowne połączenie ciała i umysłu może czasami być dezorientujące i wymagać od postaci spędzenia paru minut na dostrajaniu się. 

\subsubsection{Jaśnieje Światłością}\index{Specjalizacje!Lista!Jaśnieje Światłością}

Możesz tworzyć światło, kształtować je, naginać lub gromadzić jako broń. 

Poziom 1: Oświecony

Poziom 1: Dotyk Oświecenia

Poziom 2: Oszałamiające Światło

Poziom 3: Palące Światło lub Umiejętna Obrona

Poziom 4: Światło Słońca 

Poziom 5: Niewidzialność

Poziom 6: Żywe Światło lub Pole Obronne

Wtrącenia MG: Sprzymierzeńcy są przypadkowo oszołomieni lub oślepieni. Jasne błyski przywołują strażników. 

\subsubsection{Jest Bardzo Silny}\index{Specjalizacje!Lista!Jest Bardzo Silny}

Jesteś umięśniony, możesz podnosić wielkie ciężary i przebijać się przez drzwi.

Poziom 1: Atleta

Poziom 1: Ulepszone Skupienie w Mocy

Poziom 2: Pokaz Siły

Poziom 3: Żelazne Pięści lub Rzut

Poziom 4: Większa Ulepszona Moc

Poziom 5: Brutalne Uderzenie

Poziom 6: Większa Ulepszona Moc lub Atak z Wyskoku

Wtrącenia MG: Łatwo jest zniszczyć delikatne przedmioty lub kogoś przypadkowo zranić

\subsubsection{Jest Idolem Milionów}\index{Specjalizacje!Lista!Jest Idolem Milionów}

Jesteś celebrytą i większość ludzi Cię uwielbia.

Poziom 1: Świta

Poziom 1: Talent Celebryty

Poziom 2: Zalety Sławy

Poziom 3: Ulepszone Zdrowie lub Umiejętny Atak

Poziom 4: Zachwyt Światła Gwiazd

Poziom 4: Kompan-Ekspert

Poziom 5: Czy Ty Wiesz W Ogóle Kim Jestem?

Poziom 6: Oratorska Inspiracja lub Ulepszony Kompan

Wtrącenie GM: Fani są w niebezpieczeństwie lub odnieśli obrażenia ze względu na Ciebie. Ktośw Twojej świcie Cię zdradza. Twój show, tour, kontakt lub inne wydarzenie jest odwołane. Media publikują zdjęcia z Tobą we wstydliwych sytuacjach. 

\subsubsection{Jest Jasnowidzem}\index{Specjalizacje!Lista!Jest Jasnowidzem}

Posiadasz psioniczny dar, który  pozwala Ci widzieć to, czego inni nie mogą.

Poziom 1: Postrzeganie Niewidocznego

Poziom 2: Rentgen w Oczach

Poziom 3: Odnalezienie Ukrytych lub Sensor

Poziom 4: Widzenie na Odległość

Poziom 5: Postrzeganie Czasu

Poziom 6: Projekcja Mentalna lub Całkowita Świadomość

Wtrącenia MG: Pewne sekrety są zbyt okropne, by je poznać.

\subsubsection{Jest Jednoosobowym Bastionem}\index{Specjalizacje!Lista!Jest Jednoosobowym Bastionem}

Twoja zbroja, wraz z Twoim rozmiarem, siłą, treningiem lub wszczepami bionicznymi, czyni Ciętrudnym do przemieszczenia lub zaatakowania.

(Pewne postaci, które Są Jednoosobowym Bastionem, mogą już być ekspertami w pancerzach. Mogą one wybrać inną zdolność1-szego poziomu niż Wyszkolony w Zbroi)

Poziom 1: Wyszkolony w Zbroi

Poziom 1: Doświadczony Obrońca

Poziom 2: Odporność na Żywioły

Poziom 3: Nieporuszalny

Poziom 3: Większa Ulepszona Moc lub Wyszkolony we Wszystkich Broniach

Poziom 4: Żywa Ściana

Poziom 5: Wytrzymały

Poziom 5: Mistrzowska Biegłość w Pancerzach

Poziom 6: Śmiertelne Obrażenia lub Wyszkolony w Tarczach

Wtrącenia MG: Zbroja się uszkadza. Mali wrogowie atakują Cię w sprytny sposób. 

\subsubsection{Jest Mistrzem Obrony}\index{Specjalizacje!Lista!Jest Mistrzem Obrony}

Korzystasz z odpowiedniego ekwipunku i wyszkolenia, by uniknąć zranienia w walce.

Poziom 1: Mistrz Tarcz

Poziom 2: Twardy

Poziom 2: Wyszkolony w Zbroi

Poziom 3: Unik i Odporność lub Unik i Rewanż

Poziom 4: Wieża Siły Woli

Poziom 4: Przywykły do Noszeni Zbroi

Poziom 5: Nic Tylko Obrona

Poziom 6: Mistrz Obrony lub Jak Druga Skóra

Wtrącenia MG: Tarcza pęka przy trafieniu, jak i bronie, którymi się blokuje. Paski od pancerza pękają.

\subsubsection{Jest Poszukiwany Przez Prawo}\index{Specjalizacje!Lista!Jest Poszukiwany Przez Prawo}

Plakaty "POSZUKIWANY, ŻYWY LUB MARTWY" posiadają Twoje podobieństwo. To od Ciebie zależy, czy to koszmarna pomyłka, która się wymknęła spod kontroli, czy może potrafisz kogoś zabić, bo na Ciebie krzywo spojrzał. 

Poziom 1: Ulepszona Szybkość

Poziom 1: Zmysł Niebezpieczeństwa

Poziom 2: Atak z Zaskoczenia

Poziom 3: Reputacja Spoza Prawa lub Następny Atak

Poziom 4: Szybkie Zabójstwo

Poziom 5: Drużyna Desperados

Poziom 6: Jeszcze Żywy lub Śmiertelne Obrażenia

Wtrącenia MG: Większość ludzi nie reaguje dobrze na poszukiwanego listem gończym w swoich szeregach.

\subsubsection{Jest Stworzony z Kamienia}\index{Specjalizacje!Lista!Jest Stworzony z Kamienia}

Twoje ciało jest stworzone z twardego minerału, czyniąc się twardym, ciężkim do zranienia humanoidem.

Poziom 1: Ciało Golema

Poziom 1: Uzdrawianie Golema

Poziom 2: Chwyt Golema

Poziom 3: Wyszkolony Miażdżyciel

Poziom 3: Tupnięcie Golema lub Uzbrojenie

Poziom 4: Głębokie Rezerwy

Poziom 5: Wyspecjalizowany Pięściarz

Poziom 5: Jak Posąg

Poziom 6: Ultra Wzmocnienie lub Regeneracja Umysłu

Wtrącenia MG: Istoty z kamienia czasami zapominają o własnej sile lub wadze. Chodzący posąg może przerazić zwykłych ludzi.

\subsubsection{Jeździ Jak Maniak}\index{Specjalizacje!Lista!Jeździ Jak Maniak}

Niezależnie od tego, czy balansujesz na dwóch kołach, przeskakujesz między pojazdami lub ruszasz na przód ku niebezpiezeństwu, nie myślisz zbyt dużo o własnym bezpieczeństwie, gdy jesteś za kierownicą.  

(Ktoś to Jeździ Jak Maniak potrzebuje dostępu do pojazdu.)

Poziom 1: Kierowca

Poziom 1: Atak Podczas Kierowania

Poziom 2: Surfer Aut

Poziom 2: Pojedynek Spojrzeń

Poziom 3: Doświadczony Kierowca lub Ulepszone Skupienie w Szybkości

Poziom 4: Bystrooki

Poziom 4: Ulepszona Szybkość

Poziom 5: Coś na Drodze

Poziom 6: Uzdolniony Kierowca lub Śmiertelne Obrażenia

Wtrącenia MG: Silnik odmawia posłuszeństwa. Most na końcu drogi jest wyłączony z ruchu. Przednia szyba się roztrzaskuje. Ktoś nagle wyskakuje na przód pojazdu.

\subsubsection{Kocha Pustkę}\index{Specjalizacje!Lista!Kocha Pustkę}

Kiedy jesteś tylko Ty, Twój skafander kosmiczny i panorama niekończących się gwiazd, osiągasz stan spokoju.

Opcja do podmiany: Mam Kombinezon Kosmiczny, Będę Podróżnikiem 

Poziom 1: Umiejętności Kosmiczne

Poziom 1: Przyzwyczajony do Mikrograwitacji

Poziom 2: Ulepszone Skupienie w Szybkości

Poziom 2: Ulepszona Muskulatura

Poziom 3: Walka w Kosmosie lub Zbroja Fuzyjna

Poziom 4: Cichy jak Kosmos

Poziom 4: Odepchnięcie i Rzut

Poziom 5: Uniki w Mikrograwitacji

Poziom 6: Wystrzał Mikrograwitacyjny lub Pole Reakcyjne

Wtrącenia MG: Kombinezony kosmiczne mogą się zepsuć. Wskaźniki poziomu tlenu czasami mogą być mylące. Mikrometeoryty są powszechne w kosmosie.

\subsubsection{Kontroluje Bestie}\index{Specjalizacje!Lista!Kontroluje Bestie}

Masz rzadką zdolność komunikowania się i przewodzenia bestiom.

Poziom 1: Zwierzęcy Kompan

Poziom 2: Ukojenie Dzikiego

Poziom 2: Komunikacja

Poziom 3: Rumak lub Silniejsi Razem

Poziom 4: Oczy Bestii

Poziom 4: Ulepszony Kompan

Poziom 5: Zew Dziczy

Poziom 6: Jak Jedna Istota lub Kontrola Dzikiej Bestii

Wtrącenia MG: Społeczność jest niechętna dzikim zwierzęciom. Bestia wyrwane spod kontroli stają się prawdziwym niebezpieczeństwem. 

\subsubsection{Kontroluje Grawitację}\index{Specjalizacje!Lista!Kontroluje Grawitację}

Manipulujesz siłami grawitacyjnymi.

Opcja do podmiany: Ciężki

Poziom 1: Unoszenie się

Poziom 2: Ulepszone Skupienie w Szybkości

Poziom 3: Definiowanie Dołu lub Szarpnięcie Grawitacyjne

Poziom 4: Pole Grawitacyjne

Poziom 5: Lot

Poziom 6: Ulepszone Szarpnięcie Grawitacyjne lub Ciężar Świata

Wtrącenia MG: Świadkowie reagują nierozsądnym strachem. Dziwna interakcja posyła obiekt lub sprzymierzeńca ku przestworzom.

\subsubsection{Kopiuje Supermoce}\index{Specjalizacje!Lista!Kopiuje Supermoce}

Możesz kopiować umiejętności, zdolności i supermoce innych. 

Poziom 1: Skupienie na Umiejętności

Poziom 1: Skupienie na Umiejętności

Poziom 2: Skopiuj Moc

Poziom 3: Kradzież Mocy lub Dzikie Zdolności

Poziom 4: Ulepszone Skopiowanie Mocy

Poziom 5: Pamięć Mocy

Poziom 6: Cudowne Kopiowanie lub Wielość Kopii

Wtrącenia MG: Skopiowana moc przestaje nagle działać lub wymyka się z kontroli. Skopiowana moc nie posiada drugorzędnych mocy (np.: superszybkość bez odporności na pęd powietrza lub bycie odpornym na żar własnych ognistych pocisków). 

\subsubsection{Leci na Wspaniałych Skrzydłach}\index{Specjalizacje!Lista!Leci na Wspaniałych Skrzydłach}

Wielu superbohaterów może latać, niektórzy z nich nawet mają skrzydła. Możesz korzystać ze swoich skrzydeł do poruszania się, atakowania i obrony.

Poziom 1: Unoszenie się

Poziom 1: Krótki Lot

Poziom 2: Skrzydła-Bronie

Poziom 3: Akrobatyczny Atak lub or Latający Kompan

Poziom 4: Trudny do Trafienia

Poziom 5: Rozpęd

Poziom 6: Trudny Cel lub Mistrz Obrony

Wtrącenia MG: Skrzydło może być zranione lub nie mieć dość miejsca, przez co bohater upada. Latanie wysoko czyni postać wyraźnym celem dla niespodziewanego wroga. 

\subsubsection{Łamie Systemy}\index{Specjalizacje!Lista!Łamie Systemy}

Wykorzystujesz słabości sztucznych systemów, wliczając (ale nie ograniczając się) do programów komputerowych.

Poziom 1: Hakowanie Niemożliwości

Poziom 1: Programowanie

Poziom 2: Kontakty

Poziom 3: Sprawny Oszust lub Umiejętny Atak

Poziom 4: Skonfunduj Wroga

Poziom 5: Wsparcie Przyjaciela

Poziom 6: Przysługa lub Większy Ulepszony Potencjał

Wtrącenia MG: Twoje kontakty czasami mają swoje własne motywacje. Niekiedy urządzenia mają zabezpieczenia lub nawet pułapki.

\subsubsection{Łączy Ciało i Stal}\index{Specjalizacje!Lista!Łączy Ciało i Stal}

Twoje ciało jest częściowo maszyną.

Poziom 1: Ulepszone Ciało

Poziom 2: Interfejs

Poziom 3: Pakiet Sensoryczny lub Uzbrojenie

Poziom 4: Fuzja

Poziom 5: Głębokie Reserwy

Poziom 6: Regeneracja Umysłu lub Ultra Wzmocnienie

Wtrącenia MG: Ludzie w większości społeczności boją się kogoś, kto ma w sobie mechaniczne części.

\subsubsection{Łączy Umysł i Maszynę}\index{Specjalizacje!Lista!Łączy Umysł i Maszynę}

Elektroniczne implanty w Twoim mózgu czynią Cię supermyślicielem.

Poziom 1: Ulepszony Intelekt

Poziom 1: Umiejętności Wiedzy

Poziom 2: Kwerenda

Poziom 3: Procesor Akcji lub Telepatia Maszyn

Poziom 4: Większy Ulepszony Intelekt

Poziom 4: Umiejętności Wiedzy

Poziom 5: Wizja Przyszłości

Poziom 6: Ulepszenie Maszyny lub Regeneracja Umysłu

Wtrącenia MG: Maszyny się psują. Potężne maszyny myślące mogą przejąć kontrolę nad mniejszymi maszynami. Pewni ludzie nie ufają komuś, kto nie jest w pełni organiczny. 

\subsubsection{Ma Szlachetną Krew}\index{Specjalizacje!Lista!Ma Szlachetną Krew}

Dziedzic bogactwa i mocy, masz tytuł szlachecki i zdolności przyznae przez uprzywilejowane wychowanie. 

Opcja Zamiany Typu: Służba

Poziom 1: Przywileje Szlachty

Poziom 2: Wyszkolony Dyskutant

Poziom 3: Zaawansowany Rozkaz lub Odwaga Szlachcica

Poziom 4: Kompan-Ekspert

Poziom 5: Potwierdzenie Własnego Przywileju

Poziom 6: Przydatna Pomoc lub Umysł Lidera

Wtrącenia MG: Długi rodziny szlacheckiej są problemem bohatera. Dawno zagubione rodzeństwo chce się pozbyć swego rywala. Zabójca odnajduje postać. 

\subsubsection{Ma Tysiąc Twarzy}\index{Specjalizacje!Lista!Ma Tysiąc Twarzy}

Możesz zmienić swój wygląd, by wyglądać jak zupełnie inna osoba. 

Poziom 1: Morficzna Twarz

Poziom 1: Umiejętności Międzyludzkie

Poziom 2: Zmiana Ciała

Poziom 2: Ciało Bitewne

Poziom 3: Przebranie Innej Osoby lub Odporność

Poziom 4: Nieśmiertelny

Poziom 4: Przemyślenie Problemów

Poziom 5: Pamięć w Czyn

Poziom 6: Rozdzielenie Jaźni lub Odczytanie Myśli

Wtrącenia MG: Część przebrania zawodzi. BN myśli, że przebrana postać to ktoś, kogo zna bardzo dobrze.

\subsubsection{Mistrzowsko Posługuje się Bronią}\index{Specjalizacje!Lista!Mistrzowsko Posługuje się Bronią}

Jesteś mistrzem w używaniu pewnego rodzaju broni, czy to mieczy, biczy, noży, pistoletów, czy czegoś innego.

(Ktoś, kto Mistrzowsko Posługuje się Bronią, może mieć dodatkowy ekwipunek, wliczając broń wysokiej jakości.)

Poziom 1: Mistrz Broni

Poziom 1: Twórca Broni

Poziom 2: Obrona Bronią

Poziom 3: Szybki Atak lub Cios Rozbrajający

Poziom 4: Bez Wpadek

Poziom 5: Wyjątkowe Mistrzostwo

Poziom 6: Morderca lub Śmiertelny Cios

Wtrącenia MG: Bronie siępsują. Broniem ogą zostać ukradzione. Bronie można upuścić lub zostać rozbrojonym. 

\subsubsection{Morduje}\index{Specjalizacje!Lista!Morduje}

Jesteś asasynem, z profesji, chęci lub ponieważ na tym świecie mordujesz lub zostajesz zamordowany.

(Ktoś kto Morduje może mieć dodatkowy ekwipunek, wliczywszy 3 dawki trucizny 2 poziomu która zadaje 5 punktów obrażeń). 

Poziom 1: Atak z Zaskoczenia

Poziom 1: Umiejętności Zabójcy

Poziom 2: Szybka Śmierć

Poziom 2: Infiltrator

Poziom 3: Świadomość lub Warzenie Trucizn

Poziom 4: Lepszy Atak z Zaskoczenia

Poziom 5: Dodatkowe Obrażenia

Poziom 6: Plan Ucieczki lub Morderca

Wtrącenia MG: Większość ludzi nie reaguje dobrze na profesjonalnego zabójcę.

\subsubsection{Mówi Głosem Ziemi}\index{Specjalizacje!Lista!Mówi Głosem Ziemi}

Twoje duchowe połączenie z naturą i środowiskiem daje Ci mistyczne moce.

Poziom 1: Nasiona Furii

Poziom 1: Wiedza o Dziczy

Poziom 2: Chwytające Zielska

Poziom 3: Ukojenie Dzikiego lub Komunikacja

Poziom 4: Księżycowa Zmiana Kształtu

Poziom 5: Erupcja Insektów

Poziom 6: Wezwanie Burzy lub Trzęsienie Ziemi

Wtrącenia MG: Ranna naturalna (lecz niebezpieczna) istota jest odkryta. Ktoś poluje dla skór, zostawiając zwłoki, by gniły. Drzewo upada w lesie, jedno z ostatnich tak wielkich.

\subsubsection{Mówi do Duchów}\index{Specjalizacje!Lista!Mówi do Duchów}

Niespokojne dusze, duchy natury i żywiołaki wspomagają Cię.

(W pewnych settingach, Specjalizacja Mówi do Duchów dotyczy tylko jednego rodzaju duchów, takich jak duchy martwych, duchy natury itp.)

Poziom 1: Przepytanie Ducha

Poziom 2: Duch Kompan

Poziom 3: Rozkazywanie Duchom lub Wyczulone Zmysły

Poziom 4: Płaszcz Gniewu

Poziom 5: Wezwanie Ducha

Poziom 6: Wezwanie Międzywymiarowego Ducha lub Absorpcja Ducha

Wtrącenia MG: Niektórzy nie ufają tym, którzy się zadają z duchami. Martwi czasami wcale nie chcą rozmawiać. 

\subsubsection{Mówi do Maszyn}\index{Specjalizacje!Lista!Mówi do Maszyn}

Używasz swojego organicznego mózgu jak komputera, bezprzewodowo łącząć się z dowolnym urządzeniem elektronicznym. Możesz je kontrolować i wpływać na nie w sposób, w jaki inni nie mogą. 

Poziom 1: Umiłowanie do Maszyn

Poziom 1: Interfejs Zasięgowy

Poziom 2: Ulepszenie Maszyny

Poziom 2: Zauroczenie Maszyny

Poziom 3: Inteligentny Interfejs lub Rozkazywanie Maszynom

Poziom 4: Kompan-Maszyna

Poziom 4: Walczący z Robotami

Poziom 5: Zbieranie Informacji

Poziom 6: Kontrola Maszyny lub Ulepszony Kompan-Maszyna

Wtrącenia MG: Maszyna się psuje lub działa w nieprzewidziany sposób.

\subsubsection{Nie Potrzebuje Broni}\index{Specjalizacje!Lista!Nie Potrzebuje Broni}

Potężne ciosy, kopnięcia, zamachy łokciami i kolanami oraz ruchy całego ciała są wszystkimi broniami, których potrzebujesz. 

Poziom 1: Pięści Furii

Poziom 1: Ciało z Kamienia

Poziom 2: Atak z Rozbrojeniem

Poziom 2: Sztuki Walki

Poziom 3: Gibkość Niczym Woda lub Większy Ulepszony Potencjał

Poziom 4: Odbicie Ataków

Poziom 5: Atak Oszałamiający

Poziom 6: Mistrz Sztuk Walki lub Śmiertelne Obrażenia

Wtrącenia MG: Uderzanie pewnych wrogów boli Cię tak mocno, jak ich ranisz. Wrogowie z broniami mają większy zasięg. Skomplikowane ruchy sztuk walki mogą Cię wytrącić z równowagi.

\subsubsection{Nie Robi Zbyt Dużo}\index{Specjalizacje!Lista!Nie Robi Zbyt Dużo}

Jesteś obibokiem, ale wiesz coś o wielu rzeczach. 

Poziom 1: Lekcje Życiowe

Poziom 2: Wyluzowanie

Poziom 3: Umiejętny Atak lub Improwizacja

Poziom 4: Lekcje Życiowe

Poziom 4: Większa Umiejętność Obrony

Poziom 5: Większy Ulepszony Potencjał

Poziom 6: Korzystając z Doświadczenia Życiowego lub Szybki Umysł

Wtrącenia MG: Nowe sytuacje są konfundujące i stresujące. Przeszłe akcja (lub ich brak) wracają, by gnębić postać. 

\subsubsection{Nigdy się nie Poddaje}\index{Specjalizacje!Lista!Nigdy się nie Poddaje}

Nigdy się nie poddajesz, radzisz sobie z każdą raną, i zawsze jesteś gotowy na więcej.

Poziom 1: Ulepszone Odzyskanie Zdrowia

Poziom 1: Parcie Dalej

Poziom 2: Zignorowanie Bólu

Poziom 3: Gorączka Krwi lub Ukryta Siła

Poziom 4: Determinacja lub Wytrzymalszy Niż Wróg

Poziom 5: Jeszcze Żywy

Poziom 6: Ostateczne Zaprzeczenie lub Zignorowanie Przeszkody

Wtrącenia MG: Czasami, to ekwipunek i broń się poddają. 

\subsubsection{Nosi Egzotyczną Tarczę}\index{Specjalizacje!Lista!Nosi Egzotyczną Tarczę}

Posiadasz wspaniałą tarczę czystej mocy, która zapewnia obronę i pewne zdolności ataku.

Poziom 1: Tarcza Pola Siłowego

Poziom 1: Uderzenie Mocy

Poziom 2: Ulepszona Tarcza

Poziom 3: Leczący Puls lub Rzut Tarczą Siłową

Poziom 4: Zasilona Tarcza

Poziom 5: Ściana Mocy

Poziom 6: Skacząca Tarcza lub Wybuch Tarczy

Wtrącenia MG: Tarcza jest chwilowo nieaktywna. Wróg chwilowo przejmuje kontrolę nad tarczą.

\subsubsection{Nosi Halo Ognia}\index{Specjalizacje!Lista!Nosi Halo Ognia}

Możesz pokryć swe ciało płomieniami, co chroni Ciebie i rani Twoich wrogów. 

Poziom 1: Płaszcz Ognia

Poziom 2: Macki Płomieni

Poziom 3: Skrzydła Ognia lub Ognista Ręka Zguby

Poziom 4: Ostrze Ognia

Poziom 5: Ogniste Macki

Poziom 6: Ognisty Sługa lub Piekielny Szlak

Wtrącenia MG: Ogień pali łatwopalne materiały. Płomienie wyzwalają się spod kontroli. Prymitywne istoty boją się ognia i często atakują źródło swoich lęków. 

\subsubsection{Nosi Zasilany Pancerz}\index{Specjalizacje!Lista!Nosi Zasilany Pancerz}

Poziom 1: Zasilany Pancerz

Poziom 1: Ulepszona Moc

Poziom 2: Wyświetlacz w Hełmie

Poziom 3: Zbroja Fuzyjna lub Ulepszone Zdrowie

Poziom 4: Wystrzał Mocy

Poziom 5: Pole Mocy Zasilanego Pancerza

Poziom 6: Mistrzowska Modyfikacja Pancerza (Krótki Lot) lub Mistrzowska Modyfikacja Pancerza (Pojemnik na Cypher)

Wtrącenia MG: Nie możesz zdjąć pancerza. Pancerz działa samodzielnie. Pancerz chwilowo się wyłącza. BN-i boją się pancerza. 

\subsubsection{Oblicza Nieobliczalne}\index{Specjalizacje!Lista!Oblicza Nieobliczalne}

Nadludzkie zdolności matematyczne pozwalająci Ci na modelowanie świata na bieżąco, dając Ci przewagę nad innymi. 

Poziom 1: Prorocze Równanie

Poziom 1: Wyższa Matematyka

Poziom 2: Proroczy Model

Poziom 3: Podświadoma Obrona lub Ulepszony Intelekt

Poziom 4: Obliczenia Bitewne

Poziom 5: Większy Ulepszony Intelekt

Poziom 5: Najwyższa Matematyka

Poziom 6: Wiedza o Nieznanym lub Większy Ulepszony Intelekt 

Wtrącenia MG: Zbyt wiele przewidzianych wyników przeraża lub przeciąża i oszałamia postać. Wynik wskazuje na nadchodzącą klęskę. 

\subsubsection{Otrzymuje Boskie Błogosławieństwa}\index{Specjalizacje!Lista!Otrzymuje Boskie Błogosławieństwa}

Jako oddany wyznawca boskiej istoty, posiadasz pewne moce swego bóstwa, by czynić cuda. 

Poziom 1: Błogosławieństwo Bóstw

Poziom 2: Ulepszony Intelekt

Poziom 3: Boski Blask lub Kwiat Ognia

Poziom 4: Niebiańska Gloria

Poziom 5: Boska Interwencja

Poziom 6: Boski Symbol lub Przywołanie Demona

Wtrącenia MG: Demon bada użytkowników boskiej magii. Rywalizujący kult ma problemy z naukami postaci.

\subsubsection{Pilotuje Statki Kosmiczne}\index{Specjalizacje!Lista!Pilotuje Statki Kosmiczne}

Jesteś pilotem statku kosmicznego

Poziom 1: Pilotaż

Poziom 1: Planetarna Wiedza

Poziom 2: Kryjówka w Kosmosie

Poziom 2: Umysłowa Odporność

Poziom 3: Expert-Pilot

Poziom 3: Obeznanie ze Statkiem Kosmicznym lub Kompan-Maszyna

Poziom 4: Sensory Statku Kosmicznego

Poziom 4: Ulepszona Szybkość

Poziom 5: Znam Ten Statek Jak Własną Dłoń)

Poziom 6: Wspaniały Pilot

Poziom 6: Kontrola Zdalna lub Umiejętny Atak

Wtrącenia MG: Statek się gubi, psuje, lub zostaje zaatakowany w kosmosie. Dokonujesz odkrycia obcego pasażera na gapę. 

\subsubsection{Podróżuje przez Czas}\index{Specjalizacje!Lista!Podróżuje przez Czas}

Widzisz poprzez strumienie czasu, próbujesz w nie sięgnąć i w końcu nawet przez nie podróżować. 

(Choć wszystkie wybory postaci są zależne od zgody MG, Podróżuje przez Czas jest specjalizacją, odnośnie której MG i gracz powinni odbyć długą konwersację, by gracz znał zasady gry odnośnie podróży w czasie, jeśli istnieją w settingu MG. Postać z tą 
specjalizacją może znacząco zmienić świat gry, jeśli zasady gry na to pozwalają.) 

Poziom 1: Przebłysk

Poziom 2: Historia Przedmiotu

Poziom 3: Przyspieszenie Czasoprzestrzenne lub Pętla Czasu

Poziom 4: Czasoprzestrzenne Przesunięcie

Poziom 5: Sobowtór Czasoprzestrzenny

Poziom 6: Wezwanie przez Czas lub Podróż w Czasie

Wtrącenia MG: Powstają paradoksy. Inni pamiętają przeszłe wydarzenia inaczej.

\subsubsection{Poluje}\index{Specjalizacje!Lista!Poluje}

Jesteś wytrwałym łowcą, który potrafi upolować to, co zechce.

Poziom 1: Estetyczny Atak

Poziom 1: Łowczy

Poziom 2: Cel

Poziom 2: Skradanie się

Poziom 3: Walczący z Hordą lub Bieg i Chwyt

Poziom 4: Atak z Zaskoczena

Poziom 5: Dążenie Łowcy

Poziom 6: Większa Umiejętność Ataku lub Wiele Celów

Wtrącenia MG: Ofiara dostrzega postać. Cel nie jest taki słaby, jak się wydawał.

\subsubsection{Pomaga Swoim Przyjaciołom}\index{Specjalizacje!Lista!Pomaga Swoim Przyjaciołom}

Kochasz swoich przyjaciół i pomagasz im w każdej trudności, niezależnie od wszystkiego.

Opcja do podmiany: Porada od Przyjaciela

Poziom 1: Pomoc Przyjaciela

Poziom 1: Odwaga

Poziom 2:  Ochrona Przed Zmiennym Losem

Poziom 3: Koka lub Umiejętny Atak

Poziom 4: Obrońca Przyjaciół

Poziom 4: Ulepszona Muskulatura

Poziom 5: Zainspirowanie Akcji

Poziom 6: Głębokie Przemyślenia lub Umiejętna Obrona

Wtrącenia MG: Inni czasami mają niecne motywacje. Służby porządkowe się Tobą interesują. Nawet, gdy wszystko idzie dobrze, będzie to miało swoje konsekwencje. 

\subsubsection{Porusza się jak Kot}\index{Specjalizacje!Lista!Porusza się jak Kot}

Lekki, zwinny i pełen gracji, poruszasz się szybko i łatwo, co pozwala Ci unikać niebezpieczeństw.

Poziom 1: Większa Ulepszona Szybkość

Poziom 1: Balansowanie

Poziom 2: Umiejętności Ruchu

Poziom 2: Bezpieczny Upadek

Poziom 3: Trudny to Trafienia

Poziom 3: Ulepszone Skupienie w Szybkości lub  Większa Ulepszona Szybkość

Poziom 4: Szybki Cios

Poziom 5: Śliski

Poziom 6: Perfekcyjna Szybkość or  Większa Ulepszona Szybkość

Wtrącenia MG: Nawet kot może być niezdarny. Skok nie jest tak łatwy jak wygląda. Ucieczka jest na tyle przesadzona, że umieszcza postać w bardzo niebezpiecznym położeniu.

\subsubsection{Porusza się jak Wiatr}\index{Specjalizacje!Lista!Porusza się jak Wiatr}

Możesz się poruszać tak szybko, że rozmywasz się w oczach.

Poziom 1: Większa Ulepszona Szybkość

Poziom 1: Szybkostopy

Poziom 2: Trudny do Trafienia

Poziom 3: Perfekcyjna Szybkość lub  Większa Ulepszona Szybkość

Poziom 4: W Mgnieniu Oka

Poziom 5: Rozmazany

Poziom 6: Perfekcyjna Szybkość or Niemożliwa Szybkość

Wtrącenia MG: Powierzchnie mogą być śliskie lub oferować ukryte przeszkody. Ruch innych istot może być trudny do przewidzenia, i postać może w nie wbiec.

\subsubsection{Posiada Licencję na Broń}\index{Specjalizacje!Lista!Posiada Licencję na Broń}

Posiadasz pistolet i wiesz, jak z niego skorzystać w walce. 

(Choć Posiada Licencjęna Broń zaprojektowaniu z myślą o współczesnej broni, może także dotyczyć futurystycznych blasterów lub innych broni dystansowych.)

Poziom 1: Rewolwerowiec

Poziom 1: Wyszkolony w Broni Palnej

Poziom 2: Ostrożny Strzał

Poziom 3: Wyszkolony Rewolwerowiec lub Dodatkowe Obrażenia

Poziom 4: Podwójny Wystrzał

Poziom 5: Potrójny Wystrzał

Poziom 6: Specjalny Strzał lub Śmiertelne Obrażenia

Wtrącenia MG: Chybiony strzał lub zacięcie się broni! Atak nie odnosi stutku i akcja jest stracona, plus potrzeba dodatkowej akcji, by zająć się problemem. 

\subsubsection{Posiada Magicznego Sprzymierzeńca}\index{Specjalizacje!Lista!Posiada Magicznego Sprzymierzeńca}

Sprzymierzona magiczna istota, przywiązana do przedmiotu (np.: pomniejszy dżin w lampie lub duch w fajce) to Twój przyjaciel, obrońca i broń.

Poziom 1: Związana Magiczna Istota

Poziom 2: Więź z Obiektem

Poziom 2: Szuflada-Skrytka

Poziom 3: Mniejsze Życzenie lub Rumak

Poziom 4: Ulepszona Więź z Przedmiotem

Poziom 5: Średnie Życzenie

Poziom 6: Mistrzostwo Więzi z Obiektem lub Zaufaj Swemu Szczęściu

Wtrącenia MG: Istota nagle znika w swoim przedmiocie. Związany obiekt zostaje uszkodzony. Istota nie zgadza się i nie czyni tego, o co się nią prosi. Istota twierdzi, że odchodzi, jeśli nie wykona się dla niej pewnego zadania.

\subsubsection{Pracuje w Ciemnych Uliczkach}\index{Specjalizacje!Lista!Pracuje w Ciemnych Uliczkach}

Działasz niedostrzeżenie, kradnąc od bogatych, by osiągnąć swoje cele.

Poziom 1: Umiejętności Złodzieja

Poziom 2: Kontakty w Półświatku

Poziom 3: Niezłe Oszustwo lub Trening Gildii

Poziom 4: Złodziejski Mistrz

Poziom 5: Nieczyste Zagrania

Poziom 6: Szczur Miejski lub Wysokie Skupienie

Wtrącenia MG: Złodzieje lądują w więzieniu. Dorabiasz się potężnych wrogów.

\subsubsection{Pracuje, by Żyć}\index{Specjalizacje!Lista!Pracuje, by Żyć}

Cieszysz się, gdy możesz dobrze wykonać swoją pracę, niezależnie, czy to programowanie, budowanie domów, czy górnictwo asteroidów.

Poziom 1: Zręczny Rzemieślnik

Poziom 2: Mięśnie z Żelaza

Poziom 3: Oko do Szczegółów lub Improwizacja

Poziom 4: Ulepszona Moc

Poziom 4: Stwardniały

Poziom 5: Umiejętność Eksperta

Poziom 6: Większy Ulepszony Potencjał lub  Ciężko Zapracowana Odporność

Wtrącenia MG: Naprawy czasami zawodzą. Kable mogą być trudne do odkodowania i ciągle bbyć pod napięciem. Czasami ludzie są niegrzeczni dla tych, co pracują, by żyć. 

\subsubsection{Przebudza Sny}\index{Specjalizacje!Lista!Przebudza Sny}

Możesz wyciągnąć obrazy ze snów i umieścić je w świecie jawy.

Poziom 1: Iluzja Snów

Poziom 1: Oneiro-alchemia

Poziom 2: Złodziej Snów

Poziom 3: Sen Staje się Prawdą lub Ulepszony Intelekt

Poziom 4: Sen na Jawie

Poziom 5: Koszmar

Poziom 6: Komnata Snów lub Pole Reakcyjne

Wtrącenia MG: Niespodziewany epizod lunatykowania stawia postać w niebezpiecznej sytuacji. Koszmar wyzwala się ze snu.

\subsubsection{Przeszukuje Ruiny}\index{Specjalizacje!Lista!Przeszukuje Ruiny}

Kiedy nie biegniesz lub się chowasz, przeszukujesz ruiny cywilizacji w celu znalezienia użytecznych pozostałości, co pozwala Ci przetrwać. 

Poziom 1: Ocalały z Apokalipsy

Poziom 1: Wiedza o Ruinach

Poziom 2: Rzemieślnik Rupieci

Poziom 3: Korzystanie z Okazji lub Ulepszone Zdrowie

Poziom 4: Wiesz, Gdzie Szukać

Poziom 5: Cyphery z Odzysku

Poziom 6: Artefakty z Odzysku lub Pole Reakcyjne

Wtrącenia MG: Przedmiot stworzony z zrecyklingowanych śmieci psuje się. Ktoś pojawia się i twierdzi, że użyteczne pozostałości nalezą do niego. Zrecyklingowany cypher eksploduje. 

\subsubsection{Przewodzi}\index{Specjalizacje!Lista!Przewodzi}

Twoje naturalne zdolności przywódcze pozwalają Ci na wydawanie rozkazów, wliczając w to Twoich oddanych podległych.

Poziom 1: Naturalna Charyzma

Poziom 1: Dobra Porada

Poziom 2: Potencjał

Poziom 2:  Podstawowy Kompan

Poziom 3: Zaawansowany Rozkaz or Kompan-Ekspert

Poziom 4: Zachwyt lub Inspiracja

Poziom 5: Większy Ulepszony Potencjał

Poziom 6: Drużyna Kompanów lub Umysł Lidera

Wtrącenia MG: Kompani odnoszą klęskę, zdradzają Cię, okłamują, przechodzą na złą stronę, dają się porwać lub umierają.

\subsubsection{Przyjmuje Zwierzęcy Kształt}\index{Specjalizacje!Lista!Przyjmuje Zwierzęcy Kształt}

Możesz się zmienić w zwierzę.

Poziom 1: Zwierzęcy Kształt

Poziom 2: Komunikacja

Poziom 2: Ukojenie Dzikiego

Poziom 3: Większy Zwierzęcy Kształt lub Większa Likantropia

Poziom 4: Zwierzęce Szpiegowanie

Poziom 5: Trudny do Zamordowania

Poziom 6: Rozmazana Prędkość lub  Rozszerzony Zwierzęcy Kształt

Wtrącenia MG: Postać niespodziewanie zmienia kształt. BN jest przerażony lub agresywny w stosunku do zmiennokształtnego. Transformacja zajmuje dłużej, niż się spodziewano.

Większa Likantropia stosuje się do używania Zwierzęcego Kształtu.

\subsubsection{Przywdziewa Połyskliwy Lód}\index{Specjalizacje!Lista!Przywdziewa Połyskliwy Lód}

Rozkazujesz zimowej mocy zimna i lodu.

Poziom 1: Lodowa Zbroja

Poziom 2: Lodowy Dotyk

Poziom 3: Mrozący Dotyk lub Lodowa Kreacja

Poziom 4: Twarda Lodowa Zbroja

Poziom 5: Emisja Zimna

Poziom 6: Śniegowa Zamieć lub Lodowe Rękawice

Wtrącenia MG: Lód czyni powierzchnie śliskimi. Ektremalne zimno sprawia, że przedmioty pękają i się psują. 

\subsubsection{Pływał z Piratami}\index{Specjalizacje!Lista!Pływał z Piratami}

Pływałeś razem ze straszliwymi piratami, ale zdecydowałeś się zerwać z piractwem i poświęcićsię innemu celowi. Powstaje pytanie: czy twoja przeszłość pozwoli o sobie zapomnieć?

Poziom 1: Zignorowanie Bólu

Poziom 1: Marynarz

Poziom 2: Korzystanie z Okazji

Poziom 2: Przerażająca Reputacja

Poziom 3: Umiejętny Atak lub Umiejętna Obrona

Poziom 4: Morskie Nogi

Poziom 4: Umiejętności Ruchu

Poziom 5: Zagubieni w Chaosie

Poziom 6: Pojedynek na Śmierć i Życie lub Następny Atak

Wtrącenia MG: Istnieje wiele niebezpieczeństw Siedmiu Mórz, wliczając sztormy i zarazy. Inni piraci czasem awansują poprzez zdradę. Pirat wyśledził dawnych kompanów, by odkryć ukryty skarb.

\subsubsection{Rozciąga się}\index{Specjalizacje!Lista!Rozciąga się}

Twoje ciało jest gumowe i elastyczne, zdolne rozciągać się na wielkie długości i kompresować z powrotem.

Poziom 1: Człowiek-Guma

Poziom 1: Daleki Krok

Poziom 2: Elastyczny Chwyt

Poziom 2: Bezpieczny Upadek

Poziom 3: Przeniknięcie Przez Barierę lub  Przekierowanie Ataku

Poziom 4: Odporność

Poziom 5: Mistrz Ruchu

Poziom 6: Ruch i Multiatak lub Jeszcze Żywy

Wtrącenia MG: Atak lub efekt wchodzi w interakcję z elastycznością BG. Rozciągnięta kończyna staje się przeciążona i słaba. 

\subsubsection{Rozdziera Ściany Świata}\index{Specjalizacje!Lista!Rozdziera Ściany Świata}

Szybkość i fazowanie daje Ci unikalną zdolność unikania zagrożeń i zadawania obrażeń jednocześnie. 

Poziom 1: Bieg Fazowy

Poziom 1: Przeszkadzający Dotyk

Poziom 2: Fazowe Zadrapanie

Poziom 3: Niewidzialne Fazowanie lub Przechodzenie Przez Ściany

Poziom 4: Detonacja Fazowa

Poziom 5: Bardzo Długi Bieg Fazowy

Poziom 6: Potężniejsze Fazowanie lub Nietykalny Podczas Ruchu

Wtrącenia MG: Poruszanie się tak szybko czasami prowadzi wprost na niespodziewane, egzotyczne przeszkody.

\subsubsection{Rozwiązuje Tajemnice}\index{Specjalizacje!Lista!Rozwiązuje Tajemnice}

Jesteś mistrzem dedukcji, używającym dowodów, by odnaleźć odpowiedzi.

Poziom 1: Śledczy

Poziom 1: Detektyw

Poziom 2: Z Dala od Niebezpieczeństwa

Poziom 3: Dobrze Wykształcony lub Umiejętny Atak

Poziom 4: Wyciągnięcie Wniosków

Poziom 5: Ogarnięcie Sytuacji

Poziom 6: Przejęcie Inicjatywy lub Większa Umiejętność Obrony

Wtrącenia MG: Dowody znikają, fałszywe tropy konfundują, a świadkowie kłamią. Początkowe wnioski mogą być błędne.

\subsubsection{Rośnie do Gigantycznych Rozmiarów}\index{Specjalizacje!Lista!Rośnie do Gigantycznych Rozmiarów}

Na krótkie okresy, rośniesz większy i, z odpowiednim doświadczeniem, do prawdziwie gigantycznych rozmiarów.

Poziom 1: Wzrost

Poziom 1: Przeogromny

Poziom 2: Większy

Poziom 2:  Zalety Bycia Dużym

Poziom 3: Wielki lub Rzut

Poziom 4: Chwyt

Poziom 5: Wielgachny

Poziom 6: Kolos lub Śmiertelne Obrażenia

Wtrącenia MG: Nagły wzrost przewraca meble lub sprawia, że przebijasz sufit. Powiększona postać przebija podłogę. 

\subsubsection{Rzeźbi Twardym Światłem}\index{Specjalizacje!Lista!Rzeźbi Twardym Światłem}

Tworzysz fizyczne przedmioty z twardego światła, które możesz wykorzystać do obrony lub ataku.

Poziom 1: Automatyczny Blask

Poziom 1: Chwilowe Światło

Poziom 2: Macki Mocy

Poziom 3: Twardsze Światło lub Rzeźbienie Światłem

Poziom 4: Większy Ulepszony Intelekt

Poziom 5: Ulepszone Rzeźbienie Światłem

Poziom 6: Pole Obronne lub Lot

Wtrącenia MG: Przedmiot z twardego światła przedwcześnie znika. Przedmiot z twardego światła nie może wpłynąć na daną istotę lub kolor.

\subsubsection{Rzuca ze Śmiertelną Dokładnością}\index{Specjalizacje!Lista!Rzuca ze Śmiertelną Dokładnością}

Wszystko co opuszcza Twoją dłoń idzie dokładnie gdzie sobie tego życzysz, z prędkością i w miejsce, gdzie osiągnie najlepszy efekt.

Poziom 1: Precyzja

Poziom 2: Ostrożny Rzut

Poziom 3: Szybki Rzut lub Umiejętna Obrona

Poziom 4: Wszystko Jest Bronią

Poziom 4: Wyspecjalizowany w Rzucaniu

Poziom 5: Wir Rzutek

Poziom 6: Śmiertelne Obrażenia lub Mistrzostwo Obrony

Wtrącenia MG: Chybione ataki trafiają w zły cel. Rykoszety mogą być niebezpieczne. Improwizowane bronie się psują. 

\subsubsection{Spaceruje w Dzikich Lasach}\index{Specjalizacje!Lista!Spaceruje w Dzikich Lasach}

Posługujesz się magią natury, która czerpie z potęgi drzew. 

Poziom 1: Życie w Dziczy

Poziom 1: Dodatkowy Odpoczynek

Poziom 2: Ciało z Drewna

Poziom 3: Drzewny Kompan lub Dzika Świadomość

Poziom 4: Podróż Przez Drzewa

Poziom 5: Wielkie Drzewo

Poziom 6: Straszny Las lub Odżywczy Wykwit

Wtrącenia MG: Drewniana postać zapala się. Dziki zamach konarem drzewa uderza w sprzymierzeńca. Pewne drzewa mają mroczne serca i nienawidzą wszystkich ludzi.

\subsubsection{Stawia Umysł Ponad Materią}\index{Specjalizacje!Lista!Stawia Umysł Ponad Materią}

Możesz poruszać telekinetycznie przedmioty bez fizycznego dotykania ich.

Poziom 1: Odbicie Ataków

Poziom 2: Telekineza

Poziom 3: Chmura Ochronna lub Ulepszenie Siły

Poziom 4: Aportacja

Poziom 5: Atak Psychikinetyczny

Poziom 6: Ulepszona Aportacja lub Przebudowa

Wtrącenia MG: Jeden mentalny błąd, a poruszanie obiekty upadają, a kruche obiekty niszczeją. Czasami zły przedmiot się porusza, upada lub niszczeje.

\subsubsection{Szuka Kłopotów}\index{Specjalizacje!Lista!Szuka Kłopotów}

Jesteś niebezpieczny i lubisz dobrą walkę.

Poziom 1: Pięści Furii

Poziom 1: Opatrywanie Ran

Poziom 2: Obrońca

Poziom 2: Bezpośredni

Poziom 3: Umiejętny Atak lub Większy Ulepsozny Potencjał

Poziom 4: Pozbawienie Przytomności

Poziom 5: Mistrzostwo Ataków

Poziom 6: Większa Ulepszona Moc lub Śmiertelne Obrażenia

Wtrącenia MG: Bronie psują się lub wypadają nawet z najsilniejszego uchwytu. Atakujący mogą się potknąć i upaść. Nawet pole bitwy może działać przeciwko Tobie, gdy przedmioty upadają.

\subsubsection{Szybko się Uczy}\index{Specjalizacje!Lista!Szybko się Uczy}

Radzisz sobie z trudnymi sytuacjami w miarę, jak się pojawiają, ucząc się czegoś nowego za każdym razem.

Poziom 1: Ulepszony Intelekt

Poziom 1: Oto Twój Problem

Poziom 2: Szybka Nauka

Poziom 3: Ciężki do Rozproszenia

Poziom 3: Ulepszone Skupienie w Inteligencji lub Skupienie na Umiejętności

Poziom 4: Dzielenie się Wiedzą

Poziom 5: Ulepszony Intelekt

Poziom 5: Parę Sztuczek w Zanadrzu

Poziom 6: Dwie Sprawy na Raz lub Umiejętna Obrona

Wtrącenia MG: Wypadki i pomyłki są świetnymi nauczycielami.

\subsubsection{Tańczy z Czarną Materią}\index{Specjalizacje!Lista!Tańczy z Czarną Materią}

Możesz manipulować ciemnością i "ciemną materią".

Poziom 1:  Wstęgi Mrocznej Materii

Poziom 2: Skrzydła Pustki

Poziom 3: Płaszcz Ciemnej Materii lub Cios Ciemnej Materii

Poziom 4: Powłoka Ciemnej Materii

Poziom 5: Podróżnik Niszczącego Wiatru

Poziom 6: Budowla Ciemnej Materii lub W Objęciach Nocy

Wtrącenia MG: Czarna Materia wycofuje się, zupełnie, jakby miała swoją własną wolę. 

\subsubsection{Tworzy Dziwną Naukę}\index{Specjalizacje!Lista!Tworzy Dziwną Naukę}

Twoja nadnaturalne wejrzenie w rzeczywistość czyni z Ciebie naukowca zdolnego do wielu rzeczy.

Poziom 1: Analiza Laboratoryjna

Poziom 1: Umiejętności Wiedzy

Poziom 2: Modyfikacja Urządzenia

Poziom 3: Lepsze Życie Dzięki Chemii lub Ulepszone Zdrowie

Poziom 4: Umiejętności Wiedzy

Poziom 4: Troszkę Szalony

Poziom 5: Przełom w Badaniach Dziwnej Nauki

Poziom 6: Niemożliwe Osiągnięcie Naukowe

Poziom 6: Wynalazca lub Pole Obronne

Wtrącenia MG: Twoje twory mogą się wymknąć spod kontroli. Czasami nie można przewidzieć efektów ubocznych. Dziwna nauka przeraża ludzi i przyciąga uwagę mediów. Kiedy przedmiot stworzony lub zmodyfikowany przez dziwną naukę się 
rozładowuje, wybucha. 

\subsubsection{Tworzy Iluzje}\index{Specjalizacje!Lista!Tworzy Iluzje}

Tworzysz obrazy ze światła tak perfekcyjne, że wydają się być realne. 

Poziom 1: Mniejsza Iluzja

Poziom 2:  Iluzoryczne Przebranie

Poziom 3: Rzuć Iluzję lub Większa Iluzja

Poziom 4: Iluzyjne Ja

Poziom 5: Przerażający Obraz

Poziom 6: Wielka Iluzja lub Permanentna Iluzja

Wtrącenia MG: Ciężko uwierzyć w iluzję. Iluzja zostaje przejrzana w najgorszym możliwym momencie.

\subsubsection{Tworzy Unikalne Obiekty}\index{Specjalizacje!Lista!Tworzy Unikalne Obiekty}

Jesteś wynalazcą dziwnych i użytecznych przedmiotów. 

Poziom 1: Rzemieślnik

Poziom 1: Mistrz Identyfikacji

Poziom 2: Mechanik Artefaktów

Poziom 2: Szybka Robota

Poziom 3: Mistrzowski Rzemieślnik lub Wbudowane Bronie

Poziom 4: Twórca Cypherów

Poziom 5: Innowator

Poziom 6: Wynalazca lub Zbroja Fuzyjna

Wtrącenia MG: Przedmiot zalicza awarię, niszczeje lub kończy swój byt w katastrofalny lub niespodziewany sposób.

\subsubsection{Ucieka Precz}\index{Specjalizacje!Lista!Ucieka Precz}

Twoim pierwszym instynktem jest ucieczka od niebezpieczeństwa, i jesteś w tym bardzo dobry. 

Poziom 1: W Defensywie

Poziom 2: Ulepszona Szybkość

Poziom 2: Szybka Ucieczka

Poziom 3: Niemożliwa Szybkość lub Większa Ulepszona Szybkość

Poziom 4: Determinacja

Poziom 4: Szybki Umysł

Poziom 5: Ponowne Ukrycie się

Poziom 6: Ucieczka lub Umiejętna Obrona

Wtrącenia MG: Szybkie ruchy czasami sprawiają, że upuszczasz przedmioty, poślizgujesz się lub przypadkowo kierujesz się w złą stronę

\subsubsection{Ujeżdża Błyskawicę}\index{Specjalizacje!Lista!Ujeżdża Błyskawicę}

Generujesz i wyzwalasz energię elektryczną.

Poziom 1: Szok

Poziom 1: Naładowanie

Poziom 2: Jeździec Błyskawicy

Poziom 3: Elektryczny Pancerz lub Wyssanie Ładunku

Poziom 4: Promienie Mocy

Poziom 5: Elektryczny Lot

Poziom 6: Szybkość Błyskawicy lub Ściana Błyskawic

Wtrącenia MG: Przypadkowi ludzie zostają zaatakowani prądem. Obiekty eksplodują. 

\subsubsection{Uzdrawia}\index{Specjalizacje!Lista!Uzdrawia}

Możesz leczyć innych dotykiem, wpływać na czas, by pomagać innym, i ogólnie jesteś kochany przez wszystkich.

Poziom 1: Leczący Dotyk

Poziom 2: Uzdrowienie

Poziom 3: Uzdrawiająca Fontanna lub Cudowne Zdrowie

Poziom 4: Zainspirowanie Akcji

Poziom 5: Cofnij

Poziom 6: Większy Leczący Dotyk lub Przywrócenie Życia

Wtrącenia MG: Próby uzdrowienia mogą zamiast tego spowodować krzywdę. Społeczność lub jednostka potrzebują uzdrowiciela tam bardzo, że przetrzymują go wbrew jego woli.

\subsubsection{Walczy Nieczysto}\index{Specjalizacje!Lista!Walczy Nieczysto}

Zrobisz wszystko, by wygrać walkę: będziesz gryzł, drapał, kopał, oszukiwał i czynił jeszcze gorsze rzeczy.

Poziom 1: Łowczy

Poziom 1: Stalker

Poziom 2: Skradanie się

Poziom 2: Cel

Poziom 3: Zdrada lub Atak z Zaskoczenia

Poziom 4: Gierki Umysłowe

Poziom 4: Zręczny Wojownik

Poziom 5: Korzyści z Otoczenia

Poziom 6: Obrócenie Noża lub Morderca

Wtrącenia MG: Ludzie nie cenią tych, którzy oszukują lub walczą bez honoru. Czasami brudna sztuczka uderza w Ciebie z powrotem. 

\subsubsection{Walczy z Robotami}\index{Specjalizacje!Lista!Walczy z Robotami}

Łatwo przychodzi Ci walka z robotami, automatonami i maszynami.

Poziom 1: Słabe Punkty Maszyn

Poziom 1: Umiejętności Technologiczne

Poziom 2: Ochrona Przed Robotami

Poziom 2: Polowanie na Maszyny

Poziom 3: Rozbrojeni Urządzenia lub Atak z Zaskoczenia

Poziom 4: Walczący z Robotami

Poziom 5: Wyssanie Mocy

Poziom 6: Deaktywacja Mechanizmu lub Śmiertelne Obrażenia

Wtrącenia MG: Robot eksploduje po pokonaniu. Inne roboty szukajązemsty na postaci. 

\subsubsection{Walcząc, Porywa Tłum}\index{Specjalizacje!Lista!Walcząc, Porywa Tłum}

Jesteś ryzykantem wywijającym mieczem, który posiada zachwycający styl walki, który przyjemnie jest oglądać. 

Poziom 1: Estetyczny Atak

Poziom 2: Szybki Blok

Poziom 3: Akrobatyczny Atak lub Czcze Przechwałki

Poziom 4: Chronienie Sprzymierzeńca

Poziom 4: Szybkie Zabójstwo

Poziom 5: Korzyśći z Otoczenia

Poziom 6: Szybki Umysł lub Kontratak

Wtrącenia MG: Pokaz okazuje się głupiutki, niezdarny lub nieatrakcyjny. 

\subsubsection{Wolałby Czytać}\index{Specjalizacje!Lista!Wolałby Czytać}

Książki to Twoi przyjaciele. Co jest ważniejszego od wiedzy? Nic. 

Poziom 1: Wiedza to Potęga

Poziom 2: Większy Ulepszony Intelekt

Poziom 3: Stosowanie Swojej Wiedzy lub Skupienie na Umiejętności

Poziom 4: Wiedza to Potęga

Poziom 4: Wiedza o Nieznanym

Poziom 5: Większy Ulepszony Intelekt

Poziom 6: Wiedza to Potęga

Poziom 6: Wieża Intelektu lub Czytając Znaki

Wtrącenia MG: Książki się palą, moczą lub gubią. Komputery się psują lub tracą zasilanie. Okulary się tłuką.

\subsubsection{Wpada w Furię}\index{Specjalizacje!Lista!Wpada w Furię}

Kiedy wpadasz w furię, wszyscy wpadają w popłoch.

Poziom 1: Szał

Poziom 2: Większa Ulepszona Moc

Poziom 2: Umiejętności Ruchu

Poziom 3: Potężne Uderzenie lub Wojownik Bez Zbroi

Poziom 4: Większy Szał

Poziom 5: Atak i Ponowny Atak

Poziom 6: Większy Ulepszony Potencjał lub Śmiertelne Obrażenia

Wtrącenia MG: To łatwe dla berserka, by utracić kontrolę i zaatakować zarówno przyjaciół jak i wrogów.

\subsubsection{Wspiera Społeczność}\index{Specjalizacje!Lista!Wspiera Społeczność}

Utrzymujesz miejsce gdzie żyjesz bezpieczne od wszelkich niebezpieczeństw.

Poziom 1: Wiedza o Społeczności

Poziom 1: Lokalny Aktywista

Poziom 2: Umiejętny Atak

Poziom 3: Furia Pasterza lub Umiejętna Obrona

Poziom 4: Większy Ulepszony Potencjał

Poziom 5: Unik

Poziom 6: Większa Umiejętność Ataku lub Mur Obronny

Wtrącenia MG: Ludzie w społeczności nie rozumieją motywów postaci. Rywele próbują pozby się postaci.

\subsubsection{Wyje do Księżyca}\index{Specjalizacje!Lista!Wyje do Księżyca}

Na krótki czas stajesz się przerażającą i potężną istotą, która ma problemy, żeby się kontrolować. 

Poziom 1: Likantropia

Poziom 2: Kontrolowana Przemiana

Poziom 3: Większy Likantrop lub Większa Likantropia

Poziom 4: Większa Kontrolowana Zmiana

Poziom 5: Ulepszona Likantropia

Poziom 6: Śmiertelne Obrażenia lub Perfekcyjna Kontrola

Wtrącenia MG: Przemiana przebiega w sposób niekontrolowany. Ludzie boją się potworów. 

\subsubsection{Wysysa Energię}\index{Specjalizacje!Lista!Wysysa Energię}

Wysysasz energię tak z maszyn, jak i z istot w celu wzmocnienia samego siebie.

(Roboty i inne żywe maszyny powinny być traktowane jak istoty, a nie maszyny, dla celów wysysania z nich energii.)

Poziom 1: Wyssanie Maszyny

Poziom 2: Wyssanie Istoty

Poziom 3: Wyssanie na Zasięg lub Wysuszająca Konsupcja

Poziom 4: Przechowanie Energii

Poziom 5: Dzielona Moc

Poziom 6: Wybuchowe Rozładowanie lub Pula Słoneczna

Wtrącenia MG: Wyssana moc nosi z sobą coś niechcianego – przymusy, choroby lub obce myśli. Wyssana moc może przeciążyć postać, powodując kłopoty.

\subsubsection{Wyszedł z Obelisku}\index{Specjalizacje!Lista!Wyszedł z Obelisku}

Twoje ciało, twarde jak kryształ, daje Ci unikalne zdolności, zyskane po wejściu w interakcję z lewitującym, kryształowym obeliskiem.

Poziom 1: Kryształowe Ciało

Poziom 2: Unoszenie się

Poziom 3: Zamieszkując Kryształ lub Nieruszony

Poziom 4: Kryształowe Soczewki

Poziom 5: Częstotliwość Rezonansowa

Poziom 6: Trzęsienie Rezonansowe lub Powrót do Obelisku

Wtrącenia MG: Cyphery i artefakty działają niespodziewanie w rękach postaci. 

\subsubsection{Włada Dziką Magią}\index{Specjalizacje!Lista!Włada Dziką Magią}

Jesteś użytkownikiem magii, który uczy się różnorodnych zaklęć zamiast skupiać na jednej szkole magii.

Poziom 1: Magiczne Zasoby

Poziom 1: Przerzucanie Cypherów

Poziom 2: Zwiększenie Limitu Subtelnych Cypherów

Poziom 3: Przypływ Cyphera lub Szybsza Dzika Magia

Poziom 4: Zwiększenie Limitu Subtelnych Cypherów

Poziom 5: Wyuczone Zaklęcia

Poziom 6: Maksymalizacja Cyphera lub Dzikie Oświecenie

Wtrącenia MG: Zaklęcie działa losowo lub uderza w rzucającego. Coś przeszkadza w przygotowaniu zaklęć. Rzucanie zaklęć przyciąga uwagę potężnej istoty lub rywala.  Zaklęcie-cypher podczas rzucania zamienia się w przypadkowy cypher. 

\subsubsection{Włada Magnetyzmem}\index{Specjalizacje!Lista!Włada Magnetyzmem}

Rozkazujesz metalom i mocom magnetycznym.

Poziom 1: Poruszanie Metalu

Poziom 2: Odparcie Metalu

Poziom 3: Niszczenie Metalu lub Nakierowywany Pocisk

Poziom 4: Pole Magnetyczne

Poziom 5: Kontrola Metalu

Poziom 6: Diamagnetyzm lub Stalowy Cios

Wtrącenia MG: Metal się obraca, zgina i produkuje odpryski. Problem z koncentracją może sprawić, że coś Ci upadnie o złym czasie. 

\subsubsection{Włada Mocami Mentalnymi}\index{Specjalizacje!Lista!Włada Mocami Mentalnymi}

Wytrenowałeś swój umysł, by wykonywać zaskakujące psychiczne zadania. 

Poziom 1: Telepatia

Poziom 2: Czytanie Myśli

Poziom 3: Psioniczna Erupcja lub Psioniczna Sugestia

Poziom 4: Podłączony do Cudzych Zmysłów

Poziom 5: Wizja Przyszłości

Poziom 6: Kontrola Umysłu lub Sieć Telepatyczna

Wtrącenia MG: Coś podejrzanego w umyśle celu jest przerażające. Cel może odczytać myśli postaci.

\subsubsection{Włada Niewidzialną Mocą}\index{Specjalizacje!Lista!Włada Niewidzialną Mocą}

Naginasz światło i manipulujesz promieniami mocy dla ataku i obrony.

Poziom 1: Zniknięcie

Poziom 2: Macki Mocy

Poziom 2: Wyostrzone Zmysły

Poziom 3: Bariera Pola Siłowego lub Masowe Znikanie

Poziom 4: Niewidzialność

Poziom 5: Pole Obronne

Poziom 6: Wybuch lub Generacja Pola Siłowego

Wtrącenia MG: Niewidzialność częściowo zanika, ujawniając obecnośc postaci. Pole silowe jet przebite przez niecodzienny lub niespodziewany atak.

\subsubsection{Włada Rojem}\index{Specjalizacje!Lista!Włada Rojem}

Owady. Szczury. Nietoperze. Nawet ptaki. Władasz jednym rodzajem małych istot, które Tobie podlegają. 

Poziom 1: Wpływ na Rój

Poziom 2: Kontrola Roju

Poziom 3: Żywa Zbroja lub Umiejętny Atak

Poziom 4: Wezwanie Roju

Poziom 5: Pozyskanie Nietypowego Kompana

Poziom 6: Śmiercionośny Rój lub Umiejętna Obrona

Wtrącenia MG: Polecenie jest omylnie zinterpretowane. Kontrola jest chwilowa lub utracona. Ugryzienia i użądlenia nie są nietypowe dla władających rojami.

\subsubsection{Włada Zaklęciami}\index{Specjalizacje!Lista!Włada Zaklęciami}

Poprzez specjalizowanie się w zaklęciach i posiadanie księgi zaklęć, możesz szybko rzucać zaklęcie, takie jak błyskawicę, ogień, żywe cienie i przywoływanie. 

Poziom 1: Magiczny Błysk

Poziom 2: Promień Konfuzji

Poziom 3: Kwiat Ognia lub Przywołanie Wielkiego Pająka

Poziom 4: Przesłuchanie Duszy

Poziom 5: Ściana z Granitu

Poziom 6: Przywołanie Demona lub Słowo Śmierci

Wtrącenia MG: Zaklęcie działa źle. Przywołana istota rzuca się na czarownika. Mag-przeciwnik jest przyciągany przez magię użytkownika. 

\subsubsection{Zabawia}\index{Specjalizacje!Lista!Zabawia}

Występujesz, głównie dla innych ludzi.

Poziom 1: Beztroska

Poziom 2: Zainspirowanie Ułatwienia

Poziom 3: Umiejętności Wiedzy lub Większy Ulepszony Potencjał

Poziom 4: Uspokojenie

Poziom 5: Przydatna Pomoc

Poziom 6: Inspirujący Performer lub Okrutne Przedstawienie

Wtrącenia MG: Publiczność jest ziritowana lub obrażona. Muzyczne instrumenty się psują. Farby usychają w słoiczkach. Słowa wiersza lub piosenki wypadają Ci z pamięci.

\subsubsection{Zabija Potwory}\index{Specjalizacje!Lista!Zabija Potwory}

Zabijasz potwory.

(Choć noszenie miecza w settingu, w którym ludzie zazwyczaj nie noszą takich broni jest ok, możesz zmienić moce powiązane z mieczem Zabija Potwory na inną broń, taką jak pistolet ze srebrnymi pociskami.)

Poziom 1: Wyszkolony w Mieczach

Poziom 1: Sposób na Potwory

Poziom 1: Wiedza o Potworach

Poziom 2: Legendarna Wola

Poziom 3: Wyszkolony Morderca

Poziom 3: Ulepszony Sposób na Potwory lub Przekierowanie Ataku

Poziom 4: Niezłomny

Poziom 5: Większa Umiejętnosć Ataku (miecze)

Poziom 6: Morderca lub Heroiczny sposób na Potwory

Wtrącenia MG: Potwór stworzył pułapkę lub wziął Cię z zaskoczenia. Potwór ma zdolności, o których początkowo nie wiedziałeś. Matka potwora poprzysięga Ci zemstę. 

\subsubsection{Zadaje się z Martwymi}\index{Specjalizacje!Lista!Zadaje się z Martwym}

Martwi odpowiadają na Twoje pytania, a ich reanimowane ciała służą Tobie.

Poziom 1: Mówiący ze Zmarłymi

Poziom 2: Nekromancja

Poziom 3: Poznanie Lokacji lub Naprawa Ciała

Poziom 4: Większa Nekromancja

Poziom 5: Przerażające Spojrzenie

Poziom 6: Prawdziwa Nekromancja lub Słowo Śmierci

Wtrącenia MG: Reputacja postaci jako nekromanty ją wyprzedza. Zwłoki szukają zemsty za grzech zostania ożywionymi.

\subsubsection{Zaprowadza Sprawiedliwość}\index{Specjalizacje!Lista!Zaprowadza Sprawiedliwość}

Naprawiasz krzywdy, bronisz niewinnych i karasz winnych.

Poziom 1: Dokonanie Osądu

Poziom 1: Osąd

Poziom 2: Obrona Niewinnego 

Poziom 2: Ulepszony Osąd

Poziom 3: Chroń Wszystkich Niewinnych lub Ukaranie Winnego

Poziom 4: Odnalezienie Winnych

Poziom 4: Większy Osąd

Poziom 5: Ukaranie Wszystkich Winnych 

Poziom 6: Potępienie Winnych lub Zainspirowanie Niewinnych

Wtrącenia MG: Wina lub niewinność mogą być skomplikowane. Niektórzy ludzie gardzą samo-ustanowionymi sędziami. Dokonywanie osądów sprawia, iż zyskujemy sobie wrogów. 

\subsubsection{Zmniejsza się}\index{Specjalizacje!Lista!Zmniejsza się}

Możesz się zmniejszać do rozmiarów robaka, a z odpowiednią praktyką, być nawet mniejszym.

Poziom 1: Zmniejszenie się

Poziom 1: Niezauważalny

Poziom 2: Mniejszy

Poziom 2: Zalety Bycia Małym

Poziom 3: Wzrost lub Szybkie Skurczenie się

Poziom 4: Mały Lot

Poziom 5: Zmniejszenie Innych

Poziom 6: Większy lub Malutki

Wtrącenia MG: Istota myśli, że bohater to potencjalne pożywienie. Mała postać zostaje uwięziona w małej przestrzeni lub pod spadającym obiektem.

Postać, która Zmniejsza się, która wybiera zdolności takie jak Wzrost, nigdy nie będzie tak duża jak ktoś, kto Rośnie do Gigantycznych Rozmiarów, ale może cię cieszyć zaletami bycia dużym lub małym, w zależności od potrzeb.

\subsubsection{Został Przepowiedziany}\index{Specjalizacje!Lista!Został Przepowiedziany}

Jesteś "Wybrańcem" i przepowiednie, prognozy lub inne metody mówią, że dokonasz w pewnym momencie wielkich rzeczy.

Poziom 1: Umiejętności Międzyludzkie

Poziom 1: Wiedza

Poziom 2: Przeznaczenie Wielkości

Poziom 3:  Przezwyciężając Wszystkie Przeciwności lub Ciężko Zapracowana Odporność

Poziom 4: Centrum Uwagi

Poziom 5: Wskaż Im Drogę

Poziom 6: Jak Przepowiedziano or Większy Ulepszony Potencjał

Wtrącenia MG: Przeciwnik przepowiedziany w przepowiedni się pojawia. Niewierni grożą, że zrujnują plany postaci. Postać ma reputację w pewnych kręgach jako udawaniec. 

\subsubsection{Żyje w Dziczy}\index{Specjalizacje!Lista!Żyje w Dziczy}

Możesz przetrwać w dziczy, w której inni nie potrafią.

Poziom 1: Życie w Dziczy

Poziom 1: Ulepszona Moc

Poziom 2: Przetrwanie w Dziczy

Poziom 2: Badacz Dziczy

Poziom 3: Zwierzęce Zmysły lub Dzika Zachęta

Poziom 4: Dzika Świadomość

Poziom 5: Przyroda po Twojej Stronie

Poziom 6: Jedność z Dziczą lub Dziki Kamuflaż 

Wtrącenia MG: Ludzie w miastach i miasteczkach czasami są niechętni tym, którzy wyglądają (i pachną) jakby żyli w dziczy, uważając ich za ignorantów lub barbarzyńców.
\input{src/Tworzenie nowych Specjalizacji.tex}
\input{src/Zdolności.tex}

% below go the alphabetic abilities lists

% hyperref !!! https://tex.stackexchange.com/questions/180571/making-clickable-links-to-sections-with-hyperref

\chapter{Zdolności w kolejności alfabetycznej}

\section{A}

\textbf{Uśmiech i Słowo}\index{Zdolności!Alfabetycznie!Uśmiech i Słowo}\label{sec:Uśmiech i Słowo} - kiedy korzystasz z Wysiłku do dowolnej akcji interakcji społecznej - nawet takiej która polega na uspokajaniu zwierząt lub komunikowania się z kimś, czyim językiem nie mówisz - uzyskujesz darmowy poziom Wysiłku na tym zadaniu. Akcja.

\textbf{Przydatna Pomoc}\index{Zdolności!Alfabetycznie!Przydatna Pomoc}\label{sec:Przydatna Pomoc} - kiedy pomagasz komuś z zadaniem i stosuje on poziom wysiłku, zyskuje on darmowy poziom Wysiłku na tym zadaniu. Umożliwienie. 

\textbf{Absorpcja Energii}\index{Zdolności!Alfabetycznie!Absorpcja Energii}\label{sec:Absorpcja Energii} (7 punktów Intelektu) - dotykasz obiektu i absorbujesz jego energię. Jeśli dotykasz zamanifestowanego Cyphera, czynisz go bezużytecznym. Jeśli dotykasz artefaktu, rzuć na jego wyczerpanie. Jeśli dotykasz innego rodzaju zasilanego urządzenia lub maszyny, GM określa, czy jego moc jest w pełni wyssana. W każdym razie, absorbujesz energię z obiektu i odzyskujesz 1k10 punktów Intelektu. Jeśli to dałoby Ci więcej punktów Intelektu niż maksimum Twojej Puli, dodatkowe punkty są utracone, i musisz wykonać rzut na Obronę Mocy. Trudność tego rzutu to numer punktów powyżej Twojego maksimum, które zaabsorbowałeś. Jeśli oblejesz ten rzut, otrzymujesz 5 punktów obrażeń i nie możesz podejmować działań przez jedną rundę. Możesz wykorzystać tę zdolność jako akcję obronną kiedy jesteś celem ataku zdolnością. Taka akcja niweluje atak zdolnością, a ty absorbujesz energię, jakby pochodziła z urządzenia. Akcja.

\textbf{Absorpcja Energii Kinetycznej}\index{Zdolności!Alfabetycznie!Absorpcja Energii Kinetycznej}\label{sec:Absorpcja Energii Kinetycznej} - absorbujesz porcję energii ataku fizycznego lub uderzenia. Negujesz 1 punkt obrażeń, które normalnie byś poniósł i przechowujesz tę energię. Po tym, jak zaabsorbujesz 1 punkt energii, kontynuujesz obniżać obrażenia o 1 punkt z nadchodzących ataków, ale pozostała energia wycieka z Ciebie w formie błysku nieszkodliwego światła (nie możesz przechowywać na raz więcej niż 1 punktu energii w tym samym czasie). Umożliwienie. 

\textbf{Absorpcja Czystej Energii}\index{Zdolności!Alfabetycznie!Absorpcja Czystej Energii}\label{sec:Absorpcja Czystej Energii} - kiedy korzystasz z Absorpcji Energii Kinetycznej, możesz także absorbować i przechowywać energię ataków bazujących na czystej energii (światło, promieniowanie, energie międzywymiarowe, psioniczne itp.) lub z przekaźników owej energii, gdy masz z nimi bezpośredni kontakt. Ta zdolność nie zmienia tego, ile punktów energii możesz przechowywać. Jeśli masz również Ulepszoną Absorpcję Energii Kinetycznej, możesz również absorbować do 2 punktów obrażeń ze źródeł czystej energii. Umożliwienie. 

\textbf{Akceleracja}\index{Zdolności!Alfabetycznie!Akceleracja}\label{sec:Akceleracja} (4+ punkty Intelektu) - Twoje słowa umacniają ducha postaci w bliskim zasięgu, która jest w stanie zrozumieć Cię, przyspieszając ją, tak, że zyskuje ona atut na testach inicjatywy i rzutach na Obronę Szybkości przez 10 minut. Dodatkowo, poza zwykłymi opcjami korzystania z Wysiłku, możesz z niego skorzystać, by objąć celem tej zdolności więcej postaci - każdy poziom Wysiłku obejmuje dodatkowy cel. Musisz przemówić do dodatkowych celów, by je przyspieszyć, jeden cel na rundę. Jednak akcja na jeden cel by rozpocząć. 

\textbf{Akrobatyczny Atak}\index{Zdolności!Alfabetycznie!Akrobatyczny Atak}\label{sec:Akrobatyczny Atak} (1+ punktów Szybkości) - wyskakujesz w ataku, przesuwając się przez powietrze. Jeśli wyrzucasz naturalne 17 lub 18, możesz wybrać mniejszy efekt zamiast dodatkowych obrażeń. Jeśli zastosujesz Wysiłek do tego ataku, uzyskujesz darmowy poziom Wysiłku na zadaniu. Nie możesz skorzystać z tej zdolności, jeśli Twój Wysiłek Szybkości jest zredukowany wskutek noszenia zbroi. Umożliwienie. 

\textbf{Procesor Akcji}\index{Zdolności!Alfabetycznie!Procesor Akcji}\label{sec:Procesor Akcji} (4 punkty Intelektu) - korzystając z przechowywanych informacji i zdolności analizowania nadchodzących danych z wielką szybkością, jesteś wyszkolony w jednym fizycznym zadaniu Twojego wyboru na 10 minut. Dla przykładu, możesz wybrać bieg, wspinaczkę, pływanie, Obronę Szybkości lub atak specyficzną bronią. Akcja by rozpocząć.

\textbf{Adaptacja}\index{Zdolności!Alfabetycznie!Adaptacja}\label{sec:Adaptacja} - dzięki ukrytej mutacji, urządzeniu wbudowanemu w Twój kręgosłup, rytuałowi krwii smoka, lub jakiemuś innemu darowi, jesteś teraz w komfortowej temperaturze; nie musisz sie nigdy martwić o niebezpieczne promieniowanie, choroby lub gazy; i możesz zawsze oddychać w dowolnym środowisku (nawet w próżni kosmosu). Umożliwienie.

\textbf{Zaawansowany Użytkownik Cypherów}\index{Zdolności!Alfabetycznie!Zaawansowany Użytkownik Cypherów}\label{sec:Zaawansowany Użytkownik Cypherów} - możesz mieć przy sobie 4 Cyphery w danym czasie. Umożliwienie. 

\textbf{Zaawansowany Rozkaz}\index{Zdolności!Alfabetycznie!Zaawansowany Rozkaz}\label{sec:Zaawansowany Rozkaz} (7 punktów Intelektu) - cel w średnim zasięgu słucha każdej komendy, którą mu wydasz, tak długo, jak słyszy Cię i rozumie. Co więcej, tak długo, jak nie robisz nic innego niż wydawanie komend (nie wolno Ci wziąć żadnej innej akcji) możesz dać temu samemu celowi nową komendę. Ten efekt kończy się, gdy kończysz wydawać komendy lub gdy cel opuszcza średni zasięg względem Ciebie. Akcja by rozpocząć. 

\textbf{Atak z Rozbrojeniem}\index{Zdolności!Alfabetycznie!Atak z Rozbrojeniem}\label{sec:Atak z Rozbrojeniem} (3 punkty Szybkości) - za pomocą serii szybkich ruchów, wykonujesz atak przeciwko uzbrojonemu przeciwnikowi, zadając mu obrażenia i rozbrajając go, tak, że jego broń jest teraz w Twoich rękach lub 3 metry od niego na ziemi - Ty wybierasz. Ten atak rozbrajający jest utrudniony. Akcja.

\textbf{Zalety Bycia Dużym}\index{Zdolności!Alfabetycznie!Zalety Bycia Dużym}\label{sec:Zalety Bycia Dużym} - kiedy korzystasz ze Wzrostu, jesteś tak duży, że możesz łatwiej przenosić duże obiekty, wspinać sie na budynki korzystając z uchwytów niedostępnych dla zwykłych ludzi i skakać znacznie dalej. Kiedy korzystasz ze Wzrostu, wszystkie zadania wspinaczki, podnoszenia ciężarów i skakania są dla Ciebie ułatwione. Umożliwienie.

\textbf{Zalety Bycia Małym}\index{Zdolności!Alfabetycznie!Zalety Bycia Małym}\label{sec:Zalety Bycia Małym} - nauczyłeś się, jak wykorzystać swój rozmiar, siłę i dokładność. Twoje obrażenia już się nie dzielą na pół gdy korzystasz ze Zmniejszenia się, a zadania wspinaczki i skakania są ułatwione. Umożliwienie.

\textbf{Porada od Przyjaciela}\index{Zdolności!Alfabetycznie!Porada od Przyjaciela}\label{sec:Porada od Przyjaciela} (1 punkt Intelektu) - znasz słabe i mocne strony swojego przyjaciela, i wiesz jak go zmotywować, by osiągnął sukces. Kiedy dajesz przyjacielowi sugestię powiązaną z jego następną akcję, postać ta jest wyszkolona w tej akcji na jedną rundę. Akcja. 

\textbf{Znowu i Znowu}\index{Zdolności!Alfabetycznie!Znowu i Znowu}\label{sec:Znowu i Znowu} (8 punktów Szybkości) - możesz wziąć kolejną akcję w rundzie, w której już podjąłeś akcję. Umożliwienie.

\textbf{Nieśmiertelny}\index{Zdolności!Alfabetycznie!Nieśmiertelny}\label{sec:Nieśmiertelny} - Twoje ciało i umysł się nie starzeją. Jeśli nie zostaniesz zabity przez akt przemocy (lub jakąś zewnętrzną siłę jak trucizna lub infekcja), nigdy nie umrzesz. Umożliwienie.  

\textbf{Agent-Prowokator}\index{Zdolności!Alfabetycznie!Agent-Prowokator}\label{sec:Agent-Prowokator} - wybierz jedna z poniższych, by być wytrenowanym w: atakowanie bronią swojego wyboru, ładunki wybuchowe, lub skradanie się i otwieranie zamków (jeśli wybierzesz ostatnią opcję, posiadasz trening w dwóch umiejętnościach). Umożliwienie.

\textbf{Agresja}\index{Zdolności!Alfabetycznie!Agresja}\label{sec:Agresja} (2 punkty Mocy) - skupiasz się na atakowaniu w tak wielki sposób, że zostawiasz siebie wysuniętego na ataki wrogów. Kiedy ta zdolność jest aktywna, zyskujesz atut na atakach wręcz i Twoje rzuty na Obronę Szybkości przeciwko atakom wręcz i dystansowym są utrudnione. Ten efekt trwa tak długo, jak sobie życzysz ale kończy się, jeśli walka nie ma miejsca w zasięgu Twoich zmysłów. Umożliwienie.

\textbf{Szybki Umysł}\index{Zdolności!Alfabetycznie!Szybki Umysł}\label{sec:Szybki Umysł} - kiedy próbujesz wykonać zadanie Szybkości, możesz zamiast tego rzucić (i wydać punkty z puli) jakby to była akcja Intelektu. Jeśli stosujesz Wysiłek do tego zadania, możesz wydać punkty z Puli Intelektu zamiast Puli Szybkości (wtedy stosujesz też Skupienie w Intelekcie zamiast w Szybkości). Umożliwienie. 

\textbf{Wysokie Skupienie}\index{Zdolności!Alfabetycznie!Wysokie Skupienie}\label{sec:Wysokie Skupienie} (7 punktów Intelektu) - wkładasz w swoje zadanie wszystko. Dodajesz trzy darmowe poziomy Wysiłku to następnego zadania, które podejmujesz. Nie możesz wykorzystać tej zdolności znowu, dopóki nie zakończysz 10-godzinnego odpoczynku. Akcja.

\textbf{Uzdrowienie}\index{Zdolności!Alfabetycznie!Uzdrowienie}\label{sec:Uzdrowienie} (3 punkty Intelektu) - możesz spróbować uzdrowić jedno schorzenie (np: chorobę lub truciznę) dotyczące jednej istoty. Akcja.

\textbf{Szczur Miejski}\index{Zdolności!Alfabetycznie!Szczur Miejski}\label{sec:Szczur Miejski} (6 punktów Intelektu) - kiedy jesteś w mieście, odnajdujesz lub tworzysz znaczące skróty, sekretne wejścia lub ostateczne trasy ucieczki tam, gdzie wcześniej ich nie było. Aby to zrobićm musisz uzyskać sukces na kacji Intelektu, której trudność określa MG bazując na danej sytuacji. Powinieneś ustalić detale wraz ze swoim MG. Akcja.

\textbf{Zawsze Majsterkując}\index{Zdolności!Alfabetycznie!Zawsze Majsterkując}\label{sec:Zawsze Majsterkując} - jeśli masz narzędzia i materiały i nosisz mniej cypherów niż Twój limit, możesz stworzyć zamanifestowany cypher, jeśli poświęcisz na to godzinę. Nowy cypher jest wybierany przypadkowo i zawsze o 2 poziomy mniej niż normalnie (minimum to 1-szy poziom). Jest on także chwilowy i wrażliwy na uszkodzenia. Nazywa się go chwilowym cypherem. Jeśli dasz go komuś, by z niego korzystał, rozpada się on natychmiast w bezużyteczne śmieci. Akcja by rozpocząć; 1 godzina by ukończyć.

\textbf{Cudowne Kopiowanie}\index{Zdolności!Alfabetycznie!Cudowne Kopiowanie}\label{sec:Cudowne Kopiowanie} - możesz skorzystać ze zdolności Skopiuj Moc, aby skopiować potężniejsze zdolności. W dodatku do normalnych opcji korzystania z Wysiłku przy użyciu Skopiuj Moc, jeśli zaaplikujesz 2 poziomy Wysiłku, MG wybiera moc wysokiego poziomu, która najbardziej przypomina moc, którą pragniesz skopiować (zamiast zdolności niskiego poziomu). Umożliwienie.

\textbf{Dodatkowy Wysiłek}\index{Zdolności!Alfabetycznie!Dodatkowy Wysiłek}\label{sec:Dodatkowy Wysiłek} - kiedy stosujesz przynajmniej jeden poziom Wysiłku do akcji niebojowej, otrzymujesz darmowy, dodatkowy poziom Wysiłku na tym zadaniu. Kiedy wybierasz tę zdolność, musisz zdecydować, czy dotyczy ona Wysiłku Mocy, czy też Wysiłku Szybkości. Umożliwienie.

\textbf{Wielki Skok}\index{Zdolności!Alfabetycznie!Wielki Skok}\label{sec:Wielki Skok} (2 punkty Mocy) - skaczesz w powietrze i lądujesz bezpiecznie w pewnej odległości. Możesz skoczyć wzwyż, w dół lub w poziomie gdziekolwiek w dalekim zasięgu  jeśli masz czystą trasę do tego miejsce, bez żadnych przeszkód. Jeśli masz 3 lub więcej punktów mocy zainwestowanych w siłę, Twój zasięg się ulepsza do bardzo dalekiego. Jeśli masz 5 lub więcej punktów mocy zainwestowanych w siłę, Twój zasięg skoku zostaje ulepszony do 300 metrów. Akcja.

\textbf{Czatownik}\index{Zdolności!Alfabetycznie!Czatownik}\label{sec:Czatownik} - kiedy atakujesz istotę, która jeszcze nie wzięła swojej pierwszej rundy w walce, Twój atak jest ułatwiony. Umożliwienie.

\textbf{Wzmocnienie Dźwięku}\index{Zdolności!Alfabetycznie!Wzmocnienie Dźwięku}\label{sec:Wzmocnienie Dźwięku} (2 punkty Mocy) - na jedną minutę, możesz wzmocnić dalekie lub ciche dźwięki, tak, byś mógł je słyszeć wyraźnie, nawet jeśli jest to rozmowa lub dźwięk małego zwierzęcia poruszającego się w podziemnej norze w bardzo dalekim zasięgu. Możesz spróbować usłyszeć dźwięk, nawet jeśli istnieją bariery blokujące dźwięk lub jest on bardzo cichy, choć to wymaga paru dodatkowych rund koncentracji. Aby odróżnić dźwięk, którego poszukujesz, od głośnego środowiska, także powinieneś poświęcić parę rund na skupienie, gdy przeszukujesz słuchem swoją okolicę. Mając odpowiednio dużo czasu, możesz wyśledzić każdą konwersację, oddychającą istotę i każde urządzenie wydające dźwięk w zasięgu. Akcja by rozpocząć, do paru rund by ją zakończyć, w zależności od trudności zadania.

\textbf{Anegdota}\index{Zdolności!Alfabetycznie!Anegdota}\label{sec:Anegdota} (2 punkty Intelektu) - możesz polepszyć morale grupy istot i pomóc im w nawiązaniu więzi, poprzez zabawianie ich podnoszącą na duchu anegdotą. Przez następną godzinę, ci którzy słuchali Twojej historii są wyszkoleni w jednym zadaniu Twojego wyboru, które jest powiązane z anegdotą, tak długo, jak nie jest to atak lub obrona. Akcja by rozpocząć, jedna minuta by zakończyć.

\textbf{Zwierzęce Szpiegowanie}\index{Zdolności!Alfabetycznie!Zwierzęce Szpiegowanie}\label{sec:Zwierzęce Szpiegowanie} (4+ punkty Intelektu) - jeśli znasz ogólną lokalizację zwierzęcia, które jest przyjazne względem Ciebie i w zasięgu 1.5 km od Ciebie, możesz postrzegać świat jego zmysłami do 10 minut. Jeśli nie jesteś w formie zwierzęcej lub w formie podobnej do tego zwierzęcia, musisz zastosować poziom Wysiłku do korzystania z tej umiejętności. Akcja by rozpocząć. 

\textbf{Zwierzęcy Kształt}\index{Zdolności!Alfabetycznie!Zwierzęcy Kształt}\label{sec:Zwierzęcy Kształt} (3+ punkty Intelektu) - zmieniasz się w zwierzę tam małe jak szczur lub tak duże jak ty (np: duży pies lub mały niedźwiedź) na 10 minut. Za każdym razem, gdy zmieniasz kształt, możesz wybrać inne zwierzę. Twój ekwipunek staje się częścią owej transformacji, co czyni go nieużytecznym, o ile nie ma pasywnego efektu, takiego jak zbroja. W tej formie Twoje Statystyki pozostają takie same jak w Twojej normalnej formie, ale możesz się ruszać i atakować zgodnie z Twoim zwierzęcym kształtem (ataki większości zwierząt tego rozmiaru to bronie średnie, z których możesz korzystać bez żadnej kary). Zadania wymagające rąk - takie jak naciskanie klamek lub przycisków są utrudnione kiedy jesteś w formie zwierzęcej. Nie możesz mówić, ale dalej możesz korzystać ze zdolności, które nie polegają na ludzkiej mowie. Uzyskujesz dwie pomniejsze zdolności powiązane z istotą, w którą sie zmieniłeś (patrz tabela Mniejsze Zdolności Zwierzęcego Kształtu). Dla przykładu, jeśli zamieniasz się w nietoperza, jesteś wyszkolony w percepcji i możesz latać na daleki zasięg w każdej rundzie. Jeśli zamienisz się w ośmiornicę, jesteś wyszkolony w skradaniu się i oddychasz pod wodą. Jeśli zastosujesz poziom Wysiłku do stosowania tej zdolności, możesz albo przybrać kształt mówiącego zwierzęcia, albo hybrydowy. Kształt mówiącego zwierzęcia wygląda dokładnie jak zwykłe zwierzę, ale możesz dalej mówić i korzystać ze zdolności bazujących na ludzkiej mowie. Kształt hybrydowy wygląda jak Twoja normalna forma, ale z cechami zwierzęcia, nawet jeśli to konkretne zwierzę jest znacznie mniejsze od Ciebie (jak nietoperz lub szczur). W formie hybrydowej możesz mówić, korzystać ze swoich wszystkich zdolności, atakować jak zwierzę i wykonywać zadania przy użyciu rąk bez utrudnienia. Każdy kto dobrze się przypatrzy Tobie w formie hybrydowej nigdy nie pomyliłby Cię ze zwierzęciem. Akcja by się przemienić lub odwrócić transformację. 

``Podobieństwo'' to termin ogólnikowy. Lwy są podobne do tygrysów i leopardów, orły są podobne do kruków i łabędzi, psy są podobne do wilków i lisów itp.

Nawet jeśli Twój zwierzęcy kształt ma wiele typów ataku (np: zębami i pazurami), możesz zaatakować tylko raz w rundzie, chyba że masz jakąś zdolność, która pozwala CI na dokonywanie dodatkowych ataków w swojej turze.

Wariant Zwierzęcego Kształtu: Jeśli Twój koncept postaci sprawia, że zawsze zmienia się ona w ten sam zwierzęcy kształt zamiast wybierać z wielu, podwój czas trwania Zwierzęcego Kształtu (20 minut na jedno wykorzystanie). MG może pozwolić postaci z tym ograniczeniem na uczenie się dodatkowych zwierzęcych form poprzez wydanie 4 PD jako długotrwałą korzyść. 

\begin{table*}[t]

\centering
\caption{Tabela Mniejszych Zdolności Zwierzęcej Formy}
\label{Tabela Mniejszych Zdolności Zwierzęcej Formy}

\begin{tabularx}{\textwidth}{| X | X | X |}
\hline
 
 \textbf{Zwierzę} & \textbf{Umiejętność} & \textbf{Inne zdolności} \\ \hline

 Małpa & Wspinaczka & Ręce \\ \hline
 Borsuk & Wspinaczka & Czuły węch \\ \hline
 Nietoperz & Percepcja & Latanie \\ \hline
 Niedźwiedź & Wspinaczka & Czuły węch \\ \hline
 Ptak & Percepcja & Latanie \\ \hline
 Dzik & Obrona Mocy & Czuły węch \\ \hline
 Kot & Wspinaczka lub skradanie się & Mały \\  \hline
 Wąż dusiciel & Wspinaczka & Duszenie \\  \hline
 Krokodyl & Skradanie się lub pływanie & Duszenie \\  \hline
 Deinonych & Percepcja & Szybki \\  \hline
 Delfin & Percepcja lub pływanie & Szybki \\  \hline
 Ryba & Skradanie się lub pływanie & Wodny \\ \hline
 Żaba & Skakanie lub skradanie się & Wodny \\ \hline
 Koń & Percepcja & Szybki \\ \hline
 Leopard & Wspinaczka lub skradanie się & Szybki \\ \hline
 Jaszczurka & Wspinaczka lub skradanie się & Mały \\ \hline
 Ośmiornica & Skradanie się & Wodny \\ \hline
 Rekin & Pływanie & Wodny \\ \hline
 Żółw & Obrona Mocy & Pancerz \\ \hline
 Jadowity wąż & Wspinaczka & Trucizna \\ \hline
 Wilk & Percepcja & Czuły węch \\ \hline
 
 \end{tabularx}
 \end{table*}
 
 \begin{itemize}

\item \textbf{Wodny}: Zwierzę albo oddycha pod wodą zamiast powietrzem, albo jest w stanie oddychać wodą w dodatku do powietrza.

\item \textbf{Pancerz}: Zwierzę ma twardą skorupę lub skórę, co daje mu +1 do Pancerza.

\item \textbf{Duszenie}: Zwierzę może się szybko obwinąć wokół przeciwnika po udanym ataku wręcz (zazwyczaj ugryzieniu lub ataku pazurem), ułatwiając następne ataki przeciwko temu samemu wrogowi aż do momentu, aż kontakt nie zostanie zerwany.

\item \textbf{Szybki}: To zwierzę może się poruszać na daleki zasięg w swojej turze zamiast na średni.

\item \textbf{Latanie}: Zwierzę może latać, co (w zależności od typu zwierzęcia) może oznaczać ruch na średni lub daleki zasięg w swojej turze. 

\item \textbf{Ręce}: Zwierzę ma łapy lub ręce, które są niemal tak zwinne jak te ludzi. W przeciwieństwie do większości zwierzęcych kształtów, zadania zwierzęcia które wymagają rąk nie są utrudnione (choć MG może zdecydować, że niektóre zadania, wymagające ludzkie zwinności, np: gra na flecie, są dalej utrudnione).

\item \textbf{Czuły Węch}: Zwierzę posiada silny zmysł węchu, uzyskując atut na śledzeniu i akcjach w ciemności lub podczas oślepienia. 

\item \textbf{Mały}: Zwierzę jest znacznie mniejsze od człowieka, co ułatwia jego Obronę Szybkości ale utrudnia zadania polegające na przenoszeniu ciężkich rzeczy.

\item \textbf{Trucizna}: Zwierzę jest trujące (zazwyczaj jego ugryzienie), co zadaje dodatkowy 1 punkt obrażeń. 

\end{itemize}

\textbf{Zwierzęce Zmysły}\index{Zdolności!Alfabetycznie!Zwierzęce Zmysły}\label{sec:Zwierzęce Zmysły} - Jesteś wyszkolony w słuchaniu i dostrzeganiu rzeczy. Dodatkowo, przez większość czasu, MG powinien Cię poinformować o tym, że zaraz wkroczysz w pułapkę lub zostaniesz zaatakowany z zaskoczenia, jeśli zagrożenie jest na poziomie niższym niż 5. Umożliwienie. 

\textbf{Riposta}\index{Zdolności!Alfabetycznie!Riposta}\label{sec:Riposta} (3 punkty Szybkości) - Jeśli jesteś zaangażowany w walkę wręcz, możesz wykonać bezpośredni atak wręcz przeciwko każdemu z atakujących raz na rundę. Ten atak jest utrudniony, i ciągle możesz wykonać swoją normalną akcję podczas tej rundy. Umożliwienie.

\textbf{Uprzedzenie Ataku}\index{Zdolności!Alfabetycznie!Uprzedzenie Ataku}\label{sec:Uprzedzenie Ataku} (4 punkty Intelektu) - Możesz wyczuć jak i kiedy istoty Cię atakujące wykonają swoje ataki. Rzuty na Obronę Prędkości są ułatwione na jedną minutę. Akcja.

\textbf{Przebłysk}\index{Zdolności!Alfabetycznie!Przebłysk}\label{sec:Przebłysk} (1 punkt Intelektu) - Patrzysz w przyszłość by zobaczyć, jak Twoje akcje się zakończą. Pierwsze zadanie, które wykonasz przed końcem swojej następnej rundy uzyskuje atut. Akcja.

\textbf{Automatyczny Blask}\index{Zdolności!Alfabetycznie!Automatyczny Blask}\label{sec:Automatyczny Blask} - Przedmioty z twardego światła, które tworzysz, rzucają światło, oświecając wszystko w bliskim zasięgu. Kiedy tylko zechcesz, Twoje ciało (w całości lub tylko jego część) rzuca światło, oświecając wszystko w średnim zasięgu. Umożliwienie. 

\textbf{Stosowanie Swojej Wiedzy}\index{Zdolności!Alfabetycznie!Stosowanie Swojej Wiedzy}\label{sec:Stosowanie Swojej Wiedzy} - Kiedy pomagasz innej postaci w akcji, w której nie posiadasz wyszkolenia, jesteś traktowany jako wyszkolony w niej. Akcja.

\textbf{Aportacja}\index{Zdolności!Alfabetycznie!Aportacja}\label{sec:Aportacja} (4 punkty Intelektu) - Przywołujesz do siebie fizyczny obiekt. Możesz wybrać dowolną pozycję ze standardowej listy ekwipunku, lub (nie więcej niż raz dziennie) możesz pozwolić MG na przypadkowe określenie tego przedmiotu. Jeśli przywołujesz przypadkowy obiekt, ma on szansę 10 procent na bycie zamanifestowanym Cypherem lub artefaktem, 50 procent szans na bycie zwykłym ekwipunkiem i 40 procent szans na bycie jakimś bezużytecznym śmieciem. Nie możesz zastosować tej umiejętności, by wziąć przedmiot trzymany przez inną istotę. Akcja.

\textbf{Wodny Wojownik}\index{Zdolności!Alfabetycznie!Wodny Wojownik}\label{sec:Wodny Wojownik} - Ignorujesz wszelkie kary do akcji (wliczając walkę) w środowiskach podwodnych. Umożliwienie. 

\textbf{Potrójny Wystrzał}\index{Zdolności!Alfabetycznie!Potrójny Wystrzał}\label{sec:Potrójny Wystrzał} (3 punkty Szybkości) - Jeśli broń ma zdolność wystrzału ciągłego bez przeładowywania (zazwyczaj zwana jest bronią automatyczną), możesz wystrzelić ze swojej broni w kierunku do 3 celów (muszą stać obok siebie) na raz. Wykonaj osobny rzut na atak na każdy z celów. Każdy z tych ataków jest utrudniony. Akcja.

\textbf{Magiczny Błysk}\index{Zdolności!Alfabetycznie!Magiczny Błysk}\label{sec:Magiczny Błysk} (1 punkt Intelektu) - Ulepszasz obrażenia innego zaklęcia ofensywnego dodatkową energią, tak, że zadaje 1 dodatkowy punkt obrażeń. Alternatywnie, Twój atak sięga celu w dalekim zasięgu - jest to ognisty pocisk czystej magii, zadający 4 punkty obrażeń. Umożliwienie dla ulepszenia; akcja dla ataku na daleki zasięg.  

\textbf{Artefakty z Odzysku}\index{Zdolności!Alfabetycznie!Artefakty z Odzysku}\label{sec:Artefakty z Odzysku} (6 punktów Intelektu +2 PD) - Rozwinąłeś szósty zmysł odnośnie szukania najcenniejszych rzeczy na pustkowiach. Jeśli spędzisz czas wymagany by odnieść sukces na 2 zadaniach przeszukiwania, możesz wymienić ich rezultat na szansę pozyskania artefaktu wyboru MG jeśli zakończysz test 6 poziomu Intelektu powodzeniem. Możesz skorzystać z tej zdolności najczęściej raz dziennie i nigdy dwa razy w tym samym obszarze. Akcja by rozpocząć, parę godzin, by ją zakończyć.

\textbf{Mechanik Artefaktów}\index{Zdolności!Alfabetycznie!Mechanik Artefaktów}\label{sec:Mechanik Artefaktów} - Jeśli spędzisz przynajmniej 1 dzień majsterkując z artefaktem, który posiadasz, funkcjonuje on na poziomie o 1 wyższym niż normalnie. Stosuje się to do wszystkich artefaktów w Twoim władaniu, ale tylko Ty możesz korzystać z tego bonusa. Umożliwienie.

\textbf{Jak Przepowiedziano}\index{Zdolności!Alfabetycznie!Jak Przepowiedziano}\label{sec:Jak Przepowiedziano} - osiągasz coś, co udowadnia, że jesteś Wybrańcem. Następne zadanie, które podejmiesz, jest ułatwione o 3 stopnie. Nie możesz ponownie skorzystać z tej zdolności, aż do momentu, gdy zaznasz rzutu na odzyskanie zdrowia, który trwa 1 lub 10 godzin. Akcja.

\textbf{Jak Jedna Istota}\index{Zdolności!Alfabetycznie!Jak Jedna Istota}\label{sec:Jak Jedna Istota} - Kiedy Ty i Twoja bestia (ze zdolności Zwierzęcy Kompan) jesteście w bliskim zasięgu od siebie, możecie dzielić się obrażeniami, które uzyskujecie. Dla przykładu, jeśli jedno z Was zostanie trafione bronią za 4 punkty obrażeń, podzielcie je pomiędzy siebie jak uznacie za stosowne. Tylko Pancerz i odporności pierwotnego celu ataku wchodzą w grę. Więc jeśli masz Pancerz 2 i zostajesz zaatakowany za 4 punkty obrażeń magicznym pociskiem, Twoja bestia może wziąć 2 punkty obrażeń, które normalnie Ty byś musiał znieść, ale jej Pancerz się nie liczy, podobnie jak jej odporność na magię, jeśli jakaś. Umożliwienie. 

\textbf{Umiejętności Zabójcy}\index{Zdolności!Alfabetycznie!Umiejętności Zabójcy}\label{sec:Umiejętności Zabójcy} - Jesteś wyszkolony w skradaniu się i przebieraniu się. Umożliwienie.

\textbf{Cios Skrytobójcy}\index{Zdolności!Alfabetycznie!Cios Skrytobójcy}\label{sec:Cios Skrytobójcy} (5 punktów Intelektu) - Jeśli uda Ci sie zaatakować istotę, która nie była świadoma Twojej obecności, zadajesz dodatkowe 9 punktów obrażeń. Umożliwienie. 

\textbf{Potwierdzenie Własnego Przywileju}\index{Zdolności!Alfabetycznie!Potwierdzenie Własnego Przywileju}\label{sec:Potwierdzenie Własnego Przywileju} (3 punkty Intelektu) - Zachowując się tak, jak tylko osoba uprzywilejowana może, werbalnie ustawiasz do pionu wroga, który Cię słyszy i rozumie, wskutek czego nie może on podjąć żadnej akcji, wliczając w to ataki, przez jedną rundę. Niezależnie od tego, czy odniesiesz porażkę czy sukces, następna akcja celu jest utrudniona. Akcja.

\textbf{Przejęcie Kontroli}\index{Zdolności!Alfabetycznie!Przejęcie Kontroli}\label{sec:Przejęcie Kontroli} (6+ punktów Intelektu) - Kontrolujesz akcje innej istoty, z którą wszedłeś w interakcję lub którą badałeś przez co najmniej jedną rundę. Efekt trwa 10 minut. Cel musi być na poziomie 2 lub niższym. Kiedy już przejąłeś kontrolę, cel działa tak, jak sobie tego życzysz najlepiej jak potrafi, wolno korzystając ze swojej zdolności do czynienia osądów, chyba, że poświęcisz akcję na danie mu bardzo szczegółowych instrukcji. W ddoatku do normalnych opcji Wysiłku, możesz skorzystać z Wysiłku, by zwiększyć maksymalny poziom celu. Tak więc, aby spróbować wydać rokaz celowi 5 poziomu (3 poziomy ponad normalnym limitem), musisz zastosować 3 poziomy Wysiłku. Kiedy cel się kończy, cel pamięta wszystko, co się stało i reaguje na to zgodnie ze swoją naturą i Twojej relacji z nim - przejęcie kontroli mogło zniszczyć wcześniejszą, pozytywną relację między wami. Akcja, by rozpocząć.

\textbf{Atleta}\index{Zdolności!Alfabetycznie!Atleta}\label{sec:Atleta} - Jesteś wyszkolony w noszeniu ciężarów, wspinaczce, skakaniu i niszczeniu. Umożliwienie.

\textbf{Atak i Ponowny Atak}\index{Zdolności!Alfabetycznie!Atak i Ponowny Atak}\label{sec:Atak i Ponowny Atak} - Zamiast uzyskiwać więcej obrażeń lub mniejszy/większy efekt na naturalnej 17 lub więcej, możesz dzięki tej zdolności natychmiast wykonać następny atak. Umożliwienie.

\textbf{Estetyczny Atak}\index{Zdolności!Alfabetycznie!Estetyczny Atak}\label{sec:Estetyczny Atak} - Podczas ataku, wykonujesz stylowe ruchy, zachwycające gesty, lub coś innego co zachwyca lub rozbawia innych. Jedna istota, którą wybierasz w średnim zasięgu, która może Cię dostrzec, uzyskuje atut na swoim następnym zadaniu, jeśli jest ona wykonany w ciągu rundy lub dwóch. Umożliwienie.

\textbf{Modyfikacja Cyphera}\index{Zdolności!Alfabetycznie!Modyfikacja Cyphera}\label{sec:Modyfikacja Cyphera} (2+ punkty Intelektu) - Kiedy aktywujesz cypher, dodaj +1 do jego poziomu. W dodatku do normalnych opcji korzystania z Wysiłku, możesz wybrać korzystanie z Wysiłku, by zwiększyć poziom cyphera o dodatkowe +1 (na zastosowany poziom Wysiłku). Niem ożesz zwiększyć poziomu cyphera powyżej 10. Umożliwienie.

\textbf{Autodoktor }\index{Zdolności!Alfabetycznie!Autodoktor}\label{sec:Autodoktor} - jesteś wyszkolony w leczeniu, operacjach chirurgicznych i znoszeniu bólu. MOżesz operować samego siebie, pozostając w tym czasie w pełni świadomym. Umozliwienie. 

\textbf{Świadomość}\index{Zdolności!Alfabetycznie!Świadomość}\label{sec:Świadomość} (3 punkty Intelektu) - Stajesz się nadmiernie śiaodmym swojego otoczenia w celu lepszego odnalezienia swojego celu. Na 10 minut, jesteś świadom wszystkich żywych istot w dalekim zasięgu (wliczając ich ogólnie umiejscowienie) i poprzez koncentrację (kolejna akcja) możesz spróbować poznać informacje o stanie zdrowia i poziomie mocy każdej z nich. Akcja.

\section{B}

\textbf{Babel}\index{Zdolności!Alfabetycznie!Babel}\label{sec:Babel} - Po usłyszeniu mówionego języka przez parę minut, możesz się nim wysławiać i zostać w nim zrozumianym. Jeśli będziesz kontynuować korzystanie z tego języka, by wchodzić w interakcje z native speakerami, Twoje umiejętności z nim związane ulepszają się znacząco, aż w końcu możesz zostać pomylony z native speakerem już po paru godzinach mówienia w nowym języku. Umożliwienie.

\textbf{Balansowanie}\index{Zdolności!Alfabetycznie!Balansowanie}\label{sec:Balansowanie} - Jesteś wyszkolony w balansowaniu. Umożliwienie. 

\textbf{Drużyna Desperados}\index{Zdolności!Alfabetycznie!Drużyna Desperados}\label{sec:Drużyna Desperados} - Twoja reputacja ściąga do Ciebie grupę 6 2-poziomowych Kompanów desperado (BN-ów), którzy są Tobie kompletnie oddani. Powinieneś razem z MG określić szczegóły tych kompanów. Jeśli jeden z kompanów umrze, zyskujesz nowego w jego miejsce po przynajmniej 2 tygodniach i odpowiednim procesie rekrutacyjnym. Umożliwienie. 

\textbf{Drużyna Kompanów}\index{Zdolności!Alfabetycznie!Drużyna Kompanów}\label{sec:Drużyna Kompanów}  - Otrzymujesz 4 3-poziomowych kompanów. Nie są oni ograniczeni odnośnie ich modyfikacji. Umożliwienie.

\textbf{Ogłuszenie}\index{Zdolności!Alfabetycznie!Ogłuszenie}\label{sec:Ogłuszenie} (1 punkt Mocy) - Wykonujesz powtarzalny atak wręcz. Twój atak zadaje o 1 punkt obrażeń mniej niż zwykle, ale oszałamia cel na jedną rundę, podczas której wszystkie jego zadania są utrudnione. Akcja.

\textbf{Podstawowy Kompan}\index{Zdolności!Alfabetycznie!Podstawowy Kompan}\label{sec:Podstawowy Kompan} - Uzyskujesz kompana 2-poziomu. Jedną z jego modyfikacji musi być perswazja. Możesz wziąć tę zdolność wiele razy, za każdym razem otrzymując innego kompana 2-poziomu. Umożliwienie. (MG może określić, że musisz poszukać odpowiedniego kompana w świecie gry, zanim go zdobędziesz).

\textbf{Zarządzanie Bitwą}\index{Zdolności!Alfabetycznie!Zarządzanie Bitwą}\label{sec:Zarządzanie Bitwą} (4 punkty Intelektu) - Tak długo, jak używasz swojej akcji w każdej rundzie na wydawanie rozkazów lub dawanie porad, rzuty na atak i obronę podejmowane przez Twoich sprzymierzeńców w średnim zasięgu są ułatwione. Akcja.

\textbf{Taktyk Pola Walki}\index{Zdolności!Alfabetycznie!Taktyk Pola Walki}\label{sec:Taktyk Pola Walki} (2+ punkty Intelektu) - Oceniasz swoje otoczenia, poznając dowolne fakty, które MG uzna za ważne odnośnie atakowania, bronienia się, manewrowania i radzenia sobie z zagrożeniami środowiskowymi w średnim zasięgu. Dla przykładu, możesz zauważyć, że po wspięciu się na stos śmieci uzyskasz przewagę w walce wręcz, że w kącie najłatwiej się bronić, że gdzieś jest mniej śliska pokrywa lodowa na zamarzniętym jeziorze, lub że istnieje miejsce, gdzie trujący gaz jest rzadszy niż indziej. Jeśli Ty (lub ktoś, kogo o tym poinformujesz) przemieści sie w tamto miejsce, Ty (lub on/a) uzyska atut na zadaniach związanych z optymalną pozycją (takie jak ataki z wyższego miejsca, obrona Szybkości w łatwym do obrony rogu, rzuty na utrzymanie równowagi na lodzie lub Obrona Mocy przeciwko trującej chmurze). Zamiast uzyskać korzystną lokację, możesz się dowiedzieć o niekorzystnej lokacji, którą możesz wykorzystać przeciwko swoim wrogom, np: zapędzając ich w kozi róg który utrudnia ich ataki wręcz lub słaby punkt na zamarzniętym jeziorze, gdzie lód się pod nimi załamie. Możesz wykorzystać Wysiłek by dowiedzieć się o jednej dodatkowej dobrej lub złej lokacji w zasięgu (jedna lokacja na poziom Wysiłku), zwiększyć zasięg zdolności (dodatkowy średni zasięg na poziom Wysiłku) lub skorzystać z obydwu tych opcji. Umożliwienie. 

\textbf{Zew Dziczy}\index{Zdolności!Alfabetycznie!Zew Dziczy}\label{sec:Zew Dziczy} (5 punktów Intelektu) - Przywołujesz hordę małych zwierząt lub jedno zwierzę 4 poziom, by Tobie chwilowo pomagało. Te istoty robią ,jak rozkażesz tak długo, jak skupiasz na nich swoją uwagę, ale musisz wykorzystać swoje akcje, by nimi kierować. Istoty te naturalnie występują w tym terenie i przybywają o własnych siłach, więc jeśli jesteś w niedostępnym miejscu, to nie przybędą. Akcja.

\textbf{Zwierzęcy Kompan}\index{Zdolności!Alfabetycznie!Zwierzęcy Kompan}\label{sec:Zwierzęcy Kompan} - Istota 2 poziomu rozmiaru Twojego lub mniejszego towarzyszy Ci i słucha Twoich instrukcji. Powinieneś wypracować z MG szczegóły tej istoty, i najpewniej będziesz za nią wykonywał rzuty w czasie walki lub gdy wykonuje ona jakieś akcje. Zwierzęcy kompan działa w Twojej turze. Jako istota 2 poziomu, posiada ona stopień trudności 6 i 6 punktów życia, a zadaje 2 obrażenia. Jej zdolności ruchu bazują na jej typie istoty (ptak, istota pływająca itp.). Jeśli Twój zwierzęcy Kompan umrze, możesz przeszukać dzicz przez 1k6 dni, by odnaleźć nowego. Umożliwienie. (Poziom istoty określa jej stopień trudności, punkty życia i obrażenia, chyba, że zaznaczono inaczej. Tak więc zwierzęcy kompan 2 poziomu ma stopień trudności 6, 6 punktów życia i zadaje 2 obrażenia. Zwierzęcy kompan 4 poziomu ma stopień trudności 12, 12 punktów życia i zadaje 4 punkty obrażeń. I tak dalej.).

\textbf{Oczy Bestii}\index{Zdolności!Alfabetycznie!Oczy Bestii}\label{sec:Oczy Bestii} (3 punkty Intelektu) - Poprzez podlączenie się do istoty z Twojej zdolności Zwierzęcy Kompan, możesz postrzegać świat jego zmysłami, jeśli znajduje się w zasięgu 1.5 km od Ciebie. Ten efekt trwa 10 minut. Akcja, by ustanowić.

\textbf{Likantropia}\index{Zdolności!Alfabetycznie!Likantropia}\label{sec:Likantropia} - w ciągu 5 sąsiadujących z sobą nocy w każdym miesiącu, zamieniasz się w potworna bestię (do 1 godziny każdej nocy). W tej nowej formie, uzyskujesz +8 do Puli Mocy, +1 do Skupienia w Mocy, +2 do Puli Szybkości i +1 do Skupienia w Szybkości. Kiedy jesteś w zwierzęcej formie, nie możesz wydawać punktów Intelektu na cokolwiek innego niż próba powrotu do normalnej formy zanim minie godzina (zadanie o trudności 2). Dodatkowo, atakujesz wszystkie żywe istoty w średnim zasięgu od Ciebie. Po tym, jak wracasz do normalnej formy, otrzymujesz karę -1 do wszystkich rzutów na godzinę. Jeśli nie zabiłeś i zjadłęś przynajmniej jednej istoty w zwierzęcej formie, ta kara wzrasta do -2 i dotyczy wszystkich Twoich rzutów przez 24 godziny. Akcja, by zmienić się z powrotem.

\textbf{Niezauważalny}\index{Zdolności!Alfabetycznie!Niezauważalny}\label{sec:Niezauważalny}  - Twój obniżony wzrost sprawia, że trudno Cię znaleźć. Kiedy Zmniejszenie się jest aktywne, wszystkie Twoje próby skradania się są ułatwione. Umożliwienie.

\textbf{Wiedza z Bestiariusza}\index{Zdolności!Alfabetycznie!Wiedza z Bestiariusza}\label{sec:Wiedza z Bestiariusza} - jesteś wyszkolony w wiedzy o niehumanoidalnych istotach, które jedzą ludzkie mięso - rozpoznawaniu ich, poznawaniu ic hsłabości i poznawaniu ich zwyczajów i zachowań. Umożliwienie. 

\textbf{Zdrada}\index{Zdolności!Alfabetycznie!Zdrada}\label{sec:Zdrada} - Za każdym razem, gdy przekonasz przeciwnika, że nie jesteś zagrożeniem i następnie zaatakujesz z nienacka (bez prowokacji), ten atak zadaje 4 dodatkowy punkty obrażeń. Umożliwienie. 

\textbf{Lepsze Życie Dzięki Chemii}\index{Zdolności!Alfabetycznie!Lepsze Życie Dzięki Chemii}\label{sec:Lepsze Życie Dzięki Chemii} (4 punkty Intelektu) - Stworzyłeś koktajle chemiczne dostosowane do Twojej własnej biochemii. W zależności od tego, który z nich zażyjesz, czyni Cię to mądrzejszym, szybszym lub wytrzymalszym, ale kiedy ich działanie się kończy, masz przekichane, więc korzystasz z nich tylko wtedy, kiedy sytuacja tego wymaga. Uzyskujesz 2 punkty w Skupieniu w Mocy, Szybkości lub Intelekcie na jedna minutę, po której nie możesz ponownie skorzystać z tej zdolności przez godzinę. Podczas tej godziny, za każdym razem, gdy wydasz punkt z Puli, zwiększ kosz tej akcji o 1. Akcja.

\textbf{Lepszy Atak z Zaskoczenia}\index{Zdolności!Alfabetycznie!Lepszy Atak z Zaskoczenia}\label{sec:Lepszy Atak z Zaskoczenia} - Jeśli atakujesz z ukrycia lub przed akcją przeciwnika, otrzymujesz atut na swoim ataku (jeśli posiadasz także zdolność Atak z Zaskoczenia, dodajesz obydwa atuty). Po udanym ataku, zadajesz dodatkowe 2 punkty obrażeń (w sumie 4, jesli posiadasz Atak z Zaskoczenia. Umożliwienie.

\textbf{Większy}\index{Zdolności!Alfabetycznie!Większy}\label{sec:Większy} - Kiedy korzystasz ze Wzrostu, możesz osiągnąć rozmiar 4 metrów i dodajesz dodatkowe 3 chwilowe punkty do swojej Puli Mocy. Umożliwienie.

\textbf{Większy Zwierzęcy Kształt}\index{Zdolności!Alfabetycznie!Większy Zwierzęcy Kształt}\label{sec:Większy Zwierzęcy Kształt} - kiedy korzystasz ze Zwierzęcego Kształtu, Twoja zwierzęca forma rozrasta się do podwójnego normalnego rozmiaru. Będąć tak wielką, zwierzęca forma dodaje Ci następujące bonusy: +1 do Pancerza, +5 do Puli Mocy, jesteś także wyszkolony w używaniu naturalnych ataków swojej zwierzęcej formy jako ciężkich broni (jeśli nie byłeś wyszkolony). Jednakże, Twoja Obrona Szybkości jest utrudniona. Kiedy jestes większy, dostajesz także atut na zadaniach które są prostsze dla dużych istot, takich jak wspinaczka, zastraszanie, przepływanie rzek itp. Umożliwienie. 

\textbf{Większy Likantrop}\index{Zdolności!Alfabetycznie!Większy Likantrop}\label{sec:Większy Likantrop} - Kiedy przebywasz w formie likantropa, Twoja forma fizyczna jest większa niż wcześniej, sięgając 4 metrów. Będąc tak wielkim, otrzymujesz następujące bonusy: +1 do Pancerza, +5 do Puli Mocy, i jesteś wyszkolony w korzystaniu ze swoich pięści jak z ciężkich broni (jeśli jeszcze nie jesteś). Jednakże, Twoja Obrona Szybkości jest utrudniona. Kiedy jesteś tak wielki, uzyskujesz atut na zadaniach które są prostsze dla dużej istoty, takich jak wspinaczka, zastraszanie, przepływanie rzek itp. Umożliwienie. 

\textbf{Detonacja Biomorficzna}\index{Zdolności!Alfabetycznie!Detonacja Biomorficzna}\label{sec:Detonacja Biomorficzna} (7+ punktów Mocy) - Emitujesz impuls biomorficznej energii w średnim zasięgu, ale tak manipulujesz jego frekwencją, by przeszkadzał życiu w bliskim zasięgu. Wszyscy w promieniu detonacji otrzymują 5 punktów obrażeń, które ignorują Pancerz (chyba, że Pancerz wynika z pola siłowego). Jeśli zastosujesz dodatkowy Wysiłek by zwiększyć obrażenia, zadajesz 2 dodatkowe punkty obrażeń na poziom Wysiłku (zamiast normalnych 3 punktów); cele w zasięgu otrzymują 1 punkt obrażeń nawet jeśli nie powiedzie Ci się rzut na atak. Akcja. 

\textbf{Biomorficzne Leczenie}\index{Zdolności!Alfabetycznie!Detonacja Biomorficzna}\label{sec:Detonacja Biomorficzna} (4+ punktów Mocy) - Świadomie wysyłasz puls swojego pola biomorficznego (dziwna energia generowana przez ciało) i skupiasz go na żywej istocie w średnim zasięgu. Cel uzyskuje darmowy i natychmiastowy rzut na odzyskanie zdrowia. Nie możesz ponownie użyć tej zdolności na danej istocie aż do momentu, gdy zakończysz swój 10-godzinny odpoczynek. Akcja. 

\textbf{Spryciula}\index{Zdolności!Alfabetycznie!Spryciula}\label{sec:Spryciula} - Jesteś wyszkolony w jednej z następujących umiejętności: oszustwie, skradaniu się lub przebieraniu się. Umożliwienie. 

\textbf{Zlanie się z Tłem}\index{Zdolności!Alfabetycznie!Zlanie się z Tłem}\label{sec:Zlanie się z Tłem} (4 punkty Intelektu) - Kiedy zlewasz się z tłem, istoty dalej Cię widzą, ale nie przywiązują do Twojej obecności wagi przez około minutę. Kiedy się zlewasz z tłem, jesteś wyspecjalizowany w skradaniu się i Obronie Szybkości. Ten efekt kończy się, gdy zrobisz coś, by ujawnić swoją obecność lub pozycję - zaatakujesz, skorzystasz ze zdolności, przesuniesz wielki obiekt itp. Jeśli to się wydarzy, możesz odzyskać brakujący czas efektu poprzez poświęcenie akcji na skupieniu się, by wyglądać niewinnie i na swoim miejscu. Akcja bo ryzpocząć lub odzyskać.

\textbf{Błogosławieństwo Bóstw}\index{Zdolności!Alfabetycznie!Błogosławieństwo Bóstw}\label{sec:Błogosławieństwo Bóstw} - Jako sługa bóstw, masz różne błogosławieństwa, których Ci udzieliły. Błogosławieństwo zależy od danego bóstwa i jego specjalności. Wybierz dwie ze zdolności wymienionych poniżej.

\begin{itemize}
\item \textbf{Autorytet/Prawo/Pokój} (3 punkty Intelektu) - Powstrzymujesz wroga, który Cię słyszy i rozumie, przed zaatakowaniem kogoś lub czegoś przez jedną rundę. Akcja.
\item \textbf{Dobrotliwość/Moralność/Duch} (2+ punktów Intelektu) - Jeden demon, duch lub podobna istota 1-poziomu w średnim zasięgu zostaje zniszczona lub wygnana. W dodatku do normalnych opcji Wysiłku, możesz skorzystać z niego, by zwiększyć maksymalny poziom celu. Tak więc, aby zniszczyć lub wygnać cel 5-poziomu (4 poziomy więcej niż normalnie) musisz zastosować 4 poziomy Wysiłku. Akcja.
\item \textbf{Śmierć/Ciemność} (2 punkty Intelektu) - Cel, który wybierasz w średnim zasięgu, otrzymuje 3 punkty obrażeń. Akcja.
\item \textbf{Pragnienie/Miłość/Zdrowie} (3 punkty Intelektu) - Poprzez dotyk, możesz przywócić 1d6 punktów do jednej Puli Statystyk dowolnej istocie, wliczając siebie. Ta zdolność to zadanie Intelektu 2 poziomu. Za każdym razem, gdy chcesz uzdrowić tę samą istotę, zadanie jest utrudnione poziom więcej. Trudność wraca do 2 po tym, jak istota zakończy 10-godzinny odpoczynek. Akcja.
\item \textbf{Kamień/Ziemia} - Jesteś wyszkolony we wspinaczce, kamieniarstwie i eksploracji jaskiń. Umożliwienie.
\item \textbf{Wiedza/Mądrość} (3 punkty Intelektu) - Wybierz do 3 istot (możesz się wśród nich znaleźć). Przez minutę, konkretny typ zadania (ale nie rzut na atak lub obronę) jest ułatwiony dla tych istot, ale tylko, jeśli zostają w bliskiej odległości od Ciebie. Akcja.
\item \textbf{Natura/Zwierzęta/Rośliny} - Jesteś wyszkolony w botanice i obchodzeniu się z dzikimi zwierzętami. Umożliwienie.
\item \textbf{Ochrona/Cisza} (3 punkty Intelektu) - Tworzysz cichy bąbel ochronny wokół siebie o promieniu bliskiego zasięgu na jedną minutę. Bąbel porusza się wraz z Tobą. Wszystkie rzuty obronne dla Ciebie i istot, które wybierzesz, gdy wykonywane są wewnątrz tego bąbla, są ułatwione, i żaden hałas, niezależnie od jego natury, nie wybrzmiewa głośniej niż normalna rozmowa. Akcja, by rozpocząć. 
\item \textbf{Powietrz/Niebo} (2 punkty intelektu) - Istota, której dotkniesz, jest odporna na toksyny i choroby przenoszone drogą powietrzną na 10 minut. Akcja.
\item \textbf{Słońce/Światło/Ogień} (2 punkty Intelektu) - Sprawiasz, że jedna istota lub obiekt w średnim zasięgu się zapala, co zadaje jej 1 punkt obrażeń środowiskowych w każdej rundzie, do momentu, gdy ogień zostanie wygaszony (wymaga to akcji). Akcja.
\item \textbf{Oszustwo/Chciwość/Handel} - jesteś wyszkolony w wykrywaniu oszustw innych istot. Umożliwienie. 
\item \textbf{Wojna} (1 punkt Intelektu) - Cel, który wybierasz w średnim zasięgu (możesz to być Ty) zadaje 2 obrażenia więcej swoim następnym udanym atakiem bronią. Akcja.
\item \textbf{Woda/Morze} (2 punkty Intelektu) - Cel, który wybierasz może oddychać pod wodą przez 10 minut. Akcja.
\end{itemize}

\textbf{Oślepienie Maszyny}\index{Zdolności!Alfabetycznie!Oślepienie Maszyny}\label{sec:Oślepienie Maszyny} (6 punktów Szybkości) - Deaktywujesz sensory maszyny, czyniąc ją ślepą do czasu, aż ktoś ją naprawi. Musisz albo dotknąć maszyny, albo uderzyć w nią atakiem dystansowym (nie zadaje on obrażeń). Akcja.

\textbf{Oślepiający Atak}\index{Zdolności!Alfabetycznie!Oślepiający Atak}\label{sec:Oślepiający Atak} (3 punkty Szybkości) - Jeśli masz ze sobą źródło śWiatła, możesz go urzyć, by wykonać atakl wręcz przeciwko celowi. Jeśli atak jest udany, nie zadaje on obrażeń, ale cel jest oślepiony na jedną minutę. Akcja.

\textbf{W mgnieniu Oka}\index{Zdolności!Alfabetycznie!W Mgnieniu Oka}\label{sec:W Mgnieniu Oka} (4 punkty Szybkości) - Poruszasz się do 300 metrów w jednej undzie. Akcja.

\textbf{Blok}\index{Zdolności!Alfabetycznie!Blok}\label{sec:Blok} (3 punkty Szybkości) - Automatycznie blokujesz następny atak wręcz wykonany przeciwko Tobie w następnej minucie. Akcja by rozpocząć. 

\textbf{Chronienie Sprzymierzeńca}\index{Zdolności!Alfabetycznie!Chronienie Sprzymierzeńca}\label{sec:Chronienie Sprzymierzeńca} - Jeśli korzystasz z lekkiej lub średniej broni, możesz blokować ataki wykonywane przeciwko sprzymierzeńcowi blisko Ciebie. Wybierz jedną istotę w bliskim zasięgu. Zapewniasz mu atut do Obrony Szybkości. Nie możesz skorzystać z Szybkiego Bloku kiedy Chronisz Sprzymierzeńca. Umożliwienie.

\textbf{Gorączka Krwi}\index{Zdolności!Alfabetycznie!Gorączka Krwi}\label{sec:Gorączka Krwi} - Kiedy nie masz punktów w jednej lub dwóch Pulach, uzyskujesz atut na rzutach na atak lub obronę (Twój wybór). Umożliwienie.

\textbf{Zew Krwi}\index{Zdolności!Alfabetycznie!Zew Krwi}\label{sec:Zew Krwi} (3 punkty Mocy) - Jeśli pokonasz w walce wroga, możesz się przemieścić o średni dystans, ale tylko, jeśli się przemieszczasz w kierunku innego wroga. Nie musisz wydawać punktów aż do momentu, w którym wiesz, że pierwszy wróg poległ. Umożliwienie.

\textbf{Rozmazana Prędkość}\index{Zdolności!Alfabetycznie!Rozmazana Prędkość}\label{sec:Rozmazana Prędkość} (7 punktów Mocy) - Poruszasz się tak szybko, że do Twojej następnej tury, jesteś rozmazany. Kiedy jesteś rozmazany, jeśli wykorzystasz Wysiłek na ataku wręcz lub Obronie Szybkości, uzyskujesz darmowy poziom Wysiłku na tym zadaniu. Możesz się przemieścić na średni dystans jako część innej akcji lub na daleki dystans, jeśli poświęcisz na to całą akcję. Umożliwienie. 

\textbf{Zmiana Ciała}\index{Zdolności!Alfabetycznie!Zmiana Ciała}\label{sec:Zmiana Ciała} (3+ punkty Intelektu) - Zmieniasz swoje ciało i twarz oraz kolory na jedna godzinę, ukrywając swoją tożsamość lub poszywając się pod kogoś. Jeśli zastosujesz poziom Wysiłku, możesz udawać konkretną osobę dostatecznie dobrze, by oszukać kogoś, kto zna ją dobrze lub przebadał ją z bliska (wliczając odciski palców i porównanie głosu, ale nie wzór siatkówki lub DNA). Masz atut na wszelkich zadaniach związanych z przebieraniem się (w dodatku do atutu z Morficznej Twarzy). Musisz zastosować osobny poziom Wysiłku, jeśli chcesz udawać inny gatunek (np: gdy człowiek chce udawać humanoidalnego kosmitę). Akcja.

\textbf{Jeździec Błyskawicy}\index{Zdolności!Alfabetycznie!Jeździec Błyskawicy}\label{sec:Jeździec Błyskawicy} (4 punkty Intelektu) - Możesz przemieścić się na daleki dystans z jednego miejsca na drugie prawie natychmiastowo, przeniesiony przez błyskawicę. Musisz być w stanie dostrzec nową lokację, i nie może być między nimi żadnych przeszkadzających barier. Akcja.

\textbf{Promienie Mocy}\index{Zdolności!Alfabetycznie!Promienie Mocy}\label{sec:Promienie Mocy} (5+ punktów Intelektu) - Wystrzeliwujesz wachlarz błyskawic na średni zasięg w pióropuszu, który jest szeroki na 15 metrów na swoim końcu. Ten atak zadaje 4 punkty obrażeń. Jeśli zastosujesz Wysiłek by zwiększyć obrażenia zamiast ułatwić atak, zadajesz 2 dodatkowe punkty obrażeń na poziom Wysiłku (zamiast zwyczajowych 3); cele w zasięgu otrzymują 1 punkt obrażeń nawet jeśli nie powiedzie Ci się rzut na atak. Akcja.

\textbf{Urealnienie Iluzji}\index{Zdolności!Alfabetycznie!Urealnienie Iluzji}\label{sec:Urealnienie Iluzji} (2+ punkty Intelektu) - Dajesz jednej ze swoich wizualnych iluzji ograniczoną fizyczną realność, którą można powąchać, posmakować, usłyszeć i poczuć. Ten efekt jest powiązany z daną iluzją i zachowuje się stosownie do jej natury. Dla przykładu, może on sprawić, że iluzja cegły będzie odczuwalna jako cegła, iluzja osoby będzie pachnieć jak perfumy i być w stanie otwierać drzwi, a iluzja kominka będzie ciepła w dotyku. 

Fizyczna rzeczywistość zapewniona Twojej iluzji jest na poziomie 1 z 3 punktami zdrowia. Jeśli z iluzji się korzysta, by zaatakować, zadaje ona tylko 1 punkt obrażeń (mogą to być normalne obrażenia jak ciosy pięści i kopnięcia, lub obrażenia środowiskowe jak spadające cegły lub ognie kominka). Możesz zwiększyć poziom stworzonego efektu, poprzez dodanie poziomów Wysiłku - każdy poziom Wysiłku zwiększa realność iluzji o 1 poziom.

Możesz aktywować tę zdolność jako część akcji stworzenia iluzji (korzystając z dowolnej zdolności tworzenia iluzji, jaką dysponujesz, np: Mniejszej Iluzji) lub możesz wykorzystać osobną akcję zastosowaną względem jednej z Twoich iluzji, które już istnieją. Efekt kończy się, gdy iluzja zostaje zniszczona, pozwalasz jej na ustąpienie, punkty życia iluzji są zredukowane do 0, lub gdy upłynie 10 minut. Umożliwienie.

\textbf{Ulepsz Materialny Cypher}\index{Zdolności!Alfabetycznie!Ulepsz Materialny Cypher}\label{sec:Ulepsz Materialny Cypher} (2 punkty Intelektu) - Zamanifestowany Cypher który aktywujesz w swojej następnej akcji, działa jakby miał 2 poziomy więcej. Akcja.

\textbf{Ulepsz Funkcjonowanie Materialnego Cyphera}\index{Zdolności!Alfabetycznie!Ulepsz Funkcjonowanie Materialnego Cyphera}\label{sec:Ulepsz Funkcjonowanie Materialnego Cyphera} (4 punkty Intelektu) - Dodaj 3 do działającego poziomu zamanifestowanego cyphera, który aktywujesz w swojej następnej akcji, lub zmień jeden z aspektów jego parametrów (zasięg, czas trwania, obszar efektu itp.). Parametry możesz maksymalnie podwoić, a minimalnie obniżyć do 1/10. Akcja. 

\textbf{Skacząca Tarcza}\index{Zdolności!Alfabetycznie!Skacząca Tarcza}\label{sec:Skacząca Tarcza} - Kiedy korzystasz z Rzutu Tarczą Siłową, zamiast zanikać po jednym ataku (udanym lub nie), zaatakuje ona do dwóch dodatkowych celów w średnim zasięgu. Wysiłek lub inne modyfikatory zastosowane do pierwszego ataku stosują się również do innnych celów. Niezależnie od tego, czy trafisz wszystkie, trochę, czy zero celów, tarcza zanika, a potem reformuje się w Twoich dłoniach. (Jeśli wziąłeś Skaczącą Tarczę, a wcześniej wziąłeś Rzut Tarczą Siłową, masz opcję zamiany tamtej zdolności na Leczący Puls). Umożliwienie. 

\textbf{Związana Magiczna Istota}\index{Zdolności!Alfabetycznie!Związana Magiczna Istota}\label{sec:Związana Magiczna Istota} - Posiadasz 3-poziomowego magicznego sprzymierzeńca przywiązanego do fizycznego obiektu (może pomniejszy dżin przywiązany do lampy, mniejszy demon przywiązany do monety, albo duch przywiązany do lusterka). Magiczny sprzymierzeniec jeszcze nie posiada pełnej mocy, którą by posiadł, gdyby dojrzał. Normalnie, sprzymierzeniec pozostaje uśpiony w obiekcie, do którego jest przywiązany. Kiedy skorzystasz z akcji, by go zamanifestować, pojawia się obok Ciebie jako istota, z którą możesz porozmawiać. Istota posiada swoją własną osobowość określoną przez MG i jest o poziom wyższa w danej dziedzinie wiedzy (takiej jak np: lokalna historia). MG określa, czy magiczny sprzymierzeniec posiada jakieś długoterminowe cele.

Za każdym razem gdy magiczny sprzymierzeniec staje się fizycznie zamanifestowany, pozostaje takim na okres do jednej godziny. W tym czasie, towarzyszy Tobie i wykonuje Twoje instrukcje. Magiczny sprzymierzeniec musi pozostać w bliskiej odległości od Ciebie - jeśli się oddali bardziej, zostaje wrzucony do swojego obiektu na koniec Twojej następnej tury i nie może wrócić aż do końca Twojego następnego 10-godzinnego odpoczynku. Nie atakuje on istot, ale może skorzystać ze swojej akcji, aby dać Ci atut na każdym jednym ataku, który wykonujesz w swojej własnej turze. Poza tym, może on dokonywać akcji na własną rękę (choć to raczej Ty będziesz rzucać kością).

Jeśli atak zredukuje punkty życia istoty do 0, znika ona. Reformuje się ona w przywiązanym obiekcie w ciągu 1d6 +2 dni. 

Jeśli utracisz przywiązany obiekt, wiesz w którym kierunku musisz iść, by go znaleźć. Akcja by zamanifestować magiczną istotę. 

\textbf{Pranie Mózgu}\index{Zdolności!Alfabetycznie!Pranie Mózgu}\label{sec:Pranie Mózgu} (6+ punktów Intelektu) - Korzystasz ze sztuczek, dobrze przemyślanych kłamstw i chemikaliów wpływających na umysł (lub innych środków, takich jak magia lub hiper-technologia) by nakłonić chwilowo innych do zrobienie tego, czego chcesz. Kontrolujesz akcję innej istoty, której dotykasz. Ten efekt trwa przez minutę. Cel musi być na 3 poziomie lub niższym. Możesz pozwolić mu na swobodne działanie lub przejąć kontrolę tak długo, jak możesz ją dostrzec. W dodtku do zwykłych opcji korzystania z Wysiłku, możesz wybrać zwiększenia maksymalnego poziomu celu lub zwiększenie czasu trwania efektu na jedną minutę. Tak wię,c aby kontrolować umysł celu 6 poziomu (trzy poziomy powyżej normalny limit) lub by kontrolować cel na 4 minuty (trzy minuty powyżej normalnego czasu trwania), musisz zastosować 3 poziomy Wysiłku. Kiedy czas trwania się kończy, ta istota nie pamięta bycia kontrolowaną ani niczego, co zrobiła, kiedy była pod wpływem Twojej osoby. Akcja, by rozpocząć.

\textbf{Przełamanie Linii}\index{Zdolności!Alfabetycznie!Przełamanie Linii}\label{sec:Przełamanie Linii} - Łatwo dostrzegasz dyscyplinę grupy i hierarchie, także pośród Twoich wrogów. Jeśli grupa wrogów zyskuje dowolny rodzaj korzyści z tytułu wspólnej pracy, możesz spróbować im przeszkodzić w odniesieniu tej korzyści, poprzez wskazanie słabego punktu w formacji wrogów lub ataku grupowym. Ten efekt trwa do minuty lub dopóki dotknięci nim wrogowie nie spędzą rundy na przegrupowaniu się, by odzyskać normalną korzyść. Akcja, by zainicjować. 

\textbf{Ruch i Multiatak}\index{Zdolności!Alfabetycznie!Ruch i Multiatak}\label{sec:Ruch i Multiatak} (6 punktów Szybkości) - Poruszasz się do średniego dystansu i atakujesz do 4 różnych wrogów w jednej akcji tak długo, jak są na Twojej ścieżce. Dowolny modyfikator który stosujesz do jednego ataku stosujesz do wszystkich ataków, które wykonujesz. Jeśli masz inną specjalną zdolność, która pozwala Ci na poruszanie się i wykonanie akcji, kiedy korzystasz z Ruchu i Multiataku, uzyskujesz atut na atakowaniu tych wrogów. Akcja. 

\textbf{Przełamanie Umysłów}\index{Zdolności!Alfabetycznie!Przełamanie Umysłów}\label{sec:Przełamanie Umysłów} (7+ punktów Intelektu) - Korzystając ze swoich sprytnych słów i wiedzy o innych, i mając parę rund konwersacji by pozyskać parę konkretnych informacji o kontekście odnośnie Twojego celu, możesz wypowiedzieć zdanie zaprojektowane tak, by zraniło psychikę Twojego rozmówcy. Jeśli cel Cię słyszy i rozumie, otrzymuje on 6 punktów obrażeń Intelektu (ignorujących Pancerz) i zapomina ostatni dzień swojego życia, co może sprawić, że zapomni Ciebie i to, jak się znajduje w danym miejscu. W dodatku do zwykłych opcji korzystania z Wysiłku, możesz skorzystać z niego, by przełamać umysł jednego dodatkowego celu, który Cię słyszy i rozumie. Akcja by rozpocząć, akcja by zakończyć.

\textbf{Niszczyciel}\index{Zdolności!Alfabetycznie!Niszczyciel}\label{sec:Niszczyciel} - Jesteś wyszkolony w zadaniach polegających na uszkadzaniu obiektów. Akcja uszkodzenia obiektu to akcja Mocy, i przy sukcesie, obiekt przesuwa się o jeden krok w dół na liczniku obrażeń obiektu. Jeśli test Mocy przekroczy trudność o dwa kroki, zamiast tego obiekt przesuwa się w dół o dwa kroki w dół na liczniku obrażeń obiektu. Jeśli test Mocy przekroczy trudność o 4 kroki, obiekt przesuwa się na dół o 3 kroki na liczniku obrażeń obiektu i zostaje natychmiastowo zniszczony. Lekkie materiały redukują efektywny poziom obiektu, kiedy twarde materiały jak drewno lub kamień dodaję 1 do efektywnego poziomu lub (lub 2 dla bardzo twardych przedmiotów stworzonych z metalu). Umożliwienie.

\textbf{Brutalne Uderzenie}\index{Zdolności!Alfabetycznie!Brutalne Uderzenie}\label{sec:Brutalne Uderzenie} (4 punkty Mocy) - Zadajesz o 4 punkty obrażeń więcej wszystkimi atakami wręcz do końca swojej następnej rundy. Umożliwienie.

\textbf{Koleżka}\index{Zdolności!Alfabetycznie!Koleżka}\label{sec:Koleżka} (3 punkty Intelektu) - Wybierz jedną z postaci stojących obok Ciebie. Ta postać zostaje Twoim koleżką na 10 minut. Jesteś wyszkolony we wszelkich zadaniach polegających na odnalezieniu, leczeniu, wchodzeniu z interakcję i chronieniu Twojego koleżki. Także, kiedy stoisz obok niego, obydwoje macie atut na Ochronie Szybkości. Możesz mieć tylko jednego koleżkę w danym czasie. Akcja, by rozpocząć. 

\textbf{Wbudowane Bronie}\index{Zdolności!Alfabetycznie!Wbudowane Bronie}\label{sec:Wbudowane Bronie} - Biomechaniczne implanty, magiczny kryształ wbudowany w czoło, lub coś równie dziwnego zapewnia Cię we wbudowaną broń. Pozwala Ci to na wystrzelenie promienia energii na długi zasięg, który zadaje 5 punktów obrażeń. Ta zdolność nic Cię nie kosztuje. Akcja.

\textbf{Palące Światło}\index{Zdolności!Alfabetycznie!Palące Światło}\label{sec:Palące Światło} (3 punkty Intelektu) - Wysyłasz promień światła na daną istotę w długim zasięgu, a następne zwężasz go, aż będzie palił, zadając 5 punktów obrażeń. Akcja. 

\textbf{Ucieczka}\index{Zdolności!Alfabetycznie!Ucieczka}\label{sec:Ucieczka} (5 punktów Szybkości) - Możesz wykonać dwie osobne akcje w tej turze, tak długo jak jedna z nich to ucieczka od wroga lub ukrycie się. Umożliwienie. 

\textbf{Przeniknięcie Przez Barierę}\index{Zdolności!Alfabetycznie!Przeniknięcie Przez Barierę}\label{sec:Przeniknięcie Przez Barierę} (6+ punktów Intelektu) - Przechodzisz przez drzwi, pole siłowe lub inną barierę, która ma maksymalną grubość 1 metra. W zależności od bariery, może to oznaczać znalezienie słabego punktu, który wykorzystujesz, naciśnięcie odpowiednich guzików czystym szczęściem, po prostu użycie siły, lub nawet dziwniejsze wyjaśnienia, jak dotknięcie cieńkiej warstewki między wymiarami lub niespodziewana interakcja z Twoim ekwipunkiem. Trudność zadania to poziom bariery. Ta zdolność pozwala Tobie na przeniknięcie, nikomu innemu, a przejście zamyka się na końcu Twojej tury (co może oznacząc, że jesteś uwięziony po drugiej stronie). Masz atut na każdej próbie następnego przeniknięcia przez już raz przenikniętą barierę. W dodatku do zwykłych opcji Wysiłku, możesz skorzystać z niego, by zwiększyć maksymalną grubość bariery, na każdy poziom zwiększając ją o dodatkowy metr. Akcja.

\section{C}

\textbf{Wezwanie Ducha}\index{Zdolności!Alfabetycznie!Wezwanie Ducha}\label{sec:Wezwanie Ducha} (6 punktów Intelektu) - Pod Twoim dotykiem, istota martwa nie dłużej niż 7 dni pojawia się jako (najwyraźniej fizyczny) duch, którego poziom jest taki sam, jak za życia. Przywołany duch istnieje maksymalnie przez dzień (lub mniej, jeśli osiąga coś ważnego dla niego w tym czasie), po którym znika i nie może pojawić się ponownie. 

Wezwany duch pamięta wszystko ,co wiedział za życia, i posiada większość swoich starych zdolności (ale niekoniecznie swój ekwipunek). Dodatkowo, uzyskuje on zdolność zostanie niematerialnym jako akcję (do minuty na raz). Wezwany Duch nie jest względem Ciebie w żaden sposób zobowiązany i nie musi zostać blisko Ciebie, by pozostać zamanifestowany. Akcja by rozpocząć. 

\textbf{Przysługa}\index{Zdolności!Alfabetycznie!Przysługa}\label{sec:Przysługa} (4 punkty Intelektu) - Strażnik, doktor, technik lub najęty bandyta zatrudniony przez lub stowarzyszony z przeciwnikiem jest sekretnie Twoim sprzymierzeńcem lub wisi Ci przysługę. Kiedy się na nią powołujesz, cel robi co może, żeby pomóc Ci (rozkuwa Cię, daje Ci nóż, zostawia drzwi celi otwarte) w sposób, który minimalizuje możliwość odkrycia, co zrobił. Ta zdolność to zadanie Intelektu poziomu 3. Każdy dodatkowy raz, gdy korzystasz z tej umiejętności, zadanie jest utrudnione o dodatkowy stopień. Trudność wraca do 3 po odpoczynku trwającym 10 godzin. Akcja.

\textbf{Wezwanie Międzywymiarowego Ducha}\index{Zdolności!Alfabetycznie!Wezwanie Międzywymiarowego Ducha}\label{sec:Wezwanie Międzywymiarowego Ducha} (6 punktów Intelektu) - przywołujesz istotę-ducha, który manifestuje się przez maksymalnie dzień (lub mniej, jeśli osiągnie przedtem coś ważnego) po którym znika i nie można go ponownie przywołać. Ten duch jest istotą 6 poziomu lub niższego, i może być materialna lub nie, zgodnie z własnym życzeniem (zmiana stanu wymaga akcji). Duch nie jest Tobie winny wdzięczności, i nie potrzebuje zostać blisko Ciebie, by pozostać zamanifestowanym. Akcja, by rozpocząć.

\textbf{Wezwanie Burzy}\index{Zdolności!Alfabetycznie!Wezwanie Burzy}\label{sec:Wezwanie Burzy} (7+ punktów Intelektu) - Jeśli jesteś na zewnątrz lub w pomieszczeniu, którego sufit sięga co najmniej 90 m, przywołujesz kotłujące się warstwy oświetlonych błyskawicami chmur burzowych do 460 m w promieniu na 10 minut. Podczas dnia, naturalne oświetlenie pod burzą jest zredukowane do niskiego. Kiedy burza grzmi, możesz wykorzystać akcję, by wysłac błyskawicę zz chmury, by zaatakować cel, który dostrzegasz, zadając mu 4 punkty obrażeń (możesz normalnie korzystać z Wysiłku na tych atakach). Trzy akcje by rozpocząć, akcja, by wezwać błyskawicę. 

\textbf{Wezwanie Roju}\index{Zdolności!Alfabetycznie!Wezwanie Roju}\label{sec:Wezwanie Roju} (4 punkty Intelektu) - Jeśli znajdujesz się w lokacji, gdzie mogą przybyć istoty związane z Twoją zdolnością Wpływ na Rój, możesz wezwać je na godzinę. Podczas tej godziny, istoty te robią, co im rozkażesz telepatycznie tak długo, jak są w dalekim zasięgu od Ciebie. Mogą one sie gromadzić i utrudniać akcje Twoich wrogów. Kiedy te istoty są w dalekim zasięgu, możesz rozmawiać z nimi telepatycznie i postrzegać świat poprzez ich zmysły. Akcja by rozpocząć. 

\textbf{Wezwanie Przez Czas}\index{Zdolności!Alfabetycznie!Wezwanie Przez Czas}\label{sec:Wezwanie Przez Czas} (6+ punktów Intelektu) - Przywołujesz osobę lub istotę do 3 poziomu z niedawnej przeszłości, i pojawia się ona obok Ciebie. Możesz wybrać istotę, z którą wcześniej wszedłeś w kontakt (nawet, jeśli jest teraz martwa) lub (nie więcej niż raz na dzień) możesz pozwolić MG na określenie istoty przypadkowo. Jeśli przywołujesz przypadkową istotę, masz 10 procent szans, że będzie to istota do 5 poziomu. Ta istota nie ma pamięci niczego przed byciem wezwaną przez Ciebie, ale mimo to może mówić i ma ogólną wiedzę, którą posiada istota jej typu. Istota wezwana przez czas wykonuje swoje akcje tak długo, jak sie na niej koncentrujesz, ale musisz wykorzystać swoją akcję w każdej turze, by wydać jej rozkazy, inaczej wróci do przeszłości.

W dodatku do normalnych opcji korzystania z Wysiłku, możesz skorzystać z wysiłku, by wezwać potężniejszą istotę: każdy poziom Wysiłku zwiększa poziom istoty o 1. Dla przykładu, zastosowanie poziomu Wysiłku, wzywa specyficzną istotę do 4 poziomu lub daje Ci 10 procent szans na wezwanie przypadkowej istoty do 6 poziomu. Akcja.

\textbf{Uspokojenie}\index{Zdolności!Alfabetycznie!Uspokojenie}\label{sec:Uspokojenie} (3 punkty Intelektu) - Poprzez dowcipy, piosenkę lub inną sztukę, powstrzymujesz jednego żywego przeciwnika przeciwko zaatakowaniem kogokolwiek lub czegokolwiek przez jedną rundę. Akcja.

\textbf{Uspokojenie Nieznajomego}\index{Zdolności!Alfabetycznie!Uspokojenie Nieznajomego}\label{sec:Uspokojenie Nieznajomego} (2+ punkty Intelektu) - Możesz sprawić, że jedna inteligentna istota pozostaje spokojna w momencie, gdy mówisz. Ta istota nie musi mówić Twoim językiem, ale musi być w stanie Cię ujrzeć. Pozostaje spokojna tak długo, jak skupiasz na sobie jej uwagę i nie jest zaatakowana lub w inny sposób w niebezpiecznej sytuacji. W dodatku do normalnych opcji Wysiłku, możesz go zastosować, by uspokoić dodatkową istotę sprzymierzoną z Twoim pierwszym celem - jedna istota na poziom Wysiłku. Akcja.

\textbf{Zręczny Wojownik}\index{Zdolności!Alfabetycznie!Zręczny Wojownik}\label{sec:Zręczny Wojownik} - Twoje ataki zadają o 1 punkt obrażeń więcej. Umożliwienie. 

{\textbf{Zachwyt lub Inspiracja}\index{Zdolności!Alfabetycznie!Zachwyt lub Inspiracja}\label{sec:Zachwyt lub Inspiracja} - Możesz zastosować tę zdolność na dwa sposoby. Albo Twoje słowa utrzymują uwagę wszystkich BN-ów, którzy je słyszą tak długo, jak mówisz, albo Twoje słowa inspirują BN-ów którzy je słyszą, tak, że funkcjonują przez godzinę, jakby posiadali o poziom więcej. W dowolnym wypadku, wybierasz, którzy BN-i dostają się pod wpływ tej zdolności. Jeśli ktoś w tłumie zostanie zaatakowany, kiedy próbujesz do niego przemówić, tracisz uwagę tłumu. Akcja, by rozpocząć. 

\textbf{Zachwyt Światła Gwiazd}\index{Zdolności!Alfabetycznie!Zachwyt Światła Gwiazd}\label{sec:Zachwyt Światła Gwiazd} - Tak długo, jak mówisz, utrzymujesz uwagę wszystkich 2 poziomowych lub słabszych BN-ów, który Cię słyszą. Jeśli posiadasz także zdolność Zauroczenie, możesz w podobny sposób wpłynąć na BN-ów poziomu 3-go. Akcja, by rozpocząć.

\textbf{Surfer Aut}\index{Zdolności!Alfabetycznie!Surfer Aut}\label{sec:Surfer Aut} - możesz wstać lub porusząc się w poruszającym się pojeździe (np: jego suficie, otwartych drzwiach, masce itp.) z duża szansą, ze nie spadniesz. Jesli pojazd nie zrobi nagłego zwrotu, zatrzyma isę nagle lub w inny sposób nie wykona jakiegoś ekstremalnego manewru, wstanie lub poruszanie się po takim pojeździe to dla Ciebie zadanie rutynowe. Jeśli pojazd wykonuje jakieś ekstremalne manewry, jak te opisane wyżej, wszystkie zadania, by pozostać na powierzchni pojazdu są ułatwione. Umożliwienie. 

\textbf{Ostrożny Rzut}\index{Zdolności!Alfabetycznie!Ostrożny Rzut}\label{sec:Ostrożny Rzut} - Jesteś wyszkolony we wszystkich atakach bronią rzucaną. Umożliwienie.

\textbf{Ostrożny Strzał}\index{Zdolności!Alfabetycznie!Ostrożny Strzał}\label{sec:Ostrożny Strzał}  - Możesz wydać punkty z Puli Szybkości lub z Puli Intelektu, by zwiększać Wysiłkiem obrażenia broni palnej. Każdy poziom wysiłku dodaje 3 punkty obtażeń do udanego ataku, a jeśli spędzisz swoją turę na celowanie, każdy poziom Wysiłku zamiast tego dodaje 5 punktów obrażeń do udanego ataku. Umożliwienie.

\textbf{Rzuć Iluzję}\index{Zdolności!Alfabetycznie!Rzuć Iluzję}\label{sec:Rzuć Iluzję} - Możesz zwiększyć zasięg w którym możesz tworzyć i podtrzymywać swoje iluzje bliskiego zasięgu (np: ze zdolności Mniejsza Iluzja) do dowolnego miejsca w średnim zasięgu, które możesz dostrzec. Umożliwienie. 

\textbf{Przerażenie}\index{Zdolności!Alfabetycznie!Przerażenie}\label{sec:Przerażenie} (4 punkty Intelektu) - Przerażasz swojego oponenta w dalekim zasięgu, który rozumie mowę (choć nie musi Twojego języka) tak bardzo, że traci on swą następną akcję i na resztę swoich akcji w ciągu 1 minuty jego zadania są utrudnione. Każdy dodatkowy raz, gdy próbujesz wykorzystać tę zdolność na tym samym wrogu, musisz zastosować o poziom Wysiłku więcej, niż przy ostatniej próbie. Akcja. 

\textbf{Talent Celebryty}\index{Zdolności!Alfabetycznie!Talent Celebryty}\label{sec:Talent Celebryty} - jesteś wyszkolony w dwóch z poniższych umiejętnościach: pisaniu, dziennikarstwie, danym rodzaju sztuki, danym sporcie, szahach, komunikacji naukowej, aktorstwie, prezentacji newsów lub innej powiązanej zdolności niebojowej, która uczyniła z Ciebie celebrytę. Umożliwienie. 

\textbf{Centrum Uwagi}\index{Zdolności!Alfabetycznie!Centrum Uwagi}\label{sec:Centrum Uwagi} (5 punktów Intelektu) - Dosłowny (lub metaforyczny, w zależności od settingu) promień czystej światłości zstępuje z Niebios i Cię okala. Wszystkie istoty, który wybierzesz w swoim bliskim zasięgu padają na kolana i tracą swą następną akcję. Cele tej mocy nie mogą się bronić i są traktowane jako bezsilne. Akcja.

\textbf{Komnata Snów}\index{Zdolności!Alfabetycznie!Komnata Snów}\label{sec:Komnata Snów} (8 punktów Intelektu) - Ty i Twoi sprzymierzeńcy możecie wkroczyć w komnat snów, udekorowaną jak sobie tego zażyczysz, która zawiera pewną liczbę drzwi. Prowadzą one do lokalizacji, które odwiedziłeś lub które znasz całkiem dobrze. Przejście przez jedne z tych drzwi przenosi Cię do pożądanej lokacji. Jest to zadanie trudności 2 bazujące na Intelekcie (zadanie może być trudniejsze, jeśli lokacja jest chroniona magicznie). Akcja by wejść do komnaty snów; akcja, by przejść przez wrota w komnacie.

\textbf{Zmiana Paradygmatu}\index{Zdolności!Alfabetycznie!Zmiana Paradygmatu}\label{sec:Zmiana Paradygmatu} (6+ punktów Intelektu) - Zmieniasz światopogląd istoty, z którą spędzasz przynajmniej rundę na rozmowie (jeśli jest ona w stanie Cię zrozumieć). Ta istota zmienia swoje zdanie odnośnie ważnego poglądu lub wierzenia, co może być czymś tak prostym jak zmiana chęci zamordowania Ciebie na pomoc Ci, lub być czymś dziwniejszym. Efekt trwa przynajmniej przez 10 minut, ale może trwać to dłużej, jeśli istota nie była wcześniej Twoim wrogiem. W tym czasie, istota podejmuje akcje zgodnie z mądrością, którą się z nią podzieliłeś. Cel musi być na poziomie 2 lub niższym. W dodatku do normalnych opcji korzystania z Wysiłku, możesz z niego skorzystać, aby zwiększyć maksymalny poziom celu (o 1 więcej na każdy poziom Wysiłku). Akcja, by rozpocząć.

\textbf{Naładowanie}\index{Zdolności!Alfabetycznie!Naładowanie}\label{sec:Naładowanie} (1+ punktów Intelektu) - Możesz naładować artefakt lub inne urządzenie (ale nie cypher), tak, by skorzystać z niego raz. Koszt to 1 punkt Intelektu plus 1 punkt na poziom urządzenia. Akcja.

\textbf{Naładowanie Broni}\index{Zdolności!Alfabetycznie!Naładowanie Broni}\label{sec:Naładowanie Broni} (2+ punkty Intelektu) - Jako część ataku Twoją magiczną bronią, ładujesz ją magiczną mocą, zadając dodatkowe 2 punkty obrażeń od energii. Jeśli wykonujesz więcej niż 1 atak w swojej turze, wybierasz, czy chcesz wydać punkty na tą zdolność przed wykonaniem każdego z ataków. Umożliwienie.

\textbf{Szarża Hordy}\index{Zdolności!Alfabetycznie!Szarża Hordy}\label{sec:Szarża Hordy} (7 punktów Mocy) - Ty i dwóch lub więcej z Twoich kompanów obo kCiebie działacie jak jedna istota, by wykonać atak szarżą. Kiedy to robicie, wszyscy poruszacie się na średni dystans, w trakcie czego atakujecie wszystko w Waszym bliskim zasięgu na swojej drodze, z atutem do ataku. Cele tego ataku otrzymują dodatkowe 3 punkty obrażeń i są wywrócone. Akcja.

\textbf{Zauroczenie Maszyny}\index{Zdolności!Alfabetycznie!Zauroczenie Maszyny}\label{sec:Zauroczenie Maszyny} (2 punkty Intelektu) - Przekonujesz nieinteligentną maszynę, by Cię ``lubiła''. Maszyna, która Cię lubi, ma szanszę mniejszą o 50 procent, by funkcjonować w sposób, który mógłby Cię zranić. Tak więc, jeśli wróg chce zdetonować bombę blisko Ciebie, a jest ona kontrolowane detonatorem, który Cię lubi, istnieje 50 procent szans, że bomba nie wybuchnie. Akcja, by rozpocząć. 

\textbf{Chmura Ochronna}\index{Zdolności!Alfabetycznie!Chmura Ochronna}\label{sec:Chmura Ochronna} (5 punktów Intelektu) - Sprawiasz, że małe obiekty z Twojego otoczenia (kamienie, zepsute przedmioty, chmury pyłu itp) obracają się wokół Ciebie przed 10 minut, co daje Ci +2 do Pancerza. Akcja, by rozpocząć. 

\textbf{Umysł-Twierdza}\index{Zdolności!Alfabetycznie!Umysł-Twierdza}\label{sec:Umysł-Twierdza} - Jesteś wyszkolony w akcjach obrony Intelektu i masz Pancerz +2 do ataków, które dotyczą Twojej puli Intelektu (co normalnie ignoruje Pancerz). Umożliwienie. 

\textbf{Zamglenie Pamięci Krótkotrwałej}\index{Zdolności!Alfabetycznie!Zamglenie Pamięci Krótkotrwałej}\label{sec:Zamglenie Pamięci Krótkotrwałej} (3 punkty Intelektu) - Jeśli wchodzisz w interakcje lub studiujesz cel przez przynajmniej rundę, zyskujesz świadomość, jak działa jego umysł, co możesz wykorzystać przeciwko niemu w najgorszy możliwy sposób. Możesz spróbować go skonfundować i sprawić, że zapomni, co właśnie zaszło. W przypadku sukcesu, usuwasz do ostatnich 5 minut z jego pamięci. Akcja by przygotować, akcja by rozpocząć.
 
\textbf{Ulepszenie Maszyny}\index{Zdolności!Alfabetycznie!Ulepszenie Maszyny}\label{sec:Ulepszenie Maszyny} (2 punkty Intelektu) - POlepszasz moc lub funkcjonowanie maszyny tak, że działa na 1 poziomie więcej niż zwyklep rzez jedną godzinę. Akcja, by rozpocząć. 

\textbf{Obliczenia Bitewne}\index{Zdolności!Alfabetycznie!Obliczenia Bitewne}\label{sec:Obliczenia Bitewne} - Podczas walki, Twój mózg przełącza się na tryb bojowy, gdzie wszystkie potencja;ne ataki, które możesz wykonać, pojawiają się jako wektory w Twoim umyśle, co zawsze zapewnia najlepszą opcję. Twoje ataki są ułatwione. Umozliwienie.

\textbf{Emisja Zimna}\index{Zdolności!Alfabetycznie!Emisja Zimna}\label{sec:Emisja Zimna} (5+ punktów Intelektu) - Emitujesz zimno we wszystkich kierunkach w średnim zasięgu. Wszyscy w obszarze Twojej emisji (z wyjątkiem Ciebie) otrzymują 5 punktów obrażeń. Jeśli zastosujesz WYsiłek, by zwiększyć obrażenia, zamiast ułatwić zadanie, zadajesz 2 dodatkowe punkty obrażeń na poziom Wysiłku (zamiast 3 punktów); cele w obszarze otrzymują 1 punkt obrażeń, nawet jeśli nie uda Ci isę rzut na atak. Akcja.

\textbf{Kolos}\index{Zdolności!Alfabetycznie!Kolos}\label{sec:Kolos} - Kiedy korzystasz z Wzrostu, możesz wybrać wzrost do 18 m wysokości. Kiedy to czynisz, zadajesz dodatkowe 2 punkty obrażeń atakami wręcz (plus wszelki dodatek ze zdolności Wielki). Na każdy poziom Wysiłku, który zastosujesz, Twój wzrost zwiększa się o 3 metry, i dodajesz 1 punkt więcej do swojej Puli Mocy. Tak więc, za pierwszym razem gdy zastosujesz Wzrost po 10-godzinnym odpoczynku, jeśli zastosujesz 2 poziomy Wysiłku, Twój wzrost wynosić będzie 24 metry i dodasz 17 tymczasowych punktów do swojej Puli Mocy. Umożliwienie.

\textbf{Wyzwanie Bojowe}\index{Zdolności!Alfabetycznie!Wyzwanie Bojowe}\label{sec:Wyzwanie Bojowe} - Wszystkie zadania, których celem jest ściągnięcie na Ciebi ataków (i odciągnięcie ich od innych) są ułatwione o dwa kroki. Umożliwienie.

\textbf{Zdolności Bojowe}\index{Zdolności!Alfabetycznie!Zdolności Bojowe}\label{sec:Zdolności Bojowe} - Dodajesz +1 obrażeń do jednego typu ataku z bronią Twojego wyboru: atak bronią wręcz lub dystansowy atak bronią. Umożliwienie.

\textbf{Rozkaz}\index{Zdolności!Alfabetycznie!Rozkaz}\label{sec:Rozkaz} (3 punkty Intelektu) - Poprzez czystą moc woli, możesz wydać prosty rozkaz danej istocie, która następnie wykonuje go przez następną akcję. Istota musi być w średnim zasięgu i być w stanie Cię zrozumieć. Rozkaz nie może być bezpośrednim zagrożeniem dla istoty lub jej towarzyszy, więc ``Popełnij samobójstwo'' nie zadziała, ale ``Ucieknij'' już tak. Dodatkowo, rozkaz może wymagać od istoty pojedyńczej akcji, więc ``Otwórz drzwi'' może zadziałać, ale ``Otwórz drzwi i przebiegnij przez nie'' już nie. Istota, której rozkazano, może się bronić normalnie i odpowiedzieć na atak atakiem, jeśli zostanie zaatakowana. Jeśli posiadasz inną zdolność, którą możesz wydać rozkaz istocie, możesz efektem Rozkazu objać dwie istoty na raz (jest to ``podstawowy efekt'' obydwu zdolności) korzystając z dowolnej z tych zdolności. Akcja. 

\textbf{Rozkazywanie Bestiom}\index{Zdolności!Alfabetycznie!Rozkazywanie Bestiom}\label{sec:Rozkazywanie Bestiom} (3+ punkty Intelektu) - Możesz rozkazywać nieagresywnej bestii nie będącej człowiekiem (jak np: istota, którą uspokoiłeś poprzez Ukojenie Dzikiego) do 3 poziomu w średnim zasięgu. Jeśli osiągniesz sukces, przez następną minutę bestia słucha Twoich werbalnych komend najlepiej jak rozumie i może. GM decyduje, co liczy się jako nieludzka bestia, ale jeśli nie masz do czynienia z jakimś oszustwem, powinieneś wiedzieć, czy możesz wpłynąć na istotę zanim skorzystasz z tej zdolności. Obcy, istoty międzywymiarowe, bardzo inteligentne istoty i roboty nigdy się w to nie wliczają. 

W dodatku do normalnych opcji korzystania z Wysiłku, możesz z niego skorzystać, by zwiększyć maksymalny poziom celu. Tak więc, by rozkazywać bestii 5 poziomu (2 poziomy ponad normalny limit), musisz zastosować dwa poziomy Wysiłku. Akcja, by rozpocząć. 



% below go licenses

\chapter{Licencje}

\section{Licencja polska}

\textbf{Otwarta Licencja Cypher System}

Ta licencja („Zgoda”) obowiązuje między wydawcą lub autorem („Tobą”) i Monte Cook Games, LLC („MCG”) i nadaje Ci stałą, nie-wyłączną, darmową, obowiązującą na całym świecie licencję, by publikować i dystrybuować materiały gier fabularnych („Dzieła”) bazujące na i wykorzystujące Cypher System Reference Document („CSRD”) oraz pozwalającą na deklarowanie kompatybilności z Cypher System. Poprzez załączenie słów „kompatybilne z Cypher System” na okładce Dzieła, lub poprzez załączenie loga „kompatybilne z Cypher System” na okładce Dzieła, lub poprzez załączenie owych rzeczy w jakichkolwiek dokumentach reklamowych, promocyjnych, informacjach prasowych lub innych dokumentach powiązanych z Dziełem, wskazujesz swoje zaakceptowanie warunków Zgody.

Dzieło może w sobie zawierać dowolne lub wszystkie treści wliczone w CSRD. Nie może ono zawierać tekstu, obrazków lub innej zawartości z innych publikacji MCG. MCG może opublikować unowocześnione wersje CSRD. Możesz skorzystać z dowolnej autoryzowanej wersji CSRD w swoim Dziele.

Dzieło Musi posiadać frazę „kompatybilne z Cypher System” lub logo „kompatybilne z Cypher System” na swojej okładce. Także, musi ono zawierać w Dziele, gdziekolwiek Dzieło wymienia informacje prawne i licencyjne, następujący tekst:

\begin{displayquote}
Ten produkt jest niezależną publikacją i nie jest powiązany z Monte Cook Games, LLC. Opublikowano go zgodnie z Otwartą Licencją Cypher System, którą można znaleźć pod adresem \url{https://csol.montecookgames.com/}

CYPHER SYSTEM i jego logo są znakiem handlowym Monte Cook Games, LLC w USA i innych państwach. Wszystkie postaci Monte Cook Games i ich nazwy oraz ich cechy charakterystyczne, są znakami handlowymi Monte Cook Games, LLC.
\end{displayquote}

Dzieło nie może zawierać loga Cypher System, loga MCG, lub innego znaku handlowego MCG, z wyjątkiem loga „kompatybilne z Cypher System”.

Możesz skorzystać z każdej autoryzowanej wersji loga „kompatybilne z Cypher System”. Nie możesz zmienić loga w żaden sposób, z wyjątkiem zmiany jego rozmiaru, zachowując proporcje. 

Nie możesz sprzedawać lub reklamować Dzieła używając imienia swoich współpracownika/ów, chyba, że masz pisemną zgodę od niego/nich, by tak uczynić.

Poza uznaniem, że Dzieło jest produkowane i dystrybuowane na bazie Zgody, ani Dzieło, ani żadne dokumenty promocyjne, reklamowe, informacje prasowe lub inne dokumenty związane z Dziełem nie mogą zawierać żadnego twierdzenia, jakoby Ty lub Dzieło mielibyście zgodę lub pozwolenie MCG na publikację, lub że Ty lub Dzieło jesteście stowarzyszeni z MCG w jakikolwiek sposób.

Ani Dzieło, ani żadne materiały promocyjne, reklamowe, informacje prasowe lub inne dokumenty związane z Dziełem nie mogą zawierać poglądów rasistowskich, homofobicznych, dyskryminujących lub w inny sposób odrażających; otwarcie politycznych celów lub poglądów; opisów kryminalnej przemocy względem dzieci; gwałtu lub innych aktów perwersji kryminalnej; lub innych obscenicznych materiałów.

Dzieło nie może naruszać, nadużywać lub wykorzystywać w szkodliwy sposób praw własności intelektualnej żadnej trzeciej strony. Niniejszym zobowiązujesz się chronić MCG przeciwko wszelkim roszczeniom, pozwom, stratom i szkodom wynikającym z postulowanego naruszenia praw intelektualnych, szkodliwego wykorzystania lub nadużycia intelektualnych praw własności jakiejkolwiek trzeciej strony. Trwa to nawet po wycofaniu tej Zgody.

MCG nie bierze żadnej odpowiedzialności za Dzieło. Zgadzasz się przypilnować, by MCG i jego pracownicy, partnerzy i reprezentanci pozostali wolni od szkody w przypadku, gdy Twoja publikacja Dzieła poskutkuje pozwem sądowym.
CSRD, logo „kompatybilne z Cypher System”, logo Cypher System, logo MCG i wszystkie inne znaki handlowe MCG należą wyłącznie do MCG.

Jeśli złamiesz jakiekolwiek warunki tej Zgody, zostaje ona automatycznie Tobie wycofana. Jeśli naruszenie warunków Zgody nie zostanie wyjaśnione z MCG i udokumentowane pomiędzy Tobą a MCG w ciągu 15 dni od złamania warunków, musisz natychmiast wycofać i zniszczyć wszystkie istniejące kopie Dzieła. Dodatkowo, możesz zostać pozwany za szkody wynikłe wskutek złamania warunków Zgody.

Te Zgoda jest zarządzana przez prawa Stanu Waszyngton i prawa Stanów Zjednoczonych Ameryki. W związku z jakimikolwiek pozwami wynikłymi na bazie tej Zgody, strony sporu poddają się jurysdykcji sądów Stanu Waszyngton i Stanów Zjednoczonych Ameryki.

KONIEC DOKUMENTU

{ \color{red} \textbf{Uwaga}}: Legalnie obowiązująca wersja licencji to licencja angielska. W razie sporów prawnych, to do niej należy się odwoływać, nie zaś do powyższego tłumaczenia. Zamieszczono je tutaj w celach czysto poglądowych.

\section{Licencja angielska}

\textbf{Cypher System : : Open License}

This license (the “Agreement”) is an agreement between a publisher or author (“You”) and Monte Cook Games, LLC (“MCG”), that grants You a perpetual, non-exclusive, royalty-free, worldwide license to publish and distribute tabletop roleplaying game materials (the “Work”) based on and incorporating the Cypher System Reference Document (“CSRD”) and declaring compatibility with the Cypher System. By including the words “Compatible with the Cypher System” on the cover of the Work, or by including the “Compatible with the Cypher System” logo on the cover of the work, or by including these items on or in any advertising, promotions, press releases, or other documents affiliated with the Work, You indicate Your acceptance of the terms of this Agreement.

The Work may include any or all text included in the CSRD. It may not include text, art, or other content from other MCG publications. MCG may publish updated versions of the CSRD. You may use any authorized version of the CSRD in the Work.
The Work must include the phrase “Compatible with the Cypher System” or the “Compatible with the Cypher System” logo on the cover of the Work. And it must include within the Work, wherever the Work otherwise lists legal and copyright information, the following text:

\begin{displayquote}
This product is an independent production and is not affiliated with Monte Cook Games, LLC. It is published under the Cypher System Open License, found at \url{https://csol.montecookgames.com/}.

CYPHER SYSTEM and its logo are trademarks of Monte Cook Games, LLC in the U.S.A. and other countries. All Monte Cook Games characters and character names, and the distinctive likenesses thereof, are trademarks of Monte Cook Games, LLC.
The Work may not use or incorporate the Cypher System logo, the MCG logo, or any other trademark of MCG, except the “Compatible with the Cypher System” logo.
\end{displayquote}

You may use any authorized version of the “Compatible with the Cypher System” logo. You may not crop or alter the logo in any way, except to resize it proportionally.

You may not market or advertise the Work using the name of any contributor unless You have written permission from the contributor to do so.

Other than to acknowledge that the Work is produced and distributed under this Agreement, neither the Work nor any advertising, promotions, press releases, or other documents affiliated with the Work may contain any claim that You or the Work has been sanctioned or approved by MCG, or is affiliated with MCG in any way.

Neither the Work nor any advertising, promotions, press releases, or other documents affiliated with the Work may contain racist, homophobic, discriminatory, or other repugnant views; overt political agendas or views; depictions or descriptions of criminal violence against children; rape or other acts of criminal perversion; or other obscene material.

The Work may not infringe, wrongfully use, or misappropriate the intellectual property rights of any third party. You hereby indemnify MCG and undertake to defend MCG against and hold MCG harmless from any claims, suits, loss, and damages arising out of alleged infringement, wrongful use, or misappropriation of any third party’s intellectual property by the Work. The indemnification obligations shall survive the termination of this Agreement.

MCG takes no responsibility for the Work. You agree to hold MCG and its officers, partners, and employees harmless in the event that Your publication of the Work results in legal action.

The CSRD, the “Compatible with the Cypher System” logo, the Cypher System logo, the MCG logo, and all other trademarks of MCG belong solely and exclusively to MCG.

If You breach any of the terms of this Agreement, it results in automatic termination of the Agreement. Unless the breach is cured to MCG’s sole satisfaction and such cure is documented by a written agreement between You and MCG within 15 days of breach, You must immediately recall and destroy all existing copies of the Work. You may additionally be subject to damages as a result of breach.

This Agreement shall be construed and governed by the laws of the State of Washington and the laws of the United States. With regard to any disputes arising under this Agreement, the parties hereby submit to the jurisdiction of the courts of the State of Washington and the United States.

END OF AGREEMENT

\printindex

\listoftables

\end{document}