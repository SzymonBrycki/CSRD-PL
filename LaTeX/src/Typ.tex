\section{Typ}\index{Typ}

Typ postaci to najważniejsza cecha Twojej postaci. Twój typ pozwala określić miejsce postaci w świecie i jej relację z innymi ludźmi. Jest to rzeczownik w zdaniu “Jestem przymiotnik rzeczownik który czasownikuje”.

(W pewnych grach RPG, typ postaci może zostać nazwany klasą postaci.)

Możesz wybrać z 4 typów postaci: Wojownika, Adepta, Odkrywcy i Mówcy. Jednakże, możesz nie chcieć korzystać z tych ogólnikowych nazw na nie. Ten rozdział oferuje parę bardziej specyficznych nazw na każdy typ, które mogą być stosowne, w zależności od świata przedstawionego. Odkryjesz, że nazwy takie jak “Wojownik” czy też “Odkrywca” nie zawsze pasują do gier dziejących się w świecie współczesnym. Jak zawsze, możesz zrobić, co uznasz za stosowne. (Twój typ określa kim jest postać. Powinieneś korzystać z dowolnej nazwy na typ, tak długo, jak pasuje zarówno do postaci, jak i do settingu.)

Ponieważ typ to podstawa, na której się buduje postać, warto się zastanowić, jaka relacja łączy go z settingiem. Aby z tym pomóc, typy to w zasadzie ogólne archetypy. Wojownik, dla przykładu, może być wszystkim, od rycerza w lśniącej zbroi, przez gliniarza na ulicy po cybernetycznego weterana tysiąca futurystycznych wojen.
Aby lepiej dostosować cztery typy do różnych settingów, istnieją różne metody zwane posmakami,  zaprezentowane w stosownym rozdziale, by pomóc w dostosowaniu typów do konwencji fantasy, science fiction, lub innych (lub by dostosować typy do pomysłu na postać).

Dalej, bardziej fundamentalne opcje dla \mytext{dalszej customizacji} są dostępne na końcu tego rozdziału. 

\subsubsection{Wtrącenie Gracza}\index{Wtrącenie Gracza}

Wtrącenie gracza oznacza, że gracz wybiera zmianę czegoś w kampanii, czyniąc rzeczy łatwiejszymi dla jego postaci. Konceptualnie, jest to przeciwieństwo wtrącenia MG: zamiast MG dawać PD graczowi i wprowadzać niespodziewaną komplikacje dla jego postaci, gracz wydaje 1 PD i wprowadza rozwiązanie problemu lub komplikacji. To, co może zrobić wtrącenie gracza, to zmienić świat gry lub obecne okoliczności zamiast bezpośrednio zmieniać postać. Dla przykładu, wtrącenie mówiące, że cypher, z którego właśnie się skorzystało, ma dodatkowe użycie byłoby właściwe, ale wtrącenie uzdrawiające postać nie byłoby. Jeśli gracz nie ma PD-ków do wydania, nie może wprowadzić wtrącenia gracza. 

Parę wtrąceń gracza jest zasugerowanych pod każdym typem. Warto jednak zaznaczyć, że nie każde wtrącenie gracza jest stosowne w każdej sytuacji. MG może zezwolić graczom na inne sugestie wtrąceń, ale ostatecznie to on decyduje, czy dane wtrącenie jest stosowne do typu postaci i danej sytuacji. Jeśli MG odmawia wtrącenia, gracz nie wydaje 1 PD-ka i wtrącenie nie następuje.

Korzystanie z intruzji nie wymaga od postaci akcji, by je zastosować. Po prostu ono następuje.

(Wtrącenie gracza powinno być ograniczone do nie więcej niż jednego wtrącenia na gracza na jedną sesję.)

\subsubsection{Akcje obronne}\index{Akcje obronne}

Akcje obronne występują wtedy, gdy gracz rzuca, by uchronić się od czegoś nieporządanego, co mogłoby się wydarzyć jego BG. Rodzaj akcji obronnej ma znaczenie, gdy rozważamy Wysiłek.

\textbf{Obrona Mocy}: Używa się jej do odporności na trucizny, choroby i wszystko inne, co można przezwyciężyć siłą i zdrowiem.

\textbf{Obrona Szybkości}: Używa się jej do unikana ciosów i uciekana od niebezpieczeństw. To najczęściej wykorzystywany rodzaj akcji obronnej.

\textbf{Obrona Intelektu}: Używa się jej do odpieranie ataków mentalnych i wszystkiego, co może wpłynać na czyiś umysł.