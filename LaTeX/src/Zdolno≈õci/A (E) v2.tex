% hyperref !!! https://tex.stackexchange.com/questions/180571/making-clickable-links-to-sections-with-hyperref

\chapter{Zdolności w kolejności alfabetycznej}

\section{A}

\textbf{Uśmiech i Słowo}\index{Zdolności!Alfabetycznie!Uśmiech i Słowo}\label{sec:Uśmiech i Słowo} - kiedy korzystasz z Wysiłku do dowolnej akcji interakcji społecznej - nawet takiej która polega na uspokajaniu zwierząt lub komunikowania się z kimś, czyim językiem nie mówisz - uzyskujesz darmowy poziom Wysiłku na tym zadaniu. Akcja.

\textbf{Przydatna Pomoc}\index{Zdolności!Alfabetycznie!Przydatna Pomoc}\label{sec:Przydatna Pomoc} - kiedy pomagasz komuś z zadaniem i stosuje on poziom wysiłku, zyskuje on darmowy poziom Wysiłku na tym zadaniu. Umożliwienie. 

\textbf{Absorpcja Energii}\index{Zdolności!Alfabetycznie!Absorpcja Energii}\label{sec:Absorpcja Energii} (7 punktów Intelektu) - dotykasz obiektu i absorbujesz jego energię. Jeśli dotykasz zamanifestowanego Cyphera, czynisz go bezużytecznym. Jeśli dotykasz artefaktu, rzuć na jego wyczerpanie. Jeśli dotykasz innego rodzaju zasilanego urządzenia lub maszyny, GM określa, czy jego moc jest w pełni wyssana. W każdym razie, absorbujesz energię z obiektu i odzyskujesz 1k10 punktów Intelektu. Jeśli to dałoby Ci więcej punktów Intelektu niż maksimum Twojej Puli, dodatkowe punkty są utracone, i musisz wykonać rzut na Obronę Mocy. Trudność tego rzutu to numer punktów powyżej Twojego maksimum, które zaabsorbowałeś. Jeśli oblejesz ten rzut, otrzymujesz 5 punktów obrażeń i nie możesz podejmować działań przez jedną rundę. Możesz wykorzystać tę zdolność jako akcję obronną kiedy jesteś celem ataku zdolnością. Taka akcja niweluje atak zdolnością, a ty absorbujesz energię, jakby pochodziła z urządzenia. Akcja.

\textbf{Absorpcja Energii Kinetycznej}\index{Zdolności!Alfabetycznie!Absorpcja Energii Kinetycznej}\label{sec:Absorpcja Energii Kinetycznej} - absorbujesz porcję energii ataku fizycznego lub uderzenia. Negujesz 1 punkt obrażeń, które normalnie byś poniósł i przechowujesz tę energię. Po tym, jak zaabsorbujesz 1 punkt energii, kontynuujesz obniżać obrażenia o 1 punkt z nadchodzących ataków, ale pozostała energia wycieka z Ciebie w formie błysku nieszkodliwego światła (nie możesz przechowywać na raz więcej niż 1 punktu energii w tym samym czasie). Umożliwienie. 

\textbf{Absorpcja Czystej Energii}\index{Zdolności!Alfabetycznie!Absorpcja Czystej Energii}\label{sec:Absorpcja Czystej Energii} - kiedy korzystasz z Absorpcji Energii Kinetycznej, możesz także absorbować i przechowywać energię ataków bazujących na czystej energii (światło, promieniowanie, energie międzywymiarowe, psioniczne itp.) lub z przekaźników owej energii, gdy masz z nimi bezpośredni kontakt. Ta zdolność nie zmienia tego, ile punktów energii możesz przechowywać. Jeśli masz również Ulepszoną Absorpcję Energii Kinetycznej, możesz również absorbować do 2 punktów obrażeń ze źródeł czystej energii. Umożliwienie. 

\textbf{Akceleracja}\index{Zdolności!Alfabetycznie!Akceleracja}\label{sec:Akceleracja} (4+ punkty Intelektu) - Twoje słowa umacniają ducha postaci w bliskim zasięgu, która jest w stanie zrozumieć Cię, przyspieszając ją, tak, że zyskuje ona atut na testach inicjatywy i rzutach na Obronę Szybkości przez 10 minut. Dodatkowo, poza zwykłymi opcjami korzystania z Wysiłku, możesz z niego skorzystać, by objąć celem tej zdolności więcej postaci - każdy poziom Wysiłku obejmuje dodatkowy cel. Musisz przemówić do dodatkowych celów, by je przyspieszyć, jeden cel na rundę. Jednak akcja na jeden cel by rozpocząć. 

\textbf{Akrobatyczny Atak}\index{Zdolności!Alfabetycznie!Akrobatyczny Atak}\label{sec:Akrobatyczny Atak} (1+ punktów Szybkości) - wyskakujesz w ataku, przesuwając się przez powietrze. Jeśli wyrzucasz naturalne 17 lub 18, możesz wybrać mniejszy efekt zamiast dodatkowych obrażeń. Jeśli zastosujesz Wysiłek do tego ataku, uzyskujesz darmowy poziom Wysiłku na zadaniu. Nie możesz skorzystać z tej zdolności, jeśli Twój Wysiłek Szybkości jest zredukowany wskutek noszenia zbroi. Umożliwienie. 

\textbf{Procesor Akcji}\index{Zdolności!Alfabetycznie!Procesor Akcji}\label{sec:Procesor Akcji} (4 punkty Intelektu) - korzystając z przechowywanych informacji i zdolności analizowania nadchodzących danych z wielką szybkością, jesteś wyszkolony w jednym fizycznym zadaniu Twojego wyboru na 10 minut. Dla przykładu, możesz wybrać bieg, wspinaczkę, pływanie, Obronę Szybkości lub atak specyficzną bronią. Akcja by rozpocząć.

\textbf{Adaptacja}\index{Zdolności!Alfabetycznie!Adaptacja}\label{sec:Adaptacja} - dzięki ukrytej mutacji, urządzeniu wbudowanemu w Twój kręgosłup, rytuałowi krwii smoka, lub jakiemuś innemu darowi, jesteś teraz w komfortowej temperaturze; nie musisz sie nigdy martwić o niebezpieczne promieniowanie, choroby lub gazy; i możesz zawsze oddychać w dowolnym środowisku (nawet w próżni kosmosu). Umożliwienie.

\textbf{Zaawansowany Użytkownik Cypherów}\index{Zdolności!Alfabetycznie!Zaawansowany Użytkownik Cypherów}\label{sec:Zaawansowany Użytkownik Cypherów} - możesz mieć przy sobie 4 Cyphery w danym czasie. Umożliwienie. 

\textbf{Zaawansowany Rozkaz}\index{Zdolności!Alfabetycznie!Zaawansowany Rozkaz}\label{sec:Zaawansowany Rozkaz} (7 punktów Intelektu) - cel w średnim zasięgu słucha każdej komendy, którą mu wydasz, tak długo, jak słyszy Cię i rozumie. Co więcej, tak długo, jak nie robisz nic innego niż wydawanie komend (nie wolno Ci wziąć żadnej innej akcji) możesz dać temu samemu celowi nową komendę. Ten efekt kończy się, gdy kończysz wydawać komendy lub gdy cel opuszcza średni zasięg względem Ciebie. Akcja by rozpocząć. 

\textbf{Atak z Rozbrojeniem}\index{Zdolności!Alfabetycznie!Atak z Rozbrojeniem}\label{sec:Atak z Rozbrojeniem} (3 punkty Szybkości) - za pomocą serii szybkich ruchów, wykonujesz atak przeciwko uzbrojonemu przeciwnikowi, zadając mu obrażenia i rozbrajając go, tak, że jego broń jest teraz w Twoich rękach lub 3 metry od niego na ziemi - Ty wybierasz. Ten atak rozbrajający jest utrudniony. Akcja.

\textbf{Zalety Bycia Dużym}\index{Zdolności!Alfabetycznie!Zalety Bycia Dużym}\label{sec:Zalety Bycia Dużym} - kiedy korzystasz ze Wzrostu, jesteś tak duży, że możesz łatwiej przenosić duże obiekty, wspinać sie na budynki korzystając z uchwytów niedostępnych dla zwykłych ludzi i skakać znacznie dalej. Kiedy korzystasz ze Wzrostu, wszystkie zadania wspinaczki, podnoszenia ciężarów i skakania są dla Ciebie ułatwione. Umożliwienie.

\textbf{Zalety Bycia Małym}\index{Zdolności!Alfabetycznie!Zalety Bycia Małym}\label{sec:Zalety Bycia Małym} - nauczyłeś się, jak wykorzystać swój rozmiar, siłę i dokładność. Twoje obrażenia już się nie dzielą na pół gdy korzystasz ze Zmniejszenia się, a zadania wspinaczki i skakania są ułatwione. Umożliwienie.

\textbf{Porada od Przyjaciela}\index{Zdolności!Alfabetycznie!Porada od Przyjaciela}\label{sec:Porada od Przyjaciela} (1 punkt Intelektu) - znasz słabe i mocne strony swojego przyjaciela, i wiesz jak go zmotywować, by osiągnął sukces. Kiedy dajesz przyjacielowi sugestię powiązaną z jego następną akcję, postać ta jest wyszkolona w tej akcji na jedną rundę. Akcja. 

\textbf{Znowu i Znowu}\index{Zdolności!Alfabetycznie!Znowu i Znowu}\label{sec:Znowu i Znowu} (8 punktów Szybkości) - możesz wziąć kolejną akcję w rundzie, w której już podjąłeś akcję. Umożliwienie.

\textbf{Nieśmiertelny}\index{Zdolności!Alfabetycznie!Nieśmiertelny}\label{sec:Nieśmiertelny} - Twoje ciało i umysł się nie starzeją. Jeśli nie zostaniesz zabity przez akt przemocy (lub jakąś zewnętrzną siłę jak trucizna lub infekcja), nigdy nie umrzesz. Umożliwienie.  

\textbf{Agent-Prowokator}\index{Zdolności!Alfabetycznie!Agent-Prowokator}\label{sec:Agent-Prowokator} - wybierz jedna z poniższych, by być wytrenowanym w: atakowanie bronią swojego wyboru, ładunki wybuchowe, lub skradanie się i otwieranie zamków (jeśli wybierzesz ostatnią opcję, posiadasz trening w dwóch umiejętnościach). Umożliwienie.

\textbf{Agresja}\index{Zdolności!Alfabetycznie!Agresja}\label{sec:Agresja} (2 punkty Mocy) - skupiasz się na atakowaniu w tak wielki sposób, że zostawiasz siebie wysuniętego na ataki wrogów. Kiedy ta zdolność jest aktywna, zyskujesz atut na atakach wręcz i Twoje rzuty na Obronę Szybkości przeciwko atakom wręcz i dystansowym są utrudnione. Ten efekt trwa tak długo, jak sobie życzysz ale kończy się, jeśli walka nie ma miejsca w zasięgu Twoich zmysłów. Umożliwienie.

\textbf{Szybki Umysł}\index{Zdolności!Alfabetycznie!Szybki Umysł}\label{sec:Szybki Umysł} - kiedy próbujesz wykonać zadanie Szybkości, możesz zamiast tego rzucić (i wydać punkty z puli) jakby to była akcja Intelektu. Jeśli stosujesz Wysiłek do tego zadania, możesz wydać punkty z Puli Intelektu zamiast Puli Szybkości (wtedy stosujesz też Skupienie w Intelekcie zamiast w Szybkości). Umożliwienie. 

\textbf{Wysokie Skupienie}\index{Zdolności!Alfabetycznie!Wysokie Skupienie}\label{sec:Wysokie Skupienie} (7 punktów Intelektu) - wkładasz w swoje zadanie wszystko. Dodajesz trzy darmowe poziomy Wysiłku to następnego zadania, które podejmujesz. Nie możesz wykorzystać tej zdolności znowu, dopóki nie zakończysz 10-godzinnego odpoczynku. Akcja.

\textbf{Uzdrowienie}\index{Zdolności!Alfabetycznie!Uzdrowienie}\label{sec:Uzdrowienie} (3 punkty Intelektu) - możesz spróbować uzdrowić jedno schorzenie (np: chorobę lub truciznę) dotyczące jednej istoty. Akcja.

\textbf{Szczur Miejski}\index{Zdolności!Alfabetycznie!Szczur Miejski}\label{sec:Szczur Miejski} (6 punktów Intelektu) - kiedy jesteś w mieście, odnajdujesz lub tworzysz znaczące skróty, sekretne wejścia lub ostateczne trasy ucieczki tam, gdzie wcześniej ich nie było. Aby to zrobićm musisz uzyskać sukces na kacji Intelektu, której trudność określa MG bazując na danej sytuacji. Powinieneś ustalić detale wraz ze swoim MG. Akcja.

\textbf{Zawsze Majsterkując}\index{Zdolności!Alfabetycznie!Zawsze Majsterkując}\label{sec:Zawsze Majsterkując} - jeśli masz narzędzia i materiały i nosisz mniej cypherów niż Twój limit, możesz stworzyć zamanifestowany cypher, jeśli poświęcisz na to godzinę. Nowy cypher jest wybierany przypadkowo i zawsze o 2 poziomy mniej niż normalnie (minimum to 1-szy poziom). Jest on także chwilowy i wrażliwy na uszkodzenia. Nazywa się go chwilowym cypherem. Jeśli dasz go komuś, by z niego korzystał, rozpada się on natychmiast w bezużyteczne śmieci. Akcja by rozpocząć; 1 godzina by ukończyć.

\textbf{Cudowne Kopiowanie}\index{Zdolności!Alfabetycznie!Cudowne Kopiowanie}\label{sec:Cudowne Kopiowanie} - możesz skorzystać ze zdolności Skopiuj Moc, aby skopiować potężniejsze zdolności. W dodatku do normalnych opcji korzystania z Wysiłku przy użyciu Skopiuj Moc, jeśli zaaplikujesz 2 poziomy Wysiłku, MG wybiera moc wysokiego poziomu, która najbardziej przypomina moc, którą pragniesz skopiować (zamiast zdolności niskiego poziomu). Umożliwienie.

\textbf{Dodatkowy Wysiłek}\index{Zdolności!Alfabetycznie!Dodatkowy Wysiłek}\label{sec:Dodatkowy Wysiłek} - kiedy stosujesz przynajmniej jeden poziom Wysiłku do akcji niebojowej, otrzymujesz darmowy, dodatkowy poziom Wysiłku na tym zadaniu. Kiedy wybierasz tę zdolność, musisz zdecydować, czy dotyczy ona Wysiłku Mocy, czy też Wysiłku Szybkości. Umożliwienie.

\textbf{Wielki Skok}\index{Zdolności!Alfabetycznie!Wielki Skok}\label{sec:Wielki Skok} (2 punkty Mocy) - skaczesz w powietrze i lądujesz bezpiecznie w pewnej odległości. Możesz skoczyć wzwyż, w dół lub w poziomie gdziekolwiek w dalekim zasięgu  jeśli masz czystą trasę do tego miejsce, bez żadnych przeszkód. Jeśli masz 3 lub więcej punktów mocy zainwestowanych w siłę, Twój zasięg się ulepsza do bardzo dalekiego. Jeśli masz 5 lub więcej punktów mocy zainwestowanych w siłę, Twój zasięg skoku zostaje ulepszony do 300 metrów. Akcja.

\textbf{Czatownik}\index{Zdolności!Alfabetycznie!Czatownik}\label{sec:Czatownik} - kiedy atakujesz istotę, która jeszcze nie wzięła swojej pierwszej rundy w walce, Twój atak jest ułatwiony. Umożliwienie.

\textbf{Wzmocnienie Dźwięku}\index{Zdolności!Alfabetycznie!Wzmocnienie Dźwięku}\label{sec:Wzmocnienie Dźwięku} (2 punkty Mocy) - na jedną minutę, możesz wzmocnić dalekie lub ciche dźwięki, tak, byś mógł je słyszeć wyraźnie, nawet jeśli jest to rozmowa lub dźwięk małego zwierzęcia poruszającego się w podziemnej norze w bardzo dalekim zasięgu. Możesz spróbować usłyszeć dźwięk, nawet jeśli istnieją bariery blokujące dźwięk lub jest on bardzo cichy, choć to wymaga paru dodatkowych rund koncentracji. Aby odróżnić dźwięk, którego poszukujesz, od głośnego środowiska, także powinieneś poświęcić parę rund na skupienie, gdy przeszukujesz słuchem swoją okolicę. Mając odpowiednio dużo czasu, możesz wyśledzić każdą konwersację, oddychającą istotę i każde urządzenie wydające dźwięk w zasięgu. Akcja by rozpocząć, do paru rund by ją zakończyć, w zależności od trudności zadania.

\textbf{Anegdota}\index{Zdolności!Alfabetycznie!Anegdota}\label{sec:Anegdota} (2 punkty Intelektu) - możesz polepszyć morale grupy istot i pomóc im w nawiązaniu więzi, poprzez zabawianie ich podnoszącą na duchu anegdotą. Przez następną godzinę, ci którzy słuchali Twojej historii są wyszkoleni w jednym zadaniu Twojego wyboru, które jest powiązane z anegdotą, tak długo, jak nie jest to atak lub obrona. Akcja by rozpocząć, jedna minuta by zakończyć.

\textbf{Zwierzęce Szpiegowanie}\index{Zdolności!Alfabetycznie!Zwierzęce Szpiegowanie}\label{sec:Zwierzęce Szpiegowanie} (4+ punkty Intelektu) - jeśli znasz ogólną lokalizację zwierzęcia, które jest przyjazne względem Ciebie i w zasięgu 1.5 km od Ciebie, możesz postrzegać świat jego zmysłami do 10 minut. Jeśli nie jesteś w formie zwierzęcej lub w formie podobnej do tego zwierzęcia, musisz zastosować poziom Wysiłku do korzystania z tej umiejętności. Akcja by rozpocząć. 

\textbf{Zwierzęcy Kształt}\index{Zdolności!Alfabetycznie!Zwierzęcy Kształt}\label{sec:Zwierzęcy Kształt} (3+ punkty Intelektu) - zmieniasz się w zwierzę tam małe jak szczur lub tak duże jak ty (np: duży pies lub mały niedźwiedź) na 10 minut. Za każdym razem, gdy zmieniasz kształt, możesz wybrać inne zwierzę. Twój ekwipunek staje się częścią owej transformacji, co czyni go nieużytecznym, o ile nie ma pasywnego efektu, takiego jak zbroja. W tej formie Twoje Statystyki pozostają takie same jak w Twojej normalnej formie, ale możesz się ruszać i atakować zgodnie z Twoim zwierzęcym kształtem (ataki większości zwierząt tego rozmiaru to bronie średnie, z których możesz korzystać bez żadnej kary). Zadania wymagające rąk - takie jak naciskanie klamek lub przycisków są utrudnione kiedy jesteś w formie zwierzęcej. Nie możesz mówić, ale dalej możesz korzystać ze zdolności, które nie polegają na ludzkiej mowie. Uzyskujesz dwie pomniejsze zdolności powiązane z istotą, w którą sie zmieniłeś (patrz tabela Mniejsze Zdolności Zwierzęcego Kształtu). Dla przykładu, jeśli zamieniasz się w nietoperza, jesteś wyszkolony w percepcji i możesz latać na daleki zasięg w każdej rundzie. Jeśli zamienisz się w ośmiornicę, jesteś wyszkolony w skradaniu się i oddychasz pod wodą. Jeśli zastosujesz poziom Wysiłku do stosowania tej zdolności, możesz albo przybrać kształt mówiącego zwierzęcia, albo hybrydowy. Kształt mówiącego zwierzęcia wygląda dokładnie jak zwykłe zwierzę, ale możesz dalej mówić i korzystać ze zdolności bazujących na ludzkiej mowie. Kształt hybrydowy wygląda jak Twoja normalna forma, ale z cechami zwierzęcia, nawet jeśli to konkretne zwierzę jest znacznie mniejsze od Ciebie (jak nietoperz lub szczur). W formie hybrydowej możesz mówić, korzystać ze swoich wszystkich zdolności, atakować jak zwierzę i wykonywać zadania przy użyciu rąk bez utrudnienia. Każdy kto dobrze się przypatrzy Tobie w formie hybrydowej nigdy nie pomyliłby Cię ze zwierzęciem. Akcja by się przemienić lub odwrócić transformację. 

``Podobieństwo'' to termin ogólnikowy. Lwy są podobne do tygrysów i leopardów, orły są podobne do kruków i łabędzi, psy są podobne do wilków i lisów itp.

Nawet jeśli Twój zwierzęcy kształt ma wiele typów ataku (np: zębami i pazurami), możesz zaatakować tylko raz w rundzie, chyba że masz jakąś zdolność, która pozwala CI na dokonywanie dodatkowych ataków w swojej turze.

Wariant Zwierzęcego Kształtu: Jeśli Twój koncept postaci sprawia, że zawsze zmienia się ona w ten sam zwierzęcy kształt zamiast wybierać z wielu, podwój czas trwania Zwierzęcego Kształtu (20 minut na jedno wykorzystanie). MG może pozwolić postaci z tym ograniczeniem na uczenie się dodatkowych zwierzęcych form poprzez wydanie 4 PD jako długotrwałą korzyść. 

\begin{table*}[t]

\centering
\caption{Tabela Mniejszych Zdolności Zwierzęcej Formy}
\label{Tabela Mniejszych Zdolności Zwierzęcej Formy}

\begin{tabularx}{\textwidth}{| X | X | X |}
\hline
 
 \textbf{Zwierzę} & \textbf{Umiejętność} & \textbf{Inne zdolności} \\ \hline

 Małpa & Wspinaczka & Ręce \\ \hline
 Borsuk & Wspinaczka & Czuły węch \\ \hline
 Nietoperz & Percepcja & Latanie \\ \hline
 Niedźwiedź & Wspinaczka & Czuły węch \\ \hline
 Ptak & Percepcja & Latanie \\ \hline
 Dzik & Obrona Mocy & Czuły węch \\ \hline
 Kot & Wspinaczka lub skradanie się & Mały \\  \hline
 Wąż dusiciel & Wspinaczka & Duszenie \\  \hline
 Krokodyl & Skradanie się lub pływanie & Duszenie \\  \hline
 Deinonych & Percepcja & Szybki \\  \hline
 Delfin & Percepcja lub pływanie & Szybki \\  \hline
 Ryba & Skradanie się lub pływanie & Wodny \\ \hline
 Żaba & Skakanie lub skradanie się & Wodny \\ \hline
 Koń & Percepcja & Szybki \\ \hline
 Leopard & Wspinaczka lub skradanie się & Szybki \\ \hline
 Jaszczurka & Wspinaczka lub skradanie się & Mały \\ \hline
 Ośmiornica & Skradanie się & Wodny \\ \hline
 Rekin & Pływanie & Wodny \\ \hline
 Żółw & Obrona Mocy & Pancerz \\ \hline
 Jadowity wąż & Wspinaczka & Trucizna \\ \hline
 Wilk & Percepcja & Czuły węch \\ \hline
 
 \end{tabularx}
 \end{table*}
 
 \begin{itemize}

\item \textbf{Wodny}: Zwierzę albo oddycha pod wodą zamiast powietrzem, albo jest w stanie oddychać wodą w dodatku do powietrza.

\item \textbf{Pancerz}: Zwierzę ma twardą skorupę lub skórę, co daje mu +1 do Pancerza.

\item \textbf{Duszenie}: Zwierzę może się szybko obwinąć wokół przeciwnika po udanym ataku wręcz (zazwyczaj ugryzieniu lub ataku pazurem), ułatwiając następne ataki przeciwko temu samemu wrogowi aż do momentu, aż kontakt nie zostanie zerwany.

\item \textbf{Szybki}: To zwierzę może się poruszać na daleki zasięg w swojej turze zamiast na średni.

\item \textbf{Latanie}: Zwierzę może latać, co (w zależności od typu zwierzęcia) może oznaczać ruch na średni lub daleki zasięg w swojej turze. 

\item \textbf{Ręce}: Zwierzę ma łapy lub ręce, które są niemal tak zwinne jak te ludzi. W przeciwieństwie do większości zwierzęcych kształtów, zadania zwierzęcia które wymagają rąk nie są utrudnione (choć MG może zdecydować, że niektóre zadania, wymagające ludzkie zwinności, np: gra na flecie, są dalej utrudnione).

\item \textbf{Czuły Węch}: Zwierzę posiada silny zmysł węchu, uzyskując atut na śledzeniu i akcjach w ciemności lub podczas oślepienia. 

\item \textbf{Mały}: Zwierzę jest znacznie mniejsze od człowieka, co ułatwia jego Obronę Szybkości ale utrudnia zadania polegające na przenoszeniu ciężkich rzeczy.

\item \textbf{Trucizna}: Zwierzę jest trujące (zazwyczaj jego ugryzienie), co zadaje dodatkowy 1 punkt obrażeń. 

\end{itemize}

\textbf{Zwierzęce Zmysły}\index{Zdolności!Alfabetycznie!Zwierzęce Zmysły}\label{sec:Zwierzęce Zmysły} - Jesteś wyszkolony w słuchaniu i dostrzeganiu rzeczy. Dodatkowo, przez większość czasu, MG powinien Cię poinformować o tym, że zaraz wkroczysz w pułapkę lub zostaniesz zaatakowany z zaskoczenia, jeśli zagrożenie jest na poziomie niższym niż 5. Umożliwienie. 

\textbf{Riposta}\index{Zdolności!Alfabetycznie!Riposta}\label{sec:Riposta} (3 punkty Szybkości) - Jeśli jesteś zaangażowany w walkę wręcz, możesz wykonać bezpośredni atak wręcz przeciwko każdemu z atakujących raz na rundę. Ten atak jest utrudniony, i ciągle możesz wykonać swoją normalną akcję podczas tej rundy. Umożliwienie.

\textbf{Uprzedzenie Ataku}\index{Zdolności!Alfabetycznie!Uprzedzenie Ataku}\label{sec:Uprzedzenie Ataku} (4 punkty Intelektu) - Możesz wyczuć jak i kiedy istoty Cię atakujące wykonają swoje ataki. Rzuty na Obronę Prędkości są ułatwione na jedną minutę. Akcja.

\textbf{Przebłysk}\index{Zdolności!Alfabetycznie!Przebłysk}\label{sec:Przebłysk} (1 punkt Intelektu) - Patrzysz w przyszłość by zobaczyć, jak Twoje akcje się zakończą. Pierwsze zadanie, które wykonasz przed końcem swojej następnej rundy uzyskuje atut. Akcja.

\textbf{Automatyczny Blask}\index{Zdolności!Alfabetycznie!Automatyczny Blask}\label{sec:Automatyczny Blask} - Przedmioty z twardego światła, które tworzysz, rzucają światło, oświecając wszystko w bliskim zasięgu. Kiedy tylko zechcesz, Twoje ciało (w całości lub tylko jego część) rzuca światło, oświecając wszystko w średnim zasięgu. Umożliwienie. 

\textbf{Stosowanie Swojej Wiedzy}\index{Zdolności!Alfabetycznie!Stosowanie Swojej Wiedzy}\label{sec:Stosowanie Swojej Wiedzy} - Kiedy pomagasz innej postaci w akcji, w której nie posiadasz wyszkolenia, jesteś traktowany jako wyszkolony w niej. Akcja.

\textbf{Aportacja}\index{Zdolności!Alfabetycznie!Aportacja}\label{sec:Aportacja} (4 punkty Intelektu) - Przywołujesz do siebie fizyczny obiekt. Możesz wybrać dowolną pozycję ze standardowej listy ekwipunku, lub (nie więcej niż raz dziennie) możesz pozwolić MG na przypadkowe określenie tego przedmiotu. Jeśli przywołujesz przypadkowy obiekt, ma on szansę 10 procent na bycie zamanifestowanym Cypherem lub artefaktem, 50 procent szans na bycie zwykłym ekwipunkiem i 40 procent szans na bycie jakimś bezużytecznym śmieciem. Nie możesz zastosować tej umiejętności, by wziąć przedmiot trzymany przez inną istotę. Akcja.

\textbf{Wodny Wojownik}\index{Zdolności!Alfabetycznie!Wodny Wojownik}\label{sec:Wodny Wojownik} - Ignorujesz wszelkie kary do akcji (wliczając walkę) w środowiskach podwodnych. Umożliwienie. 

\textbf{Potrójny Wystrzał}\index{Zdolności!Alfabetycznie!Potrójny Wystrzał}\label{sec:Potrójny Wystrzał} (3 punkty Szybkości) - Jeśli broń ma zdolność wystrzału ciągłego bez przeładowywania (zazwyczaj zwana jest bronią automatyczną), możesz wystrzelić ze swojej broni w kierunku do 3 celów (muszą stać obok siebie) na raz. Wykonaj osobny rzut na atak na każdy z celów. Każdy z tych ataków jest utrudniony. Akcja.

\textbf{Magiczny Błysk}\index{Zdolności!Alfabetycznie!Magiczny Błysk}\label{sec:Magiczny Błysk} (1 punkt Intelektu) - Ulepszasz obrażenia innego zaklęcia ofensywnego dodatkową energią, tak, że zadaje 1 dodatkowy punkt obrażeń. Alternatywnie, Twój atak sięga celu w dalekim zasięgu - jest to ognisty pocisk czystej magii, zadający 4 punkty obrażeń. Umożliwienie dla ulepszenia; akcja dla ataku na daleki zasięg.  

\textbf{Artefakty z Odzysku}\index{Zdolności!Alfabetycznie!Artefakty z Odzysku}\label{sec:Artefakty z Odzysku} (6 punktów Intelektu +2 PD) - Rozwinąłeś szósty zmysł odnośnie szukania najcenniejszych rzeczy na pustkowiach. Jeśli spędzisz czas wymagany by odnieść sukces na 2 zadaniach przeszukiwania, możesz wymienić ich rezultat na szansę pozyskania artefaktu wyboru MG jeśli zakończysz test 6 poziomu Intelektu powodzeniem. Możesz skorzystać z tej zdolności najczęściej raz dziennie i nigdy dwa razy w tym samym obszarze. Akcja by rozpocząć, parę godzin, by ją zakończyć.

\textbf{Mechanik Artefaktów}\index{Zdolności!Alfabetycznie!Mechanik Artefaktów}\label{sec:Mechanik Artefaktów} - Jeśli spędzisz przynajmniej 1 dzień majsterkując z artefaktem, który posiadasz, funkcjonuje on na poziomie o 1 wyższym niż normalnie. Stosuje się to do wszystkich artefaktów w Twoim władaniu, ale tylko Ty możesz korzystać z tego bonusa. Umożliwienie.

\textbf{Jak Przepowiedziano}\index{Zdolności!Alfabetycznie!Jak Przepowiedziano}\label{sec:Jak Przepowiedziano} - osiągasz coś, co udowadnia, że jesteś Wybrańcem. Następne zadanie, które podejmiesz, jest ułatwione o 3 stopnie. Nie możesz ponownie skorzystać z tej zdolności, aż do momentu, gdy zaznasz rzutu na odzyskanie zdrowia, który trwa 1 lub 10 godzin. Akcja.

\textbf{Jak Jedna Istota}\index{Zdolności!Alfabetycznie!Jak Jedna Istota}\label{sec:Jak Jedna Istota} - Kiedy Ty i Twoja bestia (ze zdolności Zwierzęcy Kompan) jesteście w bliskim zasięgu od siebie, możecie dzielić się obrażeniami, które uzyskujecie. Dla przykładu, jeśli jedno z Was zostanie trafione bronią za 4 punkty obrażeń, podzielcie je pomiędzy siebie jak uznacie za stosowne. Tylko Pancerz i odporności pierwotnego celu ataku wchodzą w grę. Więc jeśli masz Pancerz 2 i zostajesz zaatakowany za 4 punkty obrażeń magicznym pociskiem, Twoja bestia może wziąć 2 punkty obrażeń, które normalnie Ty byś musiał znieść, ale jej Pancerz się nie liczy, podobnie jak jej odporność na magię, jeśli jakaś. Umożliwienie. 

\textbf{Umiejętności Zabójcy}\index{Zdolności!Alfabetycznie!Umiejętności Zabójcy}\label{sec:Umiejętności Zabójcy} - Jesteś wyszkolony w skradaniu się i przebieraniu się. Umożliwienie.

\textbf{Cios Skrytobójcy}\index{Zdolności!Alfabetycznie!Cios Skrytobójcy}\label{sec:Cios Skrytobójcy} (5 punktów Intelektu) - Jeśli uda Ci sie zaatakować istotę, która nie była świadoma Twojej obecności, zadajesz dodatkowe 9 punktów obrażeń. Umożliwienie. 

\textbf{Potwierdzenie Własnego Przywileju}\index{Zdolności!Alfabetycznie!Potwierdzenie Własnego Przywileju}\label{sec:Potwierdzenie Własnego Przywileju} (3 punkty Intelektu) - Zachowując się tak, jak tylko osoba uprzywilejowana może, werbalnie ustawiasz do pionu wroga, który Cię słyszy i rozumie, wskutek czego nie może on podjąć żadnej akcji, wliczając w to ataki, przez jedną rundę. Niezależnie od tego, czy odniesiesz porażkę czy sukces, następna akcja celu jest utrudniona. Akcja.

\textbf{Przejęcie Kontroli}\index{Zdolności!Alfabetycznie!Przejęcie Kontroli}\label{sec:Przejęcie Kontroli} (6+ punktów Intelektu) - Kontrolujesz akcje innej istoty, z którą wszedłeś w interakcję lub którą badałeś przez co najmniej jedną rundę. Efekt trwa 10 minut. Cel musi być na poziomie 2 lub niższym. Kiedy już przejąłeś kontrolę, cel działa tak, jak sobie tego życzysz najlepiej jak potrafi, wolno korzystając ze swojej zdolności do czynienia osądów, chyba, że poświęcisz akcję na danie mu bardzo szczegółowych instrukcji. W ddoatku do normalnych opcji Wysiłku, możesz skorzystać z Wysiłku, by zwiększyć maksymalny poziom celu. Tak więc, aby spróbować wydać rokaz celowi 5 poziomu (3 poziomy ponad normalnym limitem), musisz zastosować 3 poziomy Wysiłku. Kiedy cel się kończy, cel pamięta wszystko, co się stało i reaguje na to zgodnie ze swoją naturą i Twojej relacji z nim - przejęcie kontroli mogło zniszczyć wcześniejszą, pozytywną relację między wami. Akcja, by rozpocząć.

\textbf{Atleta}\index{Zdolności!Alfabetycznie!Atleta}\label{sec:Atleta} - Jesteś wyszkolony w noszeniu ciężarów, wspinaczce, skakaniu i niszczeniu. Umożliwienie.

\textbf{Atak i Ponowny Atak}\index{Zdolności!Alfabetycznie!Atak i Ponowny Atak}\label{sec:Atak i Ponowny Atak} - Zamiast uzyskiwać więcej obrażeń lub mniejszy/większy efekt na naturalnej 17 lub więcej, możesz dzięki tej zdolności natychmiast wykonać następny atak. Umożliwienie.

\textbf{Estetyczny Atak}\index{Zdolności!Alfabetycznie!Estetyczny Atak}\label{sec:Estetyczny Atak} - Podczas ataku, wykonujesz stylowe ruchy, zachwycające gesty, lub coś innego co zachwyca lub rozbawia innych. Jedna istota, którą wybierasz w średnim zasięgu, która może Cię dostrzec, uzyskuje atut na swoim następnym zadaniu, jeśli jest ona wykonany w ciągu rundy lub dwóch. Umożliwienie.

\textbf{Modyfikacja Cyphera}\index{Zdolności!Alfabetycznie!Modyfikacja Cyphera}\label{sec:Modyfikacja Cyphera} (2+ punkty Intelektu) - Kiedy aktywujesz cypher, dodaj +1 do jego poziomu. W dodatku do normalnych opcji korzystania z Wysiłku, możesz wybrać korzystanie z Wysiłku, by zwiększyć poziom cyphera o dodatkowe +1 (na zastosowany poziom Wysiłku). Niem ożesz zwiększyć poziomu cyphera powyżej 10. Umożliwienie.

\textbf{Autodoktor }\index{Zdolności!Alfabetycznie!Autodoktor}\label{sec:Autodoktor} - jesteś wyszkolony w leczeniu, operacjach chirurgicznych i znoszeniu bólu. MOżesz operować samego siebie, pozostając w tym czasie w pełni świadomym. Umozliwienie. 

\textbf{Świadomość}\index{Zdolności!Alfabetycznie!Świadomość}\label{sec:Świadomość} (3 punkty Intelektu) - Stajesz się nadmiernie śiaodmym swojego otoczenia w celu lepszego odnalezienia swojego celu. Na 10 minut, jesteś świadom wszystkich żywych istot w dalekim zasięgu (wliczając ich ogólnie umiejscowienie) i poprzez koncentrację (kolejna akcja) możesz spróbować poznać informacje o stanie zdrowia i poziomie mocy każdej z nich. Akcja.

\section{B}

\textbf{Babel}\index{Zdolności!Alfabetycznie!Babel}\label{sec:Babel} - Po usłyszeniu mówionego języka przez parę minut, możesz się nim wysławiać i zostać w nim zrozumianym. Jeśli będziesz kontynuować korzystanie z tego języka, by wchodzić w interakcje z native speakerami, Twoje umiejętności z nim związane ulepszają się znacząco, aż w końcu możesz zostać pomylony z native speakerem już po paru godzinach mówienia w nowym języku. Umożliwienie.

\textbf{Balansowanie}\index{Zdolności!Alfabetycznie!Balansowanie}\label{sec:Balansowanie} - Jesteś wyszkolony w balansowaniu. Umożliwienie. 

\textbf{Drużyna Desperados}\index{Zdolności!Alfabetycznie!Drużyna Desperados}\label{sec:Drużyna Desperados} - Twoja reputacja ściąga do Ciebie grupę 6 2-poziomowych Kompanów desperado (BN-ów), którzy są Tobie kompletnie oddani. Powinieneś razem z MG określić szczegóły tych kompanów. Jeśli jeden z kompanów umrze, zyskujesz nowego w jego miejsce po przynajmniej 2 tygodniach i odpowiednim procesie rekrutacyjnym. Umożliwienie. 

\textbf{Drużyna Kompanów}\index{Zdolności!Alfabetycznie!Drużyna Kompanów}\label{sec:Drużyna Kompanów}  - Otrzymujesz 4 3-poziomowych kompanów. Nie są oni ograniczeni odnośnie ich modyfikacji. Umożliwienie.

\textbf{Ogłuszenie}\index{Zdolności!Alfabetycznie!Ogłuszenie}\label{sec:Ogłuszenie} (1 punkt Mocy) - Wykonujesz powtarzalny atak wręcz. Twój atak zadaje o 1 punkt obrażeń mniej niż zwykle, ale oszałamia cel na jedną rundę, podczas której wszystkie jego zadania są utrudnione. Akcja.

\textbf{Podstawowy Kompan}\index{Zdolności!Alfabetycznie!Podstawowy Kompan}\label{sec:Podstawowy Kompan} - Uzyskujesz kompana 2-poziomu. Jedną z jego modyfikacji musi być perswazja. Możesz wziąć tę zdolność wiele razy, za każdym razem otrzymując innego kompana 2-poziomu. Umożliwienie. (MG może określić, że musisz poszukać odpowiedniego kompana w świecie gry, zanim go zdobędziesz).

\textbf{Zarządzanie Bitwą}\index{Zdolności!Alfabetycznie!Zarządzanie Bitwą}\label{sec:Zarządzanie Bitwą} (4 punkty Intelektu) - Tak długo, jak używasz swojej akcji w każdej rundzie na wydawanie rozkazów lub dawanie porad, rzuty na atak i obronę podejmowane przez Twoich sprzymierzeńców w średnim zasięgu są ułatwione. Akcja.

\textbf{Taktyk Pola Walki}\index{Zdolności!Alfabetycznie!Taktyk Pola Walki}\label{sec:Taktyk Pola Walki} (2+ punkty Intelektu) - Oceniasz swoje otoczenia, poznając dowolne fakty, które MG uzna za ważne odnośnie atakowania, bronienia się, manewrowania i radzenia sobie z zagrożeniami środowiskowymi w średnim zasięgu. Dla przykładu, możesz zauważyć, że po wspięciu się na stos śmieci uzyskasz przewagę w walce wręcz, że w kącie najłatwiej się bronić, że gdzieś jest mniej śliska pokrywa lodowa na zamarzniętym jeziorze, lub że istnieje miejsce, gdzie trujący gaz jest rzadszy niż indziej. Jeśli Ty (lub ktoś, kogo o tym poinformujesz) przemieści sie w tamto miejsce, Ty (lub on/a) uzyska atut na zadaniach związanych z optymalną pozycją (takie jak ataki z wyższego miejsca, obrona Szybkości w łatwym do obrony rogu, rzuty na utrzymanie równowagi na lodzie lub Obrona Mocy przeciwko trującej chmurze). Zamiast uzyskać korzystną lokację, możesz się dowiedzieć o niekorzystnej lokacji, którą możesz wykorzystać przeciwko swoim wrogom, np: zapędzając ich w kozi róg który utrudnia ich ataki wręcz lub słaby punkt na zamarzniętym jeziorze, gdzie lód się pod nimi załamie. Możesz wykorzystać Wysiłek by dowiedzieć się o jednej dodatkowej dobrej lub złej lokacji w zasięgu (jedna lokacja na poziom Wysiłku), zwiększyć zasięg zdolności (dodatkowy średni zasięg na poziom Wysiłku) lub skorzystać z obydwu tych opcji. Umożliwienie. 

\textbf{Zew Dziczy}\index{Zdolności!Alfabetycznie!Zew Dziczy}\label{sec:Zew Dziczy} (5 punktów Intelektu) - Przywołujesz hordę małych zwierząt lub jedno zwierzę 4 poziom, by Tobie chwilowo pomagało. Te istoty robią ,jak rozkażesz tak długo, jak skupiasz na nich swoją uwagę, ale musisz wykorzystać swoje akcje, by nimi kierować. Istoty te naturalnie występują w tym terenie i przybywają o własnych siłach, więc jeśli jesteś w niedostępnym miejscu, to nie przybędą. Akcja.

\textbf{Zwierzęcy Kompan}\index{Zdolności!Alfabetycznie!Zwierzęcy Kompan}\label{sec:Zwierzęcy Kompan} - Istota 2 poziomu rozmiaru Twojego lub mniejszego towarzyszy Ci i słucha Twoich instrukcji. Powinieneś wypracować z MG szczegóły tej istoty, i najpewniej będziesz za nią wykonywał rzuty w czasie walki lub gdy wykonuje ona jakieś akcje. Zwierzęcy kompan działa w Twojej turze. Jako istota 2 poziomu, posiada ona stopień trudności 6 i 6 punktów życia, a zadaje 2 obrażenia. Jej zdolności ruchu bazują na jej typie istoty (ptak, istota pływająca itp.). Jeśli Twój zwierzęcy Kompan umrze, możesz przeszukać dzicz przez 1k6 dni, by odnaleźć nowego. Umożliwienie. (Poziom istoty określa jej stopień trudności, punkty życia i obrażenia, chyba, że zaznaczono inaczej. Tak więc zwierzęcy kompan 2 poziomu ma stopień trudności 6, 6 punktów życia i zadaje 2 obrażenia. Zwierzęcy kompan 4 poziomu ma stopień trudności 12, 12 punktów życia i zadaje 4 punkty obrażeń. I tak dalej.).

\textbf{Oczy Bestii}\index{Zdolności!Alfabetycznie!Oczy Bestii}\label{sec:Oczy Bestii} (3 punkty Intelektu) - Poprzez podlączenie się do istoty z Twojej zdolności Zwierzęcy Kompan, możesz postrzegać świat jego zmysłami, jeśli znajduje się w zasięgu 1.5 km od Ciebie. Ten efekt trwa 10 minut. Akcja, by ustanowić.

\textbf{Likantropia}\index{Zdolności!Alfabetycznie!Likantropia}\label{sec:Likantropia} - w ciągu 5 sąsiadujących z sobą nocy w każdym miesiącu, zamieniasz się w potworna bestię (do 1 godziny każdej nocy). W tej nowej formie, uzyskujesz +8 do Puli Mocy, +1 do Skupienia w Mocy, +2 do Puli Szybkości i +1 do Skupienia w Szybkości. Kiedy jesteś w zwierzęcej formie, nie możesz wydawać punktów Intelektu na cokolwiek innego niż próba powrotu do normalnej formy zanim minie godzina (zadanie o trudności 2). Dodatkowo, atakujesz wszystkie żywe istoty w średnim zasięgu od Ciebie. Po tym, jak wracasz do normalnej formy, otrzymujesz karę -1 do wszystkich rzutów na godzinę. Jeśli nie zabiłeś i zjadłęś przynajmniej jednej istoty w zwierzęcej formie, ta kara wzrasta do -2 i dotyczy wszystkich Twoich rzutów przez 24 godziny. Akcja, by zmienić się z powrotem.

\textbf{Niezauważalny}\index{Zdolności!Alfabetycznie!Niezauważalny}\label{sec:Niezauważalny}  - Twój obniżony wzrost sprawia, że trudno Cię znaleźć. Kiedy Zmniejszenie się jest aktywne, wszystkie Twoje próby skradania się są ułatwione. Umożliwienie.

\textbf{Wiedza z Bestiariusza}\index{Zdolności!Alfabetycznie!Wiedza z Bestiariusza}\label{sec:Wiedza z Bestiariusza} - jesteś wyszkolony w wiedzy o niehumanoidalnych istotach, które jedzą ludzkie mięso - rozpoznawaniu ich, poznawaniu ic hsłabości i poznawaniu ich zwyczajów i zachowań. Umożliwienie. 

\textbf{Zdrada}\index{Zdolności!Alfabetycznie!Zdrada}\label{sec:Zdrada} - Za każdym razem, gdy przekonasz przeciwnika, że nie jesteś zagrożeniem i następnie zaatakujesz z nienacka (bez prowokacji), ten atak zadaje 4 dodatkowy punkty obrażeń. Umożliwienie. 

\textbf{Lepsze Życie Dzięki Chemii}\index{Zdolności!Alfabetycznie!Lepsze Życie Dzięki Chemii}\label{sec:Lepsze Życie Dzięki Chemii} (4 punkty Intelektu) - Stworzyłeś koktajle chemiczne dostosowane do Twojej własnej biochemii. W zależności od tego, który z nich zażyjesz, czyni Cię to mądrzejszym, szybszym lub wytrzymalszym, ale kiedy ich działanie się kończy, masz przekichane, więc korzystasz z nich tylko wtedy, kiedy sytuacja tego wymaga. Uzyskujesz 2 punkty w Skupieniu w Mocy, Szybkości lub Intelekcie na jedna minutę, po której nie możesz ponownie skorzystać z tej zdolności przez godzinę. Podczas tej godziny, za każdym razem, gdy wydasz punkt z Puli, zwiększ kosz tej akcji o 1. Akcja.

\textbf{Lepszy Atak z Zaskoczenia}\index{Zdolności!Alfabetycznie!Lepszy Atak z Zaskoczenia}\label{sec:Lepszy Atak z Zaskoczenia} - Jeśli atakujesz z ukrycia lub przed akcją przeciwnika, otrzymujesz atut na swoim ataku (jeśli posiadasz także zdolność Atak z Zaskoczenia, dodajesz obydwa atuty). Po udanym ataku, zadajesz dodatkowe 2 punkty obrażeń (w sumie 4, jesli posiadasz Atak z Zaskoczenia. Umożliwienie.

\textbf{Większy}\index{Zdolności!Alfabetycznie!Większy}\label{sec:Większy} - Kiedy korzystasz ze Wzrostu, możesz osiągnąć rozmiar 4 metrów i dodajesz dodatkowe 3 chwilowe punkty do swojej Puli Mocy. Umożliwienie.

\textbf{Większy Zwierzęcy Kształt}\index{Zdolności!Alfabetycznie!Większy Zwierzęcy Kształt}\label{sec:Większy Zwierzęcy Kształt} - kiedy korzystasz ze Zwierzęcego Kształtu, Twoja zwierzęca forma rozrasta się do podwójnego normalnego rozmiaru. Będąć tak wielką, zwierzęca forma dodaje Ci następujące bonusy: +1 do Pancerza, +5 do Puli Mocy, jesteś także wyszkolony w używaniu naturalnych ataków swojej zwierzęcej formy jako ciężkich broni (jeśli nie byłeś wyszkolony). Jednakże, Twoja Obrona Szybkości jest utrudniona. Kiedy jestes większy, dostajesz także atut na zadaniach które są prostsze dla dużych istot, takich jak wspinaczka, zastraszanie, przepływanie rzek itp. Umożliwienie. 

\textbf{Większy Likantrop}\index{Zdolności!Alfabetycznie!Większy Likantrop}\label{sec:Większy Likantrop} - Kiedy przebywasz w formie likantropa, Twoja forma fizyczna jest większa niż wcześniej, sięgając 4 metrów. Będąc tak wielkim, otrzymujesz następujące bonusy: +1 do Pancerza, +5 do Puli Mocy, i jesteś wyszkolony w korzystaniu ze swoich pięści jak z ciężkich broni (jeśli jeszcze nie jesteś). Jednakże, Twoja Obrona Szybkości jest utrudniona. Kiedy jesteś tak wielki, uzyskujesz atut na zadaniach które są prostsze dla dużej istoty, takich jak wspinaczka, zastraszanie, przepływanie rzek itp. Umożliwienie. 

\textbf{Detonacja Biomorficzna}\index{Zdolności!Alfabetycznie!Detonacja Biomorficzna}\label{sec:Detonacja Biomorficzna} (7+ punktów Mocy) - Emitujesz impuls biomorficznej energii w średnim zasięgu, ale tak manipulujesz jego frekwencją, by przeszkadzał życiu w bliskim zasięgu. Wszyscy w promieniu detonacji otrzymują 5 punktów obrażeń, które ignorują Pancerz (chyba, że Pancerz wynika z pola siłowego). Jeśli zastosujesz dodatkowy Wysiłek by zwiększyć obrażenia, zadajesz 2 dodatkowe punkty obrażeń na poziom Wysiłku (zamiast normalnych 3 punktów); cele w zasięgu otrzymują 1 punkt obrażeń nawet jeśli nie powiedzie Ci się rzut na atak. Akcja. 

\textbf{Biomorficzne Leczenie}\index{Zdolności!Alfabetycznie!Detonacja Biomorficzna}\label{sec:Detonacja Biomorficzna} (4+ punktów Mocy) - Świadomie wysyłasz puls swojego pola biomorficznego (dziwna energia generowana przez ciało) i skupiasz go na żywej istocie w średnim zasięgu. Cel uzyskuje darmowy i natychmiastowy rzut na odzyskanie zdrowia. Nie możesz ponownie użyć tej zdolności na danej istocie aż do momentu, gdy zakończysz swój 10-godzinny odpoczynek. Akcja. 

\textbf{Spryciula}\index{Zdolności!Alfabetycznie!Spryciula}\label{sec:Spryciula} - Jesteś wyszkolony w jednej z następujących umiejętności: oszustwie, skradaniu się lub przebieraniu się. Umożliwienie. 

\textbf{Zlanie się z Tłem}\index{Zdolności!Alfabetycznie!Zlanie się z Tłem}\label{sec:Zlanie się z Tłem} (4 punkty Intelektu) - Kiedy zlewasz się z tłem, istoty dalej Cię widzą, ale nie przywiązują do Twojej obecności wagi przez około minutę. Kiedy się zlewasz z tłem, jesteś wyspecjalizowany w skradaniu się i Obronie Szybkości. Ten efekt kończy się, gdy zrobisz coś, by ujawnić swoją obecność lub pozycję - zaatakujesz, skorzystasz ze zdolności, przesuniesz wielki obiekt itp. Jeśli to się wydarzy, możesz odzyskać brakujący czas efektu poprzez poświęcenie akcji na skupieniu się, by wyglądać niewinnie i na swoim miejscu. Akcja bo ryzpocząć lub odzyskać.

\textbf{Błogosławieństwo Bóstw}\index{Zdolności!Alfabetycznie!Błogosławieństwo Bóstw}\label{sec:Błogosławieństwo Bóstw} - Jako sługa bóstw, masz różne błogosławieństwa, których Ci udzieliły. Błogosławieństwo zależy od danego bóstwa i jego specjalności. Wybierz dwie ze zdolności wymienionych poniżej.

\begin{itemize}
\item \textbf{Autorytet/Prawo/Pokój} (3 punkty Intelektu) - Powstrzymujesz wroga, który Cię słyszy i rozumie, przed zaatakowaniem kogoś lub czegoś przez jedną rundę. Akcja.
\item \textbf{Dobrotliwość/Moralność/Duch} (2+ punktów Intelektu) - Jeden demon, duch lub podobna istota 1-poziomu w średnim zasięgu zostaje zniszczona lub wygnana. W dodatku do normalnych opcji Wysiłku, możesz skorzystać z niego, by zwiększyć maksymalny poziom celu. Tak więc, aby zniszczyć lub wygnać cel 5-poziomu (4 poziomy więcej niż normalnie) musisz zastosować 4 poziomy Wysiłku. Akcja.
\item \textbf{Śmierć/Ciemność} (2 punkty Intelektu) - Cel, który wybierasz w średnim zasięgu, otrzymuje 3 punkty obrażeń. Akcja.
\item \textbf{Pragnienie/Miłość/Zdrowie} (3 punkty Intelektu) - Poprzez dotyk, możesz przywócić 1d6 punktów do jednej Puli Statystyk dowolnej istocie, wliczając siebie. Ta zdolność to zadanie Intelektu 2 poziomu. Za każdym razem, gdy chcesz uzdrowić tę samą istotę, zadanie jest utrudnione poziom więcej. Trudność wraca do 2 po tym, jak istota zakończy 10-godzinny odpoczynek. Akcja.
\item \textbf{Kamień/Ziemia} - Jesteś wyszkolony we wspinaczce, kamieniarstwie i eksploracji jaskiń. Umożliwienie.
\item \textbf{Wiedza/Mądrość} (3 punkty Intelektu) - Wybierz do 3 istot (możesz się wśród nich znaleźć). Przez minutę, konkretny typ zadania (ale nie rzut na atak lub obronę) jest ułatwiony dla tych istot, ale tylko, jeśli zostają w bliskiej odległości od Ciebie. Akcja.
\item \textbf{Natura/Zwierzęta/Rośliny} - Jesteś wyszkolony w botanice i obchodzeniu się z dzikimi zwierzętami. Umożliwienie.
\item \textbf{Ochrona/Cisza} (3 punkty Intelektu) - Tworzysz cichy bąbel ochronny wokół siebie o promieniu bliskiego zasięgu na jedną minutę. Bąbel porusza się wraz z Tobą. Wszystkie rzuty obronne dla Ciebie i istot, które wybierzesz, gdy wykonywane są wewnątrz tego bąbla, są ułatwione, i żaden hałas, niezależnie od jego natury, nie wybrzmiewa głośniej niż normalna rozmowa. Akcja, by rozpocząć. 
\item \textbf{Powietrz/Niebo} (2 punkty intelektu) - Istota, której dotkniesz, jest odporna na toksyny i choroby przenoszone drogą powietrzną na 10 minut. Akcja.
\item \textbf{Słońce/Światło/Ogień} (2 punkty Intelektu) - Sprawiasz, że jedna istota lub obiekt w średnim zasięgu się zapala, co zadaje jej 1 punkt obrażeń środowiskowych w każdej rundzie, do momentu, gdy ogień zostanie wygaszony (wymaga to akcji). Akcja.
\item \textbf{Oszustwo/Chciwość/Handel} - jesteś wyszkolony w wykrywaniu oszustw innych istot. Umożliwienie. 
\item \textbf{Wojna} (1 punkt Intelektu) - Cel, który wybierasz w średnim zasięgu (możesz to być Ty) zadaje 2 obrażenia więcej swoim następnym udanym atakiem bronią. Akcja.
\item \textbf{Woda/Morze} (2 punkty Intelektu) - Cel, który wybierasz może oddychać pod wodą przez 10 minut. Akcja.
\end{itemize}

\textbf{Oślepienie Maszyny}\index{Zdolności!Alfabetycznie!Oślepienie Maszyny}\label{sec:Oślepienie Maszyny} (6 punktów Szybkości) - Deaktywujesz sensory maszyny, czyniąc ją ślepą do czasu, aż ktoś ją naprawi. Musisz albo dotknąć maszyny, albo uderzyć w nią atakiem dystansowym (nie zadaje on obrażeń). Akcja.

\textbf{Oślepiający Atak}\index{Zdolności!Alfabetycznie!Oślepiający Atak}\label{sec:Oślepiający Atak} (3 punkty Szybkości) - Jeśli masz ze sobą źródło śWiatła, możesz go urzyć, by wykonać atakl wręcz przeciwko celowi. Jeśli atak jest udany, nie zadaje on obrażeń, ale cel jest oślepiony na jedną minutę. Akcja.

\textbf{W mgnieniu Oka}\index{Zdolności!Alfabetycznie!W Mgnieniu Oka}\label{sec:W Mgnieniu Oka} (4 punkty Szybkości) - Poruszasz się do 300 metrów w jednej undzie. Akcja.

\textbf{Blok}\index{Zdolności!Alfabetycznie!Blok}\label{sec:Blok} (3 punkty Szybkości) - Automatycznie blokujesz następny atak wręcz wykonany przeciwko Tobie w następnej minucie. Akcja by rozpocząć. 

\textbf{Chronienie Sprzymierzeńca}\index{Zdolności!Alfabetycznie!Chronienie Sprzymierzeńca}\label{sec:Chronienie Sprzymierzeńca} - Jeśli korzystasz z lekkiej lub średniej broni, możesz blokować ataki wykonywane przeciwko sprzymierzeńcowi blisko Ciebie. Wybierz jedną istotę w bliskim zasięgu. Zapewniasz mu atut do Obrony Szybkości. Nie możesz skorzystać z Szybkiego Bloku kiedy Chronisz Sprzymierzeńca. Umożliwienie.

\textbf{Gorączka Krwi}\index{Zdolności!Alfabetycznie!Gorączka Krwi}\label{sec:Gorączka Krwi} - Kiedy nie masz punktów w jednej lub dwóch Pulach, uzyskujesz atut na rzutach na atak lub obronę (Twój wybór). Umożliwienie.

\textbf{Zew Krwi}\index{Zdolności!Alfabetycznie!Zew Krwi}\label{sec:Zew Krwi} (3 punkty Mocy) - Jeśli pokonasz w walce wroga, możesz się przemieścić o średni dystans, ale tylko, jeśli się przemieszczasz w kierunku innego wroga. Nie musisz wydawać punktów aż do momentu, w którym wiesz, że pierwszy wróg poległ. Umożliwienie.

\textbf{Rozmazana Prędkość}\index{Zdolności!Alfabetycznie!Rozmazana Prędkość}\label{sec:Rozmazana Prędkość} (7 punktów Mocy) - Poruszasz się tak szybko, że do Twojej następnej tury, jesteś rozmazany. Kiedy jesteś rozmazany, jeśli wykorzystasz Wysiłek na ataku wręcz lub Obronie Szybkości, uzyskujesz darmowy poziom Wysiłku na tym zadaniu. Możesz się przemieścić na średni dystans jako część innej akcji lub na daleki dystans, jeśli poświęcisz na to całą akcję. Umożliwienie. 

\textbf{Zmiana Ciała}\index{Zdolności!Alfabetycznie!Zmiana Ciała}\label{sec:Zmiana Ciała} (3+ punkty Intelektu) - Zmieniasz swoje ciało i twarz oraz kolory na jedna godzinę, ukrywając swoją tożsamość lub poszywając się pod kogoś. Jeśli zastosujesz poziom Wysiłku, możesz udawać konkretną osobę dostatecznie dobrze, by oszukać kogoś, kto zna ją dobrze lub przebadał ją z bliska (wliczając odciski palców i porównanie głosu, ale nie wzór siatkówki lub DNA). Masz atut na wszelkich zadaniach związanych z przebieraniem się (w dodatku do atutu z Morficznej Twarzy). Musisz zastosować osobny poziom Wysiłku, jeśli chcesz udawać inny gatunek (np: gdy człowiek chce udawać humanoidalnego kosmitę). Akcja.

\textbf{Jeździec Błyskawicy}\index{Zdolności!Alfabetycznie!Jeździec Błyskawicy}\label{sec:Jeździec Błyskawicy} (4 punkty Intelektu) - Możesz przemieścić się na daleki dystans z jednego miejsca na drugie prawie natychmiastowo, przeniesiony przez błyskawicę. Musisz być w stanie dostrzec nową lokację, i nie może być między nimi żadnych przeszkadzających barier. Akcja.

\textbf{Promienie Mocy}\index{Zdolności!Alfabetycznie!Promienie Mocy}\label{sec:Promienie Mocy} (5+ punktów Intelektu) - Wystrzeliwujesz wachlarz błyskawic na średni zasięg w pióropuszu, który jest szeroki na 15 metrów na swoim końcu. Ten atak zadaje 4 punkty obrażeń. Jeśli zastosujesz Wysiłek by zwiększyć obrażenia zamiast ułatwić atak, zadajesz 2 dodatkowe punkty obrażeń na poziom Wysiłku (zamiast zwyczajowych 3); cele w zasięgu otrzymują 1 punkt obrażeń nawet jeśli nie powiedzie Ci się rzut na atak. Akcja.

\textbf{Urealnienie Iluzji}\index{Zdolności!Alfabetycznie!Urealnienie Iluzji}\label{sec:Urealnienie Iluzji} (2+ punkty Intelektu) - Dajesz jednej ze swoich wizualnych iluzji ograniczoną fizyczną realność, którą można powąchać, posmakować, usłyszeć i poczuć. Ten efekt jest powiązany z daną iluzją i zachowuje się stosownie do jej natury. Dla przykładu, może on sprawić, że iluzja cegły będzie odczuwalna jako cegła, iluzja osoby będzie pachnieć jak perfumy i być w stanie otwierać drzwi, a iluzja kominka będzie ciepła w dotyku. 

Fizyczna rzeczywistość zapewniona Twojej iluzji jest na poziomie 1 z 3 punktami zdrowia. Jeśli z iluzji się korzysta, by zaatakować, zadaje ona tylko 1 punkt obrażeń (mogą to być normalne obrażenia jak ciosy pięści i kopnięcia, lub obrażenia środowiskowe jak spadające cegły lub ognie kominka). Możesz zwiększyć poziom stworzonego efektu, poprzez dodanie poziomów Wysiłku - każdy poziom Wysiłku zwiększa realność iluzji o 1 poziom.

Możesz aktywować tę zdolność jako część akcji stworzenia iluzji (korzystając z dowolnej zdolności tworzenia iluzji, jaką dysponujesz, np: Mniejszej Iluzji) lub możesz wykorzystać osobną akcję zastosowaną względem jednej z Twoich iluzji, które już istnieją. Efekt kończy się, gdy iluzja zostaje zniszczona, pozwalasz jej na ustąpienie, punkty życia iluzji są zredukowane do 0, lub gdy upłynie 10 minut. Umożliwienie.

\textbf{Ulepsz Materialny Cypher}\index{Zdolności!Alfabetycznie!Ulepsz Materialny Cypher}\label{sec:Ulepsz Materialny Cypher} (2 punkty Intelektu) - Zamanifestowany Cypher który aktywujesz w swojej następnej akcji, działa jakby miał 2 poziomy więcej. Akcja.

\textbf{Ulepsz Funkcjonowanie Materialnego Cyphera}\index{Zdolności!Alfabetycznie!Ulepsz Funkcjonowanie Materialnego Cyphera}\label{sec:Ulepsz Funkcjonowanie Materialnego Cyphera} (4 punkty Intelektu) - Dodaj 3 do działającego poziomu zamanifestowanego cyphera, który aktywujesz w swojej następnej akcji, lub zmień jeden z aspektów jego parametrów (zasięg, czas trwania, obszar efektu itp.). Parametry możesz maksymalnie podwoić, a minimalnie obniżyć do 1/10. Akcja. 

\textbf{Skacząca Tarcza}\index{Zdolności!Alfabetycznie!Skacząca Tarcza}\label{sec:Skacząca Tarcza} - Kiedy korzystasz z Rzutu Tarczą Siłową, zamiast zanikać po jednym ataku (udanym lub nie), zaatakuje ona do dwóch dodatkowych celów w średnim zasięgu. Wysiłek lub inne modyfikatory zastosowane do pierwszego ataku stosują się również do innnych celów. Niezależnie od tego, czy trafisz wszystkie, trochę, czy zero celów, tarcza zanika, a potem reformuje się w Twoich dłoniach. (Jeśli wziąłeś Skaczącą Tarczę, a wcześniej wziąłeś Rzut Tarczą Siłową, masz opcję zamiany tamtej zdolności na Leczący Puls). Umożliwienie. 

\textbf{Związana Magiczna Istota}\index{Zdolności!Alfabetycznie!Związana Magiczna Istota}\label{sec:Związana Magiczna Istota} - Posiadasz 3-poziomowego magicznego sprzymierzeńca przywiązanego do fizycznego obiektu (może pomniejszy dżin przywiązany do lampy, mniejszy demon przywiązany do monety, albo duch przywiązany do lusterka). Magiczny sprzymierzeniec jeszcze nie posiada pełnej mocy, którą by posiadł, gdyby dojrzał. Normalnie, sprzymierzeniec pozostaje uśpiony w obiekcie, do którego jest przywiązany. Kiedy skorzystasz z akcji, by go zamanifestować, pojawia się obok Ciebie jako istota, z którą możesz porozmawiać. Istota posiada swoją własną osobowość określoną przez MG i jest o poziom wyższa w danej dziedzinie wiedzy (takiej jak np: lokalna historia). MG określa, czy magiczny sprzymierzeniec posiada jakieś długoterminowe cele.

Za każdym razem gdy magiczny sprzymierzeniec staje się fizycznie zamanifestowany, pozostaje takim na okres do jednej godziny. W tym czasie, towarzyszy Tobie i wykonuje Twoje instrukcje. Magiczny sprzymierzeniec musi pozostać w bliskiej odległości od Ciebie - jeśli się oddali bardziej, zostaje wrzucony do swojego obiektu na koniec Twojej następnej tury i nie może wrócić aż do końca Twojego następnego 10-godzinnego odpoczynku. Nie atakuje on istot, ale może skorzystać ze swojej akcji, aby dać Ci atut na każdym jednym ataku, który wykonujesz w swojej własnej turze. Poza tym, może on dokonywać akcji na własną rękę (choć to raczej Ty będziesz rzucać kością).

Jeśli atak zredukuje punkty życia istoty do 0, znika ona. Reformuje się ona w przywiązanym obiekcie w ciągu 1d6 +2 dni. 

Jeśli utracisz przywiązany obiekt, wiesz w którym kierunku musisz iść, by go znaleźć. Akcja by zamanifestować magiczną istotę. 

\textbf{Pranie Mózgu}\index{Zdolności!Alfabetycznie!Pranie Mózgu}\label{sec:Pranie Mózgu} (6+ punktów Intelektu) - Korzystasz ze sztuczek, dobrze przemyślanych kłamstw i chemikaliów wpływających na umysł (lub innych środków, takich jak magia lub hiper-technologia) by nakłonić chwilowo innych do zrobienie tego, czego chcesz. Kontrolujesz akcję innej istoty, której dotykasz. Ten efekt trwa przez minutę. Cel musi być na 3 poziomie lub niższym. Możesz pozwolić mu na swobodne działanie lub przejąć kontrolę tak długo, jak możesz ją dostrzec. W dodtku do zwykłych opcji korzystania z Wysiłku, możesz wybrać zwiększenia maksymalnego poziomu celu lub zwiększenie czasu trwania efektu na jedną minutę. Tak wię,c aby kontrolować umysł celu 6 poziomu (trzy poziomy powyżej normalny limit) lub by kontrolować cel na 4 minuty (trzy minuty powyżej normalnego czasu trwania), musisz zastosować 3 poziomy Wysiłku. Kiedy czas trwania się kończy, ta istota nie pamięta bycia kontrolowaną ani niczego, co zrobiła, kiedy była pod wpływem Twojej osoby. Akcja, by rozpocząć.

\textbf{Przełamanie Linii}\index{Zdolności!Alfabetycznie!Przełamanie Linii}\label{sec:Przełamanie Linii} - Łatwo dostrzegasz dyscyplinę grupy i hierarchie, także pośród Twoich wrogów. Jeśli grupa wrogów zyskuje dowolny rodzaj korzyści z tytułu wspólnej pracy, możesz spróbować im przeszkodzić w odniesieniu tej korzyści, poprzez wskazanie słabego punktu w formacji wrogów lub ataku grupowym. Ten efekt trwa do minuty lub dopóki dotknięci nim wrogowie nie spędzą rundy na przegrupowaniu się, by odzyskać normalną korzyść. Akcja, by zainicjować. 

\textbf{Ruch i Multiatak}\index{Zdolności!Alfabetycznie!Ruch i Multiatak}\label{sec:Ruch i Multiatak} (6 punktów Szybkości) - Poruszasz się do średniego dystansu i atakujesz do 4 różnych wrogów w jednej akcji tak długo, jak są na Twojej ścieżce. Dowolny modyfikator który stosujesz do jednego ataku stosujesz do wszystkich ataków, które wykonujesz. Jeśli masz inną specjalną zdolność, która pozwala Ci na poruszanie się i wykonanie akcji, kiedy korzystasz z Ruchu i Multiataku, uzyskujesz atut na atakowaniu tych wrogów. Akcja. 

\textbf{Przełamanie Umysłów}\index{Zdolności!Alfabetycznie!Przełamanie Umysłów}\label{sec:Przełamanie Umysłów} (7+ punktów Intelektu) - Korzystając ze swoich sprytnych słów i wiedzy o innych, i mając parę rund konwersacji by pozyskać parę konkretnych informacji o kontekście odnośnie Twojego celu, możesz wypowiedzieć zdanie zaprojektowane tak, by zraniło psychikę Twojego rozmówcy. Jeśli cel Cię słyszy i rozumie, otrzymuje on 6 punktów obrażeń Intelektu (ignorujących Pancerz) i zapomina ostatni dzień swojego życia, co może sprawić, że zapomni Ciebie i to, jak się znajduje w danym miejscu. W dodatku do zwykłych opcji korzystania z Wysiłku, możesz skorzystać z niego, by przełamać umysł jednego dodatkowego celu, który Cię słyszy i rozumie. Akcja by rozpocząć, akcja by zakończyć.

\textbf{Niszczyciel}\index{Zdolności!Alfabetycznie!Niszczyciel}\label{sec:Niszczyciel} - Jesteś wyszkolony w zadaniach polegających na uszkadzaniu obiektów. Akcja uszkodzenia obiektu to akcja Mocy, i przy sukcesie, obiekt przesuwa się o jeden krok w dół na liczniku obrażeń obiektu. Jeśli test Mocy przekroczy trudność o dwa kroki, zamiast tego obiekt przesuwa się w dół o dwa kroki w dół na liczniku obrażeń obiektu. Jeśli test Mocy przekroczy trudność o 4 kroki, obiekt przesuwa się na dół o 3 kroki na liczniku obrażeń obiektu i zostaje natychmiastowo zniszczony. Lekkie materiały redukują efektywny poziom obiektu, kiedy twarde materiały jak drewno lub kamień dodaję 1 do efektywnego poziomu lub (lub 2 dla bardzo twardych przedmiotów stworzonych z metalu). Umożliwienie.

\textbf{Brutalne Uderzenie}\index{Zdolności!Alfabetycznie!Brutalne Uderzenie}\label{sec:Brutalne Uderzenie} (4 punkty Mocy) - Zadajesz o 4 punkty obrażeń więcej wszystkimi atakami wręcz do końca swojej następnej rundy. Umożliwienie.

\textbf{Koleżka}\index{Zdolności!Alfabetycznie!Koleżka}\label{sec:Koleżka} (3 punkty Intelektu) - Wybierz jedną z postaci stojących obok Ciebie. Ta postać zostaje Twoim koleżką na 10 minut. Jesteś wyszkolony we wszelkich zadaniach polegających na odnalezieniu, leczeniu, wchodzeniu z interakcję i chronieniu Twojego koleżki. Także, kiedy stoisz obok niego, obydwoje macie atut na Ochronie Szybkości. Możesz mieć tylko jednego koleżkę w danym czasie. Akcja, by rozpocząć. 

\textbf{Wbudowane Bronie}\index{Zdolności!Alfabetycznie!Wbudowane Bronie}\label{sec:Wbudowane Bronie} - Biomechaniczne implanty, magiczny kryształ wbudowany w czoło, lub coś równie dziwnego zapewnia Cię we wbudowaną broń. Pozwala Ci to na wystrzelenie promienia energii na długi zasięg, który zadaje 5 punktów obrażeń. Ta zdolność nic Cię nie kosztuje. Akcja.

\textbf{Palące Światło}\index{Zdolności!Alfabetycznie!Palące Światło}\label{sec:Palące Światło} (3 punkty Intelektu) - Wysyłasz promień światła na daną istotę w długim zasięgu, a następne zwężasz go, aż będzie palił, zadając 5 punktów obrażeń. Akcja. 

\textbf{Ucieczka}\index{Zdolności!Alfabetycznie!Ucieczka}\label{sec:Ucieczka} (5 punktów Szybkości) - Możesz wykonać dwie osobne akcje w tej turze, tak długo jak jedna z nich to ucieczka od wroga lub ukrycie się. Umożliwienie. 

\textbf{Przeniknięcie Przez Barierę}\index{Zdolności!Alfabetycznie!Przeniknięcie Przez Barierę}\label{sec:Przeniknięcie Przez Barierę} (6+ punktów Intelektu) - Przechodzisz przez drzwi, pole siłowe lub inną barierę, która ma maksymalną grubość 1 metra. W zależności od bariery, może to oznaczać znalezienie słabego punktu, który wykorzystujesz, naciśnięcie odpowiednich guzików czystym szczęściem, po prostu użycie siły, lub nawet dziwniejsze wyjaśnienia, jak dotknięcie cieńkiej warstewki między wymiarami lub niespodziewana interakcja z Twoim ekwipunkiem. Trudność zadania to poziom bariery. Ta zdolność pozwala Tobie na przeniknięcie, nikomu innemu, a przejście zamyka się na końcu Twojej tury (co może oznacząc, że jesteś uwięziony po drugiej stronie). Masz atut na każdej próbie następnego przeniknięcia przez już raz przenikniętą barierę. W dodatku do zwykłych opcji Wysiłku, możesz skorzystać z niego, by zwiększyć maksymalną grubość bariery, na każdy poziom zwiększając ją o dodatkowy metr. Akcja.

\section{C}

\textbf{Wezwanie Ducha}\index{Zdolności!Alfabetycznie!Wezwanie Ducha}\label{sec:Wezwanie Ducha} (6 punktów Intelektu) - Pod Twoim dotykiem, istota martwa nie dłużej niż 7 dni pojawia się jako (najwyraźniej fizyczny) duch, którego poziom jest taki sam, jak za życia. Przywołany duch istnieje maksymalnie przez dzień (lub mniej, jeśli osiąga coś ważnego dla niego w tym czasie), po którym znika i nie może pojawić się ponownie. 

Wezwany duch pamięta wszystko ,co wiedział za życia, i posiada większość swoich starych zdolności (ale niekoniecznie swój ekwipunek). Dodatkowo, uzyskuje on zdolność zostanie niematerialnym jako akcję (do minuty na raz). Wezwany Duch nie jest względem Ciebie w żaden sposób zobowiązany i nie musi zostać blisko Ciebie, by pozostać zamanifestowany. Akcja by rozpocząć. 

\textbf{Przysługa}\index{Zdolności!Alfabetycznie!Przysługa}\label{sec:Przysługa} (4 punkty Intelektu) - Strażnik, doktor, technik lub najęty bandyta zatrudniony przez lub stowarzyszony z przeciwnikiem jest sekretnie Twoim sprzymierzeńcem lub wisi Ci przysługę. Kiedy się na nią powołujesz, cel robi co może, żeby pomóc Ci (rozkuwa Cię, daje Ci nóż, zostawia drzwi celi otwarte) w sposób, który minimalizuje możliwość odkrycia, co zrobił. Ta zdolność to zadanie Intelektu poziomu 3. Każdy dodatkowy raz, gdy korzystasz z tej umiejętności, zadanie jest utrudnione o dodatkowy stopień. Trudność wraca do 3 po odpoczynku trwającym 10 godzin. Akcja.

\textbf{Wezwanie Międzywymiarowego Ducha}\index{Zdolności!Alfabetycznie!Wezwanie Międzywymiarowego Ducha}\label{sec:Wezwanie Międzywymiarowego Ducha} (6 punktów Intelektu) - przywołujesz istotę-ducha, który manifestuje się przez maksymalnie dzień (lub mniej, jeśli osiągnie przedtem coś ważnego) po którym znika i nie można go ponownie przywołać. Ten duch jest istotą 6 poziomu lub niższego, i może być materialna lub nie, zgodnie z własnym życzeniem (zmiana stanu wymaga akcji). Duch nie jest Tobie winny wdzięczności, i nie potrzebuje zostać blisko Ciebie, by pozostać zamanifestowanym. Akcja, by rozpocząć.

\textbf{Wezwanie Burzy}\index{Zdolności!Alfabetycznie!Wezwanie Burzy}\label{sec:Wezwanie Burzy} (7+ punktów Intelektu) - Jeśli jesteś na zewnątrz lub w pomieszczeniu, którego sufit sięga co najmniej 90 m, przywołujesz kotłujące się warstwy oświetlonych błyskawicami chmur burzowych do 460 m w promieniu na 10 minut. Podczas dnia, naturalne oświetlenie pod burzą jest zredukowane do niskiego. Kiedy burza grzmi, możesz wykorzystać akcję, by wysłac błyskawicę zz chmury, by zaatakować cel, który dostrzegasz, zadając mu 4 punkty obrażeń (możesz normalnie korzystać z Wysiłku na tych atakach). Trzy akcje by rozpocząć, akcja, by wezwać błyskawicę. 

\textbf{Wezwanie Roju}\index{Zdolności!Alfabetycznie!Wezwanie Roju}\label{sec:Wezwanie Roju} (4 punkty Intelektu) - Jeśli znajdujesz się w lokacji, gdzie mogą przybyć istoty związane z Twoją zdolnością Wpływ na Rój, możesz wezwać je na godzinę. Podczas tej godziny, istoty te robią, co im rozkażesz telepatycznie tak długo, jak są w dalekim zasięgu od Ciebie. Mogą one sie gromadzić i utrudniać akcje Twoich wrogów. Kiedy te istoty są w dalekim zasięgu, możesz rozmawiać z nimi telepatycznie i postrzegać świat poprzez ich zmysły. Akcja by rozpocząć. 

\textbf{Wezwanie Przez Czas}\index{Zdolności!Alfabetycznie!Wezwanie Przez Czas}\label{sec:Wezwanie Przez Czas} (6+ punktów Intelektu) - Przywołujesz osobę lub istotę do 3 poziomu z niedawnej przeszłości, i pojawia się ona obok Ciebie. Możesz wybrać istotę, z którą wcześniej wszedłeś w kontakt (nawet, jeśli jest teraz martwa) lub (nie więcej niż raz na dzień) możesz pozwolić MG na określenie istoty przypadkowo. Jeśli przywołujesz przypadkową istotę, masz 10 procent szans, że będzie to istota do 5 poziomu. Ta istota nie ma pamięci niczego przed byciem wezwaną przez Ciebie, ale mimo to może mówić i ma ogólną wiedzę, którą posiada istota jej typu. Istota wezwana przez czas wykonuje swoje akcje tak długo, jak sie na niej koncentrujesz, ale musisz wykorzystać swoją akcję w każdej turze, by wydać jej rozkazy, inaczej wróci do przeszłości.

W dodatku do normalnych opcji korzystania z Wysiłku, możesz skorzystać z wysiłku, by wezwać potężniejszą istotę: każdy poziom Wysiłku zwiększa poziom istoty o 1. Dla przykładu, zastosowanie poziomu Wysiłku, wzywa specyficzną istotę do 4 poziomu lub daje Ci 10 procent szans na wezwanie przypadkowej istoty do 6 poziomu. Akcja.

\textbf{Uspokojenie}\index{Zdolności!Alfabetycznie!Uspokojenie}\label{sec:Uspokojenie} (3 punkty Intelektu) - Poprzez dowcipy, piosenkę lub inną sztukę, powstrzymujesz jednego żywego przeciwnika przeciwko zaatakowaniem kogokolwiek lub czegokolwiek przez jedną rundę. Akcja.

\textbf{Uspokojenie Nieznajomego}\index{Zdolności!Alfabetycznie!Uspokojenie Nieznajomego}\label{sec:Uspokojenie Nieznajomego} (2+ punkty Intelektu) - Możesz sprawić, że jedna inteligentna istota pozostaje spokojna w momencie, gdy mówisz. Ta istota nie musi mówić Twoim językiem, ale musi być w stanie Cię ujrzeć. Pozostaje spokojna tak długo, jak skupiasz na sobie jej uwagę i nie jest zaatakowana lub w inny sposób w niebezpiecznej sytuacji. W dodatku do normalnych opcji Wysiłku, możesz go zastosować, by uspokoić dodatkową istotę sprzymierzoną z Twoim pierwszym celem - jedna istota na poziom Wysiłku. Akcja.

\textbf{Zręczny Wojownik}\index{Zdolności!Alfabetycznie!Zręczny Wojownik}\label{sec:Zręczny Wojownik} - Twoje ataki zadają o 1 punkt obrażeń więcej. Umożliwienie. 

{\textbf{Zachwyt lub Inspiracja}\index{Zdolności!Alfabetycznie!Zachwyt lub Inspiracja}\label{sec:Zachwyt lub Inspiracja} - Możesz zastosować tę zdolność na dwa sposoby. Albo Twoje słowa utrzymują uwagę wszystkich BN-ów, którzy je słyszą tak długo, jak mówisz, albo Twoje słowa inspirują BN-ów którzy je słyszą, tak, że funkcjonują przez godzinę, jakby posiadali o poziom więcej. W dowolnym wypadku, wybierasz, którzy BN-i dostają się pod wpływ tej zdolności. Jeśli ktoś w tłumie zostanie zaatakowany, kiedy próbujesz do niego przemówić, tracisz uwagę tłumu. Akcja, by rozpocząć. 

\textbf{Zachwyt Światła Gwiazd}\index{Zdolności!Alfabetycznie!Zachwyt Światła Gwiazd}\label{sec:Zachwyt Światła Gwiazd} - Tak długo, jak mówisz, utrzymujesz uwagę wszystkich 2 poziomowych lub słabszych BN-ów, który Cię słyszą. Jeśli posiadasz także zdolność Zauroczenie, możesz w podobny sposób wpłynąć na BN-ów poziomu 3-go. Akcja, by rozpocząć.

\textbf{Surfer Aut}\index{Zdolności!Alfabetycznie!Surfer Aut}\label{sec:Surfer Aut} - możesz wstać lub porusząc się w poruszającym się pojeździe (np: jego suficie, otwartych drzwiach, masce itp.) z duża szansą, ze nie spadniesz. Jesli pojazd nie zrobi nagłego zwrotu, zatrzyma isę nagle lub w inny sposób nie wykona jakiegoś ekstremalnego manewru, wstanie lub poruszanie się po takim pojeździe to dla Ciebie zadanie rutynowe. Jeśli pojazd wykonuje jakieś ekstremalne manewry, jak te opisane wyżej, wszystkie zadania, by pozostać na powierzchni pojazdu są ułatwione. Umożliwienie. 

\textbf{Ostrożny Rzut}\index{Zdolności!Alfabetycznie!Ostrożny Rzut}\label{sec:Ostrożny Rzut} - Jesteś wyszkolony we wszystkich atakach bronią rzucaną. Umożliwienie.

\textbf{Ostrożny Strzał}\index{Zdolności!Alfabetycznie!Ostrożny Strzał}\label{sec:Ostrożny Strzał}  - Możesz wydać punkty z Puli Szybkości lub z Puli Intelektu, by zwiększać Wysiłkiem obrażenia broni palnej. Każdy poziom wysiłku dodaje 3 punkty obtażeń do udanego ataku, a jeśli spędzisz swoją turę na celowanie, każdy poziom Wysiłku zamiast tego dodaje 5 punktów obrażeń do udanego ataku. Umożliwienie.

\textbf{Rzuć Iluzję}\index{Zdolności!Alfabetycznie!Rzuć Iluzję}\label{sec:Rzuć Iluzję} - Możesz zwiększyć zasięg w którym możesz tworzyć i podtrzymywać swoje iluzje bliskiego zasięgu (np: ze zdolności Mniejsza Iluzja) do dowolnego miejsca w średnim zasięgu, które możesz dostrzec. Umożliwienie. 

\textbf{Przerażenie}\index{Zdolności!Alfabetycznie!Przerażenie}\label{sec:Przerażenie} (4 punkty Intelektu) - Przerażasz swojego oponenta w dalekim zasięgu, który rozumie mowę (choć nie musi Twojego języka) tak bardzo, że traci on swą następną akcję i na resztę swoich akcji w ciągu 1 minuty jego zadania są utrudnione. Każdy dodatkowy raz, gdy próbujesz wykorzystać tę zdolność na tym samym wrogu, musisz zastosować o poziom Wysiłku więcej, niż przy ostatniej próbie. Akcja. 

\textbf{Talent Celebryty}\index{Zdolności!Alfabetycznie!Talent Celebryty}\label{sec:Talent Celebryty} - jesteś wyszkolony w dwóch z poniższych umiejętnościach: pisaniu, dziennikarstwie, danym rodzaju sztuki, danym sporcie, szahach, komunikacji naukowej, aktorstwie, prezentacji newsów lub innej powiązanej zdolności niebojowej, która uczyniła z Ciebie celebrytę. Umożliwienie. 

\textbf{Centrum Uwagi}\index{Zdolności!Alfabetycznie!Centrum Uwagi}\label{sec:Centrum Uwagi} (5 punktów Intelektu) - Dosłowny (lub metaforyczny, w zależności od settingu) promień czystej światłości zstępuje z Niebios i Cię okala. Wszystkie istoty, który wybierzesz w swoim bliskim zasięgu padają na kolana i tracą swą następną akcję. Cele tej mocy nie mogą się bronić i są traktowane jako bezsilne. Akcja.

\textbf{Komnata Snów}\index{Zdolności!Alfabetycznie!Komnata Snów}\label{sec:Komnata Snów} (8 punktów Intelektu) - Ty i Twoi sprzymierzeńcy możecie wkroczyć w komnat snów, udekorowaną jak sobie tego zażyczysz, która zawiera pewną liczbę drzwi. Prowadzą one do lokalizacji, które odwiedziłeś lub które znasz całkiem dobrze. Przejście przez jedne z tych drzwi przenosi Cię do pożądanej lokacji. Jest to zadanie trudności 2 bazujące na Intelekcie (zadanie może być trudniejsze, jeśli lokacja jest chroniona magicznie). Akcja by wejść do komnaty snów; akcja, by przejść przez wrota w komnacie.

\textbf{Zmiana Paradygmatu}\index{Zdolności!Alfabetycznie!Zmiana Paradygmatu}\label{sec:Zmiana Paradygmatu} (6+ punktów Intelektu) - Zmieniasz światopogląd istoty, z którą spędzasz przynajmniej rundę na rozmowie (jeśli jest ona w stanie Cię zrozumieć). Ta istota zmienia swoje zdanie odnośnie ważnego poglądu lub wierzenia, co może być czymś tak prostym jak zmiana chęci zamordowania Ciebie na pomoc Ci, lub być czymś dziwniejszym. Efekt trwa przynajmniej przez 10 minut, ale może trwać to dłużej, jeśli istota nie była wcześniej Twoim wrogiem. W tym czasie, istota podejmuje akcje zgodnie z mądrością, którą się z nią podzieliłeś. Cel musi być na poziomie 2 lub niższym. W dodatku do normalnych opcji korzystania z Wysiłku, możesz z niego skorzystać, aby zwiększyć maksymalny poziom celu (o 1 więcej na każdy poziom Wysiłku). Akcja, by rozpocząć.

\textbf{Naładowanie}\index{Zdolności!Alfabetycznie!Naładowanie}\label{sec:Naładowanie} (1+ punktów Intelektu) - Możesz naładować artefakt lub inne urządzenie (ale nie cypher), tak, by skorzystać z niego raz. Koszt to 1 punkt Intelektu plus 1 punkt na poziom urządzenia. Akcja.

\textbf{Naładowanie Broni}\index{Zdolności!Alfabetycznie!Naładowanie Broni}\label{sec:Naładowanie Broni} (2+ punkty Intelektu) - Jako część ataku Twoją magiczną bronią, ładujesz ją magiczną mocą, zadając dodatkowe 2 punkty obrażeń od energii. Jeśli wykonujesz więcej niż 1 atak w swojej turze, wybierasz, czy chcesz wydać punkty na tą zdolność przed wykonaniem każdego z ataków. Umożliwienie.

\textbf{Szarża Hordy}\index{Zdolności!Alfabetycznie!Szarża Hordy}\label{sec:Szarża Hordy} (7 punktów Mocy) - Ty i dwóch lub więcej z Twoich kompanów obo kCiebie działacie jak jedna istota, by wykonać atak szarżą. Kiedy to robicie, wszyscy poruszacie się na średni dystans, w trakcie czego atakujecie wszystko w Waszym bliskim zasięgu na swojej drodze, z atutem do ataku. Cele tego ataku otrzymują dodatkowe 3 punkty obrażeń i są wywrócone. Akcja.

\textbf{Zauroczenie Maszyny}\index{Zdolności!Alfabetycznie!Zauroczenie Maszyny}\label{sec:Zauroczenie Maszyny} (2 punkty Intelektu) - Przekonujesz nieinteligentną maszynę, by Cię ``lubiła''. Maszyna, która Cię lubi, ma szanszę mniejszą o 50 procent, by funkcjonować w sposób, który mógłby Cię zranić. Tak więc, jeśli wróg chce zdetonować bombę blisko Ciebie, a jest ona kontrolowane detonatorem, który Cię lubi, istnieje 50 procent szans, że bomba nie wybuchnie. Akcja, by rozpocząć. 

\textbf{Chmura Ochronna}\index{Zdolności!Alfabetycznie!Chmura Ochronna}\label{sec:Chmura Ochronna} (5 punktów Intelektu) - Sprawiasz, że małe obiekty z Twojego otoczenia (kamienie, zepsute przedmioty, chmury pyłu itp) obracają się wokół Ciebie przed 10 minut, co daje Ci +2 do Pancerza. Akcja, by rozpocząć. 

\textbf{Umysł-Twierdza}\index{Zdolności!Alfabetycznie!Umysł-Twierdza}\label{sec:Umysł-Twierdza} - Jesteś wyszkolony w akcjach obrony Intelektu i masz Pancerz +2 do ataków, które dotyczą Twojej puli Intelektu (co normalnie ignoruje Pancerz). Umożliwienie. 

\textbf{Zamglenie Pamięci Krótkotrwałej}\index{Zdolności!Alfabetycznie!Zamglenie Pamięci Krótkotrwałej}\label{sec:Zamglenie Pamięci Krótkotrwałej} (3 punkty Intelektu) - Jeśli wchodzisz w interakcje lub studiujesz cel przez przynajmniej rundę, zyskujesz świadomość, jak działa jego umysł, co możesz wykorzystać przeciwko niemu w najgorszy możliwy sposób. Możesz spróbować go skonfundować i sprawić, że zapomni, co właśnie zaszło. W przypadku sukcesu, usuwasz do ostatnich 5 minut z jego pamięci. Akcja by przygotować, akcja by rozpocząć.
 
\textbf{Ulepszenie Maszyny}\index{Zdolności!Alfabetycznie!Ulepszenie Maszyny}\label{sec:Ulepszenie Maszyny} (2 punkty Intelektu) - POlepszasz moc lub funkcjonowanie maszyny tak, że działa na 1 poziomie więcej niż zwyklep rzez jedną godzinę. Akcja, by rozpocząć. 

\textbf{Obliczenia Bitewne}\index{Zdolności!Alfabetycznie!Obliczenia Bitewne}\label{sec:Obliczenia Bitewne} - Podczas walki, Twój mózg przełącza się na tryb bojowy, gdzie wszystkie potencja;ne ataki, które możesz wykonać, pojawiają się jako wektory w Twoim umyśle, co zawsze zapewnia najlepszą opcję. Twoje ataki są ułatwione. Umozliwienie.

\textbf{Emisja Zimna}\index{Zdolności!Alfabetycznie!Emisja Zimna}\label{sec:Emisja Zimna} (5+ punktów Intelektu) - Emitujesz zimno we wszystkich kierunkach w średnim zasięgu. Wszyscy w obszarze Twojej emisji (z wyjątkiem Ciebie) otrzymują 5 punktów obrażeń. Jeśli zastosujesz WYsiłek, by zwiększyć obrażenia, zamiast ułatwić zadanie, zadajesz 2 dodatkowe punkty obrażeń na poziom Wysiłku (zamiast 3 punktów); cele w obszarze otrzymują 1 punkt obrażeń, nawet jeśli nie uda Ci isę rzut na atak. Akcja.

\textbf{Kolos}\index{Zdolności!Alfabetycznie!Kolos}\label{sec:Kolos} - Kiedy korzystasz z Wzrostu, możesz wybrać wzrost do 18 m wysokości. Kiedy to czynisz, zadajesz dodatkowe 2 punkty obrażeń atakami wręcz (plus wszelki dodatek ze zdolności Wielki). Na każdy poziom Wysiłku, który zastosujesz, Twój wzrost zwiększa się o 3 metry, i dodajesz 1 punkt więcej do swojej Puli Mocy. Tak więc, za pierwszym razem gdy zastosujesz Wzrost po 10-godzinnym odpoczynku, jeśli zastosujesz 2 poziomy Wysiłku, Twój wzrost wynosić będzie 24 metry i dodasz 17 tymczasowych punktów do swojej Puli Mocy. Umożliwienie.

\textbf{Wyzwanie Bojowe}\index{Zdolności!Alfabetycznie!Wyzwanie Bojowe}\label{sec:Wyzwanie Bojowe} - Wszystkie zadania, których celem jest ściągnięcie na Ciebi ataków (i odciągnięcie ich od innych) są ułatwione o dwa kroki. Umożliwienie.

\textbf{Zdolności Bojowe}\index{Zdolności!Alfabetycznie!Zdolności Bojowe}\label{sec:Zdolności Bojowe} - Dodajesz +1 obrażeń do jednego typu ataku z bronią Twojego wyboru: atak bronią wręcz lub dystansowy atak bronią. Umożliwienie.

\textbf{Rozkaz}\index{Zdolności!Alfabetycznie!Rozkaz}\label{sec:Rozkaz} (3 punkty Intelektu) - Poprzez czystą moc woli, możesz wydać prosty rozkaz danej istocie, która następnie wykonuje go przez następną akcję. Istota musi być w średnim zasięgu i być w stanie Cię zrozumieć. Rozkaz nie może być bezpośrednim zagrożeniem dla istoty lub jej towarzyszy, więc ``Popełnij samobójstwo'' nie zadziała, ale ``Ucieknij'' już tak. Dodatkowo, rozkaz może wymagać od istoty pojedyńczej akcji, więc ``Otwórz drzwi'' może zadziałać, ale ``Otwórz drzwi i przebiegnij przez nie'' już nie. Istota, której rozkazano, może się bronić normalnie i odpowiedzieć na atak atakiem, jeśli zostanie zaatakowana. Jeśli posiadasz inną zdolność, którą możesz wydać rozkaz istocie, możesz efektem Rozkazu objać dwie istoty na raz (jest to ``podstawowy efekt'' obydwu zdolności) korzystając z dowolnej z tych zdolności. Akcja. 

\textbf{Rozkazywanie Bestiom}\index{Zdolności!Alfabetycznie!Rozkazywanie Bestiom}\label{sec:Rozkazywanie Bestiom} (3+ punkty Intelektu) - Możesz rozkazywać nieagresywnej bestii nie będącej człowiekiem (jak np: istota, którą uspokoiłeś poprzez Ukojenie Dzikiego) do 3 poziomu w średnim zasięgu. Jeśli osiągniesz sukces, przez następną minutę bestia słucha Twoich werbalnych komend najlepiej jak rozumie i może. GM decyduje, co liczy się jako nieludzka bestia, ale jeśli nie masz do czynienia z jakimś oszustwem, powinieneś wiedzieć, czy możesz wpłynąć na istotę zanim skorzystasz z tej zdolności. Obcy, istoty międzywymiarowe, bardzo inteligentne istoty i roboty nigdy się w to nie wliczają. 

W dodatku do normalnych opcji korzystania z Wysiłku, możesz z niego skorzystać, by zwiększyć maksymalny poziom celu. Tak więc, by rozkazywać bestii 5 poziomu (2 poziomy ponad normalny limit), musisz zastosować dwa poziomy Wysiłku. Akcja, by rozpocząć. 

\textbf{Rozkazywanie Maszynom}\index{Zdolności!Alfabetycznie!Rozkazywanie Maszynom}\label{sec:Rozkazywanie Maszynom} (4 punkty Intelektu) - Jeśli zauroczyłeś nieinteligentną maszynę lub rozmawiałeś telepatycznie z inteligentną maszyną, możesz spróbować wydać jej rozkaz, który wysłucha przez jedną akcję najlepiej, jak potrafi. (Jeśli korzystasz z tej zdolności, by rozkazać inteligentnej maszynie, najpewniej będzie po fakcie nastawiona do Ciebie negatywnie i agresywnie). Akcja.

\textbf{Kontrola Metalu}\index{Zdolności!Alfabetycznie!Kontrola Metalu}\label{sec:Kontrola Metalu}  (5 punktów Intelektu) - Zmieniasz kształt metalicznego obiektu wedle swojej woli. Przedmiot musi być w zasięgu Twego wzroku i w średnim zasięgu, Jego masa nie może być większa od Twojej własnej. Możesz wpłynąć na więcej niż jeden przedmiot na raz, tak długo, jak ich połączona masa nie przekracza limitu Twojej wagi. Możesz połączyć rożne przedmioty w jeden. Możesz wykorzystać tę moc, by zniszczyć metalowy obiekt (jak w zdolności Niszczenie Metalu) lub możesz stworzyć nowy obiekt (bardzo niezręcznie, chyba, że masz odpowiednie umiejętności rzemieślnicze). Możesz wtedy przemieścić nowy obiekt gdziekolwiek w zasięgu zdolności. Dla przykładu, możesz wziąć parę metalowych tarcz, połączyć je razem, i użyć, by zablokować drzwi. Możesz użyć tej zdolności, by wykonać atak - atakując wroga jego własną zbroją, zmieniając metalowy przedmiot w odłamki, którymi rzucasz w środek pola walki itp. - przeciwko jednemu celowi w średnim zasięgu. Niezależnie od formy ataku, jest to akcja Intelektu zadająca 7 punktów obrażeń. Akcja. 

\textbf{Rozkazywanie Duchom}\index{Zdolności!Alfabetycznie!Rozkazywanie Duchom}\label{sec:Rozkazywanie Duchom} (3 punkty Intelektu) - Możesz rozkazać duchowi lub nieumarłemu do 5 poziomu w średnim zasięgu. Jeśli Ci się uda, cel nie może zaatakować Cię przez minutę, w której to minucie wykonuje Twoje instrukcje, jeśli CIę słyszy i rozumie. Akcja, by rozpocząć.

\textbf{Komunikacja}\index{Zdolności!Alfabetycznie!Komunikacja}\label{sec:Komunikacja}  (2 punkty Intelektu) - Możesz przekazać podstawowy koncept istocie, która normalnie nie może mówić lub rozumieć mowy. Ta istota może także przekazać Ci bardzo prostą odpowiedź na proste pytanie. Akcja.

\textbf{Lokalny Aktywista}\index{Zdolności!Alfabetycznie!Lokalny Aktywista}\label{sec:Lokalny Aktywista} Kiedy rozmawiasz z innymi w swojej społeczności, jesteś wyszkolony w perswazji i zastraszaniu odnośnie tematów, które są bezpośrednio powiązane ze społecznością. Umożliwienie.

\textbf{Wiedza o Społeczności}\index{Zdolności!Alfabetycznie!Wiedza o Społeczności}\label{sec:Wiedza o Społeczności} (2 punkty Intelektu) - Jeśli jesteś zaangażowany w społeczność i spędziłeś przynajmniej ostatnie parę miesięcy żyjąc w niej, możesz się dowiedzieć o rożnych rzeczach na różne sposoby. Czasami Twoje kontakty podsuwają Ci informacje. Innymi razy, wyciagasz wniosk iz tego, co słyszysz i widzisz. Kiedy korzystasz z tej zdolności, możesz zadać MG jedno pytanie o społeczności i otrzymać krótką odpowiedź. Akcja.

\textbf{Programowanie}\index{Zdolności!Alfabetycznie!Programowanie}\label{sec:Programowanie} - Jesteś wyszkolony w używaniu (i nadużywaniu) oprogramowania, znasz jeden lub więcej języków programowania dostatecznie dobrze, by pisać własne proste programy, znasz się także na korzystaniu z Internetu. Umożliwienie. 

\textbf{Wybuch}\index{Zdolności!Alfabetycznie!Wybuch}\label{sec:Wybuch} (7 punktów Intelektu) - Wywołujesz wybuchowy impuls, który wybucha w punkcie, który wybierasz w długim zasięgu. Ten impuls rozszerza się na średni zasięg we wszystkich kierunkach, zadając 5 punktów obrażeń wszystkiemu w obszarze. Nawet jeśli nie uda Ci się rzut na atak, cele w obszarze otrzymują 1 punkt obrażeń. Akcja.

\textbf{Promień Odrzucający}\index{Zdolności!Alfabetycznie!Promień Odrzucający}\label{sec:Promień Odrzucający} (2 punkty Intelektu) - Wyzwalasz promień czystej mocy, który wbija się w istotę w średnim zasięgu, zadając 5 punktów obrażeń i odrzucając ją do tyłu na bliski zasięg. Akcja.

\textbf{Sprawny Oszust}\index{Zdolności!Alfabetycznie!Sprawny Oszust}\label{sec:Sprawny Oszust} - Kiedy hakujesz system komputerowy, oszukujesz kogoś, dokonujesz kradzieży kieszonkowej, ukrywasz coś przed strażnikiem itp, zyskujesz atut na tym zadaniu. Umożliwienie.

\textbf{Konfundujące Nonsensy}\index{Zdolności!Alfabetycznie!Konfundujące Nonsensy}\label{sec:Konfundujące Nonsensy} (4 punkty Intelektu) - Wyrzucasz z siebie strumień nonsensów, by rozproszyć wroga w bliskim zasięgu. Po udanym rzucie na Intelekt, Twoje rzuty obronne przeciwko następnemu atakowi tej istoty przed końcem Twojej następnej rundy są ułatwione. Action.

\textbf{Skonfunduj Wroga}\index{Zdolności!Alfabetycznie!Skonfunduj Wroga}\label{sec:Skonfunduj Wroga} (4 punkty Intelektu) - Poprzez sprytną zmyłkę, w co wlicza się zręczne wykorzystanie płaszcza, unik we właściwym momencie lub podobną strategię, możesz spróbować przekierować fizyczny atak wręcz, który inaczej by CIę zrnaił. Kiedy to czynisz, przeniesiony atak uderza w inną istotę Twojego wyboru w bliskim zasięgu (zarówno od Ciebie, jak i od atakującego wroga) Ta zdolność to zadanie Intelektu 2 poziomu. Umożliwienie. 

\textbf{Przywołanie}\index{Zdolności!Alfabetycznie!Przywołanie}\label{sec:Przywołanie} (7 punktów Intelektu) - Powołujesz do życia, z powietrza, istotę 5 poziomu, którą wcześniej spotkałeś. Ta istota zostaje na jedną minutę, a potem wraca do domu. Kiedy jest obecna, działa ona jak ją poinstruujesz, ale nie wymaga to z Twojej strony akcji. Akcja.

\textbf{Kontakty}\index{Zdolności!Alfabetycznie!Kontakty}\label{sec:Kontakty}  - Znasz ludzi, którzy mogą zrobić coś dla Ciebie - nie tylko respektowani ludzie w pozycjach władzy, ale także różnorodnych hakerów i przestępców. Ci ludzie nie są koniecznie Twoimi przyjaciółmi i mogą nie byc wiarygodni, ale wiszą Ci przysługę. Powinieneś razem z MG określić szczegóły tych znajomości. Umożliwienie. 

\textbf{Człowiek-Guma}\index{Zdolności!Alfabetycznie!Człowiek-Guma}\label{sec:Człowiek-Guma} (2 punkty Szybkości) - Możesz się wyzwolić z okowów lub przecisnąć przez wąskie dziury. Jesteś wyszkolony w ucieczce z więzów. Kiedy korzystasz z akcji, by uciec z więzów lub przecisnąć się przez wąską przestrzeń, możesz natychmiastowo wykonać kolejną akcję. Możesz skorzystać z tej dodatkowej akcji tylko po to, by się poruszyć. Umożliwienie. 

\textbf{Kontrola Bitewna}\index{Zdolności!Alfabetycznie!Kontrola Bitewna}\label{sec:Kontrola Bitewna} (1 punkt Mocy) - Ten atak wręcz zadaje o 1 punkt obrażeń mniej niż normalnie, i niezależnie od tego, czy trafisz czy chybisz, manewrujesz go w pozycję, której pragniesz w bliskim zasięgu. Akcja.

\textbf{Kontrola Maszyny}\index{Zdolności!Alfabetycznie!Kontrola Maszyny}\label{sec:Kontrola Maszyny} (6 punktów Intelektu) - Możesz spróbować kontrolować dowolną maszynę, inteligentną lub nie, w średnim zasięgu przez 10 minut. Akcja.

\textbf{Kontrola Dzikiej Bestii}\index{Zdolności!Alfabetycznie!Kontrola Dzikiej Bestii}\label{sec:Kontrola Dzikiej Bestii}  (6 punktów Intelektu) - Możesz kontrolować spokojną, nieludzką bestię w zasięgu 9 metrów. Kontrolujesz ją tak długo, jak skupiasz na niej swoją uwagę, poświęcając na to swoją turę w każdej rundzie. Ostateczna decyzja czy możesz wpłynąć na daną istotę zależy od MG, ale jeśli nie ma jakiegoś rodzaju oszustwa, to jesteś w stanie stwierdzić przed użyciem tej zdolności, czy istota jest na nią podatna. Obcy, byty międzywymiarowe, bardzo inteligentne istoty i roboty nigdy nie są objęte tą zdolnością. Akcja.

\textbf{Kontrola Roju}\index{Zdolności!Alfabetycznie!Kontrola Roju}\label{sec:Kontrola Roju} (2 punkty Intelektu) - Przywołujesz rój istot (powiązanych z Twoją zdolnością Wpływ na Rój) w średnim zasięgu i kontrolujesz je przez 10 minut. Nawet zwykłe robaki (poziom 0) w odpowiednio wielkim roju mogą pokryć daną istotę i utrudnić jej akcje. Akcja, by rozpocząć. 

\textbf{Kontrola Pogody}\index{Zdolności!Alfabetycznie!Kontrola Pogody}\label{sec:Kontrola Pogody} (10 punktów Intelektu) - Zmieniasz pogodę w swojej okolicy. Jeśli używasz tej mocy w budynku, tworzy ona mniejsze efekty, takie jak mgła, mniejsze zmiany temperatury itp. Jeśli wykonujesz ją na zewnątrz, możesz przywołać deszcz, mglę, śnieg, wiatr lub inną normalną (niezbyt ekstremalną) pogodę. Ta zmiana pogody trwa przez naturalny czas, tak więc burza mogłaby trwać przez godzinę, mgła przez 2 lub 3 godziny, a śnieg pare godzin (lub 10 minut, jeśli nie ma zimy). Przez pierwsze 10 minut po aktywowaniu tej zdolności, możesz stworzyć bardziej dramatyczne i konkretne efekty, takie jak pioruny, wielkie wichry, huragany, itp. Te efekty muszą wystąpić w zasięGu 300 metrów od Twojej lokacji. Musisz spędzić turę, koncentrując się na stworzeniu efektu lub podtrzymaniu go w nowej rundzie. Te efekty zadają 6 punktów obrażeń w każdej rundzie. Jeśli posiadasz tę zdolność z innego źródła, jej koszt wynosi 7 punktów Intelektu zamiast 10. Jeśli już posiadasz zdolność Przywołanie Burzy, możesz ją natychmiastowo zamienić na inną zdolność tego samego poziomu. Akcja, by rozpocząć.  

\textbf{Kontrolowana Przemiana}\index{Zdolności!Alfabetycznie!Kontrolowana Przemiana}\label{sec:Kontrolowana Przemiana} - Możesz spróbować skorzystać ze swojej Likantropii by zmienić się w formę likantropa na dowolnej nocy, której zapragniesz (zadanie Intelektu trudności 3). Wszelkie transformacje z ykorzystaniem tej zdolności są dodatkowe do 5 nocy na miesiąc, gdy musisz się zmienić. Akcja by się zmienić. 

\textbf{Kontrolowany Upadek}\index{Zdolności!Alfabetycznie!Kontrolowany Upadek}\label{sec:Kontrolowany Upadek} - Gdy upadasz kiedy masz możliwość wykonania akcji i masz powierzchnię pionową w zasięgu ręki, możesz spróbować spowolnić swój upadek. Wykonaj test Szybkości o trudności 1 na każde 6 m Twojego upadku. Przy sukcesie, otrzymujesz połowę obrażeń z Upadku. Jeśli zredukujesz trudność testu do 0, nie odnosisz żadnych obrażeń. Umożliwienie. 

\textbf{Skoordynowany Wysiłek}\index{Zdolności!Alfabetycznie!Skoordynowany Wysiłek}\label{sec:Skoordynowany Wysiłek} (3 punkty Intelektu) - Kiedy Ty i Twój duplikat z mocy Kopia atakujecie tę samą istotę, możesz uczynić tę akcję jednym rzutem na atak z atutem. Jeśli trafisz, zadajesz obrażenia obydwu ataków i traktujesz je jakby były obrażeniami z jednego ataku w celach określenia ile punktów pochłania Pancerz. Akcja.

\textbf{Skopiuj Moc}\index{Zdolności!Alfabetycznie!Skopiuj Moc}\label{sec:Skopiuj Moc} (2+ punkty Intelektu) - Możesz skopiować czyjąś zdolność na godzinę, korzystając z niej, jakby była dla Ciebie naturalna. W ciągu poprzedniej godziny musiałeś dotknąć istotę, której moc pragniesz skopiować (rzut na atak) i musisz być świadkiem, jak korzysta ona z tej zdolności. Wybierz zdolność, którą chcesz skopiować, a MG wybiera odpowiednią zdolność niskiego poziomu, która najbardziej przypomina tę moc. Dla przykładu, jeśli walczysz z superzłoczyńcą, który tworzy promienie mocy, jeśli skopiujesz tę zdolność, uzyskujesz niskopoziomową zdolność, która tworzy promienie mocy. 

W dodatku do kosztów Skopiuj Moc, musisz też zapłacić koszt Mocy, Szybkości lub Intelektu (jeśli jakikolwiek) na odpowiednią zdolność wyboru MG. Dla przykładu, jeśli chcesz skopiować moc superzłoczyńcy, którą są promienie mocy, GM najpewniej zadecyduje, że otrzymujesz zdolność Pocisk. tak wiec płacisz 2 punkty intelektu za Skopiuj Moc i 1 punkt Intelektu za skorzystanie z Pocisku. Możesz mieć jedną kopią zdolności w danym czasie - skopiowanie innej kończy możliwość korzystania z wcześniejszej mocy, którą skopiowałeś przy pomocy tej zdolności.

Skopiuj Moc nie kopiuje zdolności, które permanentnie dodają punkty do Pul, takich jak Ulepszone Ciało. 

W dodatku do normalnych opcji korzystania z Wysiłku, możesz wykorzystać jego poziomy, by skopiować moc, którą widziałeś więcej niz godzinę temu - każdy poziom Wysiłku użyty w taki sposób pozwala Ci się ``cofnąć'' o dodatkową godzinę. Akcja.

\textbf{Okiełznanie Niebezpieczeństwa}\index{Zdolności!Alfabetycznie!Okiełznanie Niebezpieczeństwa}\label{sec:Okiełznanie Niebezpieczeństwa} (4 punkty Intelektu) - Negujesz źródło potencjalnego niebezpieczeństwa powiązanego z jedną istotą lub obiektem w bliskim zasięgu na 1 minutę (zamiast na 1 rundę, jak z Negacją Zagrożenia). Może to być broń lub urządzenie trzymane przez kogoś, naturalna zdolność istoty, lub pułapka aktywowana płytką w podłodze. Możesz także spróbować skontroać akcję (taką jak poruszanie się lub konwencjonalny atak bronią, pazurem itp.). Akcja.

(Korzystanie z Okiełznania Niebezpieczeństwa jest zazwyczaj kwestią stosowania szybkiego myślenia i refleksu w kontakcie z bezpośrednim zagrożeniem. Ta zdolność nie polega na nadprzyrodzonych cechach, lecz na praktycznym stosowaniu zwykłych akcji.)

\textbf{Środki Zaradcze}\index{Zdolności!Alfabetycznie!Środki Zaradcze}\label{sec:Środki Zaradcze} (4 punkty Intelektu) - Natychmiast kończysz jeden trwający efekt (taki jak efekt stworzony przez zdolność postaci) w bliskim zasięgu. Alternatywnie, możesz użyć tej zdolności jak akcji obrony, by skontrować nadchodzącą zdolność, której celem jesteś ty, lub możesz skontrować każde urządzenie lub efekt każdego urządzenia na 1k6 rund. Musisz dotknąć efektu lub urządzenia, by je skontrować. Akcja.

\textbf{Odwaga}\index{Zdolności!Alfabetycznie!Odwaga}\label{sec:Odwaga} - Jesteś wyszkolony w Obronie Intelektu i rzutach na inicjatywę. Umożliwienie. 

\textbf{Rzemieślnik}\index{Zdolności!Alfabetycznie!Rzemieślnik}\label{sec:Rzemieślnik} - jesteś wyszkolony w tworzeniu dwóch rodzajów przedmiotów. Umożliwienie. 

\textbf{Stworzenie}\index{Zdolności!Alfabetycznie!Stworzenie}\label{sec:Stworzenie} (7 punktów Intelektu) - Tworzysz coś z niczego. Możesz stworzyć dowolny przedmiot swojego wyboru, który normalnie miałby trudność stworzenia 5 lub mniej (korzystając z zasad tworzenia przedmiotów). Kiedy już go stworzysz, przedmiot ten istnieje przez liczbę godzin równą 6 minus trudność jego stworzenia. Więc, stworzenie zestawu ciężkich kajdan (trudność 5) trwałoby przez godzinę. Akcja.

\textbf{Stworzenie Śmiertelnej Trucizny}\index{Zdolności!Alfabetycznie!Stworzenie Śmiertelnej Trucizny}\label{sec:Stworzenie Śmiertelnej Trucizny} (3+ punkty Intelektu) - Tworzysz jedną dawkę trucizny 2 poziomu, która albo zadaje 5 punktów obrażeń lub utrudnia akcje zatrutej istoty na 10 minut (ty wybierasz za każdym razem, gdy tworzysz truciznę). Możesz zaaplikować truciznę na broń, jedzenie lub picie jako część akcji tworzenia jej. W dodatku do zwykłych opcji korzystania z Wysiłku, możesz wybrać zwiększenie poziomu trucizny - każdy poziom Wysiłku wykorzystany w ten sposób zwiększa poziom trucizny o 2. Jeśli się z niej nie skorzysta, trucizna traci swoją moc po godzinie. Akcja. 

\textbf{Stworzenie Wody}\index{Zdolności!Alfabetycznie!Stworzenie Wody}\label{sec:Stworzenie Wody} (2 punkty Intelektu) - Sprawiasz, że woda tryska z gruntu z miejsca, które widzisz. Woda płynie z niego przez minutę, tworząc około 4 litrów cieczy w momencie, gdy przestaje. Akcja, by rozpocząć. 

\textbf{Badanie Istoty}\index{Zdolności!Alfabetycznie!Badanie Istoty}\label{sec:Badanie Istoty} (3 punkty Intelektu) - Kiedy badasz nieludzką istotę, możesz zadać MG jedno pytanie, by pozyskać ogólne informacje o jej poziomie, zdolnościach, tym, co je, co ją motywuje jakie są jej słabości (jeśli jakiekolwiek), jak można ją naprawić i tym podobne. Stosuje się to do trudnych lub dziwnych istot, które są poza zasięgiem zwykłych zdolności. Akcja. 

\textbf{Futrzasty Kompan}\index{Zdolności!Alfabetycznie!Futrzasty Kompan}\label{sec:Futrzasty Kompan} - Istota 1 poziomu towarzyszy Ci i słucha Twoich instrukcji. Ta istota jest nie większa niż duży kot (9 kg) i jest zazwyczaj jakimś udomowionym gatunkiem. Powinniście z MG określić szczególy tej istoty, i najpewniej będziesz za nią wykonywał rzuty w walce. Futrzasty Kompan działa w Twojej turze. Jako istota 1 poziomu, posiada stopień trudności 3, 3 punkty zdrowia i zadaje 1 punkt obrażeń. Jej ruch bazuje na jej typie istoty (ptak, istota pływająca itp.). Jeśli Twój futrzasty Kompan zginie, możesz przeszukać miasto lub dzicz przez 1d6 dni, żeby znaleźć nowego. Umożliwienie. 

\textbf{Kontrola Tłumu}\index{Zdolności!Alfabetycznie!Kontrola Tłumu}\label{sec:Kontrola Tłumu} (6+ punktów Intelektu) - Kontrolujesz akcje do 5 istot w średnim zasięgu. Ten efekt trwa przez jedną minutę. Wszystkie cele muszą być na 2 poziomie lub mniejszym. Twoja kontrola ogranicza się do prostych werbalnych komend jak ``Stop'', ``Uciekaj'', ``Śledź strażnika'', ``Patrz tam'' lub ``Zejdź mi z drogi''. Wszystkie cele odpowiadają na komendę, chyba, że zaznaczysz inaczej. W dodatku do zwykłych opcji korzystania z Wysiłku, możesz z niego skorzystać, by zwiększyć maksymalny poziom celów lub wpłynąć na dodatkowe 5 ludzi. Tak więc, aby kontrolować grupę celów 4 poziomu (dwa poziomy ponad limit) lub grupę 15 ludzi, musisz zastosować 2 poziomy Wysiłku.  Kiedy ta zdolność się kończy, istoty pamiętają Twój rozkaz ale nie pamiętają bycie kontrolowanymi - Twoje komendy wydawały im się w tym czasie być rozsądne. Akcja, by rozpocząć. 

\textbf{Miażdżący Cios}\index{Zdolności!Alfabetycznie!Miażdżący Cios}\label{sec:Miażdżący Cios} (2 punkty Mocy) - Kiedy korzystasz z obuchowej lub ciętej broni w obydwu rękach i zastosujesz WYsiłek do ataku, otrzymujesz darmowy poziom Wysiłku do obrażeń ataku. (Jeśli walczysz bez broni, ten atak wykonujesz obydwoma pięściami lub stopami). Akcja.

\textbf{Kryształowe Soczewki}\index{Zdolności!Alfabetycznie!Kryształowe Soczewki}\label{sec:Kryształowe Soczewki} - Możesz skupić naturalnę enegię płynącą poprzez Twoje Kryształowe Ciało. Pozwala Ci to wystrzelić pocisk energii, który zadaje 5 punktó obrażeń celowi w bardzo dalekim zasięgu. Akcja.

\textbf{Kryształowe Ciało}\index{Zdolności!Alfabetycznie!Kryształowe Ciało}\label{sec:Kryształowe Ciało} - Twoje ciało składa się z ruchomego, przezroczystego kryształu koloru bursztynu. Określ razem z MG konkretną formę, choć zazwyczaj ma ona kształt i rozmiar umanoida. Twoje kryształowe ciało zapewnia Ci +2 do Pancerza i +4 do Puli Mocy. Jednakże, nie jesteś taki szybki i Twoje Obrona Szybkości jest utrudniona. Pewne stany, jak zwykłe choroby i trucizny, nie działają na Ciebie. Twoje kryształowe ciało regeneruje się powolniej niż zwykłe, mięsiste ciało. Posiadasz tylko 3 rzuty na odzyskanie zdrowia - 1 rundę, 1 godzinę i 10 godzin. Nie psoaidasz rzutu na odzyskanie zdrowia trwającego 10 minut. Każda zdolność, która wymaga 10-minutowego odpoczynku, zamiast tego wymaga od Ciebie odpoczynku trwającego godzinę. Umożliwienie. 

\textbf{Ciekawy}\index{Zdolności!Alfabetycznie!Ciekawy}\label{sec:Ciekawy} - Zawsze ciekawi Cię Twoje otoczenie, nawet na podświadomym poziomie. Kiedy korzystas z Wysiłku, by spróbować nawigować, dostrzec coś, lub wykonać test na inicjatywę w miejscu, w którym przebywasz rzadko lub w którym nigdy CIę nie było, możesz zastosować darmowy poziom Wysiłku. Umożliwienie.

\textbf{Tnące Światło}\index{Zdolności!Alfabetycznie!Tnące Światło}\label{sec:Tnące Światło} (2 punkty Intelektu) - Emitujesz wąski promień wysokoenergetycznego światła ze swojej dłoni. Zadaje od 5 punktów obrażeń jednemu wrogowi w bliskim zasięgu. Ten promień jest nawet bardziej efektywny przeciwko nieruchomym, nieżywym celom, penetrując na 30 cm każdy materiał poziomu 6 lub mniej. Akcja.

\textbf{Przerzucanie Cypherów}\index{Zdolności!Alfabetycznie!Przerzucanie Cypherów}\label{sec:Przerzucanie Cypherów} - Możesz przerzucić dowolny z Twoich subtelnych cypherów na inną istotę zamiast trzymać go przy sobie. Musisz dotknąć istoty, której pragniesz go przekazać. Umożliwienie. 

\textbf{Przypływ Cyphera}\index{Zdolności!Alfabetycznie!Przypływ Cyphera}\label{sec:Przypływ Cyphera} - Kiedy korzystasz z subtelnego cyphera, jako część swojej akcji, możesz wykorzystać kolejny subtelny cypher. Zamiast normalnego efektu drugiego cyphera, dodajesz jeden darmowy poziom Wysiłku do pierwszego cyphera-zaklęcia. Umożliwienie. 

\textbf{Twórca Cypherów}\index{Zdolności!Alfabetycznie!Twórca Cypherów}\label{sec:Twórca Cypherów} - Wszystkie zamanifestowane cyphery, z których korzystasz, funkcjonują na jednym poziomie wyższym niż zazwyczaj. Jeśli masz tydzień i odpowiednie narzędzia, chemikalia i części, możesz popracować nad jednym ze swoich zamanifestowanych cypherów, zamieniając go w inny cypher takiego typu, jaki miałeś w przeszłości. MG i gracz powinni współpracować i upewnić się, że transformacja jest logiczna - przykładowo, raczej nie możesz zamienić tabletki w hełm. Umożliwienie. 

\section{D}

\textbf{Dodatkowe Obrażenia}\index{Zdolności!Alfabetycznie!Dodatkowe Obrażenia}\label{sec:Dodatkowe Obrażenia} - Zadajesz dodatkowe 3 punkty obrażeń swoją wybraną bronią. Umożliwienie. 

\textbf{Przekaz Obrażeń}\index{Zdolności!Alfabetycznie!Przekaz Obrażeń}\label{sec:Przekaz Obrażeń} - Kiedy Ty lub Twój duplikat (ze zdolności Kopia) otrzymalibyście obrażenia, możesz przemieścić 1 punkt obrażeń z jednego na drugie pod warunkiem, że jesteście w zasięgu 1,5 km od siebie. Umożliwienie. 

\textbf{Potępienie Winnych}\index{Zdolności!Alfabetycznie!Potępienie Winnych}\label{sec:Potępienie Winnych} (3 punkty Intelektu) - Wymawiasz słowa objawienia i osądu do wszystkich w bliskim zasięgu. Ci, których określiłeś jako winnych swoją zdolnością Osąd, uzyskują dodatkowe 3 punkty obrażeń od ataku każdej osoby, która słyszała Twój osąd. Osąd trwa do 1 minuty lub do momentu, aż winni odsuną się na daleki zasięg od Ciebie. Akcja.

\textbf{Instynkt Niebezpieczeństwa}\index{Zdolności!Alfabetycznie!Instynkt Niebezpieczeństwa}\label{sec:Instynkt Niebezpieczeństwa} (3 punkty Szybkości) - Jeśli zostaniesz zaatakowany z zaskoczenia, niezależnie czy przez istotę, przedmiot lub zagrożenie środowiskowe (np: drzewo spadające na Ciebie) możesz się przemieścić na bliski zasięg zanim nastąpi atak. Jeśli poruszenie się uniemożliwi go, jesteś bezpieczny. Jeśli atak dalej może Cię potencjalnie dosięgnąć - jeśli atakująca istota może się poruszyć, by Cię doścignąć, jeśli atak wypełnia obszar zbyt duży, by uciec itp - ta zdolność nie oferuje żadnych korzyści. Umożliwienie.

\textbf{Zmysł Niebezpieczeństwa}\index{Zdolności!Alfabetycznie!Zmysł Niebezpieczeństwa}\label{sec:Zmysł Niebezpieczeństwa} (1 punkt Szybkości) - Twoje rzuty na inicjatywę są ułatwione. Płacisz koszt tej zdolności za każdym razem, gdy z niej korzystasz. Umożliwienie.

\textbf{Eksplorator Ciemności}\index{Zdolności!Alfabetycznie!Eksplorator Ciemności}\label{sec:Eksplorator Ciemności} - Ignorujesz kary do każdej akcji (wliczając w to walkę) w warunkach ograniczonego śWiatła lub bardzo ciasnych miejscach. Jeśli posiadasz także zdolność Dostosowanie Oczu, możesz działać bez kary nawet w absolutnej ciemności. Jesteś wyszkolony w skradaniu się, gdy przebywasz w warunkach ograniczonego śWiatła lub absolutnej ciemności. Umożliwienie. 

\textbf{Powłoka Ciemnej Materii}\index{Zdolności!Alfabetycznie!Powłoka Ciemnej Materii}\label{sec:Powłoka Ciemnej Materii} (5 punktów Intelektu) - Przez następną minutę, pokrywasz się powłoką ciemnej materii. Twój wygląd zamienia się w ciemną sylwetkę, i zyskujesz atut na testach skradania się i otrzymujesz +1 do Pancerza. Powłoka cimnej materii działa zgodnie z Twoimi życzeniami, i jeśli zastosujesz poziom Wysiłku do dowolnego fizycznego zadania kiedy powłoka istnieje, otrzymujesz dodatkowy poziom Wysiłku za darmo do tego samego zadania. Akcja, by rozpocząć. 

\textbf{Płaszcz Ciemnej Materii}\index{Zdolności!Alfabetycznie!Płaszcz Ciemnej Materii}\label{sec:Płaszcz Ciemnej Materii} (4 punkty Intelektu) - Wstęgi ciemnej materii skupiają się i wiją wokół Ciebie przez jedną minutę. Ten płaszcz ułatwia Twoje rzuty na Obronę Szybkości, zadaje 2 punkty obrażeń każdemu, kto Cię dotknie lub uderze w walce wręcz, i zapewnia Ci +1 do Pancerza. Akcja, by rozpocząć. 

\textbf{Cios Ciemnej Materii}\index{Zdolności!Alfabetycznie!Cios Ciemnej Materii}\label{sec:Cios Ciemnej Materii} (4 punkty Intelektu) - Kiedy atakujesz przeciwnika w dalekim zasięgu, ciemna materia skupia się wokół Twojego celu i chwyta jego kończyny, przytrzymując co w miejscu i ułatwiając Twój atak o dwa kroki. TA zdolność działa z dowolnym rodzajem ataku, który wykonujesz (wręcz, dystansowy, energetyczny itp.). Umożliwienie.

\textbf{Budowla Ciemnej Materii}\index{Zdolności!Alfabetycznie!Budowla Ciemnej Materii}\label{sec:Budowla Ciemnej Materii} (5 punktów Intelektu) - Możesz uformować ciemną materię w dużą strukturę skłaającą się z do 10 sześcianów o wysokości 3 meytrów. Ta struktur może być skomplikowana, ale wszystko ma ten sam matowy, czarny kolor, który nie odbija śWiatła. Poza tym, ta struktura może posiadać różne gęstości, powierzchnie i możliwości. To oznacza, że mogą w niej wystąpić okna, drzwi z kłódkami, meble, i nawet elementy ozdobne, tak długo jak wszystko jest czarne jak noc. Dla przykładu, możesz ukształtować ciemną materię w dużą, obronną strukturę; wytrzymały most o długości 30 metrów; lub cokolwiek innego. Ta struktura jest kreacją 6 poziomu i istnieje przez 24 godziny. Nie możesz utrzymać więcej niż jednej takiej struktury w danym czasie. Akcja. 

\textbf{Implant Wizualnej Identyfikacji} (1 punkt Intelektu) - Z dostępem do komputera, Twój implant podłącza się do niego natychmiastowo i zyskuje pewne informacje o tym, na co patrzysz. Otrzymujesz atut na zadaniu powiązanym z tą osobą lub obiektem. Akcja.

\textbf{Sen na Jawie}\index{Zdolności!Alfabetycznie!Sen na Jawie}\label{sec:Sen na Jawie} (4 punkty Intelektu) - Umieszczasz kogoś w śnie na jawie, kontrolując go przez jedną minutę. Możesz wpłynąć na cel w dalekim zasięgu, który możesz dostrzec, lub na cel w zasięgu 16 km, jeśli masz jego włos lub fragment skóry. Dla wszystkich z zewnątrz, cel po prostu stoi (lub leży) nieruchomo.  Ale wewnątrz, rzeczywistość kontorlowana przez Ciebie (lub sen wewnątrz snu, jeśli cel spał) jest wszystkim, czego cel doświadcza. Jeśli cel pragnie uciec, może spróbować Obrony Intelektu w każdej rundzie, by sie wyzwolić, ale cel może nie zdać sobie sprawy ze swojego stanu. Albo ten sen rozgrywa się według pewnego planu, który przygotowałeś kiedy korzystasz z tej zdolności, albo korzystasz ze swoich własnych akcji (co przymusza Cię do podobnego stanu jak cel tej zdolności), co pozwala Ci zmieniać sen z rundy na rundę. Korzystanie z tej zdolności na śpiącym celu ułatwia zadanie. Akcja by rozpocząć; aby kontrolować sen na bieżąco, akcja w każdej rundzie. 

\textbf{Oszałamiający Atak}\index{Zdolności!Alfabetycznie!Oszałamiający Atak}\label{sec:Oszałamiający Atak} (3 punkty Mocy) - Uderzasz swojego wroga we właściwe miejsce, oszałamiając go, tak, że jego akcje w następnej turze są utrudnione. Ten atak zadaje normalne obrażenia. Akcja. 

\textbf{Oszałamiające Światło}\index{Zdolności!Alfabetycznie!Oszałamiające Światło}\label{sec:Oszałamiające Światło} (2 punkty Intelektu) - Wysyłasz promień oszałamiających kolorów do istoty w średnim zasięGu i, jeśli odniesiesz sukces, zadajesz 2 punkty obrażeń celowi. Dodatkowo, ataki tej istoty są utrudnione w jej następnej turze, chyba, że cel polega przede wszystkim na innym zmyśle niż wzrok. Akcja.

\textbf{Deaktywacja Mechanizmu}\index{Zdolności!Alfabetycznie!Deaktywacja Mechanizmu}\label{sec:Deaktywacja Mechanizmu} (5+ punktów Szybkości) - Wykonujesz atak wręcz, który nie zadaje obrażeń, przeciwko maszynie. Zamiast tego, jeśli ten atak trafi, wykonaj drugi rzut bazujący na Szybkości. Jeśli jest on udany, maszyna 3 poziomu lub nizszego jest zdeaktywowana na 1 minutę. Na każdy dodatkowy poziom Wysiłku, który zastosujesz, możesz wpłynąć na maszynę o 1 poziom wyższą lub zwiększyć czas trwania o 1 minutę. Jeśli posiadasz zdolności Popsucie Maszyny lub Rozbrojenie Urządzenia (lub zdolność działającą w podobny sposób), kiedy stosujesz poziom Wysiłku do dowolnej z nich, otrzymujesz dodatkowy, darmowy poziom Wysiłku. Akcja.

\textbf{Śmiertelna Salwa} \index{Zdolności!Alfabetycznie!Śmiertelna Salwa}\label{sec:Śmiertelna Salwa} - Przez następną minutę, wszystkie ataki dystansowe, które wykonujesz, zadają dodatkowe 2 punkty obrażeń. Akcja, by rozpocząć. 

\myability{Śmiertelny Cios} (5 punktów Mocy) - Jeśli uderzysz cel poziomu 3 lub mniejszego bronią, w której jesteś wyszkolony, zabijasz cel natychmiastowo. Akcja.

\myability{Śmiercionośny Rój} (6 punktów Intelektu) - Jeśli jesteś w lokacji, w której jest możliwe, by przybył Twój rój istot ze zdolności Wpływ na Rój, możesz wezwać ten rój na 10 minut. W tym czasie, telepatycznie im rozkazujesz tak długo, jak są w dalekim zasięgu. Mogą one utrudnić zadania dowolnego lub wszystkich wrogów, lub mogą ona skupić się i zaatakować wszystkich wrogów w bliskim zasięgu od siebie (wszystcy muszą być w dalekim zasięgu od Ciebie). Atakujący rój zadaje 4 punkty obrażeń. Kiedy Twoje istoty są w dalekim zasięgu, możesz przemówić do nich telepatycznie i patrzeć na świat ich zmysłami. Akcja, by rozpocząć. (Roje zazwyczaj nie posiadają statystyk, ale jeśli zaistnieje taka potrzeba, typowy rój jest poziomu 2. Tylko ataki obszarowe wpływają na Twój rój.)

\myability{Dotyk Śmierci} (6 punktów Intelektu) - Zbierasz energię na czubku swoich palców i dotykasz istotę. Jeśli cel jest NPC lub istotą o poziomie 3 lub mniej, umiera. Jeśli cel jest BG dowolnego poziomu, spada on o poziom niżej na liczniku zdrowia. Akcja.

\myability{Debata} (3 punkty Intelektu) - W dowolnym zgromadzeniu 2 lub więcej ludzi, którzy próbują dociec prawdy lub podjąć jakąś decyzję, możesz wpłynąć na werdykt swoją mistrzowska retoryką. Jeśli masz 1 minutę lub więcej by argumentować, albo decyzja idzie po Twojej myśli, albo, jeśli ktoś inny ma dobry kontrargument, wszelkie powiązanie testy na perswazję lub oszustwo, są ułatwione o dwa kroki. Akcje by rozpocząć, jedna minuta, by zakończyć. 

\myability{Bolesne Uderzenie} (4 punkty Szybkości) - Wykonujesz atak, by zadaćć  bolesne uderzenie. Ten atak jest utrudniony. Jeśli trafi, istota otrzymuje dodatkowe 2 punkty obrażeń na końcu swojej następne rundy, a jej ataki są utrudnione do końca następnej rundy. Akcja.

\myability{Deszyfracja} (1 punkt Intelektu) - Jeśli spędzisz 1 minutę badając pismo lub kod w języku, którego nie rozumiesz, możesz wykonać test Intelektu o trudności 3 (lub więcej, bazując na skomplikowaniu języka lub kodu) to zrozumieć ogólnikowo tę wiadomość. Akcja by rozpocząć. 

\myability{Głębokie Przemyślenia} (6 punktów Intelektu) - Kiedy tworzysz plan, który angażuje Ciebie i Twoich przyjaciół, którzy pracujecie razem, by coś osiągnąć, możesz zadać MG jedno bardzo ogólne pytanie o tym, co najpewniej się wydarzy jeśli wykonacie plan, i otrzymasz prostą, krótką odpowiedź. Dodatkowo, wszyscy otrzymujecie atut na jednym rzucie związanym z wcieleniem planu w życie, tak długo, jak go przeprowadzicie w ciągu paru dni od stworzenia planu. Akcja.

\myability{Głębokie Rezerwy} - Kiedy inni są wykończeni, Ty dalej masz w sobie energię. Raz dziennie, możesz przemieścić 5 punktów pomiędzy swoimi Pulami w dowolnej kombinacji, w tempie 1 punktu na rundę. Dla przykładu, możesz przemieścić 3 punkty Mocy do Szybkości i 2 punkty Intelektu do Szybkości, co w sumie zajęłoby Ci 5 rund. Akcja.

\myability{Ulepszenie Szybkości} - Uzyskujesz 6 dodatkowych punktów w swojej Puli Szybkości. Umożliwienie.

\myability{Przewodnik Głębokich Wód} - Kiedy pod wodą, każda istota, którą wybierzesz i która Cię widzi, ma atut na testach pływania. Umożliwienie.

\myability{Chroń Wszystkich Niewinnych} - Chronisz wszystkich w bliskim zasięgu, których określiłeś jako niewinnych korzystając ze swojej zdolności Osąd. Obrona Szybkości tych istot uzyskuje atut. Umożliwienie.

\myability{Ochrona Niewinnego} (2 punkty Szybkości) - Przez następne 10 minut, jeśli ktoś, kogo kreśliłęś jako niewinnego korzystając ze swojej zdolności Osąd stanie obok Ciebie, ta istota dzieli wszelkie defensywne przewagi, które posiadasz, z wyłączeniem zwykłej zbroi. Te przewagi to między innymi Obrona Szybkości z posiadanej tarczy, Pancerz zapewniany przez pole siłowe itp. Dodatkowo, rzuty na Obronę Szybkości wykonywane przez niewinną istotę uzyskują atut. Możesz chronić tylko jedną niewinną istotę na raz. Akcja, by rozpocząć.

\myability{Broń Defensywna} - Kiedy korzystasz z swojej zaczarowanej broni, jesteś wyszkolony w Obronie Szybkości. Umożliwienie.

\myability{Ochrona Przed Robotami} - Uczyłeś się o swoich wrogach i jesteś wyszkolony w przewidywaniu posunięć, które robot lub maszyna najpewniej wykonają w walce. Rzuty Obronne, które wykonujesz przeciwko tym wrogom są ułatwione. Umożliwienie.

\myability{Mistrz Obrony} - Za każdym razem, gdy wykonujesz udany rzut Obrony Szybkości, możesz wykonać natychmiastowy atak przeciwko swojemu wrogowi. (Jeśli posiadasz zdolność Unik i Rewanż, możesz tę zdolność zamienić z tamtą.) Twój atak musi być tego samego typu (wręcz bronią, zasięgowy bronią lub nieuzbrojony) jak atak, przed którym się bronisz. Jeśli nie masz gotowego odpowiedniego typu broni, nie możesz skorzystać z tej zdolności. Umożliwienie.

\myability{Usprawnienie Defensywne} - Poprzez usprawnienie swojego systemu nerwowego i immunologicznego, jesteś wyszkolony w Obronie Mocy i Obronie Szybkości. Umożliwienie.

\myability{Obronna Teleportacja} (4 punkty Szybkości) - Wchodzisz w stan podwyższonego reagowania, tak, że kiedy jesteś uderzony dostatecznie mocno, by odnieść obrażenia, teleportujesz się o bliski zasięg w przypadkowym kierunku (nie wyżej lub niżej, niż jesteś) by pomóc Ci poradzić sobie z obrażeniami z ataku. Twoja Obrona Szybkości jest ułatwiona przez jedną minutę. Akcja.

\myability{Pole Obronne} - Dzięki implantom podskórnym, permanentnemu zaklęciu, modyfikacjom obcych lub czemuś podobnemu, posiadasz pole siłowe, które promieniuje na 2,5 cm od Twojego ciała i zapewnia Ci teraz +2 do Pancerza. Umożliwienie. 

\myability{Defensywne Znikanie} (2 punkty Intelektu) - Możesz zmieniać swoją fazę, tak, że pewne ataki przenikają przez Ciebie, nie robiąc Ci krzywdy. Przez następne 10 minut, uzyskujesz atut na Obronie Szybkości, ale w tym czasie tracisz korzyści z zbroi, którą nosisz. Akcja, by rozpocząć. 

\myability{Definiowanie Dołu} (4 punkty Intelektu) - Naturalna grawitacja w średnim zasięgu od Ciebie, w promieniu bliskiego zasięgu, zmienia się tak, że na parę sekund grawitacja działa w wybranym przez Ciebie kierunku (w górę, w górę i na południe, zachód itp.), a potem wraca do normy. Dotknięte cele mogą zostać przeniesiono do 6 metrów i otrzymać parę punktów obrażeń. Akcja.

\myability{Odbicie Ataku} (1 punkt Intelektu) - Korzystając ze swojego umysłu, chronisz się przed nadchodzącym atakiem. Przez następne 10 minut, jesteś wyszkolony w Obronie Szybkości. Akcja, by rozpocząć.

\myability{Ogarnięcie Sytuacji} - Podczas dochodzenia, Twoje pytania czasami wywołują wściekłą odpowiedź lub nawet przemoc. Poprzez udawanie, werbalne rozpraszanie lub podobne akcje, zapobiegasz, by żyjący wróg zaatakował kogoś lub cokolwiek przez jedną rundę. Akcja.

\myability{Rozkazująca Postura} (2 punkty Intelektu) - Emanujesz pewnością siebie, wiedzą i charyzmą przez następną godzinę. Twoja postura jest tak wpływowa, że każdy, kto Cię ujrzy, automatycznie rozumie, że jesteś kimś ważnym, utytułowanym i o wielkim autorytecie. Kiedy mówisz, obcy, którzy obecnie nie atakują poświęcają przynajmniej jedna rundę, by z Tobą porozmawiać i Cię wysłuchać. Jeśli przemawisz do grupy obcych, którzy Cię rozumieją, możesz spróbować, by skontaktowali Cię z ich liderem lub do niego zaprowadzili. Uzyskujesz darmowy poziom Wysiłku który możesz zastosować do jednego testu na Perswazję, który próbujesz zdać w ciągu tego okresu. Akcja, by rozpocząć. 

\myability{Osąd} - Osądzasz jedną istotę w bliskim zasięgu jako niewinną lub winną, bazując na Twojej ocenie sytuacji lub odczuciach. W innych słowach, ktoś, kto jest przez Ciebie naznaczony jako niewinny może być niewinny w pewnych okolicznościach, lub może być ogólnie niewinny pewnych zbrodni (jak morderstwo, duża kradzież itp.). Podobnie, możesz osąDzić, że istota jest winna konkretnej zbrodni lub ogólnie winna okrutnych czynów. Poprawność Twojego osądu nie jest ważna, tak długo, jak wierzysz, że osąd jest prawdziwy. MG może poprosić Cię o uzasadnienie. Od tego momentu, Twoja próby socjalnych interakcji z kimś, kto jest niewinny uzyskują atut, a ataki przeciwko tym, którzy są winni, również zyskują atut.Możesz zmienić swój osąd, ale to wymaga kolejnej akcji osądzania. Korzyści z osądu trwaja dopóki zmienisz swoje zdanie lub otrzymasz dowód, że nie masz racji. Akcja. (Korzyśći zapewnione przez Osąd stosują się do postaci korzystającej z tej zdolności, jej sprzymierzeńców i wszystkich, którzy słyszą - z pierwszej lub drugiej ręki - o osadzie i w niego wierzą.)

\myability{Przeznaczenie Wielkości} - Cieszysz się niesamowitym szczęściem, tak jakby ktoś  się Tobą opiekował i chronił Cię od nieszczęść. Kiedy miałbyś spaść na liczniku obrażeń, wykonaj rzut na Obronę Intelektu przeciwko trudności ustalonej przez wroga lub efekt. Jeśli Ci się powiedzie, nie spadasz niżej na liczniku obrażeń. Jeśli upadek miałby nastapić ze względu na obniżenie jednej z Pul do 0, ta Pula dalej wynosi 0 - po prostu nie ponosisz negatywnych skutków bycia zranionym lub krytycznie rannym. Jeśli jednak miałbyś spaść na poziom martwy, udany rzut na Obronę Intelektu utrzymuje Cię na 1 punkcie w Puli i dalej pozostajesz krytycznie ranny. Umożliwienie.

\myability{Niszczenie Metalu} (3 punkty Intelektu) - Natychmiastowo rozrywasz, rozdzierasz lub przebijasz metalowy obiekt który widzisz, który znajduje isę w średnim zasięgu i nie większy niż połowa Twojego rozmiaru. Wykonaj test na Intelekt by zniszczyć obiekt - to zadanie jest ułatwione o 3 stopnie w porównaniu z niszczeniem brutalną siłą. Akcja. 

\myability{Niszczyciel} (6 punktów Mocy) - Jeśli uda Ci się test Mocy na uszkodzenie obiektu, zamiast zmniejszyć pozycję na jego liczniku obrażeń, obiekt ten spada na sam dół licznika i jest zniszczony. Akcja.

\myability{Wykrycie Życia} (3+ punkty Mocy) - W pełni świadomie wysyłasz puls swojej energii życiowej. Wykrywasz wszelkie żywe istoty w średnim zasięgu, nawet jeśli są za ukryte, ale nie jeśli są za polem siłowym. Kiedy wykrywasz istotę, znasz jej ogólną lokalizację (z dokładnością do bliskiego zasięgu). Jeśli zastosujesz dwa dodatkowe poziomy Wysiłku, możesz zwiększyć zasięg wykrywania do długiego. Akcja.

\myability{Wgląd w Urządzenie} (3 punkty Intelektu) - Kiedy badasz nieznane, obce, lub super-technologiczne urządzenie, możesz zadać MG jedno pytanie, by zyskać wzgląd w jego możliwości, sposób funkcjonowania, jak można je aktywować lub deaktywować, jakie ma słabości (jeśli jakiekolwiek), jak można je naprawić i tym podobne. Ta zdolność dotyczy trudnych lub dziwnych cech poza tymi, które można łatwo zydentyfikować przy pomocy odpowiednich umiejętności lub wiedzy. Akcja.

\myability{Wierny Obrońca} (2 punkty Mocy lub Intelektu) - Wybierz jedną postać, którą widzisz. Ta postać zostaje pod Twoją obroną. Jesteś wyszkolony we wszelkich akcjach, by ją znaleźć, leczyć, wchodzić w interakcję i chronić. Możesz mieć tylko jedną postać pod swoją obroną w danym czasie. Akcja, by rozpocząć.

\myability{Diamagnetyzm} - Namagnesowujesz niemetaliczny przedmiot w średnim zasięgu, tak, że może być on celem innych magnetycznych mocy. Tak więc, możesz go poruszać przy pomocy Poruszania Metalu. Przy pomocy Odparcia Metalu, jesteś wyszkolony w zadaniach Obrony Szybkości, niezależnie od tego, czy nadchodzący cios korzysta z metalu. I tak dalej. Umożliwienie.

\myability{Wymiarowy Ścisk} (2+ punkty Intelektu) - Przeciskasz się przez pośredni wymiar, co pozwala Ci na natychmiastowe pojawienie się wszędzie w średnim zasięgu, gdzie masz czysty szlak przejścia. Możesz przejść przez bariery pomiędzy, jeśli posiadają otwartą przestrzeń, przez którą przeszłaby Twoje głowa - około 30 cm na 30 cm. W dodatku do zwykłych opcji korzystania z Wysiłku, możesz z niego skorzystać, by przecisnąć się przez mniejsze otwory w barierach - każdy poziom Wysiłku redukuje wtedy minimalny otwór o 1/4. Lądujesz bezpiecznie, kiedy korzystasz z tej zdolności. Akcja. 

\textit{Wymiar pośredni} - wymiar, gdzie odległości są mniejsze niż w innych wymiarach, tak, że podróż jest szybsza niż normalnie.

\myability{Nieczyste Zagrania} (2 punkty Szybkości) - Rozpraszasz, oślepiasz, wkurzasz, utrudniasz lub w inny sposób wchodzisz w interakcję z wrogiem, utrudniając jego ataki i obronę na jedną minutę. Akcja.

\myability{Rozbrojenie Urządzenia} (3 punkty Szybkości) - Z bystrym okiem i szybkimi ruchami, zakłócasz pewne funkcje robota lub maszyny i zadajesz jej jedną z poniższych przeszkód:
\begin{itemize}
\item Wszystkie jej zadania są utrudnione przez jedną minutę.
\item Ma połowę normalnej szybkości.
\item Nie może podjąć akcji przez jedną rundę.
\item Zadaje o 2 punkty  obrażeń mniej (minimum 1 punkt) przez jedną minutę. 
\end{itemize}

Musisz dotknąć robota lub maszyny, by ją rozbroić (jeśli wykonujesz atak, nie zadajesz żadnych obrażeń). Akcja.

\myability{Zanikanie} (4 punkty Intelektu) - Zaginasz światło tak, by zniknąć z oczu. Jesteś niewidoczny dla innych istot przez 10 minut. Kiedy jesteś niewidzialny, jesteś wyspecjalizowany w skradaniu się i Obronie Szybkości. Ten efekt zanika, jeśli zrobisz coś, by ujawnić swoją pozycję - zaatakujesz, skorzystasz ze zdolności, przesuniesz wielki przedmiot itp. Jeśli to się wydarzy, możesz odzyskać pozostały czas niewidzialności, poprzez podjęcie akcji, by się schować. Akcja by rozpocząć lub rozpocząć ponownie.

\myability{Rozbrojenie} (5 punktów Szybkości) - Próbujesz wykonać zadanie Szybkości, by rozbroić wroga jako część swojego ataku wręcz. Jeśli Ci się uda, Twój atak zadaje 3 dodatkowe punkty obrażeń, a broń wroga wypada z jego rąk, lądując o 6 m dalej. Jeśli Ci się nie uda, dalej możesz wykonać normalny atak, ale nie zadaje on ekstra obrażeń lub rozbraja wroga, którego trafiłeś. Akcja.

\myability{Cios Rozbrajający} (3 punkty Szybkości) - Twój atak zadaje 1 punkt obrażeń mniej i rozbraja Twojego wroga, tak, że jego broń ląduje 3 m dalej. Jeśli Twoja wybraną bronią jest bicz, możesz umiećić tę broń w swoich rękach; jeśli Twoją wybraną bronią jest łuk lub inna broń dystansowa, która strzela fizyczną amunicją, możesz zamiast tego ``przybić'' rozbrojoną broń do niedalekiego obiektu lub struktury. Wybranie jednej z tych opcji utrudnia atak.). Akcja.

\myability{Silny Umysł} - Masz +3 do Pancerza przeciwko atakom i efektom umysłowym i Intelektu, które zadają obrażenia. Rzuty Obrony, które wykonujesz przeciwko próbom konfuzjowania, perswazii ,przerażenia lub w inny sposób wpłynięcia na Ciebie są ułatwione. Umożliwienie.

\myability{Wspieranie Uważności} (7 punktów Intelektu) - Utrzymujesz uwagę swoich sprzymierzeńców w gotowości, okazjonalnie ich pytając, opowiadając dowcipy itp. Po spędzeniu 24 godzin z Tobą, Twoi sprzymierzeńcy mogą zastosować darmowy poziom Wysiłku do dowolnego zadania inicjatywy, które podejmują. Ten efekt trwa ciągle, kiedy pozostajesz w towarzystwie swoich sprzymierzeńców. Kończy się, jeśli odejdziesz, ale powraca automatycznie, jeśli wrócisz do swojej drużyny w ciągu 24 godzin. Jeśli opuścisz grupę na więcej niż 24 godziny, musisz spędzić następne 24 godziny razem z nią, by reaktywować tą korzyść.  Musisz wydać punkty z Puli Intelektu na każde 24 godziny, gdy chcesz korzystać z korzyści tej zdolności. Umożliwienie.

\myability{Przebranie} - Jesteś wyszkolony w przebieraniu się. Możesz zmienić swoją posturę, głos, manieryzmy i włosy, by wyglądać jak ktoś inny tak długo, jak nie rezygnujesz z przebrania. Jednakż,e jest niezwykle trudno przebrać się za konkretną osobę bez zestawu charakteryzacyjnego do Twojej dyspozycji. Umożliwienie. 

\myability{Przebranie Innej Osoby} (4+ punkty Intelektu) - Stosujesz swoją zdolność do zmiany kształtów do innej istoty Twojego rozmiaru lub mniejszej, dając jej formę, którą sam mógłbyś przyjąć. Działa to przez około 10 minut. W dodatku do normalnych opcji Wysiłku, możesz zastosować go, by zwiększyć czas trwania tej mocy - 1 poziom Wysiłku zwiększa go do godziny, 2 poziomy zwiększają go do dnia. Istota może wrócić do swojej normalnej formy poświęcając akcję, ale nie może potem wrócić do porzuconej formy. Akcja.

Najpewniej nie możesz skorzystać z Przebrania Innej Osoby, by przebrać się za istotę znacząco odmienną od Ciebie, jak np: człowiek przebierający się za robota, zwierzę lub kryształowego kosmitę. 

\myability{Zniechęcenie} (1 punkt Intelektu) - Utrudniasz wszelkie akcje podejmowane przez dowolną liczbę celów w średnim zasięgu, które mogą Cię zrozumieć. Wybierasz, które z celów są dotknięte. Dotknięte cele mają utrudnione akcje na jedną rundę. Umożliwienie. 

\myability{Przeszkadzający Dotyk} (1+ punkt Mocy) - Możesz zamieć swój Bieg Fazowy w atak wręcz poprzez celowe akcje przeciwko innej istocie na swojej trasie. Kiedy to robisz, Twój dotyk wyzwala emisję energii, która zadaje 2 punkty obrażeń celowi (zignoruj Pancerz). Niezależnie od tego, czy trafisz czy nie, Twój ruch (i tura) kończy się natychmiastowo, co sprawia, że jesteś w bliskim zasięgu Twojego celu. Jeśli zastosujesz Wysiłek, by zwiększyć obrażenia zamiast ułatwić atak, zadajesz 2 dodatkowe punkty obrażeń na poziom Wysiłku (zamiast 3 punktów) - cel otrzymuje 1 punkt obrażeń nawet, jeśli nie uda Ci się test na atak. Umożliwienie.

\myability{Dalekie Spojrzenie} (5 punktów Intelektu) - Wiesz, ze odległość i przestrzeń to iluzje. Koncentrujesz się, by stworzyć niewidzialny, nieruchomy sensor w lokacji, którą wcześniej odwiedziłeś lub postrzegałeś (jeśli MG uzna za stosowne, możesz musieć zdać test Intelektu, jeśli przestrzeń ta jest magicznie chroniona). Sensor działa przez 1 godzinę. Kiedy już zostanie stworzony, możesz się skoncentrować, by widzieć, słyszeć i wąchać poprzez ten sensor. Nie zapewnia on zdolności postrzegania wykraczających poza normę. Akcja by stworzyć, akcja by skorzystać. 

\myability{Interfejs Zasięgowy} (2 punkty Intelektu) - Możesz aktywować, dezaktywować lub kontrolować maszynę na dalekim zasięgu tak, jakbyś stał obok niej, nawet jeśli normalnie wymaga dotyku. Jeśli nigdy wcześniej nie miałek kontaktu z daną maszyną, zadanie jest utrudnione o dwa stopnie. By skorzystać z tej zdolności, musisz rozumieć funkcję tej maszyny, musi ona być Twojego rozmiaru lub mniejsza, i nie może być podłączona do innej inteligencji (lub sama być inteligentną). Akcja. 

\myability{Zamglenie} (2 punkty Intelektu) - Modyfikujesz jak chętna istota w średnim zasięgu odbija światło na jedną minutę. Cel nagle miga pomiędzy normalnym wyglądem a chmurą ciemności. Cel ma atut na Obronie Szybkości dopóki ten efekt nie przeminie. Akcja, by rozpocząć. 

\myability{Nurek} - Możesz bezpiecznie nurkować w wodzie skacząc z wysokości do 30 metrów i możesz znieść ciśnienie w wodzie tak głębokiej jak 30 metrów. Umożliwienie.

\myability{Odbicie Ataków} (4 punkty Szybkości) - Na jedną minutę automatycznie odbijasz lub unikasz wszystkich ataków dystansowych korzystających z pocisków. Jednakże, w swojej następnej turze po tych unikach wszystkie Twoje inne akcje są utrudnione. Akcja, by rozpocząć. 

\myability{Rozdzielenie Jaźni} (7 punktów Intelektu) - Dzielisz swoją jaźń na dwie części. Przez jedną minutę, możesz wziąć dwie akcje w swojej turze, ale tylko jedna z nich może korzystać ze specjalnej zdolności. Akcja. 

\myability{Boska Interwencja} (2 punkty Intelektu lub 2 punkty Intelektu + 4 PD) - Prosisz boga, by zainterweniował przez wzgląd na Ciebie, zazwyczaj przeciwko istocie w dalekim zasięgu, zmieniając jej życie w mały sposób poprzez wywołanie na niej większego specjalnego efektu. Większy specjalny efekt przypomina naturalne wyrzucenie 20 na k20. Jeśli pragniesz większego efektu i MG się zgodzi, możesz poprosić o Boską Interwencję z potężniejszym efektem, co przypomina wtrącenie MG zainicjowane przez MG na graczach. W takim wypadku, Boska Interwencja kosztuje także 4 PD, jej efekt może nie być dokładnie taki, jaki byś chciał, i nie możesz poprosić znowu o Boską Interwencję przez tydzień. Akcja. 

\myability{Boska Wiedza} - Jesteś wyszkolony we wszystkich zadaniach powiązanych z wiedzą o boskich istotach. Umożliwienie. 

\myability{Boski Blask} (2 punkty Intelektu) - Twoja postać wzywa boski blask z niebios by ukarać niecny cel w długim zasięgu, zadając mu 4 punkty obrażeń. Jeśli cel jest demonem, duchem lub czymś podobnym, stoi w bezruchu, gdy boska gloria przelatuje przez niego i nie może podjąć akcji w swojej następnej turze. Kiedy ktoś zostanie ofiarą tego blasku, nie można ponownie zostać sparaliżowanym przez glorię przez następnych kila godzin. Akcja. 

\myability{Boski Symbol} (5+ punktów Intelektu) - Przywołujesz boską moc poprzez narysowanie jaśniejącego symbolu w powietrzu przy pomocy palca. Poruszające się kolumny boskiej światłości przebijają do 45 celów w dalekim zasięgu. Udany test na atak zadaje mi 5 punktów obrażeń. Jeśli korzystasz z Wysiłku, by zwiększyć obrażenia, zadajesz dodatkowe 2 punkty obrażeń na poziom Wysiłku (zamiast 3 punktów); cel otrzymuje 1 punkt obrażeń nawet jeśli nie uda się test na atak. Akcja.

 \myability{Czy Ty Wiesz w Ogóle Kim Jestem?} (3 punkty Intelektu) - Zachowujesz się jak ktoś, kto jest sławny i przywykły do swojego przywileju. Werbalnie atakujesz żywego wroga, który Cię słyszy i rozumie z tak wielką siłą, że nie może on podjąć żadnej akcji, wliczając w to ataki, przez jedną rundę. Niezależnie od tego, czy odnosisz sukces, czy porażkę, następna akcja, którą cel podejmuje po Twoim użyciu zdolność jest utrudniona. Akcja.
 
 \myability(Unik i Odporność) (3 punkty Szybkości) - Możesz przerzucić dowolny rzut na Obronę Mocy, Szybkości lub Intelektu i zachować lepszy z dwóch wyników. Umożliwienie. 
 
 \myability{Unik i Rewanż} (3 punkty Mocy) - Jeśli atak wręcz Cię chybi, możesz natychmiast wykonać atak wręcz w odpowiedzi, ale nie więcej niż jeden na turę. Umożliwienie. 
 
 \myability{Podwójny Cios} (3 punkty Mocy) - Kiedy dzierżysz dwie bronie, możesz wybrać wykonanie jednego ataku przeciwko wrogowi. Jeśli trafisz, zadajesz obrażenia od dwóch broni plus 2 dodatkowe punkty obrażeń, a ponieważ wykonałeś pojedynczy atak, Pancerz celu odejmuje się od obrażeń tylko raz. Akcja.

\myability{Szczęki Smoka} (6 punktów Intelektu) - Tworzysz i kontrolujesz ``unoszący się'' półmaterialny konstrukt  magii w dalekim zasięgu, którzy przypomina głowę smoka. Ten konstrukt może istnieć przed maksymalnie godzinę, do momentu, aż zostanie zniszczony, lub gdy rzucić kolejne zaklęcie. Jest to konstrukt 4 poziomu który zadaje 6 punktów obrażeń swoi ugryzieniem (musi mu zostać wydany rozkaz). Kiedy ten konstrukt istnieje, możesz nim manipulować wielkimi obiektami, nosić ciężkie przedmioty w jego paszczy lub atakować nim wrogów. Jeśli chcesz, by zaatakował, musisz skorzystać ze swojej akcji, aby nim bezpośrednio sterować. Akcja, by rozpocząć. 

\myability{Wysysanie na Zasięg}  - Twoje zdolności Wyssanie Istoty i Wyssanie Maszyny działają na cel w średnim zasięgu. Umożliwienie.

 \myability{Wyssanie Ładunku}  - Możesz wyssać moc z artefaktu lub urządzenie, co pozwala Ci odzyskać 1 punkt Intelektu na wyssany poziom. Odzyskujesz punkty w tempie 1 punktu na rundę i musisz być w pełni na tym skoncentrowany w każdej rundzie procesu. MG określa, czy urządzenie jest w pełni wyssane (najpewniej, jeśli jest to urządzenie trzymane w dłoni) lub posiada jakieś resztki mocy (najpewniej, jeśli jest dużą maszyną), Akcja by rozpocząć; akcja w każdej rundzie wysysania.
 
 \myability{Wyssanie Istoty} (3+ punktów Intelektu) - Możesz wyssać energię z żywej istoty, której dotykasz, zadając jej 3 punkty obrażeń i odzyskując 3 punkty w puli Mocy lub Szybkości. Akcja.
 
 \myability{Wyssanie Maszyny} (3+ punktów Intelektu) - Możesz wyssać moc z artefaktu lub zasilanego urządzania, które dotykasz. Jeśli cel jest robotem, zadajesz mu 3 punkty obrażeń i odzyskujesz 3 punkty w swojej puli Mocy lub Szybkości. Jesli cel jest obiektem, odzyskujesz punkty Mocy lub Szybkości równe poziomowi celu. Jeśli cel to zamanifestowany cypher, staje się on bezużyteczny wskutek rozładowania. Artefakty i podobne urządzenia muszą natychmiast wykonać test na wyczerpanie (artefakty z wyczerpaniem ``-'' są albo odporne na tę moc, albo doznają wyczerpania po wyrzuceniu 1 na k10 kiedy zaatakowane tą mocą). Akcja.
 
 \myability{Wyssanie Mocy} (5 punktów Szybkości) - Wpływasz na główne źródło mocy maszyny lub robota, zadając mu wszystkie 4 kondycje ze zdolności Rozbrojenie Urządzenia. Musisz w tym celu dotknąć robota (jeśli wykonujesz ata, nie zadaje on zwykłych obrażeń). Akcja.
 
 \myability{Wyciągnięcie Wniosków} (3 punkty Intelektu) - Po ostrożnej obserwacji i śledztwie (pytając jednego lub więcej BN-ów w danym temacie, przeszukując strefę lub folder itp.) trwających kilka minut, poznajesz ważny fakt. Ta zdolność to zadanie Intelektu trudności 3. Za każdym razem, gdy korzystasz z tej zdolności ponownie, zadanie jest utrudnione o dodatkowy poziom. Trudność wraca do 3 po zakończeniu 10-godzinnego odpoczynku. Akcja.
 
 \myability{Korzystając z Doświadczenia Życiowego} (6 punktów Intelektu) - Widziałeś i zrobiłeś w swoim życiu dużo, i te doświadczenia Ci się przydają. Zadaj MG jedno pytanie i otrzymasz ogólnikową odpowiedź. MG daje pytaniu odpowiedni poziom, tak więć im bardziej tajemna jest wiedza, o którą pytasz, tym trudniejsze zadanie. Ogólnie rzecz ujmując, wiedza, któą można zdobyć, szukając gdzieś indziej niż Twoja obecna lokacja, to poziom 1, a tajemnicza, mała znana wiedza o przeszłości ma poziom 7. Akcja.
 
 \myability{Straszny Las} (6 punktów Intelektu) - Manipulujesz wiatrem, mglą i cieniami, by wcielić pierwotny lęk przed tajemniczym lasem. Przez następną minutę, zyskujesz atut na zadaniach zastraszania. Istoty w średnim zasięgu mogą się przerazić - wykonaj odrębny test ataku Intelektu dla każdej istoty (jeśli jesteś większy niż normalnie wskutek zdolności Wielkie Drzewo lub z innego źródła, te rzuty są ułatwione). Sukces oznacza, że istoty są sparaliżowane strachem, nie ruszając się i nie podejmując akcji przez minutę lub do momentu, gdy zostaną zaatakowane. Pewne istoty bez umysłów są odporne na ten strach. Akcja.
 
\myability{Sen Staje się Prawdą} (4 punkty Intelektu) - Tworzysz senny obiekt dowolnego kształtu który możesz sobie wyobrazić Twojego rozmiaru lub mniejszego, który przyjmuje oczywistą substancję i ciężar. Ten obiekt jest wykonany bardzo prymitywnie i nie może mieć ruchomych części, więc możesz tworzyć miecz, tarczę, krótką drabinę itp. Senny obiekt ma masę przybliżoną do prawdziwego obiektu, jeśli tak wybierzesz. Twoje senne obiekty są wytrzymałe jak żelazo, ale jeśli nie zostajesz w dalekim zasięgu od nich, funkcjonują tylko przez minutę, zanim się rozwieją w nicość. Akcja.

\myability{Złodziej Snów} (2 punkty Intelektu) - Kradniesz poprzedni sen od danej żywej istoty w średnim zasięgu. Ta istota traci 2 punkty Intelektu (ignorujesz Pancerz) a Ty uczysz się czegoś o istocie, co wybierze MG - je naturę, część jej planów, jej pamięć itp. Akcja.

\myability{Iluzja Snów} (1 punkt Intelektu) - Wyciągasz obraz ze snu i umieszczasz go w prawdziwym świecie, gdzieś w długim zasięgu. Sen tra minutę i może być mały lub wypełnić obszar o przekątnej bliskiego zasięgu. Choć wydaje się on być fizyczny i solidny, sen jest niematerialny. Ten sen (scena, istota lub obiekt) jest statyczny, chyba, że wykorzystasz swoją akcję, by go ożywić. Jako część animacji, możesz przemieścić sen na krótki dystans w każdej rundzie, tak długo, jak pozostaje on w dalekim zasięgu od Ciebie. Jeśli poruszasz sen, może on wyemitować dźwięki i głosy, ale nie zapach. Bezpośrednia fizyczna interakcja lub przedłużona interakcja ze snem roztrzaskuje go w rozwiewającą się mgłę. Dla przykładu, zaatakowanie snu go roztrzaskuje, a także ciężar utrzymywania snu kiedy BN porusza się przez senną scenę lub rozmawia z senną istotą przez więcej niż parę rund. Akcja by rozpocząć, akcja by poruszać.  

\myability{Kierowca} - Jesteś wyszkolony we wszystkich zadaniach kierowania autem, ciężarówką lub motocyklem, wliczając w to zadania naprawy mechanicznej. Umożliwienie. 

\myability{Atak Podczas Kierowania} - Możesz wykonać atak lekką lub średnią bronią dystansową i spróbować prowadzić pojazd jako jedną akcję. Umożliwienie.

\myability{Podwójna Obrona}  - Kiedy dzierżysz dwie bronie, jesteś wyszkolony w Obronie Szybkości. Umożliwienie. 

\myability{Podwójne Rozproszenie Uwagi} (1+ punkty Szybkości) - Kiedy dzierżysz dwie bronie, następny atak Twojego wroga jest utrudniony, a jeśli zastosujesz Wysiłek na swoim następny ataku przeciwko temu samemu wrogowi, uzyskujesz darmowy poziom Wysiłku. Umożliwienie.

\myability{Podwójne Władanie Lekkimi Broniami} - Możesz korzystać z dwóch lekkich broni w tym samym czasie, czyniąc dwa osobne ataki w swojej turze jako jedną akcję. Dalej ogranicza Cię ilość Wysiłku, który możesz zastosować do jednej akcji, ale ponieważ czynisz dwa osobne ataki, Pancerz wroga liczy się do obydwu.. Cokolwiek co modyfikuje obrażenia stosuje się równo do dwóch ataków, chyba, że jest specyficznie powiązane z jedną z broni. Umożliwienie. 

\myability{Podwójne Władanie Średnią Broni} - Możesz korzystać w tym samym momencie z dwóch lekkich lub średnich borni (lub jednej lekkiej i jednej średniej), wykonując nimi dwa osobne ataki w swojej turze w jednej akcji. Ta zdolność działa pod innymi względami jak Podwójne Władanie Lekkimi Broniami. Umożliwienie.

\myability{Podwójnie Bronieni} - Możesz posiadać dwie chronione osoby ze zdolności Wierny Obrońca w danym czasie. Wybór drugiej bronionej osoby może być osobną akcją, lub możesz wybiec dwie bronione osoby w jednym momencie (i zapłacić koszt umiejętności tylko jeden raz). Chronione osoby muszą pozostać w odległości bliskiego zasięgu od siebie. Korzyści zapewniane przez Wiernego Obrońca stosuje się do obydwu tych ludzi. Jeśli Twoi chronieni ludzie się rozłączą, wybierasz, który z nich jest dalej chroniony. Jeśli potem się spotkają i zbliżą, obydwoje odzyskują korzyści bycia bronionymi natychmiastowo. Umożliwienie. 

\myability{Pojedynek na Śmierć i Życie} (5 punktów Szybkości) - Wybierz cel (jedną istotę, którą widzisz). Jesteś wyszkolony we wszystkich zadaniach polegających na walce z tą istotą. Kiedy z sukcesem zaatakujesz cel, zadajesz +5 obrażeń lub +7, jeśli istota walczy z kimś innym niż Ty. Możesz walczyć na śmierć i życie tylko z jedną istotą na raz. Pojedynek trwa do 1 minuty, lub do momentu, gdy go odwołasz. Akcja, by rozpocząć. 

\myability{Kopia} (2 punkty Mocy) - Sprawiasz, że Twója kopia pojawia się w dowolnym miejscu, które widzisz w średnim zasięgu. Kopia nie posiada ani ubrań, ani przedmiotów w momencie, gdy się pojawia. Kpi to BN 2 poziomu  6 punktami życia. Kopia słucha Twouch komend i jest Tobie wierna. Kopia istnieje, dopóki jej nie odeślesz korzystając z akcji, lub do momentu, aż zostanie zabita. Kiedy kopia znika, zostawia po sobie rzeczy, co nosiła z lub na sobie.  Jeśli kopia znika, bo została zamordowana, uzyskujesz 4 punkty obrażeń które ignorują Pancerz i tracisz swoją następną akcję. Akcja, by rozpocząć. 

\myability{Pył w Pył} (7 punktów Intelektu) - Dezintegrujesz jeden przedmiot, który jest mniejszy od Ciebie i którego poziom jest mniejszy lub równy Twojemu poziomowi. Musisz dotknąć obiekt, by na niego wpłynąć. Jeśli MG uzna to za stosowne po uwzględnieniu okoliczności, możesz zdezintegrować część obiektu (która niem oże być większa od Ciebie) zamiast całą rzecz. Akcja.

\section{E}

\myability{Trzęsienie Ziemi} (7 punktów Mocy) - Wywołujesz w gruncie pod sobą destruktywny rezonans i wywołujesz trzęsienie ziemi zlokalizowane w bardzo dalekim zasięgu od siebie. Ziemia w średnim zasięgu od epicentrum trzęsie się przez 5 minut, wywołując obrażenia u struktur i w terenie w tym obszarze. Budynki i teren się zapadają i odkształcają. W każdej rundzie, istoty w tym obszarze otrzymują albo 3 punkty obrażeń wskutek trzęsienia, albo 6, jeśli spadnie na nie część budynku. Akcja, by rozpocząć.

\myability{Echolokacja}  - Jesteś szczególnie wrażliwy na dźwięk i wibracje, tak bardzo, że postrzegasz swoje środowisko w średnim zasięgu nawet w ciemności. Umożliwienie.

\myability{Ulepszenie Umiejętności} - Wybierz jedną umiejętność niebojową, gdy zyskujesz tę zdolność. Uzyskujesz mniejszy efekt z tą umiejętnością na rzucie na k20 14 lub wyżej. Uzyskujesz większy efekt z tą umiejętnością, gdy wyrzucisz naturalne 19 lub wyżej. Możesz wybrać tę zdolność więcej niż raz. Za każdym razem, gdy ją wybierasz, musisz wybrać inną umiejętność niebojową. Umożliwienie.

\myability{Elastyczny Chwyt} (3 punkty Mocy) - Twoje ataki rozciągliwymi rękoma lub ciałem sa ułatwione. Jeśli trafisz, możesz chwycić cel, zapobiegając jego ruchowi w następnej turze. Kiedy trzymasz cel, jego ataki lub próby wyzwolenia się są utrudnione. Jeśli cel pragnie się wyzwolić, zamiast zaatakować, musisz odnieść sukces w teście Mocy, by utrzymać swój uchwyt. Jeśli cel nie może się wyzwolić, możesz utrzymać swój uchwyt w następnych akcjach, automatycznie zadając 4 punkty obrażeń w każdej rundzie, w której utrzymujesz cel. Umożliwienie.

\myability{Elektryczny Pancerz} (4 punkty Intelektu) - Kiedy sobie tego zażyczysz, elektryczność przesuwa się po Twoim ciele przez 4 minuty, dając Ci +1 do Pancerza. Kiedy jesteś w tym stanie, otrzymujesz dodatkowe +2 do Pancerza przeciwko elektryczności i zadajesz 2 punkty obrażeń każdej istocie, która Cię dotyka lub atakuje bronią do walki wręcz która przewodzi prąd. Umożliwienie.

\myability{Elektryczny Lot}  (5 punktów Intelektu) - Emitujesz aurę buzującej elektryczności, która pozwala Ci latać na daleki dystans w każdej rundzie przez 10 minut. Nie możesz z sobą zabrać innych istot. Akcja, by rozpocząć. 

\myability {Obrona Przed Żywiołami} (4+ punkty Intelektu) - Ty i każde cel ,który wybierzesz w bliskim zasięgu, uzyskujecie +5 Pancerza przeciwko jednemu rodzajowi bezpośrednich obrażeń od żywiołu (np: ogień, elektryczność, cień lub ciernie) na jedną godzinę, lub do czasu, aż rzucisz to zaklęcie ponownie. Każdy poziom wysiłku zwiększą Pancerz o +2. Akcja, by rozpocząć. 

\myability{Nieuchwytny} (2 punkty Szybkości) - Kiedy uda CI się odnieść sukces w Obronie Szybkości, natychmiast uzyskujesz akcję. Możesz z niej skorzystać tylko w celu przemieszczenia się. Umożliwienie.

\myability{W Objęciach Nocy} (7+ punktów Intelektu) - Tworzysz prawdziwie przerażającą podróbkę istoty z wijących się wstęg ciemnej materii i rzucasz ją na swoich wrogów w dalekim zasięgu. W każdej rundzie, możesz zaatakować cel w dalekim zasięgu korzystając z tej istoty jak z broni. Kiedy atakujesz, istota wrzuca swoje włoso-podobne macki cienia w mózg i oczy celu. Cel otrzymuje 3 punkty obrażeń Intelektu (ignorujących pancerz) i jest wstrząśnięty na jedną rundę (więc traci swoją turę). Alternatywnie, możesz sprawić, że istota podejmie inne akcje, tak długo jak ją widzisz i mentalnie ją kontrolujesz. Ta istota rozpływa się w nicość po minucie. Akcja by rozpocząć.

\myability{W Objęciach Mroku} (6 punktów Intelektu) - Przez następną godzinę, przyjmujesz pewne cechy cienia dzięki fundamentalnej adaptacji Twojego ciała lub urządzenia, które trzymasz w tajemnicy. Twój wygląd to ciemna sylwetka. Kiedy stosujesz poziom Wysiłku do akcji skradania się, uzyskujesz darmowy poziom Wysiłku do tego zadania. W tym czasie, możesz się poruszać przez powietrze z szybkością średniego dystansu na rundę,  i możesz przemieszczać się przez fizyczne bariery (nawet te, które są wiatroszczelne i nie przepuszczają cieni lub światła), ale nie przez bariery energetyczne, z szybkością 30 cm na rundę. Posiadasz zdolność percepcji gdy przechodzisz przez barierę lub obiekt, co pozwala Ci na patrzenie przez ściany. Jako cień, nie możesz wpłynąć na normalną materie bądź być pod jej wpływem. Podobnie, nie możesz atakować, dotykać lub w inny sposób wpłynąć na cokolwiek. Jednakże, ataki i efekty bazujące na świetle mogą na Ciebie wpłynąć, a nagły błysk światła może potencjalnie pozbawić Cię Twojej następnej tury. Akcja, by rozpocząć. 

\myability{Pomoc Bez Akcji} - Możesz skorzystać z zasad pomagania, by zapewnić korzyść innej postaci próbującej wykonać fizyczną akcję. W przeciwieństwie do normalnych zasad pomagania innym, nie wymaga to od Ciebie poświęcenia akcji. Umożliwienie.

\myability{Zaczarowany Ruch} (4+ punkty Intelektu) - Korzystasz ze swojej zaczarowanej broni, by poruszyć się do dowolnej lokacji w dalekim zasięgu, którą widzisz, tak długo, jak nie ma przeszkód lub barier na Twojej drodze. W zależności od tego, jaką broń dzierżysz, ruch ten może wyglądać inaczej - możesz rzucić swoim magicznym młotem i polecieć za nim, wystrzelić strzałę ze swojego magicznego łuku i polecieć za nią itp. W dodatku do normalnych opcji korzystania z Wysiłku, możesz go użyć, by zwiększyć dystans, na jaki się przemieszczasz - każdy poziom Wysiłku zwiększa zasięg o dodatkowe 30 m. Jeśli masz inną zdolność (np: zapewnianą przez Twój typ) która pozwala Ci na przemieszczanie się na długie dystanse, zasięg obydwu tych zdolności jest bardzo daleki. Akcja.

\myability{Zaczarowana Broń} (1 punkt Intelektu) - Związujesz się z fizyczną bronią, taką jak miecz, młot lub łuk. Wiesz dokładnie, gdzie się ona znajduje, jeśli jest w średnim zasięgu od Ciebie, i znasz ogólny kierunek i dystans, jeśli jest dalej. Wszystkie inne zdolności Twojej specjalizacji wymagają od Ciebie trzymania lub dzierżenia tej broni.  Możesz być związany tylko z jedną bronią na raz - powiązanie siebie z drugą bronią sprawia, że tracisz przywiązanie do pierwszej. Akcja by rozpocząć, 10 minut by ukończyć, Umożliwienie. 

Jeśli chcesz się powiązać z inną bronią, musisz wymyślić fabularny powód, dla którego jesteś w stanie  to zrobić i dla którego wybrałeś nową broń.

 \myability{Zachęta} (1 punkt Intelektu) - Kiedy utrzymujesz tą zdolność w mocy poprzez inspirującą przemowę, Twoi sprzymierzeńcy w średnim zasięgu uzyskują ułatwienie na następujących typach akcji (Twój wybór): zadania obronne, ataki lub zadania związane z dowolną jedną umiejętnością, w której jesteś wyszkolony lub wyspecjalizowany. Akcja.
 
\myability{Zachęcająca Obecność} (2 punkty Intelektu) - Na jedną minutę, sprzymierzeńcy w średnim zasięgu zyskują atut na rzutach obronnych. Akcja.

\myability{Wytrzymałość} - Każda długość fizycznych akcji jest dla Ciebie albo podwojona, albo skrócona o połowę, w zależności, co jest dla Ciebie lepsze. Dla przykładu, jeśli zwykła osoba może wstrzymać oddech na 30 sekund, Ty możesz go wstrzymać przez minutę. Jeśli zwykły człowiek może maszerować przez 4 godziny bez przerwy, TY możesz przez 8 godzin. Jeśli chodzi o szkodliwe efekty, jak trucizna paraliżująca osobę przez jedną minutę, jesteś sparaliżowany tylko przez 30 sekund. Minimalny czas trwania to zawsze 1 runda. Umożliwienie.

\myability{Zasilenie Istoty} (6+ punktów Mocy) - Rozszerzasz zastosowanie swojej zdolności Absorpcja Energii Kinetycznej na jedną istotę w bliskim zasięgu, tak, że ona także może absorbować energię ataków fizycznych i uderzenia przez 1 godzinę. Jednakże, ta istota nie może wypuszczać nadmiaru energii jako ataku dystansowego. Na każdy poziom Wysiłku, który zastosujesz, możesz zwiększyć liczbę celów tej zdolności o 1. Jeśli posiadasz także Absorpcję Czystej Energii lub Ulepszoną Absorpcję Energi Kinetycznej, te zdolności także się duplikują dla Twoich celów Zasilenia Istoty. Akcja, by rozpocząć. 

\myability{Zasilenie Tłumu} (9 punktów Mocy) - Rozszerzasz swoją zdolność Absorpcja Energii Kinetycznej na do 30 istot w średnim zasięgu, tak, by mogły absorbować energię z ataków fizycznych i uderzeń przez 1 godzinę. Jeśli posiadasz Absorpcję Czystej Energii lub Ulepszoną Absorpcję Energii Kinetycznej, , te istoty także mogą korzystać z tych zdolności. Istoty te jednakże, nie mogą emitować nadmiarowej energii jako ataku dystansowego. Akcja, by rozpocząć.
 
 \myability{Zasilenie Przedmiotu} - Poprzez skupienie swojej zdolności Absorpcja Energii Kinetycznej na przedmiocie (np: broni), napełniasz ją mocą. Przedmiot przechowuje energię do momentu, gdy dotknie kogoś innego niż Ty, tak więc skorzystanie z tej zdolności na broni do walki wręcz lub amunicji broni dystansowej pozwala broni na wyemitowanie energii w walce. Energie zadaje 3 punkty obrażeń dotkniętej istocie w dodatku do normalnych obrażeń broni. Nie możesz mieć więcej niż jednego Zasilonego Przedmiotu jednocześnie. Akcja, by rozpocząć. 
 
\myability{Zasilona Tarcza} - Twoje pole siłowe ze zdolności Tarcza Pola Siłowego pulsuje niebezpieczną energię, kiedykolwiek je manifestujesz. Za każdym razem, gdy korzystasz z tarczy jako broni do walki wręcz lub dystansowej, zadaje ona dodatkowe 3 punkty obrażeń. Umożliwienie.

\myability{Ochrona przed Energią} (3+ punktów Intelektu) - Wybierz konkretny typ energii, z którym masz doświadczenie (taką jak ciepło, dźwięk, elektryczność itp.). Uzyskujesz +10 Pancerza przeciwko tej energii na 10 minut. Alternatywnie, uzyskujesz +1 do Pancerza przeciwko tej energii na następna 24 godziny. Musisz mieć doświadczenie z daną energią: przykładowo, jeśli nie masz doświadczenia z danym rodzajem energii międzywymiarowej, nie możesz się przed nią bronić. W dodatku do zwykłych opcji korzystania z Wysiłku, możesz skorzystać z niego, by ochronić więcej celów: każdy poziom Wysiłku wtedy dotyczy maksymalnie dwóch więcej celów. Musisz dotknąć dodatkowe cele, aby je bronić. Akcja, by rozpocząć.

\myability{Odporność na Energię} - Wybierz konkretny typ energii, z którym masz doświadczenie (taki jak ciepło, dźwięk, elektryczność itp.). Uzyskujesz +5 Pancerza przeciwko obrażeniem od tego typu energii. Musisz mieć doświadczenie z tym typem energii - przykładowo, jeśli nie posiadasz doświadczenia z daną energią międzywymiarową, nie zyskujesz od niej ochrony. Możesz wybrać tę zdolność więcej niż raz. Za każdym razem, musisz wybrać inny typ energii. Umożliwienie.

\myability{Ulepszenie Siły} (3 punkty Intelektu) - Na następne 10 minut, uzyskujesz atut na zadaniach które polegają na brutalnej sile, takich jak przenoszenie ciężkich przedmiotów, wyłamywanie drzwi z zawiasów lub zaatakowanie kogoś bronią do walki wręcz. Akcja, by rozpocząć. 

\myability{Ulepszona Likantropia} - Kiedy korzystasz z Likantropii , Twoja forma likantropa zyskuje poniższe, dodatkowe bonusy: +3 do Puli Mocy, +2 do Puli Szybkości i +2 do Pancerza, Umożliwienie.

\myability{Ulepszone Ciało} - Twoje mechaniczne części dają Ci +1 do Pancerza, +3 do Puli Mocy i +3 do Puli Szybkości. Tradycyjne umiejętności medyczne, leki i techniki lecznicze działają na Ciebie tylko z połową swej mocy. Za każdym razem, gy zaczynasz z pełnią zdrowia, pierwsze 5 punktów obrażeń nigdy nie może być uzdrowionych w tradycyjny sposób. Zamiast tego, musisz je uzdrowić przy pomocy umiejętności i zdolności technicznych. Dla przykładu, jeśli zaczynasz z pełną Pulą Mocy 1- i otrzymujesz 8 punktów obrażeń, możesz tradycyjnie odzyskać tylko 3 punkty z Puli - pozostałe 5 należy naprawić, a nie uzdrowić. Umożliwienie.

\myability{Ulepszony Intelekt} - Uzyskujesz 3 punkty w swojej Puli Intelektu. Umożliwienie.

\myability{Ulepszone Skupienie w Inteligencji} - Uzyskujesz +1 do swojego Skupienia w Intelekcie. Umożliwienie.

\myability{Ulepszona Moc} - Uzyskujesz 3 punkty w swojej Puli Mocy. Umożliwienie.

\myability{Ulepszone Skupienie w Mocy} - Uzyskujesz +1 do swojego Skupienia w Mocy. Umożliwienie.

\myability{Ulepszony Atak Fazowy} (5 punktów Intelektu) - Ta zdolność działa jak zdolność Atak Fazowy, ale Twój atak także atakuje organy celu, zadając dodatkowe 5 punktów obrażeń. Umożliwienie.

\myability{Ulepszona Muskulatura} - Uzyskujesz 3 punkty do rozdziału między Pule Mocy i Szybkości, zgodnie z własnym życzeniem. Umożliwienie.

\myability{Potencjał} - Uzyskujesz 3 punkty do podziału między swoje Pule, jakkolwiek uznasz za stosowne. Umożliwienie.

\myability{Ulepszona Szybkość} - Uzyskujesz 3 punkty w swojej Puli Szybkości. Umożliwienie.

\myability{Ulepszone Skupienie w Szybkości} - Uzyskujesz +1 do swojego Skupienia w Szybkości. Umożliwienie.

\myability{Wzrost} (1| punktów Mocy) - Zapoczątkowujesz enzymatyczną reakcję, która ściąga dodatkową masę z innego wymiaru, i Ty (oraz Twoje ubrania lub strój) się powiększacie. Uzyskujesz wysokość 3 metrów i utrzymujesz ją przez około minutę. W tym czasie, dodajesz 4 punkty do swojej Puli Mocy, +1 do Pancerza i +2 do Skupienia w Mocy. Kiedy jesteś większy niż normalnie, Twoje rzuty na ochronę Szybkości są utrudnione, i jesteś wyszkolony w korzystaniu z swoich pięści jak z ciężkich broni. 

Kiedy efekt Wzrostu się kończy, Twój pancerz i Skupienie w Mocy wracają do normy, i odejmujesz od swojej Puli Mocy punkty, które zyskałeś (jeśli obniża to wartość Puli do 0, to należy je odjąć napierw z Puli Szybkości, a potem, jeśli zajdzie potrzeba, z Puli Intelektu). Każdy dodatkowy raz, gdy skorzystasz z Wzrostu przed 10-godzinnym rzutem na odzyskanie zdrowia, musisz zastosować dodatkowy poziom Wysiłku. Tak więc drugi raz, gdy skorzystasz z Wzrostu, stosujesz 1 poziom Wysiłku, za trzecim razem 2 poziomy Wysiłku itp. 

Akcja, by rozpocząć.

\myability{Oświecony} - Jesteś wyszkolony w każdym zadaniu percepcji polegającym na wzroku. Umożliwienie. 

\myability{Macki Mocy} (1+ punktów Intelektu) - Cel w średnim zasięgu zostaje pochwycony przez mackę utworzoną z linii mocy na 1 minutę. Ta macka to konstrukt 2 poziomu. Cel schwytany przez Mackę Mocy nie może się ruszyć, ale może atakować i bronić się normalnie. Cel może także użyć swojej akcji, by spróbować się wyzwolić. Zwiększasz poziom Macki Mocy o 1 na każdy poziom Wysiłku. Akcja, by rozpocząć. 

\myability{Zauroczenie} (1 punkt Intelektu) - Kiery rozmawiasz, przyciągasz uwagę innej istoty, nawet jeśli ta istota nie może Cię zrozumieć. Tak długo, jak nie robisz niczego innego tylko mówisz (nie możesz nawet się ruszyć) ta istota nie podejmuje żadnych akcji innych niż bronienie się, nawet przez wiele rund. Jeśli ta istota zostaje zaatakowana, efekt się kończy. Akcja.

\myability{Świta} - Uzyskujesz świtę 5 istot 1-szego poziomu w wieku około 20 lat, która towarzyszy gdziekolwiek pójdziesz, chyba, że celowo im zakażesz danym razem. Możesz ją poprosić, żeby dostarczyła dla Ciebie rzeczy, wiadomości, zrobiła dla Ciebie pranie - cokolwiek zechcesz, ale w granicach rozsądku. Mogą także zająć się problemem, jeśli chcesz kogoś uniknąć, pomóc Ci ukryć się przed mediami, pomóc Ci w przejściu przez tłum itp. Z drugiej strony, jeśli sytuacja staje się pełna przemocy, uciekną w bezpieczne miejsce. Umożliwienie. 

\myability{Ulepszona Tarcza} - Twoja Tarcza Pola Siłowego generuje pole, które otula Cię, gdy trzymasz tarczę, dając Ci +1 do Pancerza. Umożliwienie.

\myability{Usunięcie Wspomnień} (3 punkty Intelektu) - Sięgasz do umysłu istoty w bliskim zasięgu i wykonujesz test Intelektu. Jeśli osiągniesz sukces, usuwasz do 5 ostatnich minut jej wspomnień. Akcja. 

\myability{Ucieczka} (2 punkty Szybkości) -  Wyzwalasz się z więzów, przeciskasz przez kraty, uciekasz z chwytu istoty, wyciągasz się z ruchomych piasków lub w inny sposób dokonujesz ucieczki. Akcja.

\myability{Plan Ucieczki} - Kiedy zabijasz przeciwnika, możesz spróbować wykonać test skradania sią, żeby natychmiast ukryć się przed wszystkimi wokół, założywszy, że odpowiednie miejsce do ukrycia się znajduje się niedaleko. Umożliwienie. 

\myability{Zniknięcie} (3 punkty Szybkości) - Wstępujesz w cień lub w inny sposób się ukrywasz, i wszyscy, którzy Cię obserwowali, tracą jakikolwiek ślad Twojej osoby. Choć nie jesteś niewidzialny, nie możesz być dostrzeżony aż się nie ujawnisz, występując z cienia lub zza zasłony (lub atakując). Akcja.

\myability{Unik} - Ciężko Cię trafić, kiedy tego nie chcesz. Jesteś wyszkolony we wszystkich akcjach obrony. Umożliwienie. 

\myability{Wszystko Jest Bronią} - Możesz wziąć dowolny mały obiekt - monetę, długopis, butelkę, kamień itp. - i rzucić nim z taką siłą i precyzją, że zadaje obrażenia jak lekka broń. Umożliwienie.

\myability{Wygnanie} (5 punktów Intelektu) - Umieszczając cel Twojego dotyku w innym, przypadkowym wymiarze lub uniwersum, gdzie zostaje on na 10 minut. Nie masz pojęcia co się dzieje z celem, gdy go nie ma, ale na koniec 10 minut, pojawia on się z powrotem w miejscu, z którego zniknął. Akcja. 

\myability{Zwiększenie Limitu Subtelnych Cypherów} - Liczba subtelnych Cypherów, które możesz nosić przy sobie, zwiększa się o 1. Umożliwienie.

\myability{Doświadczony Obrońca} - Kiedy nosisz zbroję, Twój Pancerz zwiększa się o +1.  Umożliwienie.

\myability{Doświadczony Znaleźca} (6+ punktów Intelektu) - Kiedy szukać specyficznej rzeczy, takiej jak rzadki pierwiastek, materiał chemiczny potrzebny by skompletować szczepionkę na zarazę, czesci zamiennej potrzebnej do naprawy urządzenia, śladów konkretnej bestii lub miecza, który złodziej Tobie ukradł, ta zdolność jest jak ulał. Na następne 24 godziny, jeśli znajdziesz się w średnim zasięgu  od tej rzeczy, a okoliczności są takie, że mozesz dostrzec tą rzecz (przykładowo, nie jest zamknięta w komnacie, do której nie masz klucza), znajdujesz ją. Ta zdolność zakłada, że poszukujesz rzeczy w sposób ciągły, zawsze patrząc gdzie tylko możesz, spoglądając ponad przeszkodami itp. - jeśli uciekasz, broniąc swoje życie, śpisz lub w inny sposób jesteś zajęty, ta zdolność Ci nie pomaga. Korzystasz z tej zdolności zamiast wykonywać rzuty na znalezienie danej rzeczy, ale tylko jeśli trudność to poziom 6 i mniejszy. Możesz zastosować poziom Wysiłku, aby zwiększyć maksymalny poziom rzeczy, którą próbujesz znaleźć (każdy poziom Wysiłku zwiększa maksymalny poziom o 1). Akcja, by rozpocząć. 

\myability{Przywykły do Noszenia Zbroi} - Redukcja kosztu Twojej zdolności Wyszkolony w Zbroi się zwiększa. Teraz redukujesz koszt Szybkości o 2 punkty. Umożliwienie.

\myability{Ekspert-Rzemieślnik} - Zamiast rzucać kością, możesz wybrać automatyczny sukces na zadaniu rzemieślniczym, jeśli jesteś w nim wyszkolony. Zadanie to musi być poziomu trudności 4 lub mniejszym. Jeśli jesteś w stanie zredukować trudność zadania do 4 lub mniej, ta zdolność także się stosuje do każdemu subzadania, założywszy, że coś Ci nie przeszkodzi w czasie aktu stwórczego. Umożliwienie.

\myability{Ekspercki Użytkownik Cypherów} - Możesz nosić przy sobie 3 Cyphery w danym czasie. Umożliwienie.

\myability{Doświadczony Kierowca} - Jesteś wyspecjalizowany we wszystkich zadaniach związanych z kierowaniem samochodu, ciężarówki lub motocykla, wliczając naprawy mechaniczne. Umożliwienie.

\myability{Kompan-Ekspert} - Zyskujesz kompana 3 poziomu. Nie ma on ograniczeń w swoich modyfikacjach. Możesz wziąć tą zdolność wiele razy, za każdym razem otrzymując innego kompana 3 poziomu; ALternatywnie, możesz chcieć ulepszyć kompana 2 poziomu, którego już posiadasz, i awansować go na 3 poziom, a potem pozyskać nowego kompana 2 poziomu. Umożliwienie.

\myability{Ekspert-Pilot} - Jesteś wyspecjalizowany we wszystkich zadaniach pilotowania statku kosmicznego. Umożliwienie.

\myability{Umiejętność Eksperta} - Zamiast rzucać k20, możesz wybrać automatyczny sukces za zadaniu, w którym jesteś wyszkolony. To zadanie musi być poziomu 4 lub mniejszego, i nie może to być atak albo rzut obronny. Umożliwienie. (Postać nie może zastosować Wysiłku lub innych zdolności do zadania, które wykonuje, korzystając z Umiejętności Eksperta).

\myability{Wyjaśnianie Niewysłowionego} - Poprzez anegdotki, historyczne opowieści i cytując wiedzą którą mało ludzi poza Tobą zwykło rozumieć, oświecasz swoich przyjaciół. Po spędzeniu 24 godzin razem z Tobą, raz dziennie, każdy z Twoich przyjacioł może ułatwić dany rzut o 2 stopnie. Ta korzyść trwa tak długo, jak jesteś w towarzystwie swoich przyjaciół. Kończy się, gdy odchodzisz, ale powraca automatycznie jeśli wrócisz do swoich przyjaciół przed upływem 24 godzin. Jeśli opuścisz towarzystwo swoich przyjaciół na dłużej, musisz spędzić z nimi kolejne 24 godziny, by reaktywować tą korzyść.  Umożliwienie.

\myability{Wykorzystanie Przewagi} - Nawet, jeśli robisz coś dobrze, nauczyłeś się, że zawsze możesz zrobić to lepiej. Kiedykolwiek uzyskujesz atut do rzutu, ułatwiasz ten rzut o dodatkowy stopień. Umożliwienie. 

\myability{Doświadczony Eksplorator} - Jesteś wyszkolony w dwóch dodatkowych umiejętnościach, w których nie jesteś jeszcze wyszkolony. Wybierz z listy: nawigacja, percepcja, wyczuwanie niebezpieczeństwa, inicjatywa, pokojowe rozpoczynanie rozmów z nieznajomymi, i śledzenie. Umożliwienie. 

\myability{Wybuchowe Rozładowanie} (6 punktów Intelektu) - Możesz zwiększyć energię przechowywaną w swojej Puli (ze zdolności Przechowywanie Energii) i wyrzucić z siebie potężny wybuch który dotyczy albo jednego celu w średnim zasięgu, albo wszystkiego w bliskim zasięgu. Jeśli wybierzesz pojedynczy cel, otrzymuje on 2 punkty obrażeń na każdy punkt w Puli. Jeśli wybierzesz obszar, wszystko w nim (poza Tobą) otrzymuje 1 punkt obrażeń na punkt w Puli (lub połowę, jeśli atak chybi) Obniża to liczbę punktów w Puli do 0. Akcja. 

\myability{Dodatkowy Rzut na Odzyskanie Zdrowia} - Uzyskujesz dodatkowy rzut na odzyskanie zdrowia trwający jedną akcję. Umożliwienie.

\myability{Dodatkowa Umiejętność} - Jesteś wyszkolony w jednej umiejętności własnego wyboru (innej niż atak lub obrona) w którejn ie posiadasz jeszcze treningu. Możesz wybrać tę zdolność wiele razy. Za każdym razem, wybierz odmienną umiejętność. Umożliwienie.

\myability{Dodatkowe Użycie Artefaktu} (3 punkty Intelektu) - Próbujesz uzyskać dodatkowe użycie artefaktu bez rzucenia na wyczerpanie artefaktu. Trudność tego zadania to poziom artefaktu. Jeśli stworzyłeś ten artefakt, uzyskujesz atut do rzutu. W przypadku porażki, rzut na wyczerpanie artefaktu działa normalnie. Możesz także spróbować użyć zamanifestowanego Cyphera bez jego zużywania, ale zadanie jest utrudnione. Nieudany rzut na dodatkowym użyciu Cyphera niszczy go zanim może on wywołać pożądany efekt. Akcja. 

\myability{Wyjątkowe Mistrzostwo} (6 punktów Mocy lub 6 punktów Szybkości) - Kiedy korzystasz z broni swojego wyboru, możesz przerzucić każdy rzut na atak jaki pragniesz i wybrać lepszy z dwóch wyników. Umożliwienie. 

\myability{Oko do Szczegółów} (2 punkty Intelektu) - Kiedy spędzisz około 5 minut szczegółowo badając dany obszar nie większy niż promień średniego zasięgu, możesz zadać MG jedno pytanie o nim. MG musi udzielić prawdziwej odpowiedzi. Nie możesz użyć tej umiejętności więcej niż 1 raz na obszar na 24 godziny. Akcja by rozpocząć, 5 minut by ukończyć.

\myability{Wyłupienie Oka} (2 punkty Szybkości) - Wykonujesz atak przeciwko istocie z okiem. Atak jest utrudniony, ale jeśli trafisz, istota ma problem z widzeniem na następną godzinę. W tym czasie, akcje istoty które polegają na wzroku (czyli większosć) są utrudnione. Akcja. 

\myability{Dostosowane Oczy} - Widzisz w bardzo nikłym świetle jakby było jasno, Widzisz w absolutnej ciemności, jakby było słabe światło. Umożliwienie.

\section{F}

\myability{Morficzna Twarz} (2+ punkty Intelektu) - Zmieniasz cechy swojej twarzy i kolor na jedną godzinę, ukrywając swoją tożsamość lub przebierając się za kogoś innego. Wpływa to tylko na Twoją twarz, nie resztę Twojego ciała. Nie możesz perfekcyjnie zduplikować czyjejś twarzy, ale możesz zbliżyć się do tego na tyle, by oszukać kogoś, kto zna tę osobę pobieżnie. Masz atut na wszelkich zadaniach związanych z przebieraniem się. Musisz zastosować poziom Wysiłku, by być w stanie udawać osobę innego gatunku (np: człowiek udający humanoidalnego kosmitę). Akcja.

\myability{Zapoznanie się z Terenem} - Możesz zapoznać się z terenem, jeśli spędzisz przynajmniej godzinę badając region o promieniu dalekiego zasięgu, do którego masz bezpośredni dostęp i w którym możesz się poruszać. Kiedy już się zapoznasz z tym terenem, wszystkie Twoje zadania związana z percepcją, nawigacją, poszukiwaniem i zdobywaniem przedmiotów, obroną i poruszaniem się w tym obszarze zyskują atut. Za każdym razem, gdy się zapoznasz z nowym terenem, tracisz zaznajomienie się ze starym, chyba, że wydasz 1 PD, by zachować znajomość dawnego terenu permanentnie. Akcja, by rozpocząć, jedna godzina, by ukończyć.

\myability{Daleki Krok} (2 punkty Intelektu) - Skaczesz w powietrzu i lądujesz w pewnej odległości. Możesz skoczyć w dół, górę lub na wprost gdziekolwiek wybierzesz w dalekim zasięgu, jeśli masz tam prostą drogę bez przeszkód. Lądujesz bezpiecznie. Akcja.

\myability{Szybkie Zabójstwo} (2 punkty Szybkości) - Wiesz, jak zabijąc szybko. Kiedy trafiasz atakiem wręcz lub dystansowym, zadajesz 4 dodatkowe punkty obrażeń. NIem ożesz wykonać tego ataku w dwóch turach zaraz po sobie. Akcja.

\myability{Wmawianie} (1 punkt Intelektu) - Kiedy rozmawiasz z inteligentną istotą, która Cię rozumie i nie jest wroga, przekonujesz tą istotę, by wykonała jedną rozsądną akcję w następnej rundzie. Na rozsądną akcję musi się zgodzić MG, nie powinna ona umieścić istoty lub jej sojuszników w niebezpieczeństwie i być sprzeczna z jej psychiką. Akcja.

\myability{Szybka Podróż} (7 punktów Intelektu) - Zaginasz czas i przestrzeń, tak, by Ty i do 10 innych istot w bliskim zasięgu podróżowało z prędkością 10 razy większą niż normalnie przez do 8 ofzin. W tej szybkości, większość niebezpiecznych spotkań i regionów jest ignorowanych, choć MG może ogłosić wyjątki od tej zasady. Zwykłe bariery dalej są problematyczne. Akcje, by rozpocząć.

\myability{Szybsza Dzika Magia} - Jeśli spędzisz 10 minut przygotowując swoją magię, możesz wypełnić wszystkie wolne sloty na cyphery subtelnymi cypherami wybranymi randomowo przez MG (może to być część odpoczynku 10-minutowego, 1-godzinnego lub 10-godzinnego, jeśli jesteś świadom cały ten czas). Niem ożesz skorzystać ponownie z tej zdolności, do momentu, aż wykonasz 10-godzinny rzut na odzyskanie zdrowia. Możesz dalej skorzystać z Magicznych Zasobów, by wypełnić sloty na cyphery. Akcja by rozpocząć, 10 minut, by zakończyć.

\myability{Przerażająca Reputacja} (3 punkty Intelektu) - Ty i Twoi towarzysze drogi zyskaliście przerażającą reputację w pewnych miejscach. Jeśli Twoi wrogowie słyszeli o Tobie, cele w zasięgu słuchu się boją i wszystkie ataki, które wykonują, są utrudnione do momentu, aż jeden lub więcej z nich uda się zadać obrażenia Tobie lub Twoim towarzyszom, w którym to momencie strach przemija. Akcja.

\myability{Pokaz Siły} (1 punkt Mocy) - Każde zadanie które polega na brutalnej sile jest ułatwione. Przykłady to wyłamanie drzwi z zawiasów, wyważenie zamkniętego pojemnika, podnoszenie lub przemieszczenie ciężkiego obiektu, lub uderzenie kogoś bronią do walki wręcz. Umożliwienie.

\myability{Finta} (2 punkty Szybkości) - Jeśli skorzystasz ze swojej akcji, by zmylić przeciwnika, w następnej rundzie możesz wykorzystać obniżoną obronę wroga. Wykonaj atak wręcz przeciwko wrogowi. Uzyskujesz atut na tym rzucie. Jeśli Twój atak jest udany, zadaje on dodatkowe 4 punty obrażeń. Akcja.

\myability{Kompan Eksplorator} - Uzyskujesz kompana 2 poziomu. Jedna z jego modyfikacji musi dotyczyć zadań związanych z percepcją. Umożliwienie.

\myability{Przywołanie} (3 punkty Intelektu) - Możesz sprawić, że obiekt znika i pojawia się w Twoich dłoniach lub gfzieś blisko. Wybierz jeden obiekt, który może się zmieścić w sześcianie o bok 2 metrów i który widzisz w dalekim zasięgu. Obiekt znika i pojawia się w Twoich dłoniach lub w otwartej przestrzeni gdziekolwiek w bliskim zasięgu. Akcja.

\myability{Pole Zniszczeń} (4 punkty Mocy) - Kiedy sprawiasz, że obiekt spada w doł na liczniku obrażeń o jeden stopień lub więcej, uzyskujesz 1 dodatkowy punkt Pancerza na 1 minutę. Umożliwienie.

\myability{Pole Grawitacyjne} (4 punkty Intelektu) - Kiedy sobie tego zażyczysz, pole zmienionej grawitacji wokoł Ciebie przyciąga pociski w dół. Jesteś odporny na takie ataki aż do Twojej rundy w następnej turze. Musisz być śWiadom ataku, by go zniwelować. Ta zdolność nie działa na ataki energetyczne. Umożliwienie.

\myability{Pole Mocy Zasilanego Pancerza} - Uzyskujesz +1 do Pancerza kiedy nosisz swoją zbroję ze zdolności Zasilany Pancerz. Umożliwienie.

\myability{Ognista Ręka Zguby} (3 punkty Intelektu) - Kiedy Twoja zdolność Płaszcz Ognia jest aktywna, możesz sięgnąć w swój Płaszcz i stworzyć energię stworzoną z ruchomego ognia, która jest podwójnego rozmiaru normalnej dłoni. Ręka ta czyni, co jej każesz, unosząc się w powietrzu. Kierowanie dłonią jest akcją. Bez wydania komendy, dłoń nic nie robi. Może ona się przemieścić na daleki dystans w rundzie, ale nigdy nie może być dalej od Ciebie niż daleki zasięg. Ręką może chwycić, poruszyć i nosic rzeczy, ale wszystko, czego dotyka, otrzymuje 1 punkt obrażeń na rundę od ognia. Ręka może także atakować. Jest istotą 3 poziomu i zadaje 1 dodatkowy punkt obrażeń od ognia, gdy atakuje. Kiedy zostanie stworzona, utrzymuje się przy istnieniu przez 10 minut. Akcja, by stworzyć, akcja by wydać komendę. 

\myability{Niezłomny} - Nie ponosisz normalnych kar wskutek spadania w dół na liczniku obrażeń. Jeśli jesteś krytycznie ranny, zamiast nie być w stanie wykonać większości akcji jak normalnie, możesz dalej działać - jednakże, wszystkie Twoje zadania są utrudnione. Umożliwienie. 

\myability{Ostateczne Zaprzeczenie} - W momencie, gdy normalnie byłbyś martwy, zamiast tego zachowujesz przytomność i jesteś aktywny przez jedną turę dłużej, plus dodatkowa tura za każdym razem gdy zdasz test Mocy o stopniu trudności 5. Podczas tych rund, jesteś krytycznie ranny. Jeśli nie uzyskasz leczenia lub w inny sposób odzyskasz punktów w Pulach podczas swoich finalnych rund aktywności, stosuje się do Ciebie zdolność Jeszcze Żywy. Umożliwienie.

\myability{Wyczulenie na Okazje} (1 punkt Intelektu) - Korzystasz ze sztuczek, żeby znaleźć luki w obronie przeciwnika. Jeśli odniesies sukces na rzucie na Szybkość przeciwko jednej istocie w bliskim zasięgu, Twój następny atak przeciwko tej istocie zanim skończy się Twoje następna runda jest ułatwiony. Akcja.

\myability{Odnalezienie Winnych} - Jeśli użyłeś osądu na celu, jesteś wyszkolony w śledzeniu go, poszukiwaniu go, gdy się ukrywa lub jest przebrany lub w inny sposób znajdowaniu go. Umożliwienie.

\myability{Odnalezienie Ukrytych} (4+ punkty Intelektu) - Widzisz szlaki obiektów, gdy te poruszają się przez czas i przestrzeń. Możesz wyczuć odległość i kierunek każdego specyficznego przedmiotu, który kiedykolwiek dotykałeś. Zajmuje to od jednej akcji do godzin koncentracji, w zależności od opinii MG o czasie, dystansie i innych czynnikach. Jednakże, nie wiesz na początku jak wiele czasu Ci to zajmie. Jeśli skorzystasz przynajmniej z 2 poziomów Wysiłku, kiedy ustalisz dystans i kierunek, pozostajesz w kontakcie z przedmiotem na jedną godzinę na poziom Wysiłku. Tak więc, jeśli się on poruszy, jesteś świadom nowej pozycji. Akcja by rozpocząć, akcja na każdą rundę (koncentracja).

\myability{Znajdowanie Drogi} - Kiedy korzystasz z Wysiłku na zadaniu nawigacji bo się zgubiłeś, nie znasz drogi, chcesz wyznaczyć nowy szlak, potrzebujesz wybrać między dwoma lub więcej podobnymi ścieżkami lub coś podobnego, możesz zastosować darmowy poziom Wysiłku. Umożliwienie.

\myability{Ostateczny Cios} (5 punktów Mocy) - Jeśli Twój przeciwnik jest podatny na ataki, wstrząśnięty, lub inaczej bezbronny kiedy atakujesz, zadajesz dodatkowe 7 punktów obrażeń na udanym rzucie na atak. Umożliwienie.

\myability{Ogień i Lód} (4 punkty Intelektu) - Sprawiasz, że lec w średnim zasięgu staje się albo bardzo gorący, albo bardzo zimny. Cel odnosi 3 punkty obrażeń środowiskowych (ignorujących Pancerz) w każdej rundzie do 3 rund, choć nowy test jest wymagany w każdej rundzie, by kontynuować obrażenia. Akcja, by rozpocząć. 

\myability{Kwiat Ognia} (4+ punkty Intelektu) - Kwiat ognia rozkwita w dalekim zasięgu, wypełniając obszar o promieniu 3 metrów i zadając 3 punkty obrażeń wszystkim celom w zasięgu. Wysiłek zastosowany do jednego ataku liczy się na wszystkie ataki w tym obszarze. Nawet na nieudanym rzucie na atak, cel w obszarze dalej otrzymuje 1 punkt obrażeń. Łatwopalne obiekty w obszarze mogą zapłonąć. Akcja.

\myability{Ognisty Sługa} (6 punktów Intelektu) - Kiedy Twój Płaszcz Ognia jest aktywny, sięgasz do niego i tworzysz automatona ognia w Twoim ogólnym  kształcie i rozmiaru. Działa on zgodnie z Twoimi komendami w każdej turze. Danie słudze rozkazu jest akcją, i możesz nim kierować tylko, gdy jesteś w dalekim zasięgu od niego. Bez komendy, automaton będzie podążał za Twoim ostatnim rozkazem.  Możesz także dać mu prostą zaprogramowaną akcję, taką jak ``Czekaj tutaj i atakuj każdego, kto wkroczy w ten obszar w średnim zasięgu od Ciebie, dopóki będzie on martwy''.  Sługa istnieje przez 10 minut, jest istotą 5 poziomu i zadaje 1 dodatkowu punkt obrażeń od ognia gdy atakuje. Akcja by stworzyć, akcja, by pokierować.

\myability{Ogniste Macki} (5 punktów Intelektu) - Kiedy sobie tego życzysz, Twoja ognista poświata (ze zdolności Płaszcz Ognia) wyrzuca z siebie 3 macki ognia które istnieją do 10 minut. Jako akcja, możesz użyć tych macek do ataku, czyniąc osobny test ataku każdą jedną. Każda macka zadaje 4 punkty obrażeń. Poza tym, te ataki funckjonują jak zwykłe. Jeśli nie używasz macek do ataku, pozostają one w egzystencji, ale nie czynią niczego. Umożliwienie.

\myability{Pięści Furii} - Zadajesz 2 dodatkowe punkty obrażeń atakami bez broni. Umożliwienie.

\myability{Czcze Przechwałki} (1 punkt Intelektu) - Przechwalasz się aktem, który pragniesz osiągnąć, i jako część tej samej akcji, dokonujesz jego próby. Jeśli przeciętny człowiek uznałby tą akcję za trudną (lub niemożliwą) i odniesiesz na niej sukces, istoty, które były tego świadkami i które nie są Twoimi sojusznikami, są potencjalnie oszołomione w swojej następnej turze, i wszystkie akcje, które próbują wykonać są utrudnione. MG pomoże CI określić, czy przechwałka jest czymś, co by zachwyciło świadków w taki sposób. Jeśli próbujesz wykonać akt, którym się chwaliłeś, ale odnosisz porażkę, wszystkie Twoje akcje, by wpłynąć bądź zaatakować świadków którzy to widzieli, są utrudnione na 10 minut. Umożliwienie.

\myability{Ostrze Ognia} (4 punkty Intelektu) - Kiedy sobie tego zażyczysz, z Twojego Płaszcza Ognia wysuwa się ogień i pokrywa broń na jedną godzinę. Ogień się kończy, jeśli przestaniesz trzymać lub nosić broń. Kiedy ogień istnieje, broń zadaje dodatkowe 2 punkty obrażeń. Umożliwienie. 

\myability{Błysk} (4 punkty Intelektu) - Tworzysz eksplozję energii w punkcie w bliskim zasięgu, dotyczącym obszaru w bliskim zasięgu od tego punktu. Musisz być w stanie dostrzec obszar, na którym centrujesz eksplozję. Wybuch zadaje 2 punkty obrażeń dla wszystkich istot lub obiektów w jego obszarze. Jeśli zastosujesz Wysiłek by zwiększyć obrażenia, zadajesz dodatkowe 2 punkty obrażeń na poziom Wysiłku (zamiast 3 punktów); cele w obszarze otrzymują 1 punkt obrażeń nawet, jeśli nie uda Ci się test na atak. Akcja.

\myability{Szybkość Błyskawicy} (6+ punktów Intelektu) - Możesz się przemieścić do otwartej lokacji na planecie, z którą jesteś zapoznany prawie natychmiastowo, zmieniając się w błyskawicę. Jeśli zastosujesz poziom Wysiłku, możesz spróbować dostać się do przykrytych lokacji, o których jesteś świadom tak długo, jak istnieje droga, którą od otwartego nieba elektryczność mogłaby łatwo podążać do tej zamkniętej lokacji. Akcja.

\myability{Krzyk Ucieczki} (6 pinktów Intelektu) - Wszyscy nie będący Twoimi sprzymierzeńcami w średnim zasięgu którzy słyszą Twóje przerażające słowa uciekają z pełną szybkością przez 1 minutę. Akcja.

\myability{Szybkostopy} (1+ punkt Szybkości) - Możesz przemieścić się na średni zasięg jako część innej akcji. Możesz się przemieścić na daleki zasięg jako swoją calą akcję w turze. Jeśli zastosujesz poziom Wysiłku do tej zdolności, możesz się przemieścić na daleki zasięg i wykonać atak jako swoją całą akcję w turze, ale atak jest utrudniony. Umożliwienie.

\myability{Ciało z Kamienia} - Masz +1 do Pancerza, jeśli nie nosisz fizycznego pancerza. Umożliwienie.

\myability{Planetarna Wiedza} - Po każdym 10-godzinnym rzucie na odzyskanie zdrowia, kiedy masz dostęp do wysokotechnologicznej cyfrowej biblioteki (takiej, jaka może być znaleziona na statku kosmicznych lub w centrum edukacyjnym), wybierz jeden rodzaj wiedzy związanej z specyficzną planetą lub inną lokacją. Wiedza może dotyczyć mieszkalnictwa, zwyczajów, rządów, cech głównych gatunków, ważnych osobistości itp. Jesteś wyszkolony w tej wiedzy do momentu, aż znowu skorzystasz z tej zdolności. Możesz wykorzystać tę zdolność razem z wiedzą o obszarze w której jesteś już wyszkolony, by zostać wyspecjalizowanym. Umożliwienie.

\myability{Skupienie na Umiejętności} - Na początku każdego dnia, wybierz jeden typ ataku: lekki obuchowy, lekki cięty, lekki dystansowy, średni obuchowy, średni cięty, średni dystansowy, ciężki obuchowy, ciężki cięty lub ciężki dystansowy. Na resztę dnia, jesteś wyszkolony w atakach tego typu. Nie możesz użyć tej zdolności z atakiem, w którym jesteś już wyszkolony, by zostać wyspecjalizowanym. Umożliwienie.

\myability{Lot} (4+ punkty Intelektu) - Możesz lecieć przez godzinę. Na każdy zastosowany poziom Wysiłku, możesz pozwolić lecieć jednej dodatkowej istocie Twojego rozmiaru lub mniejszej.  Musisz dotknąć istoty, by dać jej zdolność lotu. Kierujesz ruchem tej istoty, a kiedy lecisz, musi ona zostać w Twoim polu widzenia lub spadnie. Jeśli chodzi o szybkość, latająca istota leci z szybkością około 32 km/h i trudny teraz nie utrudnia jej ruchu. Akcja, by rozpocząć. 

\myability{Krótki Lot} (3 punkty Mocy lub Szybkości) - Możesz lecieć do średniego zasięgu jako swój ruch w tej turze. Jeśli ograniczysz swoją akcję tylko do lotu, możesz się przemieścić na daleki zasięg. Umożliwienie.

\myability{Ucieczka, Nie Walka} - Jeśli cała Twoja akcja to ruch, wszystkie Twoje akcje obrony Szybkości są ułatwione. Umożliwienie.

\myability{Rzut} (4 punkty Intelektu) - Brutalnie rzucasz istotą lub obiektem mniej-więcej Twojego rozmiaru lub mniejszego w średnim zasięgu i sprawiasz, że leci na średni dystans w dowolnym kierunku. Jest to atak Intelektu, który zadaje 4 punkty obrażeń rzuconemu obiektowi gdy ten ląduje lub uderza w barierę. Jeśli celujesz w inną istotę lub obiekt (i uzyskasz sukces na 2 teście ataku), drugi cel także otrzymuje 4 punkty obrażeń. Akcja.

\myability{Latający Kompan} - Uzyskujesz kompana-istotę 3 poziomu który leci z Twoją prędkością - w zależności od innych aspektów Twojej postaci, może to byc wytrenowany ptak, dron lub pomocna istota, np: chowaniec. Ta istota towarzyszy Tobie i działa zgodnie z Twoimi poleceniami. Jako kompan 3 poziomu, jego stopień trudności to 9, posiada 9 punktów zdrowia i zadaje 3 punkty obrażeń. Jeśli zostanie zabity lub zniszczony, zajmuje Ci miesiąc znalezienie lub stworzenie odpowiedniego zastępstwa. Umożliwienie.

\myability{Negacja Zagrożenia} (2 punkty Intelektu) - Negujesz jedno źródło potencjalnego zagrożenia związane z jedną istotą lub obiektem, którego jesteś świadom w bliskim zasięgu na jedną rundę. Może to być broń lub urządzenie trzymane przez kogoś, pułapka aktywowana przez płytkę w podłodze, lub naturalna zdolność istoty (coś specjalnego, wrodzonego i niebezpiecznego, jak ognisty oddech smoka lub trucizna kobry królewskiej). Możesz także spróbować zanegować zwykłą akcję przeciwnika (taką jak atak bronią lub pazurami), tak, by tej akcji nie mógł wykonać w tej rundzie. Wykonaj rzut przeciwko poziomowi ataku, niebezpieczeństwa lub istoty. Akcja.

\myability{Uzdrawiająca Fontanna} - Za Twoją zgodą, inne istoty mogą CIę dotknąć i odzyskać 1d6 punktów ze swojej Puli Mocy lub Szybkości. Kosztuje je to 2 punkty Intelektu. Jedna istota może skorzystać z tej zdolności tylko raz dziennie. Umożliwienie.

\myability{Siła i Dokładność} - Zadajesz dodatkowe 3 punkty obrażeń ataki broniami rzucanymi. Umożliwienie.

\myability{Moc na Odległość} (4+ punkty Intelektu) - Chwilowo naginasz fundamentalne prawa grawitacji wokół istoty lub obiektu (do dwukrotności Twojej własnej masy) w średnim zasięgu. Cel jest schwytany w Twój telekinetyczny chwyt i możesz go przemieścić na średni zasięg w dowolnym kierunku w każdej rundzie, w której zostaje w Twoim chwycie. Istota może podjąć akcje, ale nie może się sama z siebie poruszać. Każdej rundy po pierwszym ataku, możesz chcieć spróbować utrzymać swój chwyt, wydając dodatkowe 2 punkty Intelektu i odnosząc sukces na zadaniu Intelektu 2 poziomu. Jeśli Twoja koncentracja się zerwie, cel opada na ziemię. W dodatku do normalnych opcji korzystania z Wysiłku, możesz z niego skorzystać, by zwiększyć masę obiektu/istoty, na którą chcesz wpłynąć. Każdy poziom pozwala Ci na podwojenie owej masy. Dla przykładu, zastosowanie poziomu Wysiłku pozwala Ci wpłynąć na istotę 4 razy tak masywną jak TY, 2 poziomy - na istotę 8 razy tak masywną jak Ty, 3 poziomy - na istotę 16 rqzy tak masywną jak Ty itp. Akcja, by rozpocząć.

\myability{Uderzenie Mocy} (1 punkt Mocy) - Wykonujesz atak wręcz swoją Tarczą Pola Siłowego. Twój atak zadaje o 1 punkt obrażeń mniej, ale oszałamia Twój cel na jedną rundę, podczas której wszystkie akcje, które on podejmuje, są utrudnione. Umożliwienie. 

\myability{Wystrzał Mocy} - Uczysz się, jak wyrzucić pocisk czystej mocy z rękawic swojego Zasilanego Pancerza. Pozwala Ci to wystrzelić pocisk mocy który zadaje 5 punktów obrażeń w zasięgu do 60 m. Akcja.

\myability{Pole Siłowe} (3 punkty Intelektu) - Tworzysz niewidzialne pole energetyczne wokół istoty lub obiektu w średnim zasięgu. Pole siłowe porusza się z istotą lub obiektem i trwa przez 10 minut. Jeśli calem jest istota, otrzymuje ona +1 do Pancerza; jeśli celem jest obiekt, ataki przeciwko niemu są utrudnione. 

\myability{Bariera Pola Siłowego} (3+ punkty Intelektu) - Tworzysz nieprzezroczystą, stacjonarną barierę sztywnej energii (pole siłowe) w bliskim zasięgu. Bariera ma 3 m na 3 m i niewielką grubość. Jest to bariera poziomu 2 i trwa przez 10 minut. Może być umieszczona wszędzie tam, gdzie się zmieści, albo w otoczeniu ciał stałych, albo unosząca się w powietrzu. Każdy poziom Wysiłku zwiększa siłę bariery o 1 poziom. Dla przykładu, zastosowanie 2 poziomów Wysiłku tworzy barierę 4 poziomu. Akcja.

\myability{Tarcza Pola Siłowego} - Tworzysz małą przestrzeń czystej mocy, która przyjmuje kształt tarczy, przy pomocy samej myśli. Możesz ją równie łatwo odesłać. By korzystać z tarcy pola siłowego, musisz ją trzymać w jednej ze swych dłoni. Jesteś wyszkolony w kozystaniu ze swojej egzotycznej tarczy jako lekkiej broni do ataku wręcz, jednakże, jeśli atakujesz zarówno tarczą, jak i bronią trzymaną w drugiej ręce, obydwa ataki są utrudnione. Kiedy jesteś nieprzytomny lub śpisz, pole siłowa znika. Umożliwienie. (Tarcza, wliczając tą stworzoną z pola siłowego, zapewnia atut na rzutach obronnych na Szybkość postaci, kiedy jest trzymana w jednej z dłoni.)

\myability{Niszczyciel Pól Siłowych} - Możesz niszczyć pola siłowe i bariery energetyczne, jakby były fizycznymi ścianami. Umożliwienie.

\myability{Ściana Mocy} (5 punktów Intelektu) - Możesz sprawić, by eneegia z Twojej Tarczy Pola Siłowego rozszerzyła się we wszystkich kierunkach, by stworzyć niemobilny plan stałej energii w wymiarach 6m na 6 m do jednej godziny lub do momentu, gdy wezwiesz swoją tarczę z powrotem (Pole siłowe staje się ścianą mocy). Ściana Mocy dostosowuje się do dostępnej przestrzeni. Kiedy ściana mocy zostaje w miejscu, nie możesz korzystać z innych zdolności które wymagają Twojej Tarczy Pola Siłowego. Akcja, by rozpocząć. 

\myability{Budujący umocnienia} - Kiedykolwiek pragniesz wykonać test rzemiosła - lub pomóc w nim - by zbudować ścianą lub inną fortyfikację, ułatwiasz zadanie o dwa stopnie, do minimalnej trudności 1. Umożliwienie. 

\myability{Ufortyfikowana Pozycja} (2 punkty Mocy) - Na następną minutę, zyskujesz +1 do Pancerza i atut na testach Obrony Mocy, tak długo, jak nie poruszyłeś się więcej niż o bliski zasięg od swojej ostatniej tury. Akcja, by rozpocząć. 

\myability{Złowróżebna Aura} (5+ punktów Intelektu) - Twoje słowa, gesty i dotyk wypełniają obiekt nie większy od Ciebie aurą zguby, strachu i zwątpienia na jeden dzień. Istoty, które słuchają Cię i Cię rozumieją czują potrzebę, by odsunąć się przynajmniej na średni dystans od obiektu. Jeśli istota się nie oddali, wszystkie zadania, ataki i obrona, którą wykonuje kiedy w otoczeniu złowróżebnej aury, są utrudnione. Długość trwania aury jest rozciągnięta na jeden dzień na każdy poziom Wysiłku. Aura jest chwilowo zablokowana, kiedy obiekt jest przykryty lub zamknięty. Akcja, by rozpocząć. 

\myability{Przeogromny} - Twój wielki rozmiar przeraża większość ludzi. Kiedy korzystasz z efektów Wzrostu, wszystkie zadania zastraszania są dla Ciebie ułatwione. Umożliwienie. 

\myability{Mistrz Ruchu} - Ignorujesz wszelkie kary do przemieszczania się wynikające z terenu lub innych przeszkód. Możesz przejść przez każdą przestrzeń dostatecznie dużą, by zmieściła się w niej Twoja głowa. Zadania wyzwalania się z więzów, chwytu innych istot lub podobnych utrudnień zyskują darmowe 3 poziomy Wysiłku. Umożliwienie. 

\myability{Mrożący Dotyk} (4 punkty Intelektu) - Twoje ręce stają się tak bardzo zimne, że Twój dotyk zamraża żyjący cel Twojego rozmiaru lub mniejszy, przytrzymując go w miejscu na jedną rundę. Jeśli masz inną zdolność mrozu aktywowana przez dotyk (taką jak np: Lodowy Dotyk) możesz z niej skorzystać jako z części swojego Mrożącego Dotyku. Akcja.

\myability{Szał} (1 punkt Intelektu) - Kiedy sobie tego zażyczysz, w trakcie walki możesz wejść w stan szału. W tym stanie, nie możesz korzystać z punktów Intelektu, ale zyskujesz +1 do Skupienia w Mocy i w Szybkości. Ten efekt trwa tak długo, jak sobie zażyczysz, ale kończy się, jeśli walka nie trwa w zasięgu Twoich zmysłów. Umożliwienie.

\myability{Pomoc Przyjaciela} - Jeśli Twój przyjaciel próbuje zadania i mu się nie powodzi, może on spróbować ponownie bez wydania punktów na Wysiłek, jeśli mu pomożesz. Zapewniasz przewagę swojemu prZyjacielowi nawet, jeśli nie jesteś wyszkolony w zadaniu, które próbujecie wykonać ponownie. Umożliwienie. 

\myability{Z Cieni} - Jeśli zaatakujesz z sukcesem istotę, która wcześniej była nieświadoma Twojej obecności, zadajesz dodatkowe 3 punkty obrażeń. Umożliwienie.

\myability{Lodowy Dotyk} (1 punkt Intelektu) - Twoje ręce stają się tak zimne, że następnym razem, gdy dotkniesz istoty zadajesz 3 punkty obrażeń. Alternatywnie, możesz użyć tej zdolności na broni, i przez 10 minut, zadaje ona dodatkowy 1 punkt obrażeń od zimna. Akcja by dotknąć, umożliwienie dla broni.

\myability{Najwyższa Matematyka} - Jesteś wyspecjalizowany w wyższej matematyce. Jeśli jesteś już w niej wyspecjalizowany, wybierz inną dziedzinę wiedzy, w której zostajesz wytrenowany. Umożliwienie. 

\myability{Furia} (3 punkty Mocy) - Przez następną minutę, wszystkie ataki wręcz, które wykonujesz, zadają dodatkowe 2 punkty obrażeń. Akcja, by rozpocząć. 

\myability{Fuzja} - Możesz wbudować swoje zamanifestowane cyphery i artefakty w swoje ciało. Funkcjonują one wtedy, jakby miały o poziom więcej. Umożliwienie.

\myability{Zbroja Fuzyjna} - Medyczna procedura dała Ci bio-metalowe implanty w głównych częściach Twojego ciała; posiadasz skórę o twardości metalu; błogosławieństwo aniołów Cię chroni; lub coś podobnego występuje. Te zmiany dają Ci +1 do Pancerza nawet wtedy, gdy nie nosisz fizycznego pancerza. Umożliwienie. 

\section{G}

\myability{Pozyskanie Nietypowego Kompana} - Uzyskujesz specjalną istotę jako trwałego kompana. Jest ona na poziomie 4, najpewniej rozmiaru małego psa, i słucha Twoich telepatycznych komend. Ty i MG musicie określić szczegoły tej istoty, i wykonasz najpewniej za nią rzuty w walce lub gdy wykonuje ona akcje. Kompan działa w Twojej turze. Jeśli Twój kompan umiera, możesz polować w dziczy przez 1k6 dni, by znaleźć nowego. Umożliwienie.

\myability{Ryzykant} - Każdego dnia, wybierz dwa numery od 2 do 16. Jeden z nich jest Twoim szczęśliwym numerem, drugi pechowym. KIedy wykonasz rzut tego dnia i numer na kostce pokrywa się z szczęśliwym numerem, Twoje zadanie jest ułatwione. Jeśli rzut pokrywa się z  pechowym numerem, Twoje zadanie jest utrudnione. Umożliwienie.

\myability{Lekcje z Gier} - Grałeś w tak wiele gier, że pozyskałeś z nich trochę wiedzy. Wybierz dowolne dwie umiejętności nieprzydatne w walce. Jesteś wyszkolony w tych umiejętnościach. Umożliwienie.

\myability{Gamer} - Wybierz jeden gatunek gier, takie jak RTSy, gry losowe jak poker, gry ttRPG itp. Możesz zastosować atut do zadać związanych z graniem w te gry jeden raz pomiędzy każdym rzutem na odzyskanie zdrowia. Umożliwienie. 

\myability{Wytrzymałość Gracza} - Siedzenie i granie w grę przez 12 godzin bez przerwy nie jest czymś, co większość ludzi potrafi zrobić, ale Tobie się to udało. Jeden raz po każdym rzucie na odzyskanie zdrowia (10-godzinnym), możesz przemieścić do 5 punktów między swoimi Pulami w dowolnej kombinacji, w tempie 1 punktu na rundę. Dla przykładu, możesz przemieścić 3 punkty z Mocy w Szybkość i 2 punkty z Intelektu w Szybkość, co w sumie zajęłoby Ci 5 rund. Akcja.

\myability{Bóg Gier} - Za każdym razem, gdy zastosujesz Wysiłek na akcji Intelektu, dodaj jedno z poniższych ulepszeń do swojej akcji (Twój wybór):
\begin{itemize}
	\item Darmowy poziom Wysiłku
	\item Automatyczny mniejszy efekt.
\end{itemize}
Umożliwienie.

\myability{Wielgachny} - Kiedy korzystasz z Wzrostu, możesz wybrać wzrost do 9 metrów i dodajesz 3 dodatkowe punkty do swojej Puli Mocy (jeśli masz też zdolność Większy, to dodatkowe punkty w Puli mocy się kumulują). Umożliwienie.

\myability{Agent Wywiadu} (2 punkty Intelektu) - Kiedy jesteś w grupie ludzi (karawanie, pałacu, wiosce, mieście itp.) możesz zapytać o dowolny wybrany przez siebie temat i odejść z użyteczną informacją. Możesz zadać konkretne pytanie, lub po prostu gromadzić ogólne fakty. uzyskujesz także informacje o ogólnych cechach fizycznych lokacji, odnotowujesz obecność ważniejszych miejscówek, i może nawet odnotowujesz pewne ważne, pomniejsze detale. Dla przykładu, nie tylko dowiadujesz się, czy ktoś w pałacu widział zaginionego chłopaka, ale także rozumiesz umieszczenie pomieszczeń w pałacu, odnotowujesz wszystkie wejścia i to, które z nich są najczęściej używane, a także zauważasz, że wszyscy najwyraźniej unikają studni we wschodnim dziedzińcu.  Akcja, by rozpocząć, około 1 godziny, by zakończyć.

\myability{Generacja Pola Siłowego} (9+ punktów Intelektu) - Wytwarzasz 6 płaszczyzn czystej mocy (poziom 8), każde o boku 9 m, które istnieją przez godzinę. Te płaszczyzny muszą być ciągłe, i zachowują swoją pozycję wybraną przy ich tworzeniu. Dla przykładu, możesz umieścić te płaszczyzny w linii, tworząc ścianę długą na 55 m, lub możesz stworzyć zamkniętą kość. Płaszczyzny dostosowują się do dostępnej przestrzeni. Każdy dodatkowy poziom Wysiłku zwiększa poziom barier o 1 (do maksimum 10) lub zwiększa liczbę godzin, przez które one istnieją o 1. Akcja, by rozpocząć.

\myability{Ucieknij} (2 punkty Szybkości) - Po Twojej akcji w swojej turze, przemieszczać się o średni zasięg lub za zasłonę w bliskim zasięgu. Umożliwienie. 

\myability{Duch} (4 punkty Intelektu) - Przez następne 10 minut, uzyskujesz atut do akcji skradania się. W tym czasie, możesz się przemieszczać przez fizyczne bariery (ale nie energetyczne) w tempie 30 cm na rundę, i możesz postrzegać kiedy jesteś wfazowany w barierę lub obiekt, co pozwala Ci na widzenie przez ściany. Akcja, by rozpocząć. 

\myability{W Defensywie} (1 punkt Intelektu) - Podczas walki ,kiedy sobie tego zażyczysz, możesz wejść w stan zwiększonej świadomości niebezpieczeństwa. W tym stanie, nie możesz korzystać z punktów z Puli Intelektu, ale uzyskujesz +1 do Skupienia w Szybkości i zyskujesz 2 atuty na Obronie Szybkości. Ten efekt trwa tak długo, jak sobie zażyczysz lub dopóki nie zaatakujesz wroga lub walka przestanie się toczyć w zasięgu Twoich zmysłów. Kiedy efekt tej zdolności się skończy, nie możesz z niej skorzystać przez 1 minutę. Umożliwienie.

\myability{Ponowne Ukrycie Się} (4 punkty Szybkości) - Przemieszczasz się na daleki zasięg i próbujesz ukryć. Kiedy to robisz, zyskujesz atut na akcjach ukrywania się, znikania, lub  inny sposób unikania wykrycia przez zmysły ludzi, którzy wcześniej byli świadomi Twojej obecności. Akcja.

\myability{Prowokacja} (1 punkt Intelektu) - Próbujesz sprowokować cel tak, by podjął naglą - i najpewniej głupią - reakcję, która wymaga od celu, by zbliżył się do Ciebie i spróbował Cię fizycznie zaatakować w swojej następnej turze. Robi on to nawet, jeśli oznacza to opuszczenie formacji lub opuszczenie zasłony lub taktycznie lepszej pozycji. Niezależnie od tego, czy cios celu uderza w Ciebie lub nie, cel odzyskuje swoje zmysły natychmiastowo potem, a dalsze próby prowokacji celu są utrudnione. Akcja, by rozpocząć. 

\myability{Ciało Golema} - Uzyskujesz +1 do Pancerza, +1 do Skupienia w Mocy i 5 dodatkowych punktów do swojej Puli Mocy. Nie potrzebujesz jeść, pić lub oddychać (ale dalej potrzebujesz odpoczynku i snu). Poruszasz się powolniej niż istoty z ciała i krwii, co sprawia, że nigdy nie możesz posiadać treningu lub specjalizacji w Obronie Szybkości. Co więcej, jesteś wyszkolony w używaniu swoich kamiennych pięści jako średniej broni. Umożliwienie.

\myability{Chwyt Golema} (3 punkty Mocy) - Twój atak piesciami golema z Ciała Golema jest ułatwiony. Jeśli trafisz, przytrzymujesz cel, zapobiegając jego ruchowi w następnej turze. Kiedy przytrzymujesz cel, jego ataki lub próby wyzwolenia się są utrudnione. Jeśli cel woli się spróbować wyzwolić zamiast zaatakować, musisz wykonać test Mocy, by go przytrzymać. Jeśli cel nie wyzwoli się z Twojego chwytu, możesz kontynuować swój chwyt w następnych turach, automatycznie zadając 4 punkty obrażeń poprzez uciskanie. Umożliwienie.

\myability{Uzdrawianie Golema} -Twoje kamienne ciała ze zdolności Ciało Golema jest trudniejsze w naprawie niż zwykłe ciało, co oznacza, że nie możesz skorzystać z pierwszego rzutu na odzyskanie zdrowia (trwającego jedną akcję), w przeciwieństwie do innych BG. Tak więc, Twój pierwszy rzut na odzyskanie zdrowia zajmuje 10 minut, drugi godzinę, a trzeci 10 godzin. Umożliwienie.

\myability{Tupnięcie Golema} (4 punkty Mocy) - Tupiesz w ziemię z całej swojej mocy, tworząc falę, która atakuje wszystkie istoty w bliskim zasięgu. Istoty te otrzymują 3 punkty obrażeń i albo odsuwaja się od Ciebie na bliski zasięg, albo upadają (Twój wybór). Akcja.

\myability{Dobra Porada} - Każdy może pomóc sprzymierzeńcowi, ułatwiając dowolne zadanie, które ten próbuje wykonać. Jednakże, Ty posiadasz korzyść jasności i mądrości. Kiedy pomagasz innej postaci, otrzymuje ona dodatkowy atut. Umożliwienie.

\myability{Przeczucie} (4 punkty Intelektu) - Posiadasz niezwykłą intuicję kiedy przychodzi do znajdowania rzeczy. Kiedy badasz jakieś miejsce, możesz rozszerzyć swoje zmysły do 1,5 km w dowolnym kierunku i zadać MG bardzo proste, ogólne pytanie - zazwyczaj typu tak lub nie - o tym obszarze, takie jak ``Czy jest tutaj jakaś wioska orków'' lub ``czy w tych ruinach mogę znaleźć ciemną materię?''. Jeśli odpowiedź na Twoje pytanie nie znajduje się w tym obszarze, otrzymujesz zero informacji. Akcja.

\myability{Chwyt} - Kiedy korzystasz ze zdolności Wzrost, możesz zaatakować, próbując objąć swoimi obszernymi dłońmi cel rozmiaru zwykłego człowieka lub mniejszy. Kiedy utrzymujesz ten uchwyt jako swoją akcję, uniemożliwiasz celowi ruch lub podejmowanie fizycznych akcji (innych niż próby ucieczki). Próby ucieczki celu są utrudnione o dwa stopnie ze względu na Twój rozmiar. Jeśli sobie tego życzysz, możesz automatycznie zadać 3 punkty obrażeń w każdej turze w której trzymasz cel, ale możesz również go chronić (biorąc na siebie ataki wymierzone w cel). Akcja.

\myability{Wielkie Oszustwo} (3 punkty Intelektu) - Przekonujesz inteligentną istotę ,która Cię rozumie i nie jest wroga o czymś absolutnie wyssanym z palca i w oczywisty sposób nieprawdziwym. Akcja. 

\myability{Wielka Iluzja} (8 punktów intelektu) - Tworzysz fantastycznie skomplikowaną scenę obrazów która obejmuje sześcian o boku długości 1.5 kilometra i wszystko wewnątrz niego. Musisz być w stanie widzieć obrazy, gdy je tworzysz. Obrazy te mogą się poruszać w sześcianie i działać zgodnie z Twoimi życzeniami. Mogą one także działać logicznie (np: odpowiednie reagując na ogień lub atak) kiedy ich bezpośrednio nie obserwujesz. W iluzję wlicza się także dźwięk i zapach. Dla przykładu, armie mogą się zderzyć w bitwie, wraz z maszynami latającymi lub istotami w locie, w i ponad terenem Twojej kreacji. Iluzja trwa przez jedną godzinę (lub dłużej, jeśli się skoncentrujesz po tym czasie). Akcja.

\myability{Ściana z Granitu} (7+ punktów Intelektu) - Tworzysz ścianę poziomu 6 stworzoną z granitu, w bliskim zasięgu. Ściana ma 30 cm grubości i ma około 6 m x 6 m rozmiaru. Zdaje się ona być wybudowana na solidnym fundamencie i istnieje przez około 10 godzin. Jeśli zastosujesz 3 poziomy Wysiłku, ściana jest permanentna do momentu jej naturalnego zniszczenia. Akcja, by rozpocząć. 

\myability{Chwytające Zielska} (3+ punktów Intelektu) - Korzenie, gałęzie, trawa i inne naturalne rośliny w obszarze chwytają wroga, którego wskażesz w średnim zasięgu na jedną minutę. Wróg schwytany przez Chwytające ZIelska nie może się ruszyć, a wszystkie jego fizyczne zadania, ataki i obrony są utrudnione, wliczając próby wyzwolenia się. W dodatku do normalnych zastosować Wysiłku, możesz wybrać Wysiłek, by zadać obrażenia w swoim pierwszym ataku. Każdy poziom zastosowanego Wysiłku zadaje 2 dodatkowe punkty obrażeń kiedy Chwytające Zielska po raz pierwszy atakują i chwytają wroga. 

Możesz także użyć tej zdolności,by wyczyścić obszar obfitego porostu roślin w bliskim zasięgu, takie jak wysoka trawa, gęste zarośla, żywopłoty, liany itp. Akcja.

\myability{Szarpnięcie Grawitacyjne} (3 punkty Intelektu) - Możesz skrzywdzić cel w średnim zasięgu, poprzez nagłe zwiększenie grawitacji z jednym obszarze jego ciała i obniżenia go w innym, zadając 6 punktów obrażeń. Akcja.

\myability{Wielkie Drzewo} - Kiedy korzystasz z Ciała z Drewna, możesz wzrosnąć do 4 m wysokości. W tej większej formie, dodajesz 7 punktów do swojej Puli Mocy i +2 do Skupienia w Mocy. Jeśli wybierzesz wzrost, kiedy Ciało z Drewna się kończy, odejmujesz 7 punktów od swojej puli Mocy (jeśli obniży to Pulę do 0, odejmij punkt yz Puli Szybkości, a jeśli to dalej konieczne, z Puli Intelektu). Kiedy korzystasz z Ciała z Drewna, niezależnie od tego, czy wzrastasz czy nie, zamiast wyglądać jak drewniana wersja siebie, możesz przyjąć wygląd humanoidalnego drzewa lub zwykłego drzewa (wliczają się w to gałęzie, liście itp.). Nie wypływa to na żadne z Twoich zdolności - w kształcie drzewa, dalej możesz korzystać ze swoich zdolności typu, specjalizacji itp. W kształcie drzewa, udawanie bycia drzewem i ukrywanie się wśród normalnych drzew jest ułatwione o 2 stopnie. Umożliwienie.

\myability{Większa Likantropia} - Kiedy korzystasz z Likantropii, Twoja forma zwierzęca otrzymuje poniższe bonusy: +1 do Skupienia w Mocy, +2 do Puli Szybkości i +1 do Skupienia w Mocy. Umożliwienie.

\myability{Większa Kontrolowana Zmiana} Łatwiej jest zmienić się w kształt oferowany Ci przez Likantropię. Zamiana w dowolną stronę to teraz zadanie Intelektu trudności 2. Umożliwienie.

\myability{Większy Osąd} - Możesz określić wszystkie istoty w bliskim zasięgu jako niewinne lub winne, kiedy korzystasz z Osądu. Wszystkie istoty muszą zostać osądzone tak samo. Trwa to, aż użyjesz Większego Osądu ponownie. Akcja.

\myability{Większy Ulepszony Intelekt} - Uzyskujesz 6 punktów w Puli Intelektu. Umożliwienie.

\myability{Większa Ulepszona Moc} - Uzyskujesz 6 punktów w Puli Mocy. Umożliwienie. 

\myability{Większa Ulepszona Muskulatura} - Uzyskujesz 6 punktów do podziału pomiędzy Pule Mocy i Szybkości wedle swego uznania. Umożliwienie.

\myability{Większy Ulepszony Potencjał} - Uzyskujesz 6 punktów do podziału pomiędzy swoje Pule wedle swego uznania. Umożliwienie.

\myability{Większa Ulepszona Szybkość} - Uzyskujesz 6 punktów w Puli Szybkości. Umożliwienie.

\myability{Większy Szał} (4 punkty Intelektu) - Kiedy sobie tego życzysz, w czasie walki, możesz wejść w szał. W tym stanie nie możesz korzystać z punktów Intelektu, ale uzyskujesz +2 do Skupienia w Mocy i Szybkości. Efekt trwa tak długo, jak sobie tego życzysz, ale kończy się, jeśli nie ma żadnych walk w zasięgu Twoich zmysłów. Jeśli posiadasz zdolność Szał, możesz skorzystać z jednej z tych dwóch zdolności, ale nie z dwóch jednocześnie. Umożliwienie.

\myability{Większy Leczący Dotyk} (4 punkty Intelektu) - Dotykasz istoty i odnawiasz jej wszelkie punkty w Pulach Mocy, Szybkości i Intelektu, do maksymalnych wartości. Dana istota może korzystać z tej zdolności tylko raz na dzień. Akcja.

\myability{Większa Nekromancja} (5+ punktów Intelektu) - Ta zdolność działa jak zdolność Nekromancja, ale tworzy istotę 3 poziomu. Akcja, by stworzyć.

\myability{Większa Umiejętność Ataku} - Wybierz jeden typ ataku, nawet taki, w którym już jesteś wytrenowany: lekki obuchowy, lekki cięty, lekki dystansowy, średni obuchowy, średni cięty, średni dystansowy, ciężki obuchowy, ciężki cięty, lub ciężki dystansowy. Jesteś wyszkolony w atakach korzystających z takiej broni. Jeśli jesteś już wyszkolony w takim typie ataku, teraz jesteś w nim wyspecjalizowany. Umożliwienie. 

\myability{Większa Umiejętność Obrony} - Wybierz jeden typ Obrony, nawet taką, w której już jesteś wyszkolony: Mocy, Szybkości lub Intelektu. Jesteś wyszkolony w zadaniach obrony tego typu, lub wyspecjalizowany, jeśli już byłeś wcześniej wyszkolony. Możesz wybrać tę zdolność trzykrotnie, za każdym razem wybierając odmienną obronę. Umożliwienie.

\myability{Paczka Przyjaciół} (4 punkty Intelektu) - Przekonujesz świadomą istotę, by traktowała Cię (i do 10 istot, które wybierzesz w bliskiej odległości od Ciebie) w pozytywny sposób, tak, jak traktuje się potencjalnego przyjaciela. Akcja.

\myability{Nakierowywany Pocisk} (4+ punktów Intelektu) - Kiedy wykonujesz atak przy pomocy metalowego pocisku (np: czubka strzały) przeciwko celowi w średnim zasięgu, możesz polepszyć celność i szybkość ataku, co daje Ci atut na ataku i zadaje dodatkowe 2 punkty obrażeń. Jeśli zastosujesz poziom Wysiłku, te same korzyści odniesie sprzymierzeniec w bliskim zasięgu. W każdym wypadku, możesz skorzystać z tej zdolności tylko raz na rundę. Umożliwienie. 

\myability{Trening Gildii} - Zdolności zapewniane przez Twój typ, które mają czas trwania, trwają 2 razy tyle. Twoje zdolności typu posiadająca średni zasięg posiadają teraz daleki zasięg. Twoje zdolności typu zadające obrażenia zadają teraz 1 dodatkowy punkt obrażeń. Umożliwienie. 

\myability{Rewolwerowiec} - Zadajesz 1 dodatkowy punkt obrażeń przy pomocy broni palnej. Umożliwienie.

\section{H}

\myability{Hakowanie Niemożliwości} (3 punkty Intelektu) - Możesz przekonać roboty, maszyny i komputery, żeby robiły to, czego pragniesz. Możesz odkryć zaszyfrowane hasło, przebić się przez zabezpieczenia strony internetowej, wyłączyć na chwilę maszynę, jak np: kamerę bezpieczeństwa, lub wyłączyć robota poprzez chwilowy wysiłek. Akcja.

\myability{Haker} (2 punkty Intelektu) - Uzyskujesz szybki dostęp do pożądanej informacji w komputerze lub podobnym urządzeniu, lub dostęp do jego podstawowych funkcji. Akcja.

\myability{Koordynacja Ręka-Oko} (2 punkty Szybkości) - Ta zdolność zapewnia atut do każdego zadania bazującego na manualnej zręczności, takiego jak kradzież kieszonkowa, otwieranie zamków, gry zręcznościowe itp. Każdy wykorzystanie trwa przez minutę, a nowe wykorzystanie (by zmienić rodzaj akcji) zastępuje stare. Akcja, by rozpocząć.

\myability{Zręczny Rzemieślnik} - Pracujesz, by wyżyć i jesteś wyszkolony w zadaniach związanych ze stolarstwem, hydrauliką i naprawą elektryczną. Twoja wiedza w tym zakresie daje CI także atut by tworzyć zupełnie nowe przedmioty w zależności od Twojej wiedzy i limitów możliwości settingu. Umożliwienie.

\myability{Trudne Wybory} - Czasami, wierzysz, że musisz skłamać tym, którym ufasz dla ich własnego dobra. Jesteś wyspecjalizowany w oszustwie. Umożliwienie.

\myability{Trudny Cel} - Jeśli poruszysz się na średnią odległość lub dalej w swojej turze, wszystkie Twoje akcje Obrony Szybkości są ułatwione. Umożliwienie.

\myability{Ciężki do Rozproszenia} - Jesteś wyszkolony w Obronie Intelektu. Umożliwienie.

\myability{Trudny do Trafienia} - Jesteś wyszkolony w Obronie Szybkości. Umożliwienie.

\myability{Trudny do Zamordowania} - Możesz przerzucić dowolny rzut obronny, który wykonujesz, ale nie więcej niż ran na rundę. Umożliwienie.

\myability{Rozmazany} - Kiedy się poruszasz, jesteś rozmazany. Nie da się określić Twojej tożsamości w ruchu, a w rundzie, w której nie robisz nic innego, tylko się poruszasz, rzuty na skradanie się i Obronę Szybkości są ułatwione. Umożliwienie.

\myability{Ciężko Zapracowana Odporność} - W swoich badaniach ciemnych miejsce,zobaczyłeś całą masę okropnych rzeczy i wyrobiłeś sobie na nie odporność. Otrzymujesz +1 do Pancerza i jesteś wyszkolony w obronie Mocy. Umożliwienie. 

\myability{Twardsze Światło} - Kiedy tworzysz obiekt z twardego światła, poziom tego obiektu jest o 1 wyższy niż normalnie. Umożliwienie.

\myability{Wytrzymały} - Jesteś wyszkolony w Obronie Mocy. Umożliwienie.

\myability{Mam Kombinezon Kosmiczny, Będę Podróżnikiem} - W taki czy inny sposób, jesteś legalnym właścicielem w pełni funkcjonalnego, zaawansowanego kombinezonu kosmicznego. Kombinezon zapewnia +1 do Pancerza i, co ważniejsze, pozwala Ci przetrwać w próżni przez do 12 godzin z odpowiednią liczbą paliwa, by poruszać się w przestrzeni kosmicznej za pomocą zjonizowanego gazu przez ten czas. Po każdym użyciu, należy naładować kombinezon, albo z już dostępnymi kartridżami powietrza i gazu zjonizowanego, lub pozwoliwszy kombinezonowi na siedzenie w atmosferze przez 2 godziny, podczas których zregeneruje zarówno tlen, jak i zjonizowany gaz. Kombinezon jest zasilany radioizotopem w termo-elektrycznym generatorze, co oznacza, że będzie funkcjonował przez kilka dekad, nim trzeba w nim zmienić paliwo. Umożliwienie. 

\myability{Wyświetlacz w Hełmie} (2+ punktów Intelektu) - Twój Zasilany Pancerz posiada system pomagający Ci w zrozumieniu, zanalizowaniu i użyciu broni w Twoim środowisku. Kiedy korzystasz z tej zdolności, uzyskujesz atut na jednym teście na atak, gdy Pancerz perfekcyjnie podświetla wrogów i ulepsza Twój cel, niezależnie od tego, czy wykonujesz atak wręcz, czy dystansowy. 

Alternatywnie, możesz skorzystać z Wyświetlacza w Hełmie, by powiększyć jakiś obraz, zwiększając zasięg Twojego wzroku do 8 km na 2 rundy. Jeśli zastosujesz poziom Wysiłku, możesz także widzieć poprzez zwykłe materiały (takie jak drewno, beton, plastik i kamień) w średnim zasięgu w kolorze. Jeśli zastosujesz 2 poziomy Wysiłku, możesz widzieć poprzez specjalne materiały (takie jak ołów i inne substancje) w bliskim zasięgu w kolorze, jednakże, MG może chcieć od Ciebie sukcesu na zadaniu Intelektu, w zależności od tego, jaki materiał blokuje Twój wzrok. Umożliwienie.

\myability{Leczący Puls} (3 punkty Intelektu) - Ty i cele, które wybierzesz w bliskim zasięgu zyskujecie korzyści ze skorzystania z jednego z Waszych rzutów na odzyskanie zdrowia (nie liczy się tutaj rzut 10-godzinny) bez konieczności poświęcenia akcji, 10 minut lub 1 godziny. Cele odzyskują natychmiastowo punkty w swoich Pulach, ale zaznaczają skorzystanie z tego rzutu. BG którzy już skorzystali ze swojego rzutu na odzyskanie zdrowia trwającego 1 akcję, 10 minut i 1 godzinę nie odnoszą korzyści z tej zdolności. BN-i będący celem tej zdolności odzyskują liczbę punktów zdrowia równą swojemu poziomowi. Akcja.

\myability{Leczący Dotyk} (1 punkt Intelektu) - Za pośrednictwem dotyku, przywracasz 1k6 punktów e jednej Statystyce jednej istocie. Ta zdolność to test Intelektu o trudności 2. Za każdym razem gdy chcesz uleczyć tę samą istotę, zadanie jest utrudnione o dodatkowy stopień. Trudność wraca do 2 po tym, jak ta istota odpocznie przez 10 godzin.

\myability{Sztuczki Magiczne} (1 punkt Intelektu) - Możesz czynić proste sztuczki - chwilowo zmienić kolor lub wygląd małego obiektu, sprawić, by mały obiekt leciał w powietrzu, oczyścić mały obszar, naprawić zepsuty obiekt, przygotować (lecz nie stworzyć) jedzenie itp. Nie możesz skorzystaj z tej zdolności, by skrzywdzić istotę lub obiekt. Akcja.

\myability{Ulepszone Umiejętności} - Jesteś wyszkolony w dwóch umiejętnościach swojego wyboru (innych niż atak i obrona). Jeśli wybierzesz zdolność,w której już jesteś wytrenowany, zamiast tego jesteś w niej wyspecjalizowany. Nie możesz wybrać umiejętności, w której już jesteś wyspecjalizowany. Umożliwienie.

\myability{Krwawienie} (2+ punkty Mocy) - Wykonujesz potężny i precyzyjny cios, który zadaje dodatkowe obrażenie później. W Twojej następnej turze, cel tego ataku uzyskuje dodatkowe 3 punkty obrażeń (ignorują Pancerz). Cel może uchronić się od nich, wykonując rzut na odzyskanie zdrowia, korzystając ze zdolności leczącej, lub poświęcając akcję, by obandażować ranę. W dodatku do zwykłych opcji korzystania z Wysiłku, możesz z niego skorzystać, aby zwiększyć czas trwanie tej zdolności o jedną rundę. Akcja. 

\myability{Heroiczny Sposób na Potwory} - Kiedy zadajesz obrażenia istotom więcej niż dwa razy większym lub masywniejszym od Ciebie ,zadajesz dodatkowe 3 punkty obrażeń. Umożliwienie. 

\myability{Szuflada-Skrytka} - Twój magiczny kompan ze zdolności Związana Magiczna Istota może przechować dla Ciebie obiekty w swoim magicznym przedmiocie (tym, do którego jest uwiązana), wliczając dodatkowe ubrania, narzędzia, jedzenie itp. Wnętrze tego obiektu jest, de facto, sześcianem o boku 3 m, które normalnie tylko magiczny kompan może dosięgnąć. Umożliwienie.

\myability{Ukryta Siła} - Kiedy korzystasz z akcji, by wykonać rzut na odzyskanie zdrowia, uzyskujesz dodatkowo +1 do Skupienia w Mocy i SKupienia w Szybkości przez następne 10 minut. Umożliwienie.

\myability{Wyższa Matematyka} - Jesteś wyszkolony w zwykłej i wyższej matematyce. Umożliwienie.

\myability{Wstrzymanie Oddechu} - Możesz wstrzymać swój oddech na 5 minut. Umożliwienie.

\myability{Walczący z Hordą} - Kiedy dwóch lub więcej wrogów atakuje Cię wręcz, możesz ich nasłać na siebie. Uzyskujesz atut w Obronie Szybkości lub testach ataku (twój wybór w każdej rundzie) przeciwko nim. Umożliwienie.

\myability{Taktyka Hordy} (7 punktów mocy) - Przez go godziny na dzień, Ty i przynajmniej trzech sprzymierzeńców działacie jak jedna istota. Użys swoich statystyk, ale dodaj +8 do Puli Mocy, +1 do Skupienia w Mocy, +2 do Puli Szybkoci, +1 do Skupienia w Szybkości, i +1 do Pancerza. Umożliwienie. 

\myability{Unoszenie się} (2 punkty Intelektu) - Unosisz się powoli w powietrzu. Jeśli sie skoncentrujesz, możesz kontrolować swój lot - albo być bez ruchu w powietrzu, albo przemieścić się na średni zasięg jako akcję - inaczej unosisz się z wietrem lub pędem, który zachowałeś. Ten efekt trwa do 10 minut. Akcja, by rozpocząć. 

\myability{Jak Myślą Inni} - Rozumiesz, jak inni ludzie rozumieją. Jesteś wyszkolony w jednym z następujących zadań: perswazji, oszustwie lub wykrywaniu oszustw. Umożliwienie. 

\myability{Wielki} - Kiedy korzystasz ze Wzrostu, możesz osiągnąć wysokość 5 m, Kiedy tak robisz, otrzymujesz +1 do Pancerza (w sumie +2) i zadajesz dodatkowe 2 punkty obrażeń atakami wręcz. Umożliwienie. 

\myability{Dążenie Łowcy} (5 punktów Intelektu) - Poprzez moc swej woli, kiedy sobie tego zażyczysz, zyskujesz większą moc w celu polowania na 10 minut. W tym czasie, uzyskujesz atut na wszelkich zadaniach związanych ze swoją zdobyczą, wliczając atakowanie jej. Twoja zdobycz jest wybrana za pośrednictwem zdolności Cel. Umożliwienie.

\myability{Macki Płomieni} (2 punkty Intelektu) - Kiedy Twój Płaszcz Ognia jest aktywny, możesz sięgnąć do swojej aury ognia i zaatakować ogniem cel. Jest to atak dystansowy działający w bliskim zasięgu i zadaje 4 punkty obrażeń. Akcja. 

\section{I}

\myability{Lodowa Zbroja} (1 punkt Intelektu) - Kiedy sobie tego życzysz, Twoje ciało pokrywa się powłoką lodu na 10 minut, co daje Ci +1 do Pancerza. Kiedy Lodowa Zbroja jest aktywna, nie czujesz dyskomfortu w związku z normalnymi temperaturami ujemnymi i uzyskujesz dodatkowe +2 do Pancerza przeciwko obrażeniem od zimna. Umożliwienie.

\myability{Lodowa Kreacja} (4+ punktów Intelektu) - Tworzysz obiekt z lodu Twojego rozmiaru lub mniejszy. Ten obiekt jest bardzo ogólnikowo zrobiony i nie może mieć ruchomych części. Tak więc możesz stworzyć miecz, tarczę lub krótką drabinę, itp. Twoje lodowe przedmioty są tak twarde jak żelazo, ale jeśli nie jesteś w stałym kontakcie z nimi, istnieją tylko przez 1k6+6 tur zanim się roztopią lub zepsują. Dla przykładu, możesz stworzyć i dzierżyć lodowy miecz, aje jeśli dasz go innemu BG, tmiecz nie będzie istniał zbyt długo. W dodatku do normalnych opcji korzystania z Wysiłku, możesz z niego skorzystać, by stworzyć obiekty większe od Ciebie. Na każdy poziom Wysiłku wykorzystany w ten sposób, możesz stworzyć obiekt do dwóch razy większy od Ciebie. Akcja.

\myability{Śniegowa Zamieć} - Możesz spróbować wykonać dodatkowy test Intelektu jako część swojego ataku Emisji Zimna, i jeśli się on powiedzie, oślepiasz wrogów do 1 minuty powłoką mrożącego lodu. Wszystkie akcje oślepionych istot są utrudnione o 2 stopnie. Umożliwienie.

\myability{Zapłon} (4 punkty Intelektu) - Wybierasz istotę lub obiekt który może płonąć w średnim zasięgu. Jest to zadanie Intelektu. Cel otrzymuje 6 punktów obrażeń środowiskowych na rundę, aż ognie nie zostaną zduszone. Isttoa może to zrobić poprzez zanurzenie w wodzie, rzuceniu się na ziemię lub coś podobnego. Zazwyczaj, zduszenie płomieni zajmuje akcję. Akcja, by rozpocząć. 

 \myability{Zignorowanie Przeszkody} (5 punktów Mocy) - Jeśli dotyka się niechciana przeszkoda lub kondycja (np: zaraza, paraliż, kontrola umysłu, złamana ręka itp, ale nie obrażenia) możesz ją zignorować i działać, jakby nic się nie stało przez godzinę. Jeśli kondycja normalnie trwałaby mniej niż godzinę, jest ona kompletnie zanegowana. Akcja.
 
 \myability{Zignorowanie Bólu} - Ignorujesz stan bycia zranionym i traktujesz stan krytycznie ranny jako zraniony. Umożliwienie.
 
 \myability{Dotyk Oświecenia} (1 punkt Intelektu) - Dotykasz obiektu i ten obiekt rzuca światło, oświecając wszystko w średnim zasięgu. Światło zostaje, dopóki poświęcisz akcję i dotkniesz go ponownie, lub dopóki użyjesz tej mocy na większej liczbie obiektów, niż Twój poziom - w tym przypadku obiekt, którego dotknąłeś jako pierwszego, ciemnieje jako pierwszy. Akcja.
 
\myability{Iluzoryczne Przebranie} (2+punkty Intelektu) - Możesz wyglądać jak ktoś lub coś innego, mniej-więcej Twojego rozmiaru i kształtu, przed godzinę. Kiedy już zostanie stworzone, przebranie nie wymaga koncentracji. Na każdy dodatkowy wydany punkt Intelektu, możesz przebrać jedną inną istotę. Wszystkie przebrane istoty muszą zostać w zasięgu Twojego wzroku lub stracić swoje przebranie. Akcja, by stworzyć. 

\myability{Iluzyjny Duplikat} (2 punkty Intelektu) - Tworzysz pojedynczy obraz samego siebie w bliskim zasięgu. Obraz wygląda jak Ty teraz (wliczając Twój strój). Obraz może się poruszać (np: imitując chodzenie lub atak), ale nie może sie przemieścić na więcej niż bliski zasięg od miejsca, w którym powstał. Ta iluzja posiada swój zapach i wydaje dźwięki. Istnieje ona przez 10 minut i zmienia sie zgodnie z Twoim życzeniem (nie trzeba na niej skupiać swojej uwagi). Jeśli poruszysz się na więcej niż średni zasięg od iluzji, zanika ona. Akcja, by stworzyć.

\myability{Iluzoryczny Unik} (5 punktów Intelektu) - Kiedy zostałbyś trafiony przez atak, teleportujesz się na bliski zasięg, pozostawiając za sobą tylko iluzyjną kopię, która zostaje trafiona zamiast Ciebie. Niszczy to iluzję, ale zostawia Cię wolnego od obrażeń. Jeśli atak jest atakiem obszarowym o teleportacja nie usuwa Cię z jego obszaru, atak dalej dotyczy Cię w normalny sposób. Umożliwienie.

\myability{Iluzyjne Ja} (4 punkty Intelektu) - Tworzysz 4 holograficzne duplikaty samego siebie w średnim zasięgu. Te duplikaty istnieją przez 1 minutę. Mentalnie kierujesz ich akcjami - każdy duplikat może robić coś innego. Jeśli zostaną trafione w walce, albo się rozpadają, albą zamierają w bezruchu (Twój wybór). Akcja, by stworzyć. 

\myability{Nieruszony} - Uzyskujesz +3 do swojej Puli Mocy. Możesz spróbować wykonać test Mocy, by uniknąć padnięcia na ziemię, wycofania się lub przemieszczenia wbrew Twojej woli, nawet jeśli efekt poruszający Cię normalnie na to nie zezwala. Jeśli zastosujesz Wysiłek do tego zadania, możesz zastosować dwa darmowe poziomy Wysiłku. Umożliwienie.

\myability{Zasianie Idei} (3 punkty Intelektu) - Po przynajmniej minucie interakcji, z istotą, która Cię słyszy i rozumie, możesz spróbować chwilowo zasiać w jej umyśle ideę, której nie zaakceptowała w normalnych warunkach. Idea ta jest inna od zwykłej sugestii lub polecenia - idea to pewna wartość, taka jak ``Wszelkie życie jest święte'', ``moja partia polityczna jest najlepsza', ``dzieci powinni być widocznie, ale ciche'' itp. Idea wpływa na akcje istoty, ale go nie kontroluje. Idea istnieje tak długo, jak MG uzna za stosowne, zazwyczaj do paru godzin. Idea zostaje zagrożona, jeśli ktoś przyjacielski względem istoty spędzi minutę lub więcej, przekonując ją o tym, by wróciła do swojego starego ja. Akcja.

\myability{Zasianie Zrozumienia} - Twoje zdolność Nauka w Biegu działa bardziej efektywnie, pozwalając Ci na ułatwienie zadania o 2 stopnie lub zapewniając 2 atuty do akcji przyjaciela, zamiast po prostu to ułatwiać. Umożliwienie. 

\myability{Impersonacja} (2 punkty Intelektu) -Przez 1 godzinę, zmieniasz swój głos, posturę i manieryzmy, tworzysz przebranie i zyskujesz atut na próbach udawania kogoś innego, niezależnie od tego, czy jest to specyficzna osoba (Bob policjant) czy ogólna rola (jakiś policjant). Akcja, by rozpocząć. 

\myability{Przyciągnięcie} (2 punkty Intelektu) - Wolno stojący obiekt w średnim zasięgu, który mógłbyś unieść w jednej ręce zostaje przyciągnięty do Twojej dłoni. Jeśli obiekt jest uwiązany lub trzyma go inna istota, musisz odnieść sukces na teście Mocy, by go wyrwać, lub obiekt zostaje tam, gdzie jest. Akcja. 

\myability{Niemożliwy Spacer} (5+ punktów Szybkości) - Możesz chodzić (lub biec/przeczołgać się) po stromych i horyzontalnych powierzchniach (takich jak ścianach i klify) przez następną minutę, jakby była to zwykła ziemia pod stopami. Kirzy korzystasz z tej zdolności ``dół'' to dla Ciebie albo powierzchnia, po której stąpasz, lub normalny kierunek działania grawitacji (Ty wybierasz). Jeśli zastosujesz 1 poziom Wysiłku, możesz także się poruszać po suficie lub płynach, takich jak woda, błoto, ruchome piaski lub nawet lawa (choć dotykanie niebezpiecznej powierzchni jak lawa dalej może Cię zranić). Jeśli zastosujesz dwa poziomy Wysiłku, możesz także stąpać w powietrzu. Umożliwienie.

\myability{Niemożliwy Pokaz} (2 punkty Mocy) - Prezentujesz pokaz siły, szybkości lub walki, zachwycając ludzi wokół Ciebie. Przez następną minutę, otrzymujesz atut na wszystkich zadaniach interakcji z ludźmi, którzy widzieli, jak korzystasz z tej zdolności. Akcja.

\myability{Ulepszona Absorpcja Energii Kinetycznej} - Kiedy korzystasz z Absorpcji Energii Kinetycznej, zamiast absorbować 1 punkt obrażeń z fizycznego ataku lub uderzenia, możesz zaabsorbować 2. Możesz także przechować do 2 punktów energii z dowolnego źródła. Jednakże, dalej możesz wydatkować energię tylko w ilości 1 punktu na raz. Umożliwienie. 

\myability{Ulepszona Aportacja} (6 punktów Intelektu) - Przywołujesz istotę o maksymalnym poziomie 3, która pojawia się obok Ciebie. Możesz wybrać istotę, którą wcześniej spotkałeś, lub (nie więcej niż 1 raz dziennie) pozwolić MG na losowe jej wybranie. Jeśli wzywasz przypadkową istotę, ma ona 10\% szans na bycie istotą o maksymalnym poziomie 5. Ta istota nie ma wspomnień o czymkolwiek przed byciem wezwaną, ale jest w stanie rozmawiać i ma ogólną wiedzę, którą istota jej typu by miała. Ta istota jest w stanie się komunikować i pomaga Tobie (chyba, że chcesz, czy czyniła inaczej). Akcja.

\myability{Ulepszone Rozkazywanie Duchom} - Kiedy korzystasz ze zdolności Rozkazywanie Duchom, możesz rozkazywać duchowi lub animowanemu nieumarłemu o maksymalnym poziomie 7. Umożliwienie.

\myability{Ulepszony Kompan} - Twój kompan (taki jak kontrolowana bestia) lub wierny sługa zwiększa swój poziom do 4. Jako istota 4 poziomu, ma stopień trudności 12, 12 punktów zdrowia i zadaje 4 punkty obrażeń (choć zazwyczaj, zamiast atakować, zapewnia atut do Twoich ataków). Możesz zdobyć tę zdolność raz ma poziom. Za każdym razem, gdy ją zdobywasz, zwiększasz poziom swojego kompana o 1. Umożliwienie.

\myability{Ulepszone Kopiowanie Mocy} - Możesz wykorzystać Kopiowanie Mocy, by skopiować potężniejsze zdolności. W dodatku do zwykłych opcji korzystania z Wysiłku z Kopiowaniem Mocy, jeśli zastosujesz 1 poziom Wysiłku,MG wybiera zdolność średniego poziomu, która najbardziej przypomina tę moc (zamiast zdolności niskiego poziomu). Umożliwienie.

Kiedy korzystasz z Ulepszonego Kopiowania Mocy, skopiowana moc musi być poziomu niskiego, średniego lub wysokiego,  co zaznaczono w kategoriach zdolności. Nie ma znaczenia, czy typ lub specjalizacja czyni ją dostępną na wyższym lub niższym poziomie.

\myability{Ulepszony Osąd} - Kiedy korzystasz z Osądu, możesz osąDzić jedną dodatkową istotę jako niewinną lub winną, co oznacza że dwie istoty za jednym razem mogą być niewinne, winne, lub mieszanka tych dwóch opcji. Umożliwienie. 

\myability{Ulepszone Skupienie} - Wybierz jedno za Skupień swojej postaci, które wynosi 0. Teraz wynosi ona 1. Umożliwienie.

\myability{Ulepszone Szarpnięcie Grawitacyjne} (9 punktów Intelektu) - Możesz zranić grupę celów w dalekim zasięgu, przez nagłe zwiększenie grawitacji w jednym rejonie ich ciał i zmniejszenie w innym, zadając 6 punktów obrażeń. Cele muszą buć w bliskim zasięgu jeden od drugiego. Akcja.

\myability{Ulepszony Kompan Maszyna} - Maszyna z Twojej zdolności Kompan Maszyna się polepsza, stając się istotę 5 poziomu ze zdolność do lotu na daleki dystans w każdej rundzie (może Cię przenieść) przez do 10 minut na raz, *LUB* do przenoszenia ekstra cyphera, który nie wlicza się do Twojego limitu cypherów. Umożliwienie. 

\myability{Ulepszony Sposób na Potwory} - Kiedy atakujesz istote więcej niż dwa razy tak masywną lub dużą jak Ty, zadajesz jej dodatkowe 3 punkty obrażeń. Umożliwienie. 

\myability{Ulepszona Więź z Przedmiotem} (5 punktów Intelektu) - Kiedy manifestujesz sprzymierzeńca ze zdolności Związana Magiczna Istota, jest to teraz istota poziomu 4. Co więcej, istota uzyskuje atak pulsem, który sprawia, że wszystkie artefakty, maszyny, zamanifestowane cyphery i pomniejsze magiczne przedmioty w średnim zasięgu są nie do obsługi przez minutę. Po tym, jak istota skorzysta z tej zdolności, musi wycofać się do obiektu i odpoczywać przez 3 godziny. Umożliwienie.

\myability{Ulepszone Odzyskanie Zdrowia} - Twój test na odzyskanie zdrowia wymagający 10-minutowej akcji trwa teraz tylko 1 akcję, więc masz 2 rzuty na odzyskanie zdrowia trwające akcję, trzeci zajmuje godzinę, a czwarty 10 godzin. Umożliwienie.

\myability{Ulepszone Rzeźbienie Światłem} (7+ punktów Intelektu) - Tworzysz obiekt twardego światła w kształcie, który jesteś sobie w stanie wyobrazić, który mieści się w sześcianie o boku długości 3 m. obiekt pojawia się w obszarze wokół Ciebie lub unosi się swobodnie do dalekiego dystansu od Ciebie, i istnieje on przez kilka dni. Obiekt jest bardzo ogólnikowy i nie może posiadać poruszających się cześci, więc możesz stworzyć fragment ściany, blok, pudełko, schody itp. obiekt ma masę zbliżoną do masy prawdziwego obiektu, który naśladuje, i jest na poziomie 6. Jeśli zastosujesz Wysiłek by zwiększyć rozmiar obiektu, każdy dodatkowy poziom Wysiłku zwiększa rozmiar obiektu o kolejny sześcian o boku długości 3m. Akcja.

\myability{Ulepszony Sensor} (2 punkty Intelektu) - Kiedy korzystasz z Sensora, możesz umieścić go gdziekolwiek w dalekim zasięgu od Ciebie. Umożliwienie.

\myability{Ulepszony Sukces} - Kiedy wyrzucisz 17 lub więcej na teście ataku, który zadaje obrażenia, zadajesz 1 dodatkowy punkt obrażeń. Dla przykładu, gdy wyrzucisz naturalną 18, która normalnie zadaje 2 dodatkowe punkty obrażeń, teraz zadajesz ich 3 punkty. Kiedy wyrzucisz naturalną 20, i wybierzesz zadawanie obrażeń zamiast specjalnego efektu, zadajesz dodatkowe 5 punktów obrażeń. Umożliwienie.

\myability{Improwizacja} (3 punkty Intelektu) - Kiedy wykonujesz zadanie, w którym nie jesteś wyszkolony, możesz zaimprowizować by uzyskać atut na tym zadaniu. Ten atut może wynikać z narzędzia, które tworzysz na boku, nagłego olśnienia, lub przypływu szczęścia. Umożliwienie. 

(Improwizacja może być użyta na zadaniu, na którym postać posiada nieumiejętność, ale zamiast uzyskać atut, postać traci nieumiejętność).

\myability{Obrońca Przyjaciół} (3 punkty Intelektu) - Kiedy stawiasz przyjaciół ponad sobą w swojej akcji, ułatwiasz wszystkie zadania obrony dla wszystkich postaci, które wybierasz, a które stoją obok Ciebie. Trwa to do końca Twojej następnej tury. Jeśli jedna z postaci zostałaby ranna, możesz zaakceptować połowę jej obrażeń ale tylko jeśli nie jesteś sam zraniony lub krytycznie ranny. Umożliwienie.

\myability{Wspaniały Pilot} - Kiedy znajdujesz się na statku, który posiadasz lub który dużo dla Ciebie znaczy, Twoje Skupienie w Mocy, Szybkości i Intelekcie zwiększa się o 1. Kiedy wykonujesz rzut na odzyskanie zdrowia na statku kosmicznym, z którym jesteś zaznajomiony, odzyskujesz dodatkowe 5 punktów w Pulach. Umożliwienie.

\myability{Zwiększony Efekt} - Traktujesz naturalne 19 jak naturalne 20 dla akcji Mocy lub Szybkości (wybierz, gdy zdobywasz tę zdolność). Pozwala Ci to na uzyskanie większego efektu na rzucie 19 lub 20. Umożliwienie. 

\myability{Determinacja} - Gdy osiągniesz klęskę na fizyczny zadaniu niebędącym walką (np: wyważanie drzwi lub wspinanie się na klif) i spróbujesz ponownie, zadanie jest ułatwione. Jeśli znowu odniesiesz klęskę, nie uzyskujesz specjalnych korzyści. Umożliwienie. 

\myability{Niemożliwe Osiągnięcie Naukowe} (12 punktów Intelektu) - Osiągasz coś wspaniałego w laboratorium. Wymaga to części i materiałów równych ceną 3 drogim przedmiotom. Możliwe osiagnięcia naukowe to m.in:
\begin{itemize}
\item Reanimacja i rozkazywanie martwemu ciału przez 1 godzinę
\item Stworzenie silnika-perpetum mobile
\item Stworzenie wrót wymiarowych działających przez 1 minutę
\item Transmutacja jednej substancji w inną
\item Wyleczenie jednej osoby z niemożliwej do wyleczenia choroby
\item Stworzenie broni raniącej coś, czego normalnie nie można zranić
\item Stworzenie obrony przed czymś, przed czym normalnie nie można się bronić
\end{itemize}
Akcja, by rozpocząć; pełen dzień pracy, aby zakończyć.

\myability{Ulepszone Zdrowie} - Dzięki wykąpaniu się w magicznym źródle, zastrzykowi ze sztucznymi antyciałami i immuno-nanobotami, wystawieniu na dziwne promieniowanie, lub czemuś innemu, jesteś teraz odporny na choroby, wirusy i mutacje wszelkiego rodzaju. Umożliwienie, 

\myability{Niemożliwe Odzyskanie Zdrowia} (6 punktów Mocy) - Wchodzisz o poziom wyżej na liczniku obrażeń lub odrzucasz dowolną niechcianą przypadłość. Akcja.

\myability{Niemożliwa Szybkość} - Poruszasz się znacznie szybciej niż normalni ludzie w rundzie. Oznacza to, że jako część innej akcji, możesz poruszyć się na daleki dystans. Jako akcję, możesz przemieścić się o 60 m, lub 150, jeśli zdasz test Szybkości o trudności 4. Umożliwienie. 

\myability{Odczytanie Myśli} (4 punkty Intelektu) - Jeśli wejdziesz w interakcje lub obserwujesz cel przez przynajmniej rundę, możesz postarać się odczytać jego powierzchowne myśli, nwet jeśli cel tego nie chce. Musisz patrzeć na cel. Kiedy uzyskasz świadomość tego, o czym on myśli- poprzez język ciała, mowę, i to, co mówi, a czego nie - możesz kontynuować czytanie myśli do minuty tak długo, jak widzisz i słyszysz cel. Akcja, by przygotować się, akcja by rozpocząć. 

\myability{Piekielny Szlak} (6 punktów Intelektu) - Przez następną minutę, zostawiasz za sobą szlak ognia. Ten szlak ognia podąża za Tobą i trwa do minuty, tworząc ścianę ognia wysokości 2 m, która zadaje 5 punktów obrażeń każdej istocie, która przez niego przechodzi, potencjalnie podpalając go za 1 dodatkowy punkt obrażeń w każdej rundzie (jeśli jest łatwopalny) do momentu, aż spędzi rundę na wygaszaniu ognia. Akcja.

\myability{Infiltrator} - Jesteś wyszkolony w interakcja bazujących na kłamstwach bądź oszustwach. Umożliwienie. 

\myability{Wpływ na Rój} (1 punkt Intelektu) - Rządzisz jednym rodzajem małych istot (takich jak insekty, szczury, nietoperze lub nawet ptaki) i odpowiadają one masowe na Twoje wezwanie. Twoje istoty w średnim zasięgu nie zranią Ciebie lub tych, których wskażesz przez jedną godzinę. Akcja, by rozpocząć. 

\myability{Zbieranie Informacji} (5 punktów Intelektu) - Rozmawiasz telepatycznie z jedną lub wszystkimi maszynami w zasięgu 1.5 hm. Możesz zadać im jedno podstawowe pytanie o nie same lub to, co się wokół nich dzieje i odebrać prosta odpowiedź. Dla przykładu, w obszarze z wielką ilością maszyn, możesz zapytać o lokację specyficznej istoty lub człowieka, i jeśli jest on w zasięgu 1.5 km od Ciebie, jedna lub więcej maszyn najpewniej odpowiedzą na pytanie. Akcja.

\myability{Informator} - Pozyskujesz informatora w sprzymierzonej społeczności. Działa on jako sekretny (lub jawny) informator. Jeśli coś wartego uwagi się wydarzy w lokacji Twojego informatora, wykorzysta on wszelkie możliwe środki, by Ci o tym powiedzieć. Umożliwienie.

\myability{Absorpcja Ducha} - Kidy zabijasz istotę lub niszczysz ducha atakiem, wybierasz, czy duch (jeśli nie jest chroniony) zostaje przez Ciebie zaabsorbowany (i odzyskujesz wtedy 1k6 punktów w jednej Puli swojego wyboru). Ten duch jest przechowany w Tobie, co oznacza, że nie może być przepytany lub wskrzeszony w żadne sposób, chyba, że na to zezwolisz. Umożliwienie. 

\myability{Zamieszkując Kryształ} (4 punkty Intelektu) - Przenosisz swoje ciało i to, co masz przy sobie do kryształu rozmiaru co najmniej Twojego małego palca. Gdy jesteś w tym krysztale, jesteś świadom tego, co się dzieje wokół niego, słysząc i widząc poprzez kryształ. Możesz nawet mówić i odbywać konwersacje. Nie możesz brać jakichkolwiek innych akcji niż opuszczenie kryształu. Pozostajesz w jego wnętrzu tak długo, jak sobie tego życzysz, ale nie jesteś w hibernacji i musisz wyjść, by jeść, pić, spać itp jak zwykle (z wyjątkiem oddychania, która wydarza się we wnętrzu kryształu). Jeśli kryształ zostanie zniszczony lub uzyskuje duże obrażenia, gdy jesteś w jego środku, natychmiast go opuszczasz, nie możesz działać przez 3 rundy, i spadasz o 2 stopnie na liczniku obrażeń. Akcja, by wejść i wyjść. (Postać powinna określić, gdzie umieszcza kryształ, w który chce wejść, nawet jeśli jest to grunt przed jej stopami).

\myability{Wrodzona Moc} - Wybierz abo swoją Pulę Mocy, albo Szybkości. Kiedy wydajesz punkty z niej, by aktywować swoje zdolności specjalizacji, możesz wydać punkt yz tej Puli zamiast punktów z Puli Intelektu (w takim wypadku korzystasz ze swojego Skupienia w Mocy lub Szybkości zamiast Skupienia w Intelekcie). Umożliwienie.

\myability{Wewnętrzna Obrona} - Trudy życia uczyniły Cię odpornym i trudniejszym do odczytania. Jesteś wyszkolony w dowolnym zadaniu polegającym na ukrywaniu swoich prawdziwych uczuć, wierzeń i planów. Jesteś też wyszkolony w odpieraniu tortur, telepatycznych włamań i kontroli umysłu. Umożliwienie.

\myability{Innowator} - Możesz zmodyfikować dowolny artefakt, by dać mu inne lub lepsze zdolności, tak, jakby był o poziom mniejszy niż normalnie, a modyfikacja zajmuje połowę normalnego czasu. Umożliwienie.

\myability{Erupcja Insektów} (6 punktów Intelektu) - Przywołujesz rój insektów tam, gdzie mogłyby normalnie się pojawić. Pozostają przez minutę, i w tym czasie, słuchają Twoich rozkazów gdy zostają w dalekim zasięgu. Mogą one utrudnić akcje dowolnych lub wszystkich istot, lub możesz skupić uwagę roju i zaatakować wszystkie cely w bliskim zasięgu od siebie (wszystkie w dalekim zasięgu od Ciebie). Atakujący rój zadaje 2 punkty obrażeń na rundę. Możesz także kontrolować rój, by przenosił ciężkie obiekty poprzez kolektywny wysiłek, przegryzał się przed drewniane ściany i wykonywał inne akcje odpowiednie dla nadnaturalnego roju. Akcja, by rozpocząć.

\myability{Empatia} - Jesteś wyszkolony w akcjach polegających na rozumieniu motywów innych ludzi i zrozumienia ich ogólnej natury. Posiadasz talent do określania, czy ktoś jest naprawdę niewinny. Umożliwienie.

\myability{Inspiracja} (6 punktów Intelektu) - Wymawiasz słowa zachęty i inspiracji. Wszyscy sprzymierzeńcy w średnim zasięgu, którzy mogą Cię słyszeć, uzyskują natychmiastowy rzut na odzyskanie zdrowia, uzyskują natychmiastową darmową akcję i mają atut na tej darmowej akcji. Rzut na odzyskanie zdrowia nie liczy się jako część normalnych rzutów tego typu. Akcja.

\myability{Zainspirowanie Akcji} (4 punkty Intelektu) - Jeśli jeden sprzymierzeniec widzi Cię i rozumie z łatwością, możesz go poinstruować, by wykonał akcję. Jeśli sprzymierzeniec tak zrobi, wykonuje to jako dodatkową, natychmiastową akcję. Nie przeszkadza to sprzymierzeńcowi w wykonaniu normalnej akcji w swojej turze. Akcja.

\myability{Zainspirowanie Agresji} (2 punkty Intelektu) -Twoje słowa wykrzywiają umysł postaci w średnim zasięgu, która jest w stanie Cię zrozumieć, odblokowując jej bardziej prymitywne instynkty. W wyniku tego, zyskuje ona atut na atakach bazujących na Mocy przez jedną minutę. Akcja, by rozpocząć.

\myability{Zainspirowanie Skoordynowanych Akcji} (9 punktów Intelektu) - Jeśli Twoi sprzymierzeńcy Cię widzą i z łatwością rozumieją, możesz poinstruować każdego z nich, by podjęli jedna specyficzną akcję (taką samą dla nich wszystkich). Jeśli dowolna z nich wykona tę akcję, jest to dodatkowa, natychmiastowa akcja. Nie wpływa to na ich normalną akcję w turze. Akcja.

\myability{Zainspirowanie Niewinnych} (3 punkty Intelektu) - Wymawiasz słowa zachęty i inspiracji do każdego w bliskim zasięgu, kogo oznaczyłeś jako niewinnego swoją mocą Osąd. Uzyskują oni wszyscy darmowy rzut na odzyskanie zdrowia. Jedna osoba, którą wybierasz może uzyskać natychmiastową darmową akcję  zmiast rzutu na odzyskanie zdrowia. Jeśli posiadasz także zdolność Inspiracja, cel, który uzyskuje darmową akcję otrzymuje również na niej atut. Akcja. 

\myability{Zainspirowanie Ułatwienia} - Poprzez historie, pieśni, sztukę lub inną formę rozrywki, inspirujesz swoich przyjaciół. Po spędzeniu z nimi 24 godzin, jeden raz dziennie każdy z Twoich przyjaciół może ułatwić jeden rzut. Ten efekt trwa tak długo, jak jesteś w towarzystwie swoich przyjaciół. Kończy się, jeśli ich opuścisz, ale natychmiast powraca wraz z Tobą (do 24 godzin). Jeśli opuścisz swoich towarzyszy na więcej niż 24 godziny, musisz spędzić z nimi kolejną dobę, by reaktywować tą korzyść. Umożliwienie. 

\myability{Zainspirowanie Sukcesu} (6 punktów Intelektu) - Kiedy odnosisz sukces na teście danej akcji związanej ze Statystyką, którą wybrałeś przy selekcji tej zdolności, i zastosowałeś przynajmniej poziom Wysiłku, możesz wybrać inną postać w średnim zasięgu. Ta postać ma atut na następnym zadaniu, która korzysta z tej Statystyki w swojej następnej turze. Umożliwienie.

\myability{Inteligentny Interfejs} (3 punkty Intelektu) - Możesz rozmawiać telepatycznie z każdą inteligentną maszyną w dalekim zasięgu. Co więcej, jesteś wyszkolony we wszelkich interakcjach społecznych z inteligentnymi maszynami. Takie maszyny i roboty normalnie nigdy nie komunikowałyby się z istotą ludzką, ale dla Ciebie zrobią wyjątek. Umożliwienie.

\myability{Intensywna Interakcja} (3 punkty Intelektu) - Uzyskujesz atut na zastraszaniu, perswazji i wpływania na ludzi przez następne 10 minut. Akcja.

\myability{Umiejętności Międzyludzkie} - Jesteś wyszkolony w dwóch umiejętnościach, w których jeszcze nie posiadasz treningu. Wybierz dwie z następujących: kłamstwo, perswazja, publiczne przemowy, dostrzeganie kłamstw lub zastraszanie. Możesz wybrać tę zdolność wiele razy. Z każdym razem, musisz wybrać dwie nowe umiejętności. Umożliwienie.

\myability{Interfejs} - Poprzez bezpośrednie podłączenie się do urządzenia, możesz je zidentyfikować i nauczyć się nim operować tak, by Twoje testy były o poziom mniejsze. Umożliwienie.

\myability{Przeszkodzenie} ( punkty Intelektu) - Twoja komenda przeszkadza istocie w dokonaniu jakichkolwiek akcji na jedną rundę. Może ona bronić siebie, jeśli została zaatakowana, ale kiedy to czyni, jej obrona jest utrudniona o 2 stopnie. Każdy dodatkowy raz, gdy wykorzystujesz tę zdolność przeciwko tej samej istocie, musisz zastosować jeden poziom Wysiłku więcej niż na poprzedniej próbie. Akcja.  

\myability{Wynalazca} - Możesz tworzyć nowe artefakty w połowie normalnego czasu, jakby były o 2 poziomy niżej, wydając połowę normalnych PD. Umożliwienie.

\myability{Śledztwo} - Jesteś wyszkolony w percepcji, kryptografii, kłamaniu i hakowaniu komputerów. Umożliwienie.

\myability{Umiejętności Śledcze} - Jesteś wyszkolony w dwóch umiejętnościach, w których jeszcze nie masz treningu. Wybierz dwie z poniższych: percepcja, identyfikowanie, otwieranie zamków, ocena zagrożenia lub majstrowanie przy urządzeniach. Możesz wybrać tę zdolność wiele razy. Za każdym razem, musisz wybrać dwie nowe umiejętności. Umożliwienie. 

\myability{Śledczy} - By naprawdę lśnić jako śledczy, musisz zaangażować swoje ciało i umysł w swoich wnioskach. Możesz wydawać punkty ze swoich Pul Mocy, Szybkości i Intelektu by zastosować poziomy Wysiłku do dowolnego zadania bazującego na Inteligencji. Umożliwienie. 

\myability{Niewidzialność} (4 punkty Intelektu) - Stajesz się niewidzialny na 10 minut. Kiedy jesteś niewidzialny, jesteś wyspecjalizowany w skradaniu się i obronie Szybkości. Ten efekt kończy się, gdy zrobić coś, by ujawnić swoją obecność lub pozycję - jak, użycie zdolności, przesunięcie ciężkiego obiektu itp. Jeśli masz inną zdolność, która także zapewnia niewidzialność, skorzystanie z dowolnej z nich zapewnia niewidzialność przez podwojoną ilość czasu. Akcja by rozpocząć lub wykonać ponownie.

\myability{Niewidzialne Fazowanie} (4 punkty Mocy) - Stajesz się niewidzialny kiedy korzystasz z Biegu Fazowego i w następnej rundzie. Kiedy niewidzialny, skradanie się jest ułatwione o 2 stopnie, a obrana Szybkości także o 2 (to zastępuje atuty do obrony Szybkości zapewniane przez Bieg Fazowy). Pierwszy atak, który wykonujesz korzystając z zdolności specjalności Rozdziera Ściany Świata jest ułatwiony o dwa stopnie; jednakże, jeśli atakujesz istotę, Niewidzialne Fazowanie kończy się natychmiast zamiast trwać przez jedną dodatkową rundę. Jeśli posiadasz zdolność niewidzialność, możesz zostać niewidzialny przez całą rundę, co oznacza, że jeśli korzystasz z Fazowego Zadrapania lub Potężniejszego Fazowania, atakowanie każdego celuna Twojej drodze jest ułatwione o dwa stopnie. Umożliwienie. 

\myability{Żelazne Pięści} - Twoje ataki bez broni zadają 4 punkty obrażeń. Umożliwienie. 

\myability{Stalowy Cios} (5+ punktów Intelektu) - Magnetycznie podnosisz ciężki, metalowy obiekt w średnim zasięgu i rzucasz go na kogoś w średnim zasięgu. Na każdy dodatkowy zastosowany poziom Wysiłku, możesz podnieść trochę cięższy obiekt, co pozwala Ci zaatakować dodatkowy cel w średnim zasięgu, tak długo, jak stoi obok pierwszego celu. Akcja. 

\section{J}

\myability{Dalsza Teleportacja} (5+ punktów Intelektu) - Natychmiast teleportujesz się do dowolnej lokacji w dalekim zasięgu, którą widzisz. W dodatku do normalnych opcji korzystania z Wysiłku, możesz wybrać Wysiłek, by zwiększyć zasięg, na który się teleportujesz- każdy poziom zwiększa zasięg teleportacji o 30 m. Akcja.

\myability{Juggernaunt} (5 punktów Mocy) - Do końca następnej mocy, możesz się przemieszczać przez ciała stałe takie jak drzwi i ściany. Tylko 60 cm drewna, 30cm kamienia lub 15 cm metalu może zatrzymać Twój ruch. Umożliwienie.

\myability{Atak z Wyskoku} (5+ punktów Mocy) - Próbujesz wykonać skok o trudności poziomu 4 (Moc) by wyskoczyć wysoko w powietrze jako część Twojego ataku wręcz. Jeśli uda Ci się ten skok i Twój atak trafi, zadajesz dodatkowe 3 punkty obrażeń i przewracasz przeciwnika. Jeśli nie uda Ci się skok, ale uda Ci się cios, to wtedy, choć trafiasz, nie zadajesz dodatkowych obrażeń lub przewracasz przeciwnika. W dodatku do zwykłych opcji korzystania z Wysiłku, możesz wybrać ulepszenie Twojego skoku - każdy poziom Wysiłku dodaje +60 cm do wysokości Twojego skoku i +1 do obrażeń ataku. Akcja.

\myability{Rzemieślnik Rupieci} (2 punkty Intelektu) - Jesteś wyszkolony w tworzeniu dwóch kategorii przedmiotów korzystając z zebranych rupieci. Jeśli zebrałeś przynajmniej dwa kawałki rupieci w odmiennych kategoriach (elektroniczne, plastikowe, niebezpieczne, metalowe, szklane lub tekstylia), posiadasz materiały, których potrzebujesz, by stworzyć nowy przedmiot w jednej ze swoich kategorii treningu (chyba, że MG określi inaczej). Umożliwienie.

\myability{Naprawa na Oko} (5 punktów Intelektu) - Szybko tworzysz obiekt z tego, co wygląda na pierwszy rzut oka na kompletnie niewłaściwe materiały. Możesz stworzyć bombę z metalowej puszki i środków czystości, wtrych z folii aluminiowej, lub miecz ze zniszczonego mebla. Poziom przedmiotu określa trudność zadania, ale stosowność materiału obniża go lub utrudnia. Ogólnie rzecz ujmując, obiekt nie może być większy niż coś, co trzymasz w jednej dłoni, i funkcjonuje tylko jeden raz (w przypadku broni i podobnych przedmiotów, tylko przez jedno spotkanie). Jeśli spędzisz przynajmniej 10 minut na zadaniu, możesz stworzyć przedmiot na poziomie 5 lub niższym. Nie możesz zmienić natury materiałów z których korzystasz. Przykładow, nie możesz wziąć żelaznych prętów i zamienić je w złote sztabki lub wiklinowy koszyk. Akcja.

\myability{Troszkę Szalony} - Jesteś wyszkolony w zadaniach obrony Intelektu. Umożliwienie.

\section{K}

\myability{Pozbawienie Przytomności} (5+ punktów Intelektu) - Wykonujesz atak wręcz, który nie zadaje obrażeń. Zamiast tego, jeśli atak trafi, wykonaj drugi rzut bazujący na Mocy. Jeśli jest on udany, przeciwnik poziomu 3 lub niższego jest pozbawiony przytomności na jedną minutę. Na każdy poziom Wysiłku, możesz zwiększyć maksymalny poziom wroga o 1, lub możesz zwiększyć długość trwania efektu przez kolejną minutę. Akcja.

\myability{Poznanie Słabości} - Jeśli istota, którą widzisz, ma jakąś specjalną słabość, taką jak wrażliwość na głośne dźwięki, negatywny modyfikator do percepcji itp, wiesz o tym. Zapytaj i MG Ci powie - zazwyczaj, nie wymaga to rzutu, ale w pewnych przypadkach MG może zadecydować, że jest szansa, że nie wiesz. W takim wypadku, jesteś wyspecjalizowany w wiedzy o słabościach istoty. Umożliwienie.

\myability{Wiesz, Gdzie Szukać} - Za każdym razem, gdy MG rzuca w tabeli Użytecznych Rzeczy, otrzymujesz dwa wyniki zamiast 1. Jeśli MG korzysta z jakiejś innej metody określania nagród/znajdywania cennych rzeczy, uzyskujesz podwójny wynik niż to, co normalnie byś otrzymał. Umożliwienie.

\myability{Wiedza} - Jesteś wyszkolony w jednej dziedzinie wiedzy własnego wyboru. Umożliwienie.

\myability{Wiedza o Nieznanym} (6 punktów Intelektu) - Poprzez dostęp do odpowiednich źródeł wiedzy, możesz zadać MG jedno pytanie i uzyskać ogólnikową odpowiedź. MG określa poziom pytania, tak, że im bardziej tajemnicze pytanie, tym trudniejsze zadanie. Ogólnie rzecz ujmując, wiedza, którą można znaleźć gdzieś indziej niż Twoja obecna lokalizacja jest na poziomie 1, a wielkie tajemnice przeszłości to poziom 7. Pozyskanie wiedzy o przyszłości nie jest możliwe. Akcja.

\myability{Wiedza Prawnicza} - Jesteś wyszkolony w prawie danego kraju. Jeśli nie znasz odpowiedzi na pytanie prawne, wiesz, gdzie i jak to przebadać (biblioteka uniwersytecka to dobre miejsce, by zacząć, ale masz także Internet). Umożliwienie.

\myability{Wiedza to Potęga} - Wybierz dwie umiejętności niezwiązane z walką, w których nie jesteś wyszkolony. Jesteś teraz w nich wyszkolony. Umożliwienie.

\myability{Umiejętności Wiedzy} - Jesteś wyszkolony w dwóch umiejętnościach, których jeszcze nie posiadasz. Wybierz dwie dziedziny wiedzy takie jak historia, geografia, archeologia itp. Możesz wybrać tę zdolność wiele razy. Za każdym razem, musisz wybrać nowe umiejętności. Umożliwienie.

\section{L}

\myability{Analiza Laboratoryjna} (3 punkty Intelektu) - Analizujesz miejsce zbrodni, tajemniczego wypadku lub serię niewyjaśnionych zjawisk, i może zdobywasz zaskakująco dużo wiadomości o odpowiedzialnych, ludziach, którzy w tym uczestniczyli, lub siłach, które za tym stoją. Aby to uczynić, musisz pobrać próbki z danego miejsca. Próbki to odpryski farby lub drewna, brud, fotografie miejsca, włosy, całe zwłoki itp. Z próbkami w posiadaniu, możesz odkryć do 3 ważnych informacji o wydarzeniach, być może rozwiązując pomniejszą tajemnicę i kierując się w stronie większej. GM decyduje, jakie informacje zdobywasz i jaki poziom trudności ma zadanie zdobycia ich. (Przykładowo, odkrycie, że ofiara umarła nie wskutek upadku, a porażenie prądem, to zadanie 3 poziomu). Rzut jest ułatwiony, jeśli przeniesiesz próbki do stałego laboratorium (jeśli masz do niego dostęp), w przeciwieństwie do zrobienia analizy przy pomocy przenośnego zestawu naukowca. Akcja by rozpocząć, 2d20 minut, by zakończyć. 

\myability{Spóźniona Inspiracja} (3 punkty Intelektu) - Ponawiasz zadanie, które Ci się nie powiodło w ciągu ostatniej minuty, korzystając z tej samej trudności i modyfikatorów, ale tym razem uzyskujesz atut na teście. Jeśli ta ponowna próba zawiedzie, nie możesz ponownie jej podjąć przy wykorzystaniu tej zdolności. Umożliwienie.

\myability{Lider Spostrzegawczości}  - Utrzymujesz uwagę swoich sprzymierzeńców, przy pomocy okazjonalnych pytań, żartów itp. Po spędzeniu z Tobą 24 godzin, Twoi sprzymierzeńcy są traktowani, jakby byli wyszkoleni w zadaniach związanych z percepcją. Ten efekt trwa tak długo, jak jesteś w towarzystwie swoich sprzymierzeńców. Kończy się ,jeśli ich opuścisz, ale ponawia się, jeśli do nich wrócisz w ciągu 24 godzin. Jeśli opuścisz swoich sprzymierzeńców na więcej niż 24 godziny, muszą spędzić z Tobą kolejna 24 godziny, by odzyskali tą korzyść. Umożliwienie.

\myability{Na linii Frontu} - Uzyskujesz 3 punkty do rozdysponowania w swoich Pulach jak uznasz za stosowne. Umożliwienie.

\myability{Parę Sztuczek W Zanadrzu} - Jesteś wyszkolony w dwóch dziedzinach wiedzy swojego wyboru, lub wyspecjalizowany w jednej. Umożliwienie.

\myability{Nauka w Biegu} (2 punkty Intelektu) - Obserwujesz lub badasz istotę, obiekt lub lokację przez przynajmniej jedną rundę. Następnym razem, gdy wejdziesz w interakcję z nią (możliwe, że w następnej rundzie), powiązanie zadanie (takie jak przekonywanie istoty, atakowanie jej lub obrona) jest ułatwione. Akcja.

\myability{Prawnik-Stażysta} - Uzyskujesz kompana 4 poziomu, które głownie zajmuje się pomaganiem Ci w zadaniach powiązanych z prawem, ale który może także być pomocny w innych sytuacjach. Umożliwienie.

\myability{Zwinne Dłonie} (1 punkt Szybkości) - Możesz wykonać małe, lecz najwyraźniej niemożliwe triki. Dla przykładu, możesz sprawić, by mały obiekt w Twojej dłoni zniknął i pojawił się w miejscu w Twoim zasięgu (np: w Twojej kieszeni). Możesz sprawić, że ktoś uwierzy, że ma coś,czego nie ma (i vice versa). Możesz podmienić podobne obiekty przed czyimiś oczami. Akcja.

\myability{Rozszerzony Zwierzęcy Kształt} (6+ punktów Intelektu) - Możesz się zmienić w zwierzę, i jedna chętna istota w bliskim zasięgu także zmienia się w zwierzę tego typu (niedźwiedzia, tygrysa, wilka itp.) na 10 minut, tak jakby korzystała ze zdolności Zwierzęcy Kształt. Na każdy zastosowany poziom Wysiłku, możesz przemienić jedną dodatkową istotę. Wszystkie przemienione istoty muszą być Twojego rozmiaru lub mniejsze. Istota może owrócić do zwykłego kształtu poświęcając akcję, ale nie może potem wrócić do zwierzęcego kształtu. Zmiana kształtu jednej istoty (Ty lub kogoś innego) nie wpływa na inne istoty, których dotyczy ta zdolność. Akcja.

Istota, która przyjęła zwierzęcy kształt przy wykorzystaniu tej zdolności liczy się jako zwierzę na rzecz skorzystania ze zdolności Zwierzęce Szpiegowanie.

Postać może przyjąć kształt istoty podobnej do zwykłego zwierzęcia, takiego jak jednorożec zamiast konia lub bazyliszek zamiast jaszczurki, ale wymaga to zastosowania przynajmniej jednego poziomu Wysiłku, i postać nie otrzymuje żadnych magicznych zdolności takiej istoty.

\myability{Śmiertelne Obrażenia} - Wybierz jeden ze swoich ataków, który zadaje obrażenia (w zależności od Twojego typu i specjalizacji, może to być specyficzny atak, taki jak wybuch ognia, lub nieuzbrojone ataki). Kiedy trafiasz takim atakiem, zadajesz dodatkowe 5 punktów obrażeń. Umożliwienie.

\myability{Śmiertelny Trik} (5+ punktów Intelektu) - Długie doświadczenie nauczyło Cię, że podstępność to Twój przyjaciel w desperackich sytuacjach. Pchasz, atakujesz, lub rozpraszasz cel w pewien najwyraźniej niekonsekwentny sposób, który prowadzi do śmierci celu. Cel musi być na poziomie 2 lub niższym. W dodatku do normalnych opcji korzystania z Wysiłku, możesz z niego skorzystać, by zwiększyć maksymalny poziom celu o 1. Tak więc, aby zabić cel 5 poziomu (3 poziomy powyżej normalnego limitu) musisz zastosować 3 poziomy Wysiłku. Akcja.

\myability{Śmiertelna Wibracja} (7 punktów Mocy) - Ustanawiasz w swoim ciele śmiertelną wibrację i przekazujesz ją istocie dotykiem poprzez udany atak. Jesli cel jest poziomu 2 lub mniej, umiera, eksplodując z wielkim hukiem. Jesli cel jest poziomu 3 lub wyższego, otrzymuje on 6 punktów obrażeń i jest wstrząśnięty przez swoją następną akcję. Jesli cel to BG jakiegokolwiek poziomu, spada on o 1 poziom na liczniku obrażeń W dodatku do normalnych opcji korzystania z Wysiłku, możesz zastosować Wysiłek, by wpłynąć na potężniejszy cel (jeden poziom Wysiłku ozncvza, że cel o maksymalnym poziomie 3 eksploduje, a cel o poziomie 4 i wyższym otrzymuje obrażenia i jest wstrząśnięty itp.) Akcja.

\myability{Beztroska} - Poprzez spryt, urok, humor i grację, jesteś wyszkolony we wszystkich społecznych interakcjach innych niż te związane z zastraszaniem i przymuszaniem. Podczas odpoczynku, wpływasz tak dobrze na swoich przyjaciół i sprzymierzeńców, że uzyskują oni +1 do swoich rzutów na odzyskanie zdrowia. Umożliwienie.

\myability{Lekcje Życiowe} - Wybierz dwie umiejętności niezwiązane z walką. Jesteś wyszkolony w tych umiejętnościach. Umożliwienie.

\myability{Znam Ten Statek Jak Własną Dłoń} - Wszystkie testy bezpośrednio związane ze statkiem, który posiadasz lub z którym jesteś związany, są ułatwione. Te zadania obejmują naprawy, tankowanie, znajdowanie luk w poszyciu, znajdowanie pasażerów na gapę itp. Liczy się to także do ataków lub obrony wykonywanych przy pomocy statku przeciwko wrogim pasażerom, jak i ataków i obrony przeciwko statkom wrogów. Umożliwienie.

\myability{Link Sensoryczny} (2 punkty intelektu) - Dotykasz chętnej istoty i podłączasz jej zmysły do Twoich przez 1 minutę. W dowolnym momencie podczas trwania tej zdolności, możesz się skoncentrować by widzieć, słyszeć i wąchać to, czego doświadcza istota, zamiast korzystać z własnych zmysłów. Jeśli Ty lub istota opuścicie daleki zasięg, połączenie jest zerwane.  Akcja, by rozpocząć.

\myability{Żywa Zbroja} - Jeśli jesteś w lokacji, w której jest możliwe, by istoty z Twojej zdolności Wpływ na Rój przybyły, przywołujesz rój wokół siebie na 1 godzinę. Przesuwa się on po Twoim ciele lub fruwa obok CIebie w formie chmury. W czasie trwania tej zdolności, Twoje zadania Obrony Szybkości są ułatwione i uzyskujesz +1 do Pancerza. Akcja, by rozpocząć. 

