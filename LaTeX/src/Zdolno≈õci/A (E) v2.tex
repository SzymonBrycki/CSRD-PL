% hyperref !!! https://tex.stackexchange.com/questions/180571/making-clickable-links-to-sections-with-hyperref

\chapter{Zdolności w kolejności alfabetycznej}

\section{A}

\textbf{Uśmiech i Słowo}\index{Zdolności!Alfabetycznie!Uśmiech i Słowo}\label{sec:Uśmiech i Słowo} - kiedy korzystasz z Wysiłku do dowolnej akcji interakcji społecznej - nawet takiej która polega na uspokajaniu zwierząt lub komunikowania się z kimś, czyim językiem nie mówisz - uzyskujesz darmowy poziom Wysiłku na tym zadaniu. Akcja.

\textbf{Przydatna Pomoc}\index{Zdolności!Alfabetycznie!Przydatna Pomoc}\label{sec:Przydatna Pomoc} - kiedy pomagasz komuś z zadaniem i stosuje on poziom wysiłku, zyskuje on darmowy poziom Wysiłku na tym zadaniu. Umożliwienie. 

\textbf{Absorpcja Energii}\index{Zdolności!Alfabetycznie!Absorpcja Energii}\label{sec:Absorpcja Energii} (7 punktów Intelektu) - dotykasz obiektu i absorbujesz jego energię. Jeśli dotykasz zamanifestowanego Cyphera, czynisz go bezużytecznym. Jeśli dotykasz artefaktu, rzuć na jego wyczerpanie. Jeśli dotykasz innego rodzaju zasilanego urządzenia lub maszyny, GM określa, czy jego moc jest w pełni wyssana. W każdym razie, absorbujesz energię z obiektu i odzyskujesz 1k10 punktów Intelektu. Jeśli to dałoby Ci więcej punktów Intelektu niż maksimum Twojej Puli, dodatkowe punkty są utracone, i musisz wykonać rzut na Obronę Mocy. Trudność tego rzutu to numer punktów powyżej Twojego maksimum, które zaabsorbowałeś. Jeśli oblejesz ten rzut, otrzymujesz 5 punktów obrażeń i nie możesz podejmować działań przez jedną rundę. Możesz wykorzystać tę zdolność jako akcję obronną kiedy jesteś celem ataku zdolnością. Taka akcja niweluje atak zdolnością, a ty absorbujesz energię, jakby pochodziła z urządzenia. Akcja.

\textbf{Absorpcja Energii Kinetycznej}\index{Zdolności!Alfabetycznie!Absorpcja Energii Kinetycznej}\label{sec:Absorpcja Energii Kinetycznej} - absorbujesz porcję energii ataku fizycznego lub uderzenia. Negujesz 1 punkt obrażeń, które normalnie byś poniósł i przechowujesz tę energię. Po tym, jak zaabsorbujesz 1 punkt energii, kontynuujesz obniżać obrażenia o 1 punkt z nadchodzących ataków, ale pozostała energia wycieka z Ciebie w formie błysku nieszkodliwego światła (nie możesz przechowywać na raz więcej niż 1 punktu energii w tym samym czasie). Umożliwienie. 

\textbf{Absorpcja Czystej Energii}\index{Zdolności!Alfabetycznie!Absorpcja Czystej Energii}\label{sec:Absorpcja Czystej Energii} - kiedy korzystasz z Absorpcji Energii Kinetycznej, możesz także absorbować i przechowywać energię ataków bazujących na czystej energii (światło, promieniowanie, energie międzywymiarowe, psioniczne itp.) lub z przekaźników owej energii, gdy masz z nimi bezpośredni kontakt. Ta zdolność nie zmienia tego, ile punktów energii możesz przechowywać. Jeśli masz również Ulepszoną Absorpcję Energii Kinetycznej, możesz również absorbować do 2 punktów obrażeń ze źródeł czystej energii. Umożliwienie. 

\textbf{Akceleracja}\index{Zdolności!Alfabetycznie!Akceleracja}\label{sec:Akceleracja} (4+ punkty Intelektu) - Twoje słowa umacniają ducha postaci w bliskim zasięgu, która jest w stanie zrozumieć Cię, przyspieszając ją, tak, że zyskuje ona atut na testach inicjatywy i rzutach na Obronę Szybkości przez 10 minut. Dodatkowo, poza zwykłymi opcjami korzystania z Wysiłku, możesz z niego skorzystać, by objąć celem tej zdolności więcej postaci - każdy poziom Wysiłku obejmuje dodatkowy cel. Musisz przemówić do dodatkowych celów, by je przyspieszyć, jeden cel na rundę. Jednak akcja na jeden cel by rozpocząć. 

\textbf{Akrobatyczny Atak}\index{Zdolności!Alfabetycznie!Akrobatyczny Atak}\label{sec:Akrobatyczny Atak} (1+ punktów Szybkości) - wyskakujesz w ataku, przesuwając się przez powietrze. Jeśli wyrzucasz naturalne 17 lub 18, możesz wybrać mniejszy efekt zamiast dodatkowych obrażeń. Jeśli zastosujesz Wysiłek do tego ataku, uzyskujesz darmowy poziom Wysiłku na zadaniu. Nie możesz skorzystać z tej zdolności, jeśli Twój Wysiłek Szybkości jest zredukowany wskutek noszenia zbroi. Umożliwienie. 

\textbf{Procesor Akcji}\index{Zdolności!Alfabetycznie!Procesor Akcji}\label{sec:Procesor Akcji} (4 punkty Intelektu) - korzystając z przechowywanych informacji i zdolności analizowania nadchodzących danych z wielką szybkością, jesteś wyszkolony w jednym fizycznym zadaniu Twojego wyboru na 10 minut. Dla przykładu, możesz wybrać bieg, wspinaczkę, pływanie, Obronę Szybkości lub atak specyficzną bronią. Akcja by rozpocząć.

\textbf{Adaptacja}\index{Zdolności!Alfabetycznie!Adaptacja}\label{sec:Adaptacja} - dzięki ukrytej mutacji, urządzeniu wbudowanemu w Twój kręgosłup, rytuałowi krwii smoka, lub jakiemuś innemu darowi, jesteś teraz w komfortowej temperaturze; nie musisz sie nigdy martwić o niebezpieczne promieniowanie, choroby lub gazy; i możesz zawsze oddychać w dowolnym środowisku (nawet w próżni kosmosu). Umożliwienie.

\textbf{Zaawansowany Użytkownik Cypherów}\index{Zdolności!Alfabetycznie!Zaawansowany Użytkownik Cypherów}\label{sec:Zaawansowany Użytkownik Cypherów} - możesz mieć przy sobie 4 Cyphery w danym czasie. Umożliwienie. 

\textbf{Zaawansowany Rozkaz}\index{Zdolności!Alfabetycznie!Zaawansowany Rozkaz}\label{sec:Zaawansowany Rozkaz} (7 punktów Intelektu) - cel w średnim zasięgu słucha każdej komendy, którą mu wydasz, tak długo, jak słyszy Cię i rozumie. Co więcej, tak długo, jak nie robisz nic innego niż wydawanie komend (nie wolno Ci wziąć żadnej innej akcji) możesz dać temu samemu celowi nową komendę. Ten efekt kończy się, gdy kończysz wydawać komendy lub gdy cel opuszcza średni zasięg względem Ciebie. Akcja by rozpocząć. 

\textbf{Atak z Rozbrojeniem}\index{Zdolności!Alfabetycznie!Atak z Rozbrojeniem}\label{sec:Atak z Rozbrojeniem} (3 punkty Szybkości) - za pomocą serii szybkich ruchów, wykonujesz atak przeciwko uzbrojonemu przeciwnikowi, zadając mu obrażenia i rozbrajając go, tak, że jego broń jest teraz w Twoich rękach lub 3 metry od niego na ziemi - Ty wybierasz. Ten atak rozbrajający jest utrudniony. Akcja.

\textbf{Zalety Bycia Dużym}\index{Zdolności!Alfabetycznie!Zalety Bycia Dużym}\label{sec:Zalety Bycia Dużym} - kiedy korzystasz ze Wzrostu, jesteś tak duży, że możesz łatwiej przenosić duże obiekty, wspinać sie na budynki korzystając z uchwytów niedostępnych dla zwykłych ludzi i skakać znacznie dalej. Kiedy korzystasz ze Wzrostu, wszystkie zadania wspinaczki, podnoszenia ciężarów i skakania są dla Ciebie ułatwione. Umożliwienie.

\textbf{Zalety Bycia Małym}\index{Zdolności!Alfabetycznie!Zalety Bycia Małym}\label{sec:Zalety Bycia Małym} - nauczyłeś się, jak wykorzystać swój rozmiar, siłę i dokładność. Twoje obrażenia już się nie dzielą na pół gdy korzystasz ze Zmniejszenia się, a zadania wspinaczki i skakania są ułatwione. Umożliwienie.

\textbf{Porada od Przyjaciela}\index{Zdolności!Alfabetycznie!Porada od Przyjaciela}\label{sec:Porada od Przyjaciela} (1 punkt Intelektu) - znasz słabe i mocne strony swojego przyjaciela, i wiesz jak go zmotywować, by osiągnął sukces. Kiedy dajesz przyjacielowi sugestię powiązaną z jego następną akcję, postać ta jest wyszkolona w tej akcji na jedną rundę. Akcja. 

\textbf{Znowu i Znowu}\index{Zdolności!Alfabetycznie!Znowu i Znowu}\label{sec:Znowu i Znowu} (8 punktów Szybkości) - możesz wziąć kolejną akcję w rundzie, w której już podjąłeś akcję. Umożliwienie.

\textbf{Nieśmiertelny}\index{Zdolności!Alfabetycznie!Nieśmiertelny}\label{sec:Nieśmiertelny} - Twoje ciało i umysł się nie starzeją. Jeśli nie zostaniesz zabity przez akt przemocy (lub jakąś zewnętrzną siłę jak trucizna lub infekcja), nigdy nie umrzesz. Umożliwienie.  

\textbf{Agent-Prowokator}\index{Zdolności!Alfabetycznie!Agent-Prowokator}\label{sec:Agent-Prowokator} - wybierz jedna z poniższych, by być wytrenowanym w: atakowanie bronią swojego wyboru, ładunki wybuchowe, lub skradanie się i otwieranie zamków (jeśli wybierzesz ostatnią opcję, posiadasz trening w dwóch umiejętnościach). Umożliwienie.

\textbf{Agresja}\index{Zdolności!Alfabetycznie!Agresja}\label{sec:Agresja} (2 punkty Mocy) - skupiasz się na atakowaniu w tak wielki sposób, że zostawiasz siebie wysuniętego na ataki wrogów. Kiedy ta zdolność jest aktywna, zyskujesz atut na atakach wręcz i Twoje rzuty na Obronę Szybkości przeciwko atakom wręcz i dystansowym są utrudnione. Ten efekt trwa tak długo, jak sobie życzysz ale kończy się, jeśli walka nie ma miejsca w zasięgu Twoich zmysłów. Umożliwienie.

\textbf{Szybki Umysł}\index{Zdolności!Alfabetycznie!Szybki Umysł}\label{sec:Szybki Umysł} - kiedy próbujesz wykonać zadanie Szybkości, możesz zamiast tego rzucić (i wydać punkty z puli) jakby to była akcja Intelektu. Jeśli stosujesz Wysiłek do tego zadania, możesz wydać punkty z Puli Intelektu zamiast Puli Szybkości (wtedy stosujesz też Skupienie w Intelekcie zamiast w Szybkości). Umożliwienie. 

\textbf{Wysokie Skupienie}\index{Zdolności!Alfabetycznie!Wysokie Skupienie}\label{sec:Wysokie Skupienie} (7 punktów Intelektu) - wkładasz w swoje zadanie wszystko. Dodajesz trzy darmowe poziomy Wysiłku to następnego zadania, które podejmujesz. Nie możesz wykorzystać tej zdolności znowu, dopóki nie zakończysz 10-godzinnego odpoczynku. Akcja.

\textbf{Uzdrowienie}\index{Zdolności!Alfabetycznie!Uzdrowienie}\label{sec:Uzdrowienie} (3 punkty Intelektu) - możesz spróbować uzdrowić jedno schorzenie (np: chorobę lub truciznę) dotyczące jednej istoty. Akcja.

\textbf{Szczur Miejski}\index{Zdolności!Alfabetycznie!Szczur Miejski}\label{sec:Szczur Miejski} (6 punktów Intelektu) - kiedy jesteś w mieście, odnajdujesz lub tworzysz znaczące skróty, sekretne wejścia lub ostateczne trasy ucieczki tam, gdzie wcześniej ich nie było. Aby to zrobićm musisz uzyskać sukces na kacji Intelektu, której trudność określa MG bazując na danej sytuacji. Powinieneś ustalić detale wraz ze swoim MG. Akcja.

\textbf{Zawsze Majsterkując}\index{Zdolności!Alfabetycznie!Zawsze Majsterkując}\label{sec:Zawsze Majsterkując} - jeśli masz narzędzia i materiały i nosisz mniej cypherów niż Twój limit, możesz stworzyć zamanifestowany cypher, jeśli poświęcisz na to godzinę. Nowy cypher jest wybierany przypadkowo i zawsze o 2 poziomy mniej niż normalnie (minimum to 1-szy poziom). Jest on także chwilowy i wrażliwy na uszkodzenia. Nazywa się go chwilowym cypherem. Jeśli dasz go komuś, by z niego korzystał, rozpada się on natychmiast w bezużyteczne śmieci. Akcja by rozpocząć; 1 godzina by ukończyć.

\textbf{Cudowne Kopiowanie}\index{Zdolności!Alfabetycznie!Cudowne Kopiowanie}\label{sec:Cudowne Kopiowanie} - możesz skorzystać ze zdolności Skopiuj Moc, aby skopiować potężniejsze zdolności. W dodatku do normalnych opcji korzystania z Wysiłku przy użyciu Skopiuj Moc, jeśli zaaplikujesz 2 poziomy Wysiłku, MG wybiera moc wysokiego poziomu, która najbardziej przypomina moc, którą pragniesz skopiować (zamiast zdolności niskiego poziomu). Umożliwienie.

\textbf{Dodatkowy Wysiłek}\index{Zdolności!Alfabetycznie!Dodatkowy Wysiłek}\label{sec:Dodatkowy Wysiłek} - kiedy stosujesz przynajmniej jeden poziom Wysiłku do akcji niebojowej, otrzymujesz darmowy, dodatkowy poziom Wysiłku na tym zadaniu. Kiedy wybierasz tę zdolność, musisz zdecydować, czy dotyczy ona Wysiłku Mocy, czy też Wysiłku Szybkości. Umożliwienie.

\textbf{Wielki Skok}\index{Zdolności!Alfabetycznie!Wielki Skok}\label{sec:Wielki Skok} (2 punkty Mocy) - skaczesz w powietrze i lądujesz bezpiecznie w pewnej odległości. Możesz skoczyć wzwyż, w dół lub w poziomie gdziekolwiek w dalekim zasięgu  jeśli masz czystą trasę do tego miejsce, bez żadnych przeszkód. Jeśli masz 3 lub więcej punktów mocy zainwestowanych w siłę, Twój zasięg się ulepsza do bardzo dalekiego. Jeśli masz 5 lub więcej punktów mocy zainwestowanych w siłę, Twój zasięg skoku zostaje ulepszony do 300 metrów. Akcja.

\textbf{Czatownik}\index{Zdolności!Alfabetycznie!Czatownik}\label{sec:Czatownik} - kiedy atakujesz istotę, która jeszcze nie wzięła swojej pierwszej rundy w walce, Twój atak jest ułatwiony. Umożliwienie.

\textbf{Wzmocnienie Dźwięku}\index{Zdolności!Alfabetycznie!Wzmocnienie Dźwięku}\label{sec:Wzmocnienie Dźwięku} (2 punkty Mocy) - na jedną minutę, możesz wzmocnić dalekie lub ciche dźwięki, tak, byś mógł je słyszeć wyraźnie, nawet jeśli jest to rozmowa lub dźwięk małego zwierzęcia poruszającego się w podziemnej norze w bardzo dalekim zasięgu. Możesz spróbować usłyszeć dźwięk, nawet jeśli istnieją bariery blokujące dźwięk lub jest on bardzo cichy, choć to wymaga paru dodatkowych rund koncentracji. Aby odróżnić dźwięk, którego poszukujesz, od głośnego środowiska, także powinieneś poświęcić parę rund na skupienie, gdy przeszukujesz słuchem swoją okolicę. Mając odpowiednio dużo czasu, możesz wyśledzić każdą konwersację, oddychającą istotę i każde urządzenie wydające dźwięk w zasięgu. Akcja by rozpocząć, do paru rund by ją zakończyć, w zależności od trudności zadania.

\textbf{Anegdota}\index{Zdolności!Alfabetycznie!Anegdota}\label{sec:Anegdota} (2 punkty Intelektu) - możesz polepszyć morale grupy istot i pomóc im w nawiązaniu więzi, poprzez zabawianie ich podnoszącą na duchu anegdotą. Przez następną godzinę, ci którzy słuchali Twojej historii są wyszkoleni w jednym zadaniu Twojego wyboru, które jest powiązane z anegdotą, tak długo, jak nie jest to atak lub obrona. Akcja by rozpocząć, jedna minuta by zakończyć.

\textbf{Zwierzęce Szpiegowanie}\index{Zdolności!Alfabetycznie!Zwierzęce Szpiegowanie}\label{sec:Zwierzęce Szpiegowanie} (4+ punkty Intelektu) - jeśli znasz ogólną lokalizację zwierzęcia, które jest przyjazne względem Ciebie i w zasięgu 1.5 km od Ciebie, możesz postrzegać świat jego zmysłami do 10 minut. Jeśli nie jesteś w formie zwierzęcej lub w formie podobnej do tego zwierzęcia, musisz zastosować poziom Wysiłku do korzystania z tej umiejętności. Akcja by rozpocząć. 

\textbf{Zwierzęcy Kształt}\index{Zdolności!Alfabetycznie!Zwierzęcy Kształt}\label{sec:Zwierzęcy Kształt} (3+ punkty Intelektu) - zmieniasz się w zwierzę tam małe jak szczur lub tak duże jak ty (np: duży pies lub mały niedźwiedź) na 10 minut. Za każdym razem, gdy zmieniasz kształt, możesz wybrać inne zwierzę. Twój ekwipunek staje się częścią owej transformacji, co czyni go nieużytecznym, o ile nie ma pasywnego efektu, takiego jak zbroja. W tej formie Twoje Statystyki pozostają takie same jak w Twojej normalnej formie, ale możesz się ruszać i atakować zgodnie z Twoim zwierzęcym kształtem (ataki większości zwierząt tego rozmiaru to bronie średnie, z których możesz korzystać bez żadnej kary). Zadania wymagające rąk - takie jak naciskanie klamek lub przycisków są utrudnione kiedy jesteś w formie zwierzęcej. Nie możesz mówić, ale dalej możesz korzystać ze zdolności, które nie polegają na ludzkiej mowie. Uzyskujesz dwie pomniejsze zdolności powiązane z istotą, w którą sie zmieniłeś (patrz tabela Mniejsze Zdolności Zwierzęcego Kształtu). Dla przykładu, jeśli zamieniasz się w nietoperza, jesteś wyszkolony w percepcji i możesz latać na daleki zasięg w każdej rundzie. Jeśli zamienisz się w ośmiornicę, jesteś wyszkolony w skradaniu się i oddychasz pod wodą. Jeśli zastosujesz poziom Wysiłku do stosowania tej zdolności, możesz albo przybrać kształt mówiącego zwierzęcia, albo hybrydowy. Kształt mówiącego zwierzęcia wygląda dokładnie jak zwykłe zwierzę, ale możesz dalej mówić i korzystać ze zdolności bazujących na ludzkiej mowie. Kształt hybrydowy wygląda jak Twoja normalna forma, ale z cechami zwierzęcia, nawet jeśli to konkretne zwierzę jest znacznie mniejsze od Ciebie (jak nietoperz lub szczur). W formie hybrydowej możesz mówić, korzystać ze swoich wszystkich zdolności, atakować jak zwierzę i wykonywać zadania przy użyciu rąk bez utrudnienia. Każdy kto dobrze się przypatrzy Tobie w formie hybrydowej nigdy nie pomyliłby Cię ze zwierzęciem. Akcja by się przemienić lub odwrócić transformację. 

``Podobieństwo'' to termin ogólnikowy. Lwy są podobne do tygrysów i leopardów, orły są podobne do kruków i łabędzi, psy są podobne do wilków i lisów itp.

Nawet jeśli Twój zwierzęcy kształt ma wiele typów ataku (np: zębami i pazurami), możesz zaatakować tylko raz w rundzie, chyba że masz jakąś zdolność, która pozwala CI na dokonywanie dodatkowych ataków w swojej turze.

Wariant Zwierzęcego Kształtu: Jeśli Twój koncept postaci sprawia, że zawsze zmienia się ona w ten sam zwierzęcy kształt zamiast wybierać z wielu, podwój czas trwania Zwierzęcego Kształtu (20 minut na jedno wykorzystanie). MG może pozwolić postaci z tym ograniczeniem na uczenie się dodatkowych zwierzęcych form poprzez wydanie 4 PD jako długotrwałą korzyść. 

\begin{table*}[t]

\centering
\caption{Tabela Mniejszych Zdolności Zwierzęcej Formy}
\label{Tabela Mniejszych Zdolności Zwierzęcej Formy}

\begin{tabularx}{\textwidth}{| X | X | X |}
\hline
 
 \textbf{Zwierzę} & \textbf{Umiejętność} & \textbf{Inne zdolności} \\ \hline

 Małpa & Wspinaczka & Ręce \\ \hline
 Borsuk & Wspinaczka & Czuły węch \\ \hline
 Nietoperz & Percepcja & Latanie \\ \hline
 Niedźwiedź & Wspinaczka & Czuły węch \\ \hline
 Ptak & Percepcja & Latanie \\ \hline
 Dzik & Obrona Mocy & Czuły węch \\ \hline
 Kot & Wspinaczka lub skradanie się & Mały \\  \hline
 Wąż dusiciel & Wspinaczka & Duszenie \\  \hline
 Krokodyl & Skradanie się lub pływanie & Duszenie \\  \hline
 Deinonych & Percepcja & Szybki \\  \hline
 Delfin & Percepcja lub pływanie & Szybki \\  \hline
 Ryba & Skradanie się lub pływanie & Wodny \\ \hline
 Żaba & Skakanie lub skradanie się & Wodny \\ \hline
 Koń & Percepcja & Szybki \\ \hline
 Leopard & Wspinaczka lub skradanie się & Szybki \\ \hline
 Jaszczurka & Wspinaczka lub skradanie się & Mały \\ \hline
 Ośmiornica & Skradanie się & Wodny \\ \hline
 Rekin & Pływanie & Wodny \\ \hline
 Żółw & Obrona Mocy & Pancerz \\ \hline
 Jadowity wąż & Wspinaczka & Trucizna \\ \hline
 Wilk & Percepcja & Czuły węch \\ \hline
 
 \end{tabularx}
 \end{table*}
 
 \begin{itemize}

\item \textbf{Wodny}: Zwierzę albo oddycha pod wodą zamiast powietrzem, albo jest w stanie oddychać wodą w dodatku do powietrza.

\item \textbf{Pancerz}: Zwierzę ma twardą skorupę lub skórę, co daje mu +1 do Pancerza.

\item \textbf{Duszenie}: Zwierzę może się szybko obwinąć wokół przeciwnika po udanym ataku wręcz (zazwyczaj ugryzieniu lub ataku pazurem), ułatwiając następne ataki przeciwko temu samemu wrogowi aż do momentu, aż kontakt nie zostanie zerwany.

\item \textbf{Szybki}: To zwierzę może się poruszać na daleki zasięg w swojej turze zamiast na średni.

\item \textbf{Latanie}: Zwierzę może latać, co (w zależności od typu zwierzęcia) może oznaczać ruch na średni lub daleki zasięg w swojej turze. 

\item \textbf{Ręce}: Zwierzę ma łapy lub ręce, które są niemal tak zwinne jak te ludzi. W przeciwieństwie do większości zwierzęcych kształtów, zadania zwierzęcia które wymagają rąk nie są utrudnione (choć MG może zdecydować, że niektóre zadania, wymagające ludzkie zwinności, np: gra na flecie, są dalej utrudnione).

\item \textbf{Czuły Węch}: Zwierzę posiada silny zmysł węchu, uzyskując atut na śledzeniu i akcjach w ciemności lub podczas oślepienia. 

\item \textbf{Mały}: Zwierzę jest znacznie mniejsze od człowieka, co ułatwia jego Obronę Szybkości ale utrudnia zadania polegające na przenoszeniu ciężkich rzeczy.

\item \textbf{Trucizna}: Zwierzę jest trujące (zazwyczaj jego ugryzienie), co zadaje dodatkowy 1 punkt obrażeń. 

\end{itemize}

\textbf{Zwierzęce Zmysły}\index{Zdolności!Alfabetycznie!Zwierzęce Zmysły}\label{sec:Zwierzęce Zmysły} - Jesteś wyszkolony w słuchaniu i dostrzeganiu rzeczy. Dodatkowo, przez większość czasu, MG powinien Cię poinformować o tym, że zaraz wkroczysz w pułapkę lub zostaniesz zaatakowany z zaskoczenia, jeśli zagrożenie jest na poziomie niższym niż 5. Umożliwienie. 

\textbf{Riposta}\index{Zdolności!Alfabetycznie!Riposta}\label{sec:Riposta} (3 punkty Szybkości) - Jeśli jesteś zaangażowany w walkę wręcz, możesz wykonać bezpośredni atak wręcz przeciwko każdemu z atakujących raz na rundę. Ten atak jest utrudniony, i ciągle możesz wykonać swoją normalną akcję podczas tej rundy. Umożliwienie.

\textbf{Uprzedzenie Ataku}\index{Zdolności!Alfabetycznie!Uprzedzenie Ataku}\label{sec:Uprzedzenie Ataku} (4 punkty Intelektu) - Możesz wyczuć jak i kiedy istoty Cię atakujące wykonają swoje ataki. Rzuty na Obronę Prędkości są ułatwione na jedną minutę. Akcja.

\textbf{Przebłysk}\index{Zdolności!Alfabetycznie!Przebłysk}\label{sec:Przebłysk} (1 punkt Intelektu) - Patrzysz w przyszłość by zobaczyć, jak Twoje akcje się zakończą. Pierwsze zadanie, które wykonasz przed końcem swojej następnej rundy uzyskuje atut. Akcja.

\textbf{Automatyczny Blask}\index{Zdolności!Alfabetycznie!Automatyczny Blask}\label{sec:Automatyczny Blask} - Przedmioty z twardego światła, które tworzysz, rzucają światło, oświecając wszystko w bliskim zasięgu. Kiedy tylko zechcesz, Twoje ciało (w całości lub tylko jego część) rzuca światło, oświecając wszystko w średnim zasięgu. Umożliwienie. 

\textbf{Stosowanie Swojej Wiedzy}\index{Zdolności!Alfabetycznie!Stosowanie Swojej Wiedzy}\label{sec:Stosowanie Swojej Wiedzy} - Kiedy pomagasz innej postaci w akcji, w której nie posiadasz wyszkolenia, jesteś traktowany jako wyszkolony w niej. Akcja.

\textbf{Aportacja}\index{Zdolności!Alfabetycznie!Aportacja}\label{sec:Aportacja} (4 punkty Intelektu) - Przywołujesz do siebie fizyczny obiekt. Możesz wybrać dowolną pozycję ze standardowej listy ekwipunku, lub (nie więcej niż raz dziennie) możesz pozwolić MG na przypadkowe określenie tego przedmiotu. Jeśli przywołujesz przypadkowy obiekt, ma on szansę 10 procent na bycie zamanifestowanym Cypherem lub artefaktem, 50 procent szans na bycie zwykłym ekwipunkiem i 40 procent szans na bycie jakimś bezużytecznym śmieciem. Nie możesz zastosować tej umiejętności, by wziąć przedmiot trzymany przez inną istotę. Akcja.

\textbf{Wodny Wojownik}\index{Zdolności!Alfabetycznie!Wodny Wojownik}\label{sec:Wodny Wojownik} - Ignorujesz wszelkie kary do akcji (wliczając walkę) w środowiskach podwodnych. Umożliwienie. 

\textbf{Potrójny Wystrzał}\index{Zdolności!Alfabetycznie!Potrójny Wystrzał}\label{sec:Potrójny Wystrzał} (3 punkty Szybkości) - Jeśli broń ma zdolność wystrzału ciągłego bez przeładowywania (zazwyczaj zwana jest bronią automatyczną), możesz wystrzelić ze swojej broni w kierunku do 3 celów (muszą stać obok siebie) na raz. Wykonaj osobny rzut na atak na każdy z celów. Każdy z tych ataków jest utrudniony. Akcja.

\textbf{Magiczny Błysk}\index{Zdolności!Alfabetycznie!Magiczny Błysk}\label{sec:Magiczny Błysk} (1 punkt Intelektu) - Ulepszasz obrażenia innego zaklęcia ofensywnego dodatkową energią, tak, że zadaje 1 dodatkowy punkt obrażeń. Alternatywnie, Twój atak sięga celu w dalekim zasięgu - jest to ognisty pocisk czystej magii, zadający 4 punkty obrażeń. Umożliwienie dla ulepszenia; akcja dla ataku na daleki zasięg.  

\textbf{Artefakty z Odzysku}\index{Zdolności!Alfabetycznie!Artefakty z Odzysku}\label{sec:Artefakty z Odzysku} (6 punktów Intelektu +2 PD) - Rozwinąłeś szósty zmysł odnośnie szukania najcenniejszych rzeczy na pustkowiach. Jeśli spędzisz czas wymagany by odnieść sukces na 2 zadaniach przeszukiwania, możesz wymienić ich rezultat na szansę pozyskania artefaktu wyboru MG jeśli zakończysz test 6 poziomu Intelektu powodzeniem. Możesz skorzystać z tej zdolności najczęściej raz dziennie i nigdy dwa razy w tym samym obszarze. Akcja by rozpocząć, parę godzin, by ją zakończyć.

\textbf{Mechanik Artefaktów}\index{Zdolności!Alfabetycznie!Mechanik Artefaktów}\label{sec:Mechanik Artefaktów} - Jeśli spędzisz przynajmniej 1 dzień majsterkując z artefaktem, który posiadasz, funkcjonuje on na poziomie o 1 wyższym niż normalnie. Stosuje się to do wszystkich artefaktów w Twoim władaniu, ale tylko Ty możesz korzystać z tego bonusa. Umożliwienie.

\textbf{Jak Przepowiedziano}\index{Zdolności!Alfabetycznie!Jak Przepowiedziano}\label{sec:Jak Przepowiedziano} - osiągasz coś, co udowadnia, że jesteś Wybrańcem. Następne zadanie, które podejmiesz, jest ułatwione o 3 stopnie. Nie możesz ponownie skorzystać z tej zdolności, aż do momentu, gdy zaznasz rzutu na odzyskanie zdrowia, który trwa 1 lub 10 godzin. Akcja.

\textbf{Jak Jedna Istota}\index{Zdolności!Alfabetycznie!Jak Jedna Istota}\label{sec:Jak Jedna Istota} - Kiedy Ty i Twoja bestia (ze zdolności Zwierzęcy Kompan) jesteście w bliskim zasięgu od siebie, możecie dzielić się obrażeniami, które uzyskujecie. Dla przykładu, jeśli jedno z Was zostanie trafione bronią za 4 punkty obrażeń, podzielcie je pomiędzy siebie jak uznacie za stosowne. Tylko Pancerz i odporności pierwotnego celu ataku wchodzą w grę. Więc jeśli masz Pancerz 2 i zostajesz zaatakowany za 4 punkty obrażeń magicznym pociskiem, Twoja bestia może wziąć 2 punkty obrażeń, które normalnie Ty byś musiał znieść, ale jej Pancerz się nie liczy, podobnie jak jej odporność na magię, jeśli jakaś. Umożliwienie. 

\textbf{Umiejętności Zabójcy}\index{Zdolności!Alfabetycznie!Umiejętności Zabójcy}\label{sec:Umiejętności Zabójcy} - Jesteś wyszkolony w skradaniu się i przebieraniu się. Umożliwienie.

\textbf{Cios Skrytobójcy}\index{Zdolności!Alfabetycznie!Cios Skrytobójcy}\label{sec:Cios Skrytobójcy} (5 punktów Intelektu) - Jeśli uda Ci sie zaatakować istotę, która nie była świadoma Twojej obecności, zadajesz dodatkowe 9 punktów obrażeń. Umożliwienie. 

\textbf{Potwierdzenie Własnego Przywileju}\index{Zdolności!Alfabetycznie!Potwierdzenie Własnego Przywileju}\label{sec:Potwierdzenie Własnego Przywileju} (3 punkty Intelektu) - Zachowując się tak, jak tylko osoba uprzywilejowana może, werbalnie ustawiasz do pionu wroga, który Cię słyszy i rozumie, wskutek czego nie może on podjąć żadnej akcji, wliczając w to ataki, przez jedną rundę. Niezależnie od tego, czy odniesiesz porażkę czy sukces, następna akcja celu jest utrudniona. Akcja.

\textbf{Przejęcie Kontroli}\index{Zdolności!Alfabetycznie!Przejęcie Kontroli}\label{sec:Przejęcie Kontroli} (6+ punktów Intelektu) - Kontrolujesz akcje innej istoty, z którą wszedłeś w interakcję lub którą badałeś przez co najmniej jedną rundę. Efekt trwa 10 minut. Cel musi być na poziomie 2 lub niższym. Kiedy już przejąłeś kontrolę, cel działa tak, jak sobie tego życzysz najlepiej jak potrafi, wolno korzystając ze swojej zdolności do czynienia osądów, chyba, że poświęcisz akcję na danie mu bardzo szczegółowych instrukcji. W ddoatku do normalnych opcji Wysiłku, możesz skorzystać z Wysiłku, by zwiększyć maksymalny poziom celu. Tak więc, aby spróbować wydać rokaz celowi 5 poziomu (3 poziomy ponad normalnym limitem), musisz zastosować 3 poziomy Wysiłku. Kiedy cel się kończy, cel pamięta wszystko, co się stało i reaguje na to zgodnie ze swoją naturą i Twojej relacji z nim - przejęcie kontroli mogło zniszczyć wcześniejszą, pozytywną relację między wami. Akcja, by rozpocząć.

\textbf{Atleta}\index{Zdolności!Alfabetycznie!Atleta}\label{sec:Atleta} - Jesteś wyszkolony w noszeniu ciężarów, wspinaczce, skakaniu i niszczeniu. Umożliwienie.

\textbf{Atak i Ponowny Atak}\index{Zdolności!Alfabetycznie!Atak i Ponowny Atak}\label{sec:Atak i Ponowny Atak} - Zamiast uzyskiwać więcej obrażeń lub mniejszy/większy efekt na naturalnej 17 lub więcej, możesz dzięki tej zdolności natychmiast wykonać następny atak. Umożliwienie.

\textbf{Estetyczny Atak}\index{Zdolności!Alfabetycznie!Estetyczny Atak}\label{sec:Estetyczny Atak} - Podczas ataku, wykonujesz stylowe ruchy, zachwycające gesty, lub coś innego co zachwyca lub rozbawia innych. Jedna istota, którą wybierasz w średnim zasięgu, która może Cię dostrzec, uzyskuje atut na swoim następnym zadaniu, jeśli jest ona wykonany w ciągu rundy lub dwóch. Umożliwienie.

\textbf{Modyfikacja Cyphera}\index{Zdolności!Alfabetycznie!Modyfikacja Cyphera}\label{sec:Modyfikacja Cyphera} (2+ punkty Intelektu) - Kiedy aktywujesz cypher, dodaj +1 do jego poziomu. W dodatku do normalnych opcji korzystania z Wysiłku, możesz wybrać korzystanie z Wysiłku, by zwiększyć poziom cyphera o dodatkowe +1 (na zastosowany poziom Wysiłku). Niem ożesz zwiększyć poziomu cyphera powyżej 10. Umożliwienie.

\textbf{Autodoktor }\index{Zdolności!Alfabetycznie!Autodoktor}\label{sec:Autodoktor} - jesteś wyszkolony w leczeniu, operacjach chirurgicznych i znoszeniu bólu. MOżesz operować samego siebie, pozostając w tym czasie w pełni świadomym. Umozliwienie. 

\textbf{Świadomość}\index{Zdolności!Alfabetycznie!Świadomość}\label{sec:Świadomość} (3 punkty Intelektu) - Stajesz się nadmiernie śiaodmym swojego otoczenia w celu lepszego odnalezienia swojego celu. Na 10 minut, jesteś świadom wszystkich żywych istot w dalekim zasięgu (wliczając ich ogólnie umiejscowienie) i poprzez koncentrację (kolejna akcja) możesz spróbować poznać informacje o stanie zdrowia i poziomie mocy każdej z nich. Akcja.

\section{B}

\textbf{Babel}\index{Zdolności!Alfabetycznie!Babel}\label{sec:Babel} - Po usłyszeniu mówionego języka przez parę minut, możesz się nim wysławiać i zostać w nim zrozumianym. Jeśli będziesz kontynuować korzystanie z tego języka, by wchodzić w interakcje z native speakerami, Twoje umiejętności z nim związane ulepszają się znacząco, aż w końcu możesz zostać pomylony z native speakerem już po paru godzinach mówienia w nowym języku. Umożliwienie.

\textbf{Balansowanie}\index{Zdolności!Alfabetycznie!Balansowanie}\label{sec:Balansowanie} - Jesteś wyszkolony w balansowaniu. Umożliwienie. 

\textbf{Drużyna Desperados}\index{Zdolności!Alfabetycznie!Drużyna Desperados}\label{sec:Drużyna Desperados} - Twoja reputacja ściąga do Ciebie grupę 6 2-poziomowych Kompanów desperado (BN-ów), którzy są Tobie kompletnie oddani. Powinieneś razem z MG określić szczegóły tych kompanów. Jeśli jeden z kompanów umrze, zyskujesz nowego w jego miejsce po przynajmniej 2 tygodniach i odpowiednim procesie rekrutacyjnym. Umożliwienie. 

\textbf{Drużyna Kompanów}\index{Zdolności!Alfabetycznie!Drużyna Kompanów}\label{sec:Drużyna Kompanów}  - Otrzymujesz 4 3-poziomowych kompanów. Nie są oni ograniczeni odnośnie ich modyfikacji. Umożliwienie.

\textbf{Ogłuszenie}\index{Zdolności!Alfabetycznie!Ogłuszenie}\label{sec:Ogłuszenie} (1 punkt Mocy) - Wykonujesz powtarzalny atak wręcz. Twój atak zadaje o 1 punkt obrażeń mniej niż zwykle, ale oszałamia cel na jedną rundę, podczas której wszystkie jego zadania są utrudnione. Akcja.

\textbf{Podstawowy Kompan}\index{Zdolności!Alfabetycznie!Podstawowy Kompan}\label{sec:Podstawowy Kompan} - Uzyskujesz kompana 2-poziomu. Jedną z jego modyfikacji musi być perswazja. Możesz wziąć tę zdolność wiele razy, za każdym razem otrzymując innego kompana 2-poziomu. Umożliwienie. (MG może określić, że musisz poszukać odpowiedniego kompana w świecie gry, zanim go zdobędziesz).

\textbf{Zarządzanie Bitwą}\index{Zdolności!Alfabetycznie!Zarządzanie Bitwą}\label{sec:Zarządzanie Bitwą} (4 punkty Intelektu) - Tak długo, jak używasz swojej akcji w każdej rundzie na wydawanie rozkazów lub dawanie porad, rzuty na atak i obronę podejmowane przez Twoich sprzymierzeńców w średnim zasięgu są ułatwione. Akcja.

\textbf{Taktyk Pola Walki}\index{Zdolności!Alfabetycznie!Taktyk Pola Walki}\label{sec:Taktyk Pola Walki} (2+ punkty Intelektu) - Oceniasz swoje otoczenia, poznając dowolne fakty, które MG uzna za ważne odnośnie atakowania, bronienia się, manewrowania i radzenia sobie z zagrożeniami środowiskowymi w średnim zasięgu. Dla przykładu, możesz zauważyć, że po wspięciu się na stos śmieci uzyskasz przewagę w walce wręcz, że w kącie najłatwiej się bronić, że gdzieś jest mniej śliska pokrywa lodowa na zamarzniętym jeziorze, lub że istnieje miejsce, gdzie trujący gaz jest rzadszy niż indziej. Jeśli Ty (lub ktoś, kogo o tym poinformujesz) przemieści sie w tamto miejsce, Ty (lub on/a) uzyska atut na zadaniach związanych z optymalną pozycją (takie jak ataki z wyższego miejsca, obrona Szybkości w łatwym do obrony rogu, rzuty na utrzymanie równowagi na lodzie lub Obrona Mocy przeciwko trującej chmurze). Zamiast uzyskać korzystną lokację, możesz się dowiedzieć o niekorzystnej lokacji, którą możesz wykorzystać przeciwko swoim wrogom, np: zapędzając ich w kozi róg który utrudnia ich ataki wręcz lub słaby punkt na zamarzniętym jeziorze, gdzie lód się pod nimi załamie. Możesz wykorzystać Wysiłek by dowiedzieć się o jednej dodatkowej dobrej lub złej lokacji w zasięgu (jedna lokacja na poziom Wysiłku), zwiększyć zasięg zdolności (dodatkowy średni zasięg na poziom Wysiłku) lub skorzystać z obydwu tych opcji. Umożliwienie. 

\textbf{Zew Dziczy}\index{Zdolności!Alfabetycznie!Zew Dziczy}\label{sec:Zew Dziczy} (5 punktów Intelektu) - Przywołujesz hordę małych zwierząt lub jedno zwierzę 4 poziom, by Tobie chwilowo pomagało. Te istoty robią ,jak rozkażesz tak długo, jak skupiasz na nich swoją uwagę, ale musisz wykorzystać swoje akcje, by nimi kierować. Istoty te naturalnie występują w tym terenie i przybywają o własnych siłach, więc jeśli jesteś w niedostępnym miejscu, to nie przybędą. Akcja.

\textbf{Zwierzęcy Kompan}\index{Zdolności!Alfabetycznie!Zwierzęcy Kompan}\label{sec:Zwierzęcy Kompan} - Istota 2 poziomu rozmiaru Twojego lub mniejszego towarzyszy Ci i słucha Twoich instrukcji. Powinieneś wypracować z MG szczegóły tej istoty, i najpewniej będziesz za nią wykonywał rzuty w czasie walki lub gdy wykonuje ona jakieś akcje. Zwierzęcy kompan działa w Twojej turze. Jako istota 2 poziomu, posiada ona stopień trudności 6 i 6 punktów życia, a zadaje 2 obrażenia. Jej zdolności ruchu bazują na jej typie istoty (ptak, istota pływająca itp.). Jeśli Twój zwierzęcy Kompan umrze, możesz przeszukać dzicz przez 1k6 dni, by odnaleźć nowego. Umożliwienie. (Poziom istoty określa jej stopień trudności, punkty życia i obrażenia, chyba, że zaznaczono inaczej. Tak więc zwierzęcy kompan 2 poziomu ma stopień trudności 6, 6 punktów życia i zadaje 2 obrażenia. Zwierzęcy kompan 4 poziomu ma stopień trudności 12, 12 punktów życia i zadaje 4 punkty obrażeń. I tak dalej.).

\textbf{Oczy Bestii}\index{Zdolności!Alfabetycznie!Oczy Bestii}\label{sec:Oczy Bestii} (3 punkty Intelektu) - Poprzez podlączenie się do istoty z Twojej zdolności Zwierzęcy Kompan, możesz postrzegać świat jego zmysłami, jeśli znajduje się w zasięgu 1.5 km od Ciebie. Ten efekt trwa 10 minut. Akcja, by ustanowić.

\textbf{Likantropia}\index{Zdolności!Alfabetycznie!Likantropia}\label{sec:Likantropia} - w ciągu 5 sąsiadujących z sobą nocy w każdym miesiącu, zamieniasz się w potworna bestię (do 1 godziny każdej nocy). W tej nowej formie, uzyskujesz +8 do Puli Mocy, +1 do Skupienia w Mocy, +2 do Puli Szybkości i +1 do Skupienia w Szybkości. Kiedy jesteś w zwierzęcej formie, nie możesz wydawać punktów Intelektu na cokolwiek innego niż próba powrotu do normalnej formy zanim minie godzina (zadanie o trudności 2). Dodatkowo, atakujesz wszystkie żywe istoty w średnim zasięgu od Ciebie. Po tym, jak wracasz do normalnej formy, otrzymujesz karę -1 do wszystkich rzutów na godzinę. Jeśli nie zabiłeś i zjadłęś przynajmniej jednej istoty w zwierzęcej formie, ta kara wzrasta do -2 i dotyczy wszystkich Twoich rzutów przez 24 godziny. Akcja, by zmienić się z powrotem.

\textbf{Niezauważalny}\index{Zdolności!Alfabetycznie!Niezauważalny}\label{sec:Niezauważalny}  - Twój obniżony wzrost sprawia, że trudno Cię znaleźć. Kiedy Zmniejszenie się jest aktywne, wszystkie Twoje próby skradania się są ułatwione. Umożliwienie.

\textbf{Wiedza z Bestiariusza}\index{Zdolności!Alfabetycznie!Wiedza z Bestiariusza}\label{sec:Wiedza z Bestiariusza} - jesteś wyszkolony w wiedzy o niehumanoidalnych istotach, które jedzą ludzkie mięso - rozpoznawaniu ich, poznawaniu ic hsłabości i poznawaniu ich zwyczajów i zachowań. Umożliwienie. 

\textbf{Zdrada}\index{Zdolności!Alfabetycznie!Zdrada}\label{sec:Zdrada} - Za każdym razem, gdy przekonasz przeciwnika, że nie jesteś zagrożeniem i następnie zaatakujesz z nienacka (bez prowokacji), ten atak zadaje 4 dodatkowy punkty obrażeń. Umożliwienie. 

\textbf{Lepsze Życie Dzięki Chemii}\index{Zdolności!Alfabetycznie!Lepsze Życie Dzięki Chemii}\label{sec:Lepsze Życie Dzięki Chemii} (4 punkty Intelektu) - Stworzyłeś koktajle chemiczne dostosowane do Twojej własnej biochemii. W zależności od tego, który z nich zażyjesz, czyni Cię to mądrzejszym, szybszym lub wytrzymalszym, ale kiedy ich działanie się kończy, masz przekichane, więc korzystasz z nich tylko wtedy, kiedy sytuacja tego wymaga. Uzyskujesz 2 punkty w Skupieniu w Mocy, Szybkości lub Intelekcie na jedna minutę, po której nie możesz ponownie skorzystać z tej zdolności przez godzinę. Podczas tej godziny, za każdym razem, gdy wydasz punkt z Puli, zwiększ kosz tej akcji o 1. Akcja.

\textbf{Lepszy Atak z Zaskoczenia}\index{Zdolności!Alfabetycznie!Lepszy Atak z Zaskoczenia}\label{sec:Lepszy Atak z Zaskoczenia} - Jeśli atakujesz z ukrycia lub przed akcją przeciwnika, otrzymujesz atut na swoim ataku (jeśli posiadasz także zdolność Atak z Zaskoczenia, dodajesz obydwa atuty). Po udanym ataku, zadajesz dodatkowe 2 punkty obrażeń (w sumie 4, jesli posiadasz Atak z Zaskoczenia. Umożliwienie.

\textbf{Większy}\index{Zdolności!Alfabetycznie!Większy}\label{sec:Większy} - Kiedy korzystasz ze Wzrostu, możesz osiągnąć rozmiar 4 metrów i dodajesz dodatkowe 3 chwilowe punkty do swojej Puli Mocy. Umożliwienie.

\textbf{Większy Zwierzęcy Kształt}\index{Zdolności!Alfabetycznie!Większy Zwierzęcy Kształt}\label{sec:Większy Zwierzęcy Kształt} - kiedy korzystasz ze Zwierzęcego Kształtu, Twoja zwierzęca forma rozrasta się do podwójnego normalnego rozmiaru. Będąć tak wielką, zwierzęca forma dodaje Ci następujące bonusy: +1 do Pancerza, +5 do Puli Mocy, jesteś także wyszkolony w używaniu naturalnych ataków swojej zwierzęcej formy jako ciężkich broni (jeśli nie byłeś wyszkolony). Jednakże, Twoja Obrona Szybkości jest utrudniona. Kiedy jestes większy, dostajesz także atut na zadaniach które są prostsze dla dużych istot, takich jak wspinaczka, zastraszanie, przepływanie rzek itp. Umożliwienie. 

\textbf{Większy Likantrop}\index{Zdolności!Alfabetycznie!Większy Likantrop}\label{sec:Większy Likantrop} - Kiedy przebywasz w formie likantropa, Twoja forma fizyczna jest większa niż wcześniej, sięgając 4 metrów. Będąc tak wielkim, otrzymujesz następujące bonusy: +1 do Pancerza, +5 do Puli Mocy, i jesteś wyszkolony w korzystaniu ze swoich pięści jak z ciężkich broni (jeśli jeszcze nie jesteś). Jednakże, Twoja Obrona Szybkości jest utrudniona. Kiedy jesteś tak wielki, uzyskujesz atut na zadaniach które są prostsze dla dużej istoty, takich jak wspinaczka, zastraszanie, przepływanie rzek itp. Umożliwienie. 

\textbf{Detonacja Biomorficzna}\index{Zdolności!Alfabetycznie!Detonacja Biomorficzna}\label{sec:Detonacja Biomorficzna} (7+ punktów Mocy) - Emitujesz impuls biomorficznej energii w średnim zasięgu, ale tak manipulujesz jego frekwencją, by przeszkadzał życiu w bliskim zasięgu. Wszyscy w promieniu detonacji otrzymują 5 punktów obrażeń, które ignorują Pancerz (chyba, że Pancerz wynika z pola siłowego). Jeśli zastosujesz dodatkowy Wysiłek by zwiększyć obrażenia, zadajesz 2 dodatkowe punkty obrażeń na poziom Wysiłku (zamiast normalnych 3 punktów); cele w zasięgu otrzymują 1 punkt obrażeń nawet jeśli nie powiedzie Ci się rzut na atak. Akcja. 

\textbf{Biomorficzne Leczenie}\index{Zdolności!Alfabetycznie!Detonacja Biomorficzna}\label{sec:Detonacja Biomorficzna} (4+ punktów Mocy) - Świadomie wysyłasz puls swojego pola biomorficznego (dziwna energia generowana przez ciało) i skupiasz go na żywej istocie w średnim zasięgu. Cel uzyskuje darmowy i natychmiastowy rzut na odzyskanie zdrowia. Nie możesz ponownie użyć tej zdolności na danej istocie aż do momentu, gdy zakończysz swój 10-godzinny odpoczynek. Akcja. 

\textbf{Spryciula}\index{Zdolności!Alfabetycznie!Spryciula}\label{sec:Spryciula} - Jesteś wyszkolony w jednej z następujących umiejętności: oszustwie, skradaniu się lub przebieraniu się. Umożliwienie. 

\textbf{Zlanie się z Tłem}\index{Zdolności!Alfabetycznie!Zlanie się z Tłem}\label{sec:Zlanie się z Tłem} (4 punkty Intelektu) - Kiedy zlewasz się z tłem, istoty dalej Cię widzą, ale nie przywiązują do Twojej obecności wagi przez około minutę. Kiedy się zlewasz z tłem, jesteś wyspecjalizowany w skradaniu się i Obronie Szybkości. Ten efekt kończy się, gdy zrobisz coś, by ujawnić swoją obecność lub pozycję - zaatakujesz, skorzystasz ze zdolności, przesuniesz wielki obiekt itp. Jeśli to się wydarzy, możesz odzyskać brakujący czas efektu poprzez poświęcenie akcji na skupieniu się, by wyglądać niewinnie i na swoim miejscu. Akcja bo ryzpocząć lub odzyskać.

\textbf{Błogosławieństwo Bóstw}\index{Zdolności!Alfabetycznie!Błogosławieństwo Bóstw}\label{sec:Błogosławieństwo Bóstw} - Jako sługa bóstw, masz różne błogosławieństwa, których Ci udzieliły. Błogosławieństwo zależy od danego bóstwa i jego specjalności. Wybierz dwie ze zdolności wymienionych poniżej.

\begin{itemize}
\item \textbf{Autorytet/Prawo/Pokój} (3 punkty Intelektu) - Powstrzymujesz wroga, który Cię słyszy i rozumie, przed zaatakowaniem kogoś lub czegoś przez jedną rundę. Akcja.
\item \textbf{Dobrotliwość/Moralność/Duch} (2+ punktów Intelektu) - Jeden demon, duch lub podobna istota 1-poziomu w średnim zasięgu zostaje zniszczona lub wygnana. W dodatku do normalnych opcji Wysiłku, możesz skorzystać z niego, by zwiększyć maksymalny poziom celu. Tak więc, aby zniszczyć lub wygnać cel 5-poziomu (4 poziomy więcej niż normalnie) musisz zastosować 4 poziomy Wysiłku. Akcja.
\item \textbf{Śmierć/Ciemność} (2 punkty Intelektu) - Cel, który wybierasz w średnim zasięgu, otrzymuje 3 punkty obrażeń. Akcja.
\item \textbf{Pragnienie/Miłość/Zdrowie} (3 punkty Intelektu) - Poprzez dotyk, możesz przywócić 1d6 punktów do jednej Puli Statystyk dowolnej istocie, wliczając siebie. Ta zdolność to zadanie Intelektu 2 poziomu. Za każdym razem, gdy chcesz uzdrowić tę samą istotę, zadanie jest utrudnione poziom więcej. Trudność wraca do 2 po tym, jak istota zakończy 10-godzinny odpoczynek. Akcja.
\item \textbf{Kamień/Ziemia} - Jesteś wyszkolony we wspinaczce, kamieniarstwie i eksploracji jaskiń. Umożliwienie.
\item \textbf{Wiedza/Mądrość} (3 punkty Intelektu) - Wybierz do 3 istot (możesz się wśród nich znaleźć). Przez minutę, konkretny typ zadania (ale nie rzut na atak lub obronę) jest ułatwiony dla tych istot, ale tylko, jeśli zostają w bliskiej odległości od Ciebie. Akcja.
\item \textbf{Natura/Zwierzęta/Rośliny} - Jesteś wyszkolony w botanice i obchodzeniu się z dzikimi zwierzętami. Umożliwienie.
\item \textbf{Ochrona/Cisza} (3 punkty Intelektu) - Tworzysz cichy bąbel ochronny wokół siebie o promieniu bliskiego zasięgu na jedną minutę. Bąbel porusza się wraz z Tobą. Wszystkie rzuty obronne dla Ciebie i istot, które wybierzesz, gdy wykonywane są wewnątrz tego bąbla, są ułatwione, i żaden hałas, niezależnie od jego natury, nie wybrzmiewa głośniej niż normalna rozmowa. Akcja, by rozpocząć. 
\item \textbf{Powietrz/Niebo} (2 punkty intelektu) - Istota, której dotkniesz, jest odporna na toksyny i choroby przenoszone drogą powietrzną na 10 minut. Akcja.
\item \textbf{Słońce/Światło/Ogień} (2 punkty Intelektu) - Sprawiasz, że jedna istota lub obiekt w średnim zasięgu się zapala, co zadaje jej 1 punkt obrażeń środowiskowych w każdej rundzie, do momentu, gdy ogień zostanie wygaszony (wymaga to akcji). Akcja.
\item \textbf{Oszustwo/Chciwość/Handel} - jesteś wyszkolony w wykrywaniu oszustw innych istot. Umożliwienie. 
\item \textbf{Wojna} (1 punkt Intelektu) - Cel, który wybierasz w średnim zasięgu (możesz to być Ty) zadaje 2 obrażenia więcej swoim następnym udanym atakiem bronią. Akcja.
\item \textbf{Woda/Morze} (2 punkty Intelektu) - Cel, który wybierasz może oddychać pod wodą przez 10 minut. Akcja.
\end{itemize}

\textbf{Oślepienie Maszyny}\index{Zdolności!Alfabetycznie!Oślepienie Maszyny}\label{sec:Oślepienie Maszyny} (6 punktów Szybkości) - Deaktywujesz sensory maszyny, czyniąc ją ślepą do czasu, aż ktoś ją naprawi. Musisz albo dotknąć maszyny, albo uderzyć w nią atakiem dystansowym (nie zadaje on obrażeń). Akcja.

\textbf{Oślepiający Atak}\index{Zdolności!Alfabetycznie!Oślepiający Atak}\label{sec:Oślepiający Atak} (3 punkty Szybkości) - Jeśli masz ze sobą źródło śWiatła, możesz go urzyć, by wykonać atakl wręcz przeciwko celowi. Jeśli atak jest udany, nie zadaje on obrażeń, ale cel jest oślepiony na jedną minutę. Akcja.

\textbf{W mgnieniu Oka}\index{Zdolności!Alfabetycznie!W Mgnieniu Oka}\label{sec:W Mgnieniu Oka} (4 punkty Szybkości) - Poruszasz się do 300 metrów w jednej undzie. Akcja.

\textbf{Blok}\index{Zdolności!Alfabetycznie!Blok}\label{sec:Blok} (3 punkty Szybkości) - Automatycznie blokujesz następny atak wręcz wykonany przeciwko Tobie w następnej minucie. Akcja by rozpocząć. 

\textbf{Chronienie Sprzymierzeńca}\index{Zdolności!Alfabetycznie!Chronienie Sprzymierzeńca}\label{sec:Chronienie Sprzymierzeńca} - Jeśli korzystasz z lekkiej lub średniej broni, możesz blokować ataki wykonywane przeciwko sprzymierzeńcowi blisko Ciebie. Wybierz jedną istotę w bliskim zasięgu. Zapewniasz mu atut do Obrony Szybkości. Nie możesz skorzystać z Szybkiego Bloku kiedy Chronisz Sprzymierzeńca. Umożliwienie.

\textbf{Gorączka Krwi}\index{Zdolności!Alfabetycznie!Gorączka Krwi}\label{sec:Gorączka Krwi} - Kiedy nie masz punktów w jednej lub dwóch Pulach, uzyskujesz atut na rzutach na atak lub obronę (Twój wybór). Umożliwienie.

\textbf{Zew Krwi}\index{Zdolności!Alfabetycznie!Zew Krwi}\label{sec:Zew Krwi} (3 punkty Mocy) - Jeśli pokonasz w walce wroga, możesz się przemieścić o średni dystans, ale tylko, jeśli się przemieszczasz w kierunku innego wroga. Nie musisz wydawać punktów aż do momentu, w którym wiesz, że pierwszy wróg poległ. Umożliwienie.

\textbf{Rozmazana Prędkość}\index{Zdolności!Alfabetycznie!Rozmazana Prędkość}\label{sec:Rozmazana Prędkość} (7 punktów Mocy) - Poruszasz się tak szybko, że do Twojej następnej tury, jesteś rozmazany. Kiedy jesteś rozmazany, jeśli wykorzystasz Wysiłek na ataku wręcz lub Obronie Szybkości, uzyskujesz darmowy poziom Wysiłku na tym zadaniu. Możesz się przemieścić na średni dystans jako część innej akcji lub na daleki dystans, jeśli poświęcisz na to całą akcję. Umożliwienie. 

\textbf{Zmiana Ciała}\index{Zdolności!Alfabetycznie!Zmiana Ciała}\label{sec:Zmiana Ciała} (3+ punkty Intelektu) - Zmieniasz swoje ciało i twarz oraz kolory na jedna godzinę, ukrywając swoją tożsamość lub poszywając się pod kogoś. Jeśli zastosujesz poziom Wysiłku, możesz udawać konkretną osobę dostatecznie dobrze, by oszukać kogoś, kto zna ją dobrze lub przebadał ją z bliska (wliczając odciski palców i porównanie głosu, ale nie wzór siatkówki lub DNA). Masz atut na wszelkich zadaniach związanych z przebieraniem się (w dodatku do atutu z Morficznej Twarzy). Musisz zastosować osobny poziom Wysiłku, jeśli chcesz udawać inny gatunek (np: gdy człowiek chce udawać humanoidalnego kosmitę). Akcja.

\textbf{Jeździec Błyskawicy}\index{Zdolności!Alfabetycznie!Jeździec Błyskawicy}\label{sec:Jeździec Błyskawicy} (4 punkty Intelektu) - Możesz przemieścić się na daleki dystans z jednego miejsca na drugie prawie natychmiastowo, przeniesiony przez błyskawicę. Musisz być w stanie dostrzec nową lokację, i nie może być między nimi żadnych przeszkadzających barier. Akcja.

\textbf{Promienie Mocy}\index{Zdolności!Alfabetycznie!Promienie Mocy}\label{sec:Promienie Mocy} (5+ punktów Intelektu) - Wystrzeliwujesz wachlarz błyskawic na średni zasięg w pióropuszu, który jest szeroki na 15 metrów na swoim końcu. Ten atak zadaje 4 punkty obrażeń. Jeśli zastosujesz Wysiłek by zwiększyć obrażenia zamiast ułatwić atak, zadajesz 2 dodatkowe punkty obrażeń na poziom Wysiłku (zamiast zwyczajowych 3); cele w zasięgu otrzymują 1 punkt obrażeń nawet jeśli nie powiedzie Ci się rzut na atak. Akcja.

\textbf{Urealnienie Iluzji}\index{Zdolności!Alfabetycznie!Urealnienie Iluzji}\label{sec:Urealnienie Iluzji} (2+ punkty Intelektu) - Dajesz jednej ze swoich wizualnych iluzji ograniczoną fizyczną realność, którą można powąchać, posmakować, usłyszeć i poczuć. Ten efekt jest powiązany z daną iluzją i zachowuje się stosownie do jej natury. Dla przykładu, może on sprawić, że iluzja cegły będzie odczuwalna jako cegła, iluzja osoby będzie pachnieć jak perfumy i być w stanie otwierać drzwi, a iluzja kominka będzie ciepła w dotyku. 

Fizyczna rzeczywistość zapewniona Twojej iluzji jest na poziomie 1 z 3 punktami zdrowia. Jeśli z iluzji się korzysta, by zaatakować, zadaje ona tylko 1 punkt obrażeń (mogą to być normalne obrażenia jak ciosy pięści i kopnięcia, lub obrażenia środowiskowe jak spadające cegły lub ognie kominka). Możesz zwiększyć poziom stworzonego efektu, poprzez dodanie poziomów Wysiłku - każdy poziom Wysiłku zwiększa realność iluzji o 1 poziom.

Możesz aktywować tę zdolność jako część akcji stworzenia iluzji (korzystając z dowolnej zdolności tworzenia iluzji, jaką dysponujesz, np: Mniejszej Iluzji) lub możesz wykorzystać osobną akcję zastosowaną względem jednej z Twoich iluzji, które już istnieją. Efekt kończy się, gdy iluzja zostaje zniszczona, pozwalasz jej na ustąpienie, punkty życia iluzji są zredukowane do 0, lub gdy upłynie 10 minut. Umożliwienie.

\textbf{Ulepsz Materialny Cypher}\index{Zdolności!Alfabetycznie!Ulepsz Materialny Cypher}\label{sec:Ulepsz Materialny Cypher} (2 punkty Intelektu) - Zamanifestowany Cypher który aktywujesz w swojej następnej akcji, działa jakby miał 2 poziomy więcej. Akcja.

\textbf{Ulepsz Funkcjonowanie Materialnego Cyphera}\index{Zdolności!Alfabetycznie!Ulepsz Funkcjonowanie Materialnego Cyphera}\label{sec:Ulepsz Funkcjonowanie Materialnego Cyphera} (4 punkty Intelektu) - Dodaj 3 do działającego poziomu zamanifestowanego cyphera, który aktywujesz w swojej następnej akcji, lub zmień jeden z aspektów jego parametrów (zasięg, czas trwania, obszar efektu itp.). Parametry możesz maksymalnie podwoić, a minimalnie obniżyć do 1/10. Akcja. 

\textbf{Skacząca Tarcza}\index{Zdolności!Alfabetycznie!Skacząca Tarcza}\label{sec:Skacząca Tarcza} - Kiedy korzystasz z Rzutu Tarczą Siłową, zamiast zanikać po jednym ataku (udanym lub nie), zaatakuje ona do dwóch dodatkowych celów w średnim zasięgu. Wysiłek lub inne modyfikatory zastosowane do pierwszego ataku stosują się również do innnych celów. Niezależnie od tego, czy trafisz wszystkie, trochę, czy zero celów, tarcza zanika, a potem reformuje się w Twoich dłoniach. (Jeśli wziąłeś Skaczącą Tarczę, a wcześniej wziąłeś Rzut Tarczą Siłową, masz opcję zamiany tamtej zdolności na Leczący Puls). Umożliwienie. 

\textbf{Związana Magiczna Istota}\index{Zdolności!Alfabetycznie!Związana Magiczna Istota}\label{sec:Związana Magiczna Istota} - Posiadasz 3-poziomowego magicznego sprzymierzeńca przywiązanego do fizycznego obiektu (może pomniejszy dżin przywiązany do lampy, mniejszy demon przywiązany do monety, albo duch przywiązany do lusterka). Magiczny sprzymierzeniec jeszcze nie posiada pełnej mocy, którą by posiadł, gdyby dojrzał. Normalnie, sprzymierzeniec pozostaje uśpiony w obiekcie, do którego jest przywiązany. Kiedy skorzystasz z akcji, by go zamanifestować, pojawia się obok Ciebie jako istota, z którą możesz porozmawiać. Istota posiada swoją własną osobowość określoną przez MG i jest o poziom wyższa w danej dziedzinie wiedzy (takiej jak np: lokalna historia). MG określa, czy magiczny sprzymierzeniec posiada jakieś długoterminowe cele.

Za każdym razem gdy magiczny sprzymierzeniec staje się fizycznie zamanifestowany, pozostaje takim na okres do jednej godziny. W tym czasie, towarzyszy Tobie i wykonuje Twoje instrukcje. Magiczny sprzymierzeniec musi pozostać w bliskiej odległości od Ciebie - jeśli się oddali bardziej, zostaje wrzucony do swojego obiektu na koniec Twojej następnej tury i nie może wrócić aż do końca Twojego następnego 10-godzinnego odpoczynku. Nie atakuje on istot, ale może skorzystać ze swojej akcji, aby dać Ci atut na każdym jednym ataku, który wykonujesz w swojej własnej turze. Poza tym, może on dokonywać akcji na własną rękę (choć to raczej Ty będziesz rzucać kością).

Jeśli atak zredukuje punkty życia istoty do 0, znika ona. Reformuje się ona w przywiązanym obiekcie w ciągu 1d6 +2 dni. 

Jeśli utracisz przywiązany obiekt, wiesz w którym kierunku musisz iść, by go znaleźć. Akcja by zamanifestować magiczną istotę. 

\textbf{Pranie Mózgu}\index{Zdolności!Alfabetycznie!Pranie Mózgu}\label{sec:Pranie Mózgu} (6+ punktów Intelektu) - Korzystasz ze sztuczek, dobrze przemyślanych kłamstw i chemikaliów wpływających na umysł (lub innych środków, takich jak magia lub hiper-technologia) by nakłonić chwilowo innych do zrobienie tego, czego chcesz. Kontrolujesz akcję innej istoty, której dotykasz. Ten efekt trwa przez minutę. Cel musi być na 3 poziomie lub niższym. Możesz pozwolić mu na swobodne działanie lub przejąć kontrolę tak długo, jak możesz ją dostrzec. W dodtku do zwykłych opcji korzystania z Wysiłku, możesz wybrać zwiększenia maksymalnego poziomu celu lub zwiększenie czasu trwania efektu na jedną minutę. Tak wię,c aby kontrolować umysł celu 6 poziomu (trzy poziomy powyżej normalny limit) lub by kontrolować cel na 4 minuty (trzy minuty powyżej normalnego czasu trwania), musisz zastosować 3 poziomy Wysiłku. Kiedy czas trwania się kończy, ta istota nie pamięta bycia kontrolowaną ani niczego, co zrobiła, kiedy była pod wpływem Twojej osoby. Akcja, by rozpocząć.

\textbf{Przełamanie Linii}\index{Zdolności!Alfabetycznie!Przełamanie Linii}\label{sec:Przełamanie Linii} - Łatwo dostrzegasz dyscyplinę grupy i hierarchie, także pośród Twoich wrogów. Jeśli grupa wrogów zyskuje dowolny rodzaj korzyści z tytułu wspólnej pracy, możesz spróbować im przeszkodzić w odniesieniu tej korzyści, poprzez wskazanie słabego punktu w formacji wrogów lub ataku grupowym. Ten efekt trwa do minuty lub dopóki dotknięci nim wrogowie nie spędzą rundy na przegrupowaniu się, by odzyskać normalną korzyść. Akcja, by zainicjować. 

\textbf{Ruch i Multiatak}\index{Zdolności!Alfabetycznie!Ruch i Multiatak}\label{sec:Ruch i Multiatak} (6 punktów Szybkości) - Poruszasz się do średniego dystansu i atakujesz do 4 różnych wrogów w jednej akcji tak długo, jak są na Twojej ścieżce. Dowolny modyfikator który stosujesz do jednego ataku stosujesz do wszystkich ataków, które wykonujesz. Jeśli masz inną specjalną zdolność, która pozwala Ci na poruszanie się i wykonanie akcji, kiedy korzystasz z Ruchu i Multiataku, uzyskujesz atut na atakowaniu tych wrogów. Akcja. 

\textbf{Przełamanie Umysłów}\index{Zdolności!Alfabetycznie!Przełamanie Umysłów}\label{sec:Przełamanie Umysłów} (7+ punktów Intelektu) - Korzystając ze swoich sprytnych słów i wiedzy o innych, i mając parę rund konwersacji by pozyskać parę konkretnych informacji o kontekście odnośnie Twojego celu, możesz wypowiedzieć zdanie zaprojektowane tak, by zraniło psychikę Twojego rozmówcy. Jeśli cel Cię słyszy i rozumie, otrzymuje on 6 punktów obrażeń Intelektu (ignorujących Pancerz) i zapomina ostatni dzień swojego życia, co może sprawić, że zapomni Ciebie i to, jak się znajduje w danym miejscu. W dodatku do zwykłych opcji korzystania z Wysiłku, możesz skorzystać z niego, by przełamać umysł jednego dodatkowego celu, który Cię słyszy i rozumie. Akcja by rozpocząć, akcja by zakończyć.

\textbf{Niszczyciel}\index{Zdolności!Alfabetycznie!Niszczyciel}\label{sec:Niszczyciel} - Jesteś wyszkolony w zadaniach polegających na uszkadzaniu obiektów. Akcja uszkodzenia obiektu to akcja Mocy, i przy sukcesie, obiekt przesuwa się o jeden krok w dół na liczniku obrażeń obiektu. Jeśli test Mocy przekroczy trudność o dwa kroki, zamiast tego obiekt przesuwa się w dół o dwa kroki w dół na liczniku obrażeń obiektu. Jeśli test Mocy przekroczy trudność o 4 kroki, obiekt przesuwa się na dół o 3 kroki na liczniku obrażeń obiektu i zostaje natychmiastowo zniszczony. Lekkie materiały redukują efektywny poziom obiektu, kiedy twarde materiały jak drewno lub kamień dodaję 1 do efektywnego poziomu lub (lub 2 dla bardzo twardych przedmiotów stworzonych z metalu). Umożliwienie.

\textbf{Brutalne Uderzenie}\index{Zdolności!Alfabetycznie!Brutalne Uderzenie}\label{sec:Brutalne Uderzenie} (4 punkty Mocy) - Zadajesz o 4 punkty obrażeń więcej wszystkimi atakami wręcz do końca swojej następnej rundy. Umożliwienie.

\textbf{Koleżka}\index{Zdolności!Alfabetycznie!Koleżka}\label{sec:Koleżka} (3 punkty Intelektu) - Wybierz jedną z postaci stojących obok Ciebie. Ta postać zostaje Twoim koleżką na 10 minut. Jesteś wyszkolony we wszelkich zadaniach polegających na odnalezieniu, leczeniu, wchodzeniu z interakcję i chronieniu Twojego koleżki. Także, kiedy stoisz obok niego, obydwoje macie atut na Ochronie Szybkości. Możesz mieć tylko jednego koleżkę w danym czasie. Akcja, by rozpocząć. 

\textbf{Wbudowane Bronie}\index{Zdolności!Alfabetycznie!Wbudowane Bronie}\label{sec:Wbudowane Bronie} - Biomechaniczne implanty, magiczny kryształ wbudowany w czoło, lub coś równie dziwnego zapewnia Cię we wbudowaną broń. Pozwala Ci to na wystrzelenie promienia energii na długi zasięg, który zadaje 5 punktów obrażeń. Ta zdolność nic Cię nie kosztuje. Akcja.

\textbf{Palące Światło}\index{Zdolności!Alfabetycznie!Palące Światło}\label{sec:Palące Światło} (3 punkty Intelektu) - Wysyłasz promień światła na daną istotę w długim zasięgu, a następne zwężasz go, aż będzie palił, zadając 5 punktów obrażeń. Akcja. 

\textbf{Ucieczka}\index{Zdolności!Alfabetycznie!Ucieczka}\label{sec:Ucieczka} (5 punktów Szybkości) - Możesz wykonać dwie osobne akcje w tej turze, tak długo jak jedna z nich to ucieczka od wroga lub ukrycie się. Umożliwienie. 

\textbf{Przeniknięcie Przez Barierę}\index{Zdolności!Alfabetycznie!Przeniknięcie Przez Barierę}\label{sec:Przeniknięcie Przez Barierę} (6+ punktów Intelektu) - Przechodzisz przez drzwi, pole siłowe lub inną barierę, która ma maksymalną grubość 1 metra. W zależności od bariery, może to oznaczać znalezienie słabego punktu, który wykorzystujesz, naciśnięcie odpowiednich guzików czystym szczęściem, po prostu użycie siły, lub nawet dziwniejsze wyjaśnienia, jak dotknięcie cieńkiej warstewki między wymiarami lub niespodziewana interakcja z Twoim ekwipunkiem. Trudność zadania to poziom bariery. Ta zdolność pozwala Tobie na przeniknięcie, nikomu innemu, a przejście zamyka się na końcu Twojej tury (co może oznacząc, że jesteś uwięziony po drugiej stronie). Masz atut na każdej próbie następnego przeniknięcia przez już raz przenikniętą barierę. W dodatku do zwykłych opcji Wysiłku, możesz skorzystać z niego, by zwiększyć maksymalną grubość bariery, na każdy poziom zwiększając ją o dodatkowy metr. Akcja.

\section{C}

\textbf{Wezwanie Ducha}\index{Zdolności!Alfabetycznie!Wezwanie Ducha}\label{sec:Wezwanie Ducha} (6 punktów Intelektu) - Pod Twoim dotykiem, istota martwa nie dłużej niż 7 dni pojawia się jako (najwyraźniej fizyczny) duch, którego poziom jest taki sam, jak za życia. Przywołany duch istnieje maksymalnie przez dzień (lub mniej, jeśli osiąga coś ważnego dla niego w tym czasie), po którym znika i nie może pojawić się ponownie. 

Wezwany duch pamięta wszystko ,co wiedział za życia, i posiada większość swoich starych zdolności (ale niekoniecznie swój ekwipunek). Dodatkowo, uzyskuje on zdolność zostanie niematerialnym jako akcję (do minuty na raz). Wezwany Duch nie jest względem Ciebie w żaden sposób zobowiązany i nie musi zostać blisko Ciebie, by pozostać zamanifestowany. Akcja by rozpocząć. 

\textbf{Przysługa}\index{Zdolności!Alfabetycznie!Przysługa}\label{sec:Przysługa} (4 punkty Intelektu) - Strażnik, doktor, technik lub najęty bandyta zatrudniony przez lub stowarzyszony z przeciwnikiem jest sekretnie Twoim sprzymierzeńcem lub wisi Ci przysługę. Kiedy się na nią powołujesz, cel robi co może, żeby pomóc Ci (rozkuwa Cię, daje Ci nóż, zostawia drzwi celi otwarte) w sposób, który minimalizuje możliwość odkrycia, co zrobił. Ta zdolność to zadanie Intelektu poziomu 3. Każdy dodatkowy raz, gdy korzystasz z tej umiejętności, zadanie jest utrudnione o dodatkowy stopień. Trudność wraca do 3 po odpoczynku trwającym 10 godzin. Akcja.

\textbf{Wezwanie Międzywymiarowego Ducha}\index{Zdolności!Alfabetycznie!Wezwanie Międzywymiarowego Ducha}\label{sec:Wezwanie Międzywymiarowego Ducha} (6 punktów Intelektu) - przywołujesz istotę-ducha, który manifestuje się przez maksymalnie dzień (lub mniej, jeśli osiągnie przedtem coś ważnego) po którym znika i nie można go ponownie przywołać. Ten duch jest istotą 6 poziomu lub niższego, i może być materialna lub nie, zgodnie z własnym życzeniem (zmiana stanu wymaga akcji). Duch nie jest Tobie winny wdzięczności, i nie potrzebuje zostać blisko Ciebie, by pozostać zamanifestowanym. Akcja, by rozpocząć.

\textbf{Wezwanie Burzy}\index{Zdolności!Alfabetycznie!Wezwanie Burzy}\label{sec:Wezwanie Burzy} (7+ punktów Intelektu) - Jeśli jesteś na zewnątrz lub w pomieszczeniu, którego sufit sięga co najmniej 90 m, przywołujesz kotłujące się warstwy oświetlonych błyskawicami chmur burzowych do 460 m w promieniu na 10 minut. Podczas dnia, naturalne oświetlenie pod burzą jest zredukowane do niskiego. Kiedy burza grzmi, możesz wykorzystać akcję, by wysłac błyskawicę zz chmury, by zaatakować cel, który dostrzegasz, zadając mu 4 punkty obrażeń (możesz normalnie korzystać z Wysiłku na tych atakach). Trzy akcje by rozpocząć, akcja, by wezwać błyskawicę. 

\textbf{Wezwanie Roju}\index{Zdolności!Alfabetycznie!Wezwanie Roju}\label{sec:Wezwanie Roju} (4 punkty Intelektu) - Jeśli znajdujesz się w lokacji, gdzie mogą przybyć istoty związane z Twoją zdolnością Wpływ na Rój, możesz wezwać je na godzinę. Podczas tej godziny, istoty te robią, co im rozkażesz telepatycznie tak długo, jak są w dalekim zasięgu od Ciebie. Mogą one sie gromadzić i utrudniać akcje Twoich wrogów. Kiedy te istoty są w dalekim zasięgu, możesz rozmawiać z nimi telepatycznie i postrzegać świat poprzez ich zmysły. Akcja by rozpocząć. 

\textbf{Wezwanie Przez Czas}\index{Zdolności!Alfabetycznie!Wezwanie Przez Czas}\label{sec:Wezwanie Przez Czas} (6+ punktów Intelektu) - Przywołujesz osobę lub istotę do 3 poziomu z niedawnej przeszłości, i pojawia się ona obok Ciebie. Możesz wybrać istotę, z którą wcześniej wszedłeś w kontakt (nawet, jeśli jest teraz martwa) lub (nie więcej niż raz na dzień) możesz pozwolić MG na określenie istoty przypadkowo. Jeśli przywołujesz przypadkową istotę, masz 10 procent szans, że będzie to istota do 5 poziomu. Ta istota nie ma pamięci niczego przed byciem wezwaną przez Ciebie, ale mimo to może mówić i ma ogólną wiedzę, którą posiada istota jej typu. Istota wezwana przez czas wykonuje swoje akcje tak długo, jak sie na niej koncentrujesz, ale musisz wykorzystać swoją akcję w każdej turze, by wydać jej rozkazy, inaczej wróci do przeszłości.

W dodatku do normalnych opcji korzystania z Wysiłku, możesz skorzystać z wysiłku, by wezwać potężniejszą istotę: każdy poziom Wysiłku zwiększa poziom istoty o 1. Dla przykładu, zastosowanie poziomu Wysiłku, wzywa specyficzną istotę do 4 poziomu lub daje Ci 10 procent szans na wezwanie przypadkowej istoty do 6 poziomu. Akcja.

\textbf{Uspokojenie}\index{Zdolności!Alfabetycznie!Uspokojenie}\label{sec:Uspokojenie} (3 punkty Intelektu) - Poprzez dowcipy, piosenkę lub inną sztukę, powstrzymujesz jednego żywego przeciwnika przeciwko zaatakowaniem kogokolwiek lub czegokolwiek przez jedną rundę. Akcja.

\textbf{Uspokojenie Nieznajomego}\index{Zdolności!Alfabetycznie!Uspokojenie Nieznajomego}\label{sec:Uspokojenie Nieznajomego} (2+ punkty Intelektu) - Możesz sprawić, że jedna inteligentna istota pozostaje spokojna w momencie, gdy mówisz. Ta istota nie musi mówić Twoim językiem, ale musi być w stanie Cię ujrzeć. Pozostaje spokojna tak długo, jak skupiasz na sobie jej uwagę i nie jest zaatakowana lub w inny sposób w niebezpiecznej sytuacji. W dodatku do normalnych opcji Wysiłku, możesz go zastosować, by uspokoić dodatkową istotę sprzymierzoną z Twoim pierwszym celem - jedna istota na poziom Wysiłku. Akcja.

\textbf{Zręczny Wojownik}\index{Zdolności!Alfabetycznie!Zręczny Wojownik}\label{sec:Zręczny Wojownik} - Twoje ataki zadają o 1 punkt obrażeń więcej. Umożliwienie. 

{\textbf{Zachwyt lub Inspiracja}\index{Zdolności!Alfabetycznie!Zachwyt lub Inspiracja}\label{sec:Zachwyt lub Inspiracja} - Możesz zastosować tę zdolność na dwa sposoby. Albo Twoje słowa utrzymują uwagę wszystkich BN-ów, którzy je słyszą tak długo, jak mówisz, albo Twoje słowa inspirują BN-ów którzy je słyszą, tak, że funkcjonują przez godzinę, jakby posiadali o poziom więcej. W dowolnym wypadku, wybierasz, którzy BN-i dostają się pod wpływ tej zdolności. Jeśli ktoś w tłumie zostanie zaatakowany, kiedy próbujesz do niego przemówić, tracisz uwagę tłumu. Akcja, by rozpocząć. 

\textbf{Zachwyt Światła Gwiazd}\index{Zdolności!Alfabetycznie!Zachwyt Światła Gwiazd}\label{sec:Zachwyt Światła Gwiazd} - Tak długo, jak mówisz, utrzymujesz uwagę wszystkich 2 poziomowych lub słabszych BN-ów, który Cię słyszą. Jeśli posiadasz także zdolność Zauroczenie, możesz w podobny sposób wpłynąć na BN-ów poziomu 3-go. Akcja, by rozpocząć.

\textbf{Surfer Aut}\index{Zdolności!Alfabetycznie!Surfer Aut}\label{sec:Surfer Aut} - możesz wstać lub porusząc się w poruszającym się pojeździe (np: jego suficie, otwartych drzwiach, masce itp.) z duża szansą, ze nie spadniesz. Jesli pojazd nie zrobi nagłego zwrotu, zatrzyma isę nagle lub w inny sposób nie wykona jakiegoś ekstremalnego manewru, wstanie lub poruszanie się po takim pojeździe to dla Ciebie zadanie rutynowe. Jeśli pojazd wykonuje jakieś ekstremalne manewry, jak te opisane wyżej, wszystkie zadania, by pozostać na powierzchni pojazdu są ułatwione. Umożliwienie. 

\textbf{Ostrożny Rzut}\index{Zdolności!Alfabetycznie!Ostrożny Rzut}\label{sec:Ostrożny Rzut} - Jesteś wyszkolony we wszystkich atakach bronią rzucaną. Umożliwienie.

\textbf{Ostrożny Strzał}\index{Zdolności!Alfabetycznie!Ostrożny Strzał}\label{sec:Ostrożny Strzał}  - Możesz wydać punkty z Puli Szybkości lub z Puli Intelektu, by zwiększać Wysiłkiem obrażenia broni palnej. Każdy poziom wysiłku dodaje 3 punkty obtażeń do udanego ataku, a jeśli spędzisz swoją turę na celowanie, każdy poziom Wysiłku zamiast tego dodaje 5 punktów obrażeń do udanego ataku. Umożliwienie.

\textbf{Rzuć Iluzję}\index{Zdolności!Alfabetycznie!Rzuć Iluzję}\label{sec:Rzuć Iluzję} - Możesz zwiększyć zasięg w którym możesz tworzyć i podtrzymywać swoje iluzje bliskiego zasięgu (np: ze zdolności Mniejsza Iluzja) do dowolnego miejsca w średnim zasięgu, które możesz dostrzec. Umożliwienie. 

\textbf{Przerażenie}\index{Zdolności!Alfabetycznie!Przerażenie}\label{sec:Przerażenie} (4 punkty Intelektu) - Przerażasz swojego oponenta w dalekim zasięgu, który rozumie mowę (choć nie musi Twojego języka) tak bardzo, że traci on swą następną akcję i na resztę swoich akcji w ciągu 1 minuty jego zadania są utrudnione. Każdy dodatkowy raz, gdy próbujesz wykorzystać tę zdolność na tym samym wrogu, musisz zastosować o poziom Wysiłku więcej, niż przy ostatniej próbie. Akcja. 

\textbf{Talent Celebryty}\index{Zdolności!Alfabetycznie!Talent Celebryty}\label{sec:Talent Celebryty} - jesteś wyszkolony w dwóch z poniższych umiejętnościach: pisaniu, dziennikarstwie, danym rodzaju sztuki, danym sporcie, szahach, komunikacji naukowej, aktorstwie, prezentacji newsów lub innej powiązanej zdolności niebojowej, która uczyniła z Ciebie celebrytę. Umożliwienie. 

\textbf{Centrum Uwagi}\index{Zdolności!Alfabetycznie!Centrum Uwagi}\label{sec:Centrum Uwagi} (5 punktów Intelektu) - Dosłowny (lub metaforyczny, w zależności od settingu) promień czystej światłości zstępuje z Niebios i Cię okala. Wszystkie istoty, który wybierzesz w swoim bliskim zasięgu padają na kolana i tracą swą następną akcję. Cele tej mocy nie mogą się bronić i są traktowane jako bezsilne. Akcja.

\textbf{Komnata Snów}\index{Zdolności!Alfabetycznie!Komnata Snów}\label{sec:Komnata Snów} (8 punktów Intelektu) - Ty i Twoi sprzymierzeńcy możecie wkroczyć w komnat snów, udekorowaną jak sobie tego zażyczysz, która zawiera pewną liczbę drzwi. Prowadzą one do lokalizacji, które odwiedziłeś lub które znasz całkiem dobrze. Przejście przez jedne z tych drzwi przenosi Cię do pożądanej lokacji. Jest to zadanie trudności 2 bazujące na Intelekcie (zadanie może być trudniejsze, jeśli lokacja jest chroniona magicznie). Akcja by wejść do komnaty snów; akcja, by przejść przez wrota w komnacie.

\textbf{Zmiana Paradygmatu}\index{Zdolności!Alfabetycznie!Zmiana Paradygmatu}\label{sec:Zmiana Paradygmatu} (6+ punktów Intelektu) - Zmieniasz światopogląd istoty, z którą spędzasz przynajmniej rundę na rozmowie (jeśli jest ona w stanie Cię zrozumieć). Ta istota zmienia swoje zdanie odnośnie ważnego poglądu lub wierzenia, co może być czymś tak prostym jak zmiana chęci zamordowania Ciebie na pomoc Ci, lub być czymś dziwniejszym. Efekt trwa przynajmniej przez 10 minut, ale może trwać to dłużej, jeśli istota nie była wcześniej Twoim wrogiem. W tym czasie, istota podejmuje akcje zgodnie z mądrością, którą się z nią podzieliłeś. Cel musi być na poziomie 2 lub niższym. W dodatku do normalnych opcji korzystania z Wysiłku, możesz z niego skorzystać, aby zwiększyć maksymalny poziom celu (o 1 więcej na każdy poziom Wysiłku). Akcja, by rozpocząć.

\textbf{Naładowanie}\index{Zdolności!Alfabetycznie!Naładowanie}\label{sec:Naładowanie} (1+ punktów Intelektu) - Możesz naładować artefakt lub inne urządzenie (ale nie cypher), tak, by skorzystać z niego raz. Koszt to 1 punkt Intelektu plus 1 punkt na poziom urządzenia. Akcja.

\textbf{Naładowanie Broni}\index{Zdolności!Alfabetycznie!Naładowanie Broni}\label{sec:Naładowanie Broni} (2+ punkty Intelektu) - Jako część ataku Twoją magiczną bronią, ładujesz ją magiczną mocą, zadając dodatkowe 2 punkty obrażeń od energii. Jeśli wykonujesz więcej niż 1 atak w swojej turze, wybierasz, czy chcesz wydać punkty na tą zdolność przed wykonaniem każdego z ataków. Umożliwienie.

\textbf{Szarża Hordy}\index{Zdolności!Alfabetycznie!Szarża Hordy}\label{sec:Szarża Hordy} (7 punktów Mocy) - Ty i dwóch lub więcej z Twoich kompanów obo kCiebie działacie jak jedna istota, by wykonać atak szarżą. Kiedy to robicie, wszyscy poruszacie się na średni dystans, w trakcie czego atakujecie wszystko w Waszym bliskim zasięgu na swojej drodze, z atutem do ataku. Cele tego ataku otrzymują dodatkowe 3 punkty obrażeń i są wywrócone. Akcja.

\textbf{Zauroczenie Maszyny}\index{Zdolności!Alfabetycznie!Zauroczenie Maszyny}\label{sec:Zauroczenie Maszyny} (2 punkty Intelektu) - Przekonujesz nieinteligentną maszynę, by Cię ``lubiła''. Maszyna, która Cię lubi, ma szanszę mniejszą o 50 procent, by funkcjonować w sposób, który mógłby Cię zranić. Tak więc, jeśli wróg chce zdetonować bombę blisko Ciebie, a jest ona kontrolowane detonatorem, który Cię lubi, istnieje 50 procent szans, że bomba nie wybuchnie. Akcja, by rozpocząć. 

\textbf{Chmura Ochronna}\index{Zdolności!Alfabetycznie!Chmura Ochronna}\label{sec:Chmura Ochronna} (5 punktów Intelektu) - Sprawiasz, że małe obiekty z Twojego otoczenia (kamienie, zepsute przedmioty, chmury pyłu itp) obracają się wokół Ciebie przed 10 minut, co daje Ci +2 do Pancerza. Akcja, by rozpocząć. 

\textbf{Umysł-Twierdza}\index{Zdolności!Alfabetycznie!Umysł-Twierdza}\label{sec:Umysł-Twierdza} - Jesteś wyszkolony w akcjach obrony Intelektu i masz Pancerz +2 do ataków, które dotyczą Twojej puli Intelektu (co normalnie ignoruje Pancerz). Umożliwienie. 

\textbf{Zamglenie Pamięci Krótkotrwałej}\index{Zdolności!Alfabetycznie!Zamglenie Pamięci Krótkotrwałej}\label{sec:Zamglenie Pamięci Krótkotrwałej} (3 punkty Intelektu) - Jeśli wchodzisz w interakcje lub studiujesz cel przez przynajmniej rundę, zyskujesz świadomość, jak działa jego umysł, co możesz wykorzystać przeciwko niemu w najgorszy możliwy sposób. Możesz spróbować go skonfundować i sprawić, że zapomni, co właśnie zaszło. W przypadku sukcesu, usuwasz do ostatnich 5 minut z jego pamięci. Akcja by przygotować, akcja by rozpocząć.
 
\textbf{Ulepszenie Maszyny}\index{Zdolności!Alfabetycznie!Ulepszenie Maszyny}\label{sec:Ulepszenie Maszyny} (2 punkty Intelektu) - POlepszasz moc lub funkcjonowanie maszyny tak, że działa na 1 poziomie więcej niż zwyklep rzez jedną godzinę. Akcja, by rozpocząć. 

\textbf{Obliczenia Bitewne}\index{Zdolności!Alfabetycznie!Obliczenia Bitewne}\label{sec:Obliczenia Bitewne} - Podczas walki, Twój mózg przełącza się na tryb bojowy, gdzie wszystkie potencja;ne ataki, które możesz wykonać, pojawiają się jako wektory w Twoim umyśle, co zawsze zapewnia najlepszą opcję. Twoje ataki są ułatwione. Umozliwienie.

\textbf{Emisja Zimna}\index{Zdolności!Alfabetycznie!Emisja Zimna}\label{sec:Emisja Zimna} (5+ punktów Intelektu) - Emitujesz zimno we wszystkich kierunkach w średnim zasięgu. Wszyscy w obszarze Twojej emisji (z wyjątkiem Ciebie) otrzymują 5 punktów obrażeń. Jeśli zastosujesz WYsiłek, by zwiększyć obrażenia, zamiast ułatwić zadanie, zadajesz 2 dodatkowe punkty obrażeń na poziom Wysiłku (zamiast 3 punktów); cele w obszarze otrzymują 1 punkt obrażeń, nawet jeśli nie uda Ci isę rzut na atak. Akcja.

\textbf{Kolos}\index{Zdolności!Alfabetycznie!Kolos}\label{sec:Kolos} - Kiedy korzystasz z Wzrostu, możesz wybrać wzrost do 18 m wysokości. Kiedy to czynisz, zadajesz dodatkowe 2 punkty obrażeń atakami wręcz (plus wszelki dodatek ze zdolności Wielki). Na każdy poziom Wysiłku, który zastosujesz, Twój wzrost zwiększa się o 3 metry, i dodajesz 1 punkt więcej do swojej Puli Mocy. Tak więc, za pierwszym razem gdy zastosujesz Wzrost po 10-godzinnym odpoczynku, jeśli zastosujesz 2 poziomy Wysiłku, Twój wzrost wynosić będzie 24 metry i dodasz 17 tymczasowych punktów do swojej Puli Mocy. Umożliwienie.

\textbf{Wyzwanie Bojowe}\index{Zdolności!Alfabetycznie!Wyzwanie Bojowe}\label{sec:Wyzwanie Bojowe} - Wszystkie zadania, których celem jest ściągnięcie na Ciebi ataków (i odciągnięcie ich od innych) są ułatwione o dwa kroki. Umożliwienie.

\textbf{Zdolności Bojowe}\index{Zdolności!Alfabetycznie!Zdolności Bojowe}\label{sec:Zdolności Bojowe} - Dodajesz +1 obrażeń do jednego typu ataku z bronią Twojego wyboru: atak bronią wręcz lub dystansowy atak bronią. Umożliwienie.

\textbf{Rozkaz}\index{Zdolności!Alfabetycznie!Rozkaz}\label{sec:Rozkaz} (3 punkty Intelektu) - Poprzez czystą moc woli, możesz wydać prosty rozkaz danej istocie, która następnie wykonuje go przez następną akcję. Istota musi być w średnim zasięgu i być w stanie Cię zrozumieć. Rozkaz nie może być bezpośrednim zagrożeniem dla istoty lub jej towarzyszy, więc ``Popełnij samobójstwo'' nie zadziała, ale ``Ucieknij'' już tak. Dodatkowo, rozkaz może wymagać od istoty pojedyńczej akcji, więc ``Otwórz drzwi'' może zadziałać, ale ``Otwórz drzwi i przebiegnij przez nie'' już nie. Istota, której rozkazano, może się bronić normalnie i odpowiedzieć na atak atakiem, jeśli zostanie zaatakowana. Jeśli posiadasz inną zdolność, którą możesz wydać rozkaz istocie, możesz efektem Rozkazu objać dwie istoty na raz (jest to ``podstawowy efekt'' obydwu zdolności) korzystając z dowolnej z tych zdolności. Akcja. 

\textbf{Rozkazywanie Bestiom}\index{Zdolności!Alfabetycznie!Rozkazywanie Bestiom}\label{sec:Rozkazywanie Bestiom} (3+ punkty Intelektu) - Możesz rozkazywać nieagresywnej bestii nie będącej człowiekiem (jak np: istota, którą uspokoiłeś poprzez Ukojenie Dzikiego) do 3 poziomu w średnim zasięgu. Jeśli osiągniesz sukces, przez następną minutę bestia słucha Twoich werbalnych komend najlepiej jak rozumie i może. GM decyduje, co liczy się jako nieludzka bestia, ale jeśli nie masz do czynienia z jakimś oszustwem, powinieneś wiedzieć, czy możesz wpłynąć na istotę zanim skorzystasz z tej zdolności. Obcy, istoty międzywymiarowe, bardzo inteligentne istoty i roboty nigdy się w to nie wliczają. 

W dodatku do normalnych opcji korzystania z Wysiłku, możesz z niego skorzystać, by zwiększyć maksymalny poziom celu. Tak więc, by rozkazywać bestii 5 poziomu (2 poziomy ponad normalny limit), musisz zastosować dwa poziomy Wysiłku. Akcja, by rozpocząć. 

\textbf{Rozkazywanie Maszynom}\index{Zdolności!Alfabetycznie!Rozkazywanie Maszynom}\label{sec:Rozkazywanie Maszynom} (4 punkty Intelektu) - Jeśli zauroczyłeś nieinteligentną maszynę lub rozmawiałeś telepatycznie z inteligentną maszyną, możesz spróbować wydać jej rozkaz, który wysłucha przez jedną akcję najlepiej, jak potrafi. (Jeśli korzystasz z tej zdolności, by rozkazać inteligentnej maszynie, najpewniej będzie po fakcie nastawiona do Ciebie negatywnie i agresywnie). Akcja.

\textbf{Kontrola Metalu}\index{Zdolności!Alfabetycznie!Kontrola Metalu}\label{sec:Kontrola Metalu}  (5 punktów Intelektu) - Zmieniasz kształt metalicznego obiektu wedle swojej woli. Przedmiot musi być w zasięgu Twego wzroku i w średnim zasięgu, Jego masa nie może być większa od Twojej własnej. Możesz wpłynąć na więcej niż jeden przedmiot na raz, tak długo, jak ich połączona masa nie przekracza limitu Twojej wagi. Możesz połączyć rożne przedmioty w jeden. Możesz wykorzystać tę moc, by zniszczyć metalowy obiekt (jak w zdolności Niszczenie Metalu) lub możesz stworzyć nowy obiekt (bardzo niezręcznie, chyba, że masz odpowiednie umiejętności rzemieślnicze). Możesz wtedy przemieścić nowy obiekt gdziekolwiek w zasięgu zdolności. Dla przykładu, możesz wziąć parę metalowych tarcz, połączyć je razem, i użyć, by zablokować drzwi. Możesz użyć tej zdolności, by wykonać atak - atakując wroga jego własną zbroją, zmieniając metalowy przedmiot w odłamki, którymi rzucasz w środek pola walki itp. - przeciwko jednemu celowi w średnim zasięgu. Niezależnie od formy ataku, jest to akcja Intelektu zadająca 7 punktów obrażeń. Akcja. 

\textbf{Rozkazywanie Duchom}\index{Zdolności!Alfabetycznie!Rozkazywanie Duchom}\label{sec:Rozkazywanie Duchom} (3 punkty Intelektu) - Możesz rozkazać duchowi lub nieumarłemu do 5 poziomu w średnim zasięgu. Jeśli Ci się uda, cel nie może zaatakować Cię przez minutę, w której to minucie wykonuje Twoje instrukcje, jeśli CIę słyszy i rozumie. Akcja, by rozpocząć.

\textbf{Komunikacja}\index{Zdolności!Alfabetycznie!Komunikacja}\label{sec:Komunikacja}  (2 punkty Intelektu) - Możesz przekazać podstawowy koncept istocie, która normalnie nie może mówić lub rozumieć mowy. Ta istota może także przekazać Ci bardzo prostą odpowiedź na proste pytanie. Akcja.

\textbf{Lokalny Aktywista}\index{Zdolności!Alfabetycznie!Lokalny Aktywista}\label{sec:Lokalny Aktywista} Kiedy rozmawiasz z innymi w swojej społeczności, jesteś wyszkolony w perswazji i zastraszaniu odnośnie tematów, które są bezpośrednio powiązane ze społecznością. Umożliwienie.

\textbf{Wiedza o Społeczności}\index{Zdolności!Alfabetycznie!Wiedza o Społeczności}\label{sec:Wiedza o Społeczności} (2 punkty Intelektu) - Jeśli jesteś zaangażowany w społeczność i spędziłeś przynajmniej ostatnie parę miesięcy żyjąc w niej, możesz się dowiedzieć o rożnych rzeczach na różne sposoby. Czasami Twoje kontakty podsuwają Ci informacje. Innymi razy, wyciagasz wniosk iz tego, co słyszysz i widzisz. Kiedy korzystasz z tej zdolności, możesz zadać MG jedno pytanie o społeczności i otrzymać krótką odpowiedź. Akcja.

\textbf{Programowanie}\index{Zdolności!Alfabetycznie!Programowanie}\label{sec:Programowanie} - Jesteś wyszkolony w używaniu (i nadużywaniu) oprogramowania, znasz jeden lub więcej języków programowania dostatecznie dobrze, by pisać własne proste programy, znasz się także na korzystaniu z Internetu. Umożliwienie. 

\textbf{Wybuch}\index{Zdolności!Alfabetycznie!Wybuch}\label{sec:Wybuch} (7 punktów Intelektu) - Wywołujesz wybuchowy impuls, który wybucha w punkcie, który wybierasz w długim zasięgu. Ten impuls rozszerza się na średni zasięg we wszystkich kierunkach, zadając 5 punktów obrażeń wszystkiemu w obszarze. Nawet jeśli nie uda Ci się rzut na atak, cele w obszarze otrzymują 1 punkt obrażeń. Akcja.

\textbf{Promień Odrzucający}\index{Zdolności!Alfabetycznie!Promień Odrzucający}\label{sec:Promień Odrzucający} (2 punkty Intelektu) - Wyzwalasz promień czystej mocy, który wbija się w istotę w średnim zasięgu, zadając 5 punktów obrażeń i odrzucając ją do tyłu na bliski zasięg. Akcja.

\textbf{Sprawny Oszust}\index{Zdolności!Alfabetycznie!Sprawny Oszust}\label{sec:Sprawny Oszust} - Kiedy hakujesz system komputerowy, oszukujesz kogoś, dokonujesz kradzieży kieszonkowej, ukrywasz coś przed strażnikiem itp, zyskujesz atut na tym zadaniu. Umożliwienie.

\textbf{Konfundujące Nonsensy}\index{Zdolności!Alfabetycznie!Konfundujące Nonsensy}\label{sec:Konfundujące Nonsensy} (4 punkty Intelektu) - Wyrzucasz z siebie strumień nonsensów, by rozproszyć wroga w bliskim zasięgu. Po udanym rzucie na Intelekt, Twoje rzuty obronne przeciwko następnemu atakowi tej istoty przed końcem Twojej następnej rundy są ułatwione. Action.

\textbf{Skonfunduj Wroga}\index{Zdolności!Alfabetycznie!Skonfunduj Wroga}\label{sec:Skonfunduj Wroga} (4 punkty Intelektu) - Poprzez sprytną zmyłkę, w co wlicza się zręczne wykorzystanie płaszcza, unik we właściwym momencie lub podobną strategię, możesz spróbować przekierować fizyczny atak wręcz, który inaczej by CIę zrnaił. Kiedy to czynisz, przeniesiony atak uderza w inną istotę Twojego wyboru w bliskim zasięgu (zarówno od Ciebie, jak i od atakującego wroga) Ta zdolność to zadanie Intelektu 2 poziomu. Umożliwienie. 

\textbf{Przywołanie}\index{Zdolności!Alfabetycznie!Przywołanie}\label{sec:Przywołanie} (7 punktów Intelektu) - Powołujesz do życia, z powietrza, istotę 5 poziomu, którą wcześniej spotkałeś. Ta istota zostaje na jedną minutę, a potem wraca do domu. Kiedy jest obecna, działa ona jak ją poinstruujesz, ale nie wymaga to z Twojej strony akcji. Akcja.

\textbf{Kontakty}\index{Zdolności!Alfabetycznie!Kontakty}\label{sec:Kontakty}  - Znasz ludzi, którzy mogą zrobić coś dla Ciebie - nie tylko respektowani ludzie w pozycjach władzy, ale także różnorodnych hakerów i przestępców. Ci ludzie nie są koniecznie Twoimi przyjaciółmi i mogą nie byc wiarygodni, ale wiszą Ci przysługę. Powinieneś razem z MG określić szczegóły tych znajomości. Umożliwienie. 

\textbf{Człowiek-Guma}\index{Zdolności!Alfabetycznie!Człowiek-Guma}\label{sec:Człowiek-Guma} (2 punkty Szybkości) - Możesz się wyzwolić z okowów lub przecisnąć przez wąskie dziury. Jesteś wyszkolony w ucieczce z więzów. Kiedy korzystasz z akcji, by uciec z więzów lub przecisnąć się przez wąską przestrzeń, możesz natychmiastowo wykonać kolejną akcję. Możesz skorzystać z tej dodatkowej akcji tylko po to, by się poruszyć. Umożliwienie. 

\textbf{Kontrola Bitewna}\index{Zdolności!Alfabetycznie!Kontrola Bitewna}\label{sec:Kontrola Bitewna} (1 punkt Mocy) - Ten atak wręcz zadaje o 1 punkt obrażeń mniej niż normalnie, i niezależnie od tego, czy trafisz czy chybisz, manewrujesz go w pozycję, której pragniesz w bliskim zasięgu. Akcja.

\textbf{Kontrola Maszyny}\index{Zdolności!Alfabetycznie!Kontrola Maszyny}\label{sec:Kontrola Maszyny} (6 punktów Intelektu) - Możesz spróbować kontrolować dowolną maszynę, inteligentną lub nie, w średnim zasięgu przez 10 minut. Akcja.

\textbf{Kontrola Dzikiej Bestii}\index{Zdolności!Alfabetycznie!Kontrola Dzikiej Bestii}\label{sec:Kontrola Dzikiej Bestii}  (6 punktów Intelektu) - Możesz kontrolować spokojną, nieludzką bestię w zasięgu 9 metrów. Kontrolujesz ją tak długo, jak skupiasz na niej swoją uwagę, poświęcając na to swoją turę w każdej rundzie. Ostateczna decyzja czy możesz wpłynąć na daną istotę zależy od MG, ale jeśli nie ma jakiegoś rodzaju oszustwa, to jesteś w stanie stwierdzić przed użyciem tej zdolności, czy istota jest na nią podatna. Obcy, byty międzywymiarowe, bardzo inteligentne istoty i roboty nigdy nie są objęte tą zdolnością. Akcja.

\textbf{Kontrola Roju}\index{Zdolności!Alfabetycznie!Kontrola Roju}\label{sec:Kontrola Roju} (2 punkty Intelektu) - Przywołujesz rój istot (powiązanych z Twoją zdolnością Wpływ na Rój) w średnim zasięgu i kontrolujesz je przez 10 minut. Nawet zwykłe robaki (poziom 0) w odpowiednio wielkim roju mogą pokryć daną istotę i utrudnić jej akcje. Akcja, by rozpocząć. 

\textbf{Kontrola Pogody}\index{Zdolności!Alfabetycznie!Kontrola Pogody}\label{sec:Kontrola Pogody} (10 punktów Intelektu) - Zmieniasz pogodę w swojej okolicy. Jeśli używasz tej mocy w budynku, tworzy ona mniejsze efekty, takie jak mgła, mniejsze zmiany temperatury itp. Jeśli wykonujesz ją na zewnątrz, możesz przywołać deszcz, mglę, śnieg, wiatr lub inną normalną (niezbyt ekstremalną) pogodę. Ta zmiana pogody trwa przez naturalny czas, tak więc burza mogłaby trwać przez godzinę, mgła przez 2 lub 3 godziny, a śnieg pare godzin (lub 10 minut, jeśli nie ma zimy). Przez pierwsze 10 minut po aktywowaniu tej zdolności, możesz stworzyć bardziej dramatyczne i konkretne efekty, takie jak pioruny, wielkie wichry, huragany, itp. Te efekty muszą wystąpić w zasięGu 300 metrów od Twojej lokacji. Musisz spędzić turę, koncentrując się na stworzeniu efektu lub podtrzymaniu go w nowej rundzie. Te efekty zadają 6 punktów obrażeń w każdej rundzie. Jeśli posiadasz tę zdolność z innego źródła, jej koszt wynosi 7 punktów Intelektu zamiast 10. Jeśli już posiadasz zdolność Przywołanie Burzy, możesz ją natychmiastowo zamienić na inną zdolność tego samego poziomu. Akcja, by rozpocząć.  

\textbf{Kontrolowana Przemiana}\index{Zdolności!Alfabetycznie!Kontrolowana Przemiana}\label{sec:Kontrolowana Przemiana} - Możesz spróbować skorzystać ze swojej Likantropii by zmienić się w formę likantropa na dowolnej nocy, której zapragniesz (zadanie Intelektu trudności 3). Wszelkie transformacje z ykorzystaniem tej zdolności są dodatkowe do 5 nocy na miesiąc, gdy musisz się zmienić. Akcja by się zmienić. 

\textbf{Kontrolowany Upadek}\index{Zdolności!Alfabetycznie!Kontrolowany Upadek}\label{sec:Kontrolowany Upadek} - Gdy upadasz kiedy masz możliwość wykonania akcji i masz powierzchnię pionową w zasięgu ręki, możesz spróbować spowolnić swój upadek. Wykonaj test Szybkości o trudności 1 na każde 6 m Twojego upadku. Przy sukcesie, otrzymujesz połowę obrażeń z Upadku. Jeśli zredukujesz trudność testu do 0, nie odnosisz żadnych obrażeń. Umożliwienie. 

\textbf{Skoordynowany Wysiłek}\index{Zdolności!Alfabetycznie!Skoordynowany Wysiłek}\label{sec:Skoordynowany Wysiłek} (3 punkty Intelektu) - Kiedy Ty i Twój duplikat z mocy Kopia atakujecie tę samą istotę, możesz uczynić tę akcję jednym rzutem na atak z atutem. Jeśli trafisz, zadajesz obrażenia obydwu ataków i traktujesz je jakby były obrażeniami z jednego ataku w celach określenia ile punktów pochłania Pancerz. Akcja.

\textbf{Skopiuj Moc}\index{Zdolności!Alfabetycznie!Skopiuj Moc}\label{sec:Skopiuj Moc} (2+ punkty Intelektu) - Możesz skopiować czyjąś zdolność na godzinę, korzystając z niej, jakby była dla Ciebie naturalna. W ciągu poprzedniej godziny musiałeś dotknąć istotę, której moc pragniesz skopiować (rzut na atak) i musisz być świadkiem, jak korzysta ona z tej zdolności. Wybierz zdolność, którą chcesz skopiować, a MG wybiera odpowiednią zdolność niskiego poziomu, która najbardziej przypomina tę moc. Dla przykładu, jeśli walczysz z superzłoczyńcą, który tworzy promienie mocy, jeśli skopiujesz tę zdolność, uzyskujesz niskopoziomową zdolność, która tworzy promienie mocy. 

W dodatku do kosztów Skopiuj Moc, musisz też zapłacić koszt Mocy, Szybkości lub Intelektu (jeśli jakikolwiek) na odpowiednią zdolność wyboru MG. Dla przykładu, jeśli chcesz skopiować moc superzłoczyńcy, którą są promienie mocy, GM najpewniej zadecyduje, że otrzymujesz zdolność Pocisk. tak wiec płacisz 2 punkty intelektu za Skopiuj Moc i 1 punkt Intelektu za skorzystanie z Pocisku. Możesz mieć jedną kopią zdolności w danym czasie - skopiowanie innej kończy możliwość korzystania z wcześniejszej mocy, którą skopiowałeś przy pomocy tej zdolności.

Skopiuj Moc nie kopiuje zdolności, które permanentnie dodają punkty do Pul, takich jak Ulepszone Ciało. 

W dodatku do normalnych opcji korzystania z Wysiłku, możesz wykorzystać jego poziomy, by skopiować moc, którą widziałeś więcej niz godzinę temu - każdy poziom Wysiłku użyty w taki sposób pozwala Ci się ``cofnąć'' o dodatkową godzinę. Akcja.

\textbf{Okiełznanie Niebezpieczeństwa}\index{Zdolności!Alfabetycznie!Okiełznanie Niebezpieczeństwa}\label{sec:Okiełznanie Niebezpieczeństwa} (4 punkty Intelektu) - Negujesz źródło potencjalnego niebezpieczeństwa powiązanego z jedną istotą lub obiektem w bliskim zasięgu na 1 minutę (zamiast na 1 rundę, jak z Negacją Zagrożenia). Może to być broń lub urządzenie trzymane przez kogoś, naturalna zdolność istoty, lub pułapka aktywowana płytką w podłodze. Możesz także spróbować skontroać akcję (taką jak poruszanie się lub konwencjonalny atak bronią, pazurem itp.). Akcja.

(Korzystanie z Okiełznania Niebezpieczeństwa jest zazwyczaj kwestią stosowania szybkiego myślenia i refleksu w kontakcie z bezpośrednim zagrożeniem. Ta zdolność nie polega na nadprzyrodzonych cechach, lecz na praktycznym stosowaniu zwykłych akcji.)

\textbf{Środki Zaradcze}\index{Zdolności!Alfabetycznie!Środki Zaradcze}\label{sec:Środki Zaradcze} (4 punkty Intelektu) - Natychmiast kończysz jeden trwający efekt (taki jak efekt stworzony przez zdolność postaci) w bliskim zasięgu. Alternatywnie, możesz użyć tej zdolności jak akcji obrony, by skontrować nadchodzącą zdolność, której celem jesteś ty, lub możesz skontrować każde urządzenie lub efekt każdego urządzenia na 1k6 rund. Musisz dotknąć efektu lub urządzenia, by je skontrować. Akcja.

\textbf{Odwaga}\index{Zdolności!Alfabetycznie!Odwaga}\label{sec:Odwaga} - Jesteś wyszkolony w Obronie Intelektu i rzutach na inicjatywę. Umożliwienie. 

\textbf{Rzemieślnik}\index{Zdolności!Alfabetycznie!Rzemieślnik}\label{sec:Rzemieślnik} - jesteś wyszkolony w tworzeniu dwóch rodzajów przedmiotów. Umożliwienie. 

\textbf{Stworzenie}\index{Zdolności!Alfabetycznie!Stworzenie}\label{sec:Stworzenie} (7 punktów Intelektu) - Tworzysz coś z niczego. Możesz stworzyć dowolny przedmiot swojego wyboru, który normalnie miałby trudność stworzenia 5 lub mniej (korzystając z zasad tworzenia przedmiotów). Kiedy już go stworzysz, przedmiot ten istnieje przez liczbę godzin równą 6 minus trudność jego stworzenia. Więc, stworzenie zestawu ciężkich kajdan (trudność 5) trwałoby przez godzinę. Akcja.

\textbf{Stworzenie Śmiertelnej Trucizny}\index{Zdolności!Alfabetycznie!Stworzenie Śmiertelnej Trucizny}\label{sec:Stworzenie Śmiertelnej Trucizny} (3+ punkty Intelektu) - Tworzysz jedną dawkę trucizny 2 poziomu, która albo zadaje 5 punktów obrażeń lub utrudnia akcje zatrutej istoty na 10 minut (ty wybierasz za każdym razem, gdy tworzysz truciznę). Możesz zaaplikować truciznę na broń, jedzenie lub picie jako część akcji tworzenia jej. W dodatku do zwykłych opcji korzystania z Wysiłku, możesz wybrać zwiększenie poziomu trucizny - każdy poziom Wysiłku wykorzystany w ten sposób zwiększa poziom trucizny o 2. Jeśli się z niej nie skorzysta, trucizna traci swoją moc po godzinie. Akcja. 

\textbf{Stworzenie Wody}\index{Zdolności!Alfabetycznie!Stworzenie Wody}\label{sec:Stworzenie Wody} (2 punkty Intelektu) - Sprawiasz, że woda tryska z gruntu z miejsca, które widzisz. Woda płynie z niego przez minutę, tworząc około 4 litrów cieczy w momencie, gdy przestaje. Akcja, by rozpocząć. 

\textbf{Badanie Istoty}\index{Zdolności!Alfabetycznie!Badanie Istoty}\label{sec:Badanie Istoty} (3 punkty Intelektu) - Kiedy badasz nieludzką istotę, możesz zadać MG jedno pytanie, by pozyskać ogólne informacje o jej poziomie, zdolnościach, tym, co je, co ją motywuje jakie są jej słabości (jeśli jakiekolwiek), jak można ją naprawić i tym podobne. Stosuje się to do trudnych lub dziwnych istot, które są poza zasięgiem zwykłych zdolności. Akcja. 

\textbf{Futrzasty Kompan}\index{Zdolności!Alfabetycznie!Futrzasty Kompan}\label{sec:Futrzasty Kompan} - Istota 1 poziomu towarzyszy Ci i słucha Twoich instrukcji. Ta istota jest nie większa niż duży kot (9 kg) i jest zazwyczaj jakimś udomowionym gatunkiem. Powinniście z MG określić szczególy tej istoty, i najpewniej będziesz za nią wykonywał rzuty w walce. Futrzasty Kompan działa w Twojej turze. Jako istota 1 poziomu, posiada stopień trudności 3, 3 punkty zdrowia i zadaje 1 punkt obrażeń. Jej ruch bazuje na jej typie istoty (ptak, istota pływająca itp.). Jeśli Twój futrzasty Kompan zginie, możesz przeszukać miasto lub dzicz przez 1d6 dni, żeby znaleźć nowego. Umożliwienie. 

\textbf{Kontrola Tłumu}\index{Zdolności!Alfabetycznie!Kontrola Tłumu}\label{sec:Kontrola Tłumu} (6+ punktów Intelektu) - Kontrolujesz akcje do 5 istot w średnim zasięgu. Ten efekt trwa przez jedną minutę. Wszystkie cele muszą być na 2 poziomie lub mniejszym. Twoja kontrola ogranicza się do prostych werbalnych komend jak ``Stop'', ``Uciekaj'', ``Śledź strażnika'', ``Patrz tam'' lub ``Zejdź mi z drogi''. Wszystkie cele odpowiadają na komendę, chyba, że zaznaczysz inaczej. W dodatku do zwykłych opcji korzystania z Wysiłku, możesz z niego skorzystać, by zwiększyć maksymalny poziom celów lub wpłynąć na dodatkowe 5 ludzi. Tak więc, aby kontrolować grupę celów 4 poziomu (dwa poziomy ponad limit) lub grupę 15 ludzi, musisz zastosować 2 poziomy Wysiłku.  Kiedy ta zdolność się kończy, istoty pamiętają Twój rozkaz ale nie pamiętają bycie kontrolowanymi - Twoje komendy wydawały im się w tym czasie być rozsądne. Akcja, by rozpocząć. 

\textbf{Miażdżący Cios}\index{Zdolności!Alfabetycznie!Miażdżący Cios}\label{sec:Miażdżący Cios} (2 punkty Mocy) - Kiedy korzystasz z obuchowej lub ciętej broni w obydwu rękach i zastosujesz WYsiłek do ataku, otrzymujesz darmowy poziom Wysiłku do obrażeń ataku. (Jeśli walczysz bez broni, ten atak wykonujesz obydwoma pięściami lub stopami). Akcja.

\textbf{Kryształowe Soczewki}\index{Zdolności!Alfabetycznie!Kryształowe Soczewki}\label{sec:Kryształowe Soczewki} - Możesz skupić naturalnę enegię płynącą poprzez Twoje Kryształowe Ciało. Pozwala Ci to wystrzelić pocisk energii, który zadaje 5 punktó obrażeń celowi w bardzo dalekim zasięgu. Akcja.

\textbf{Kryształowe Ciało}\index{Zdolności!Alfabetycznie!Kryształowe Ciało}\label{sec:Kryształowe Ciało} - Twoje ciało składa się z ruchomego, przezroczystego kryształu koloru bursztynu. Określ razem z MG konkretną formę, choć zazwyczaj ma ona kształt i rozmiar umanoida. Twoje kryształowe ciało zapewnia Ci +2 do Pancerza i +4 do Puli Mocy. Jednakże, nie jesteś taki szybki i Twoje Obrona Szybkości jest utrudniona. Pewne stany, jak zwykłe choroby i trucizny, nie działają na Ciebie. Twoje kryształowe ciało regeneruje się powolniej niż zwykłe, mięsiste ciało. Posiadasz tylko 3 rzuty na odzyskanie zdrowia - 1 rundę, 1 godzinę i 10 godzin. Nie psoaidasz rzutu na odzyskanie zdrowia trwającego 10 minut. Każda zdolność, która wymaga 10-minutowego odpoczynku, zamiast tego wymaga od Ciebie odpoczynku trwającego godzinę. Umożliwienie. 

