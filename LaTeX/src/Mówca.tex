\cleardoublepage

\subsection{Mówca}\index{Typ!Mówca}

Fantasy/Baśń: bard, mówca, skald, emisariusz, kapłan, rzecznik.

Współczesność/Horror/Romans: dyplomata, lider, manipulator, minister, mediator, prawnik.

Science fiction: dyplomata, empata, konsul, legat.

Superbohaterowie/Post-Apokalipsa: władca marionetek, mesmerysta.

Jesteś dobry, jeśli chodzi o słowa i ludzi. Wyplątujesz się z niebezpieczeństw przy pomocy języka, i sprawiasz, że ludzie robią to, czego pragniesz.

Rola w grze: Mówcy są bystrzy i charyzmatyczni. Lubią ludzi i, co ważniejsze, rozumieją ich. To pomaga Mówcą sprawić, by inni zrobili to, co musi być zrobione.

Rola w drużynie: Mówca to bardzo często “twarz” drużyny – jest osobą, którą mówi za wszystkich i negocjuje z innymi. Walka i akcja nie są silną stroną Mówcy, więc inne postaci muszą czasami chronić Mówcę w chwili kryzysu.
Rola społeczna: Mówcy to często liderzy polityczni lub religijni. Równie często, jednakże, są oszustami lub kryminalistami.
Zaawansowani Mówcy: Wysokopoziomowi Mówcy korzystają ze swoich zdolności, by kontrolować i manipulować ludźmi, a także wspierać swoich przyjaciół. Mogą oni dzięki rozmowie uniknąć niebezpieczeństwa, a nawet użyć słów jak broni.

\subsubsection{Mówca - Wtrącenia Gracza}

Kiedy grasz Mówcą, możesz wydać 1 PD, by skorzystać z poniższych \mytext{wtrąceń gracza}, jeśli sytuacja jest odpowiednia i MG się zgadza.

Przyjazny BN: BN którego nie znasz, ktoś, kogo znasz słabo, lub ktoś, kogo znasz, a kto nie był szczególnie przyjazny w przeszłości postanawia Ci pomóc, lecz nie musi on wyjaśnić czemu. Może poproszi Cię potem o zwrot przysługi, w zależności od tego, w jak wielki kłopoty się wpakuje. 

Perfekcyjna Sugestia: Kompan lub inny przyjazny BN sugeruje, co zrobić w kontekście ważkiego pytania, problemu lub przeszkody na Twojej drodze.

Niespodziewany Prezent: BN wręcza Ci fizyczny dar, którego sięnie spodziewałeś, który ułatwia Twoje problemy, lub zapewnia nowy wzgląd i oświeca w kontekście sytuacji, której nie pojmujesz jak należy.

\begin{table*}[t]
 \centering
 \begin{tabularx}{\textwidth}{ | X | X  |}
  \hline
    \textbf{Statystyka} & \textbf{Początkowa Wartość Puli} \\ \hline
    Moc & 8  \\ \hline
    Szybkość & 9  \\ \hline
    Intelekt & 11  \\ \hline
 \end{tabularx}
  \caption {Pula Statystyk Mówcy}
  \label {Pula Statystyk Mówcy}
 \end{table*}
 
Otrzymujesz dodatkowe 6 punktów do rozdzielenia pomiędzy Twoje Pule statystyk, zgodnie z własnym życzeniem.
 
\subsubsection{Historia Mówcy}

Twój typ pomaga określić Twoje połączenie z settingiem. Rzuć k20 lub wybierz z poniższej listy konkretny fakt o swojej historii, który zapewnia połączenie zresztą świata. Możesz także stworzyć swój własny fakt.

