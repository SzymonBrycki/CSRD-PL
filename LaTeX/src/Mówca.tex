\cleardoublepage

\subsection{Mówca}\index{Typ!Mówca}

Fantasy/Baśń: bard, mówca, skald, emisariusz, kapłan, rzecznik.

Współczesność/Horror/Romans: dyplomata, lider, manipulator, minister, mediator, prawnik.

Science fiction: dyplomata, empata, konsul, legat.

Superbohaterowie/Post-Apokalipsa: władca marionetek, mesmerysta.

Jesteś dobry, jeśli chodzi o słowa i ludzi. Wyplątujesz się z niebezpieczeństw przy pomocy języka, i sprawiasz, że ludzie robią to, czego pragniesz.

Rola w grze: Mówcy są bystrzy i charyzmatyczni. Lubią ludzi i, co ważniejsze, rozumieją ich. To pomaga Mówcą sprawić, by inni zrobili to, co musi być zrobione.

Rola w drużynie: Mówca to bardzo często “twarz” drużyny – jest osobą, którą mówi za wszystkich i negocjuje z innymi. Walka i akcja nie są silną stroną Mówcy, więc inne postaci muszą czasami chronić Mówcę w chwili kryzysu.
Rola społeczna: Mówcy to często liderzy polityczni lub religijni. Równie często, jednakże, są oszustami lub kryminalistami.
Zaawansowani Mówcy: Wysokopoziomowi Mówcy korzystają ze swoich zdolności, by kontrolować i manipulować ludźmi, a także wspierać swoich przyjaciół. Mogą oni dzięki rozmowie uniknąć niebezpieczeństwa, a nawet użyć słów jak broni.

\subsubsection{Mówca - Wtrącenia Gracza}

Kiedy grasz Mówcą, możesz wydać 1 PD, by skorzystać z poniższych \mytext{wtrąceń gracza}, jeśli sytuacja jest odpowiednia i MG się zgadza.

Przyjazny BN: BN którego nie znasz, ktoś, kogo znasz słabo, lub ktoś, kogo znasz, a kto nie był szczególnie przyjazny w przeszłości postanawia Ci pomóc, lecz nie musi on wyjaśnić czemu. Może poproszi Cię potem o zwrot przysługi, w zależności od tego, w jak wielki kłopoty się wpakuje. 

Perfekcyjna Sugestia: Kompan lub inny przyjazny BN sugeruje, co zrobić w kontekście ważkiego pytania, problemu lub przeszkody na Twojej drodze.

Niespodziewany Prezent: BN wręcza Ci fizyczny dar, którego sięnie spodziewałeś, który ułatwia Twoje problemy, lub zapewnia nowy wzgląd i oświeca w kontekście sytuacji, której nie pojmujesz jak należy.

\begin{table*}[t]
 \centering
 \begin{tabularx}{\textwidth}{ | X | X  |}
  \hline
    \textbf{Statystyka} & \textbf{Początkowa Wartość Puli} \\ \hline
    Moc & 8  \\ \hline
    Szybkość & 9  \\ \hline
    Intelekt & 11  \\ \hline
 \end{tabularx}
  \caption {Pula Statystyk Mówcy}
  \label {Pula Statystyk Mówcy}
 \end{table*}
 
Otrzymujesz dodatkowe 6 punktów do rozdzielenia pomiędzy Twoje Pule statystyk, zgodnie z własnym życzeniem.
 
\subsubsection{Historia Mówcy}

Twój typ pomaga określić Twoje połączenie z settingiem. Rzuć k20 lub wybierz z poniższej listy konkretny fakt o swojej historii, który zapewnia połączenie zresztą świata. Możesz także stworzyć swój własny fakt.
 \begin{table*}[t]
 \centering
 \begin{tabularx}{\textwidth}{| p{0.10\textwidth} | X |}
  \hline
    \textbf{k20} & \textbf{Historia Mówcy}  \\ \hline
    1 & Jeden z Twoich rodziców był słynnym performerem za młodu i miał nadzieję, że Ty osiągniesz to samo. \\ \hline
    2 & Kiedy byłeś nastolatkiem, jedno z Twojego rodzeństwa zaginęło i uznaje się je za zmarłe. Szok wstrząsnął Twoją rodziną, i nigdy w pełni nie dałeś sobie z tym rady. \\ \hline
    3 & Zostałeś wprowadzony do sekretnego stowarzyszenia, które ponoć posiada i chroni ezoteryczną wiedzę, przeciwstawiając się siłom zła. \\ \hline
    4 & Jedno z Twoich rodziców przegrało walkę z alkoholizmem. Może ciągle jest on żywy, ale bardzo ciężko Ci mu przebaczyć.  \\ \hline
    5 & Nie posiadasz żadnej pamięci o tym, co się zdarzyło przed Twoimi 18 urodzinami. \\ \hline
    6 & Twoi dziadkowie wychowali Cię na farmie z dala od tłoku miast. Lubisz myśleć, że ich nauki przygotowały Cię na wszystko. \\ \hline
    7 & Jako sierota, miałeś trudne dzieciństwo, a Twoje wejście w dorosłość było wyzwaniem. \\ \hline
    8 & Wychowałeś się w ekstremalnej biedzie, pośród kryminalistów. Dalej masz kontakty z tym środowiskiem. \\ \hline
    9 & W przeszłości służyłeś jako wysłannik dla potężnej i wpływowej osoby – dalej myśli ona o Tobie ciepło. \\ \hline
    10 & Masz wkurzającego rywala, który zawsze Ci przeszkadza i psuje Twoje plany. \\ \hline
    11 & Zapracowałeś na pozycję rzecznika prasowego organizacji lub firmy o pewnym znaczeniu. \\ \hline
    12 & Twoich sąsiadów zamordowano, a ich tajemnicze morderstwo dalej nie zostało wyjaśnione. \\ \hline
    13 & Bardzo dużo podróżowałeś i w tym czasie nagromadziłeś kolekcję  dziwnych pamiątek. \\ \hline
    14 & Twoja pierwsza miłość z dzieciństwa skończyła związana z Twoim najlepszym przyjacielem (teraz już byłym). \\ \hline
    15 & Jesteś członkiem dyskryminowanej mniejszości, ale pracujesz nad zwróceniem publicznej uwagi na niesprawiedliwość, z którą się mierzysz. \\ \hline
    16 & Jesteś częściowo właścicielem lokalnego baru, gdzie Twoją specjalnością są koktajle. \\ \hline
    17 & Kiedyś byłeś oszustem, który pozbawił ważnych ludzi pieniędzy, a teraz pragną oni zemsty. \\ \hline
    18 & Zwykłeś występować w podróżnym teatrze, i jego członkowie miło Cię wspominają (jak i ludzie w miejscach, które odwiedziłeś). \\ \hline
    19 & Jesteś w bliskiej relacji romantycznej z kimś, kto działa w lokalnej polityce. \\ \hline
    20 & Ktoś próbuje się podawać za Ciebie, używając Twojej tożsamości do złych celów. Nigdy nie spotkałeś tej osoby, ale z chęcią byś to zrobił.  \\ \hline
 \end{tabularx}
  \caption {Historia Mówcy}
  \label {Historia Mówcy}
 \end{table*}
 
  \afterpage{\clearpage}
 
 \subsubsection{Mówca Pierwszego Poziomu}

Pierwszopoziomowy Mówca ma poniższe zdolności:

Wysiłek: Twój Wysiłek to 1.

Genius: Masz Skupienie w Intelekcie 1, a w Mocy i Szybkości – 0.

Korzystanie z Cypherów: Możesz mieć przy sobie dwa cyphery w danym momencie.

Bronie: Możesz korzystać z lekkich broni bez kary. Masz nieumiejętność w średnich i ciężkich broniach; Twoje ataki średnimi i ciężkimi broniami są utrudnione.

Początkowy Ekwipunek: Odpowiednie ubranie i lekka broń Twojego wyboru, plus dwa drogie przedmioty, dwa przedmioty średniej ceny i do 4 niedrogich przedmiotów.

Specjalne Zdolności: Wybierz cztery zdolności z poniższej listy. Nie możesz wybrać tej samej zdolności więcej niż jeden raz, chyba, że jej opis stanowi inaczej. Pełen opis wszystkich zdolności znajduje się w rozdziale \mytext{Zdolności}, razem z opisem Posmaków i specjalizacji w jednym dużym katalogu. (Pewne zdolności Mówcy, jak Czytanie Myśli lub Prawdziwe Zmysły, sugerują element nadnaturalny. Jeśli nie jest to stosowne w Twoim settingu, można je zamienić na coś z Posmaku skradanie się, lub MG może je zmodyfikować, by bazowały na supertalencie i wglądzie w ludzką psychikę zamiast na mocach nadprzyrodzonych.). 

\begin{itemize}
\item Anegdota
\item Babel
\item Fałszywa Tożsamość 
\item Przerażająca Obecność
\item Rozkazująca Postura
\item Umiejętności Międzyludzkie
\item Usunięcie Wspomnień
\item Wmawianie
\item Wyszkolony w Średnich Broniach
\item Zachęta
\item Zainspirowanie Agresji
\item Zauroczenie
\item Zrozumienie 
\end{itemize}

\subsubsection{Mówca Drugiego Poziomu}

Wybierz dwie z poniższych zdolności (lub niższego poziomu) i dodaj do swoich zdolności. Dodatkowo, możesz zamienić jedną z ze zdolności niższego poziomu na inną zdolność niższego poziomu.

\begin{itemize}
\item Agent Wywiadu
\item Niespodziewana Zdrada
\item Podstawowy Kompan
\item Szybkie Zdrowienie
\item Umiejętna Obrona
\item Umiejętności Międzyludzkie
\item Uspokojenie Nieznajomego
\item Wyszkolony w Zbroi
\item Zainspirowanie Ułatwienia
\item Zniechęcenie
\item Zasianie Idei
\end{itemize}

\subsubsection{Mówca Trzeciego Poziomu}

Wybierz trzy z poniższych zdolności (lub niższego poziomu) i dodaj do swoich zdolności. Dodatkowo, możesz zamienić jedną ze zdolności niższego poziomu na inną zdolność niższego poziomu.

\begin{itemize}
\item Akceleracja
\item Czytanie Myśli
\item Ekspercki Użytkownich Cypherów 
\item Kompan-Ekspert
\item Lider Spostrzegawczości
\item Oracja
\item Perfekcyjny Nieznajomy
\item Rozmowny
\item Silny Umysł
\item Szybki Umysł
\item Wielkie Oszustwo
\item Zlanie się z Tłem
\end{itemize}

\subsubsection{Mówca Czwartego Poziomu}

Wybierz dwie z poniższych zdolności (lub z niższego poziomu) i dodaj do swoich zdolności. Dodatkowo, możesz zamienić jedną ze zdolności niższego poziomu na inną zdolność niższego poziomu.

\begin{itemize}
\item Czytając Znaki
\item Finta
\item Konfundujące Nonsensy
\item Psychoza
\item Strategia
\item Sugestia
\item Ulepszone Umiejętności
\item Uprzedzenie Ataku
\item Wspólny Wysiłek
\end{itemize}

\subsubsection{Mówca Piątego Poziomu}

Wybierz trzy z poniższych zdolności (lub z niższego poziomu) i dodaj do swoich zdolności. Dodatkowo, możesz zamienić jedną ze zdolności niższego poziomu na inną z niższego poziomu.

\begin{itemize}
\item Przywykły do Noszenia Zbroi
\item Regeneracja
\item Stymulacja
\item Ucieczka
\item Umiejętny Atak
\item Wiedza o Nieznanym
\item Wspieranie Uważności
\item Zaawansowany Użytkownik Cypherów
\item Złowróżna Aura
\end{itemize}

\subsubsection{Mówca Szóstego Poziomu}

Wybierz dwie zdolności z poniższej listy (lub z niższego poziomu) i dodaj do swoich zdolności. Dodatkowo, możesz zamienić jedną z umiejętności niższego poziomu na inną zdolność niższego poziomu.

\begin{itemize}
\item Kontrola Tłumu
\item Prawdziwe Zmysły
\item Rekrutowanie Delegata
\item Przejęcie Kontroli
\item Słowo Rozkazu
\item Strzaskanie Umysłu
\item Zainspirowanie Sukcesu
\item Zarządzanie Bitwą
\end{itemize}

\subsubsection{Przykładowy Mówca}

Mary chce stworzyć Mówcę do kampanii Lovecraftowskiego horroru. Przeznacza 3 punkty wolne na Intelekt i 3 na Szybkość. Jej Pule wyglądają teraz następująco: Moc 8, Szybkość 12 i Intelekt 14. Jako postać pierwszego poziomu, jej Wysiłek wynosi 1, jej Skupienie w Mocy i Szybkości 0, a jej Skupienie w Intelekcie 1. Jest ona charyzmatyczna i mądra, ale niekoniecznie silna. 

Mary wybiera \mytext{Wmawianie} i  \mytext{Fałszywą Tożsamość} aby mogła się łatwiej dostać w różne miejsca i zdobyć wiedzę, której pragnie. Jest nieco oszustką. Jest jednak dobra dla swoich przyjaciół, więc wybiera  \mytext{Zachętę}. Mary kończy, wybierając  \mytext{Umiejętności Międzyludzkie} (oszustwo i perswazję).

Mówca normalnie zaczyna grę z dwoma cypherami, ale MG podejmuje decyzję, że w te kampanii postaci zaczynają tylko z jednym – czymś mrocznym i dziwnym, co ma związek z ich historią. Cypher Mary to dziwny zegarek kieszonkowy, który otrzymała od swojego ojca. Nie wie jak ani czemu, ale kiedy się go aktywuje, ten zegarek pozwala jej na podwójną liczbę akcji w ciągu trzech rund.
Postać Mary nosi przy sobie mały nóż ukryty w jej torebce w przypadku kłopotów. Jako lekka broń, zadaje on 2 punkty obrażeń, ale ataki nim są ułatwione. 

Mary jako swój deskryptor wybiera  \mytext{Odporny} i decyduje, że jej postać może najpewniej poznać prawdę o dziwnych rzeczach, o których słyszała i nie otrzymać z tego tytuły zbyt wiele traumy. Odporny zwiększa jej Pulę Mocy do 10, a jej Intelekt do 16. Jest ona wyszkolona w Obronie Mocy i Intelektu i otrzymuje dodatkowy rzut na odzyskanie zdrowia każdego dnia. Na początku, Mary jest smutna, że jej deskryptor daje jej  \mytext{nieumiejętność} w wiedzy i łamigłówkach, ale potem zdała sobie sprawę z tego, że dobrze to pasuje – jej postać woli uzyskiwać informacje i wiedzę od innych ludzi, niż samej ją zdobywać. 

Na swoją specjalizację, Mary wybiera  \mytext{Porusza się jak Kot}, Decyduje, że miała obsesję z dziwnym tomem, który był w jej rodzinie przez pokolenia, a jej postać jest ciekawa dziwnych języków i rytuałów.