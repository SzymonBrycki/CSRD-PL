\cleardoublepage

\subsection{Odkrywca}\index{Typ!Odkrywca}

Fantasy/Baśń: Odkrywca, poszukiwacz przygód, badacz tajemnic.

Współczesność/Horror/Romans: atleta, odkrywca, poszukiwacz przygód, detektyw, badacz, pionier, reporter śledczy.

Science fiction: odkrywca, poszukiwacz przygód, podróżnik, planetolog, ksenobiolog.

Superbohaterowie/Post-apokalipsa: poszukiwacz przygód, stróż prawa.

Jesteś osobą akcji i fizycznych zdolności, bez lęku patrzącą ku nieodkrytemu. Podróżujesz do dziwnych, egzotycznych i niebezpiecznych miejsc, i odkrywasz nowe rzeczy. Oznacza to, że masz duże zdolności fizyczne, ale zapewne także jesteś dobrze wykształcony. 

Rola w grze: Choć Odkrywcy mogą być uczonymi i dobrze wykształconymi, są przede wszystkim zainteresowani akcją. Mierzą się ze śmiertelnymi niebezpieczeństwami i okropnymi przeszkodami praktycznie codziennie.

Rola w drużynie: Odkrywcy czasami pracują sami, ale częściej są częścią zespołu z innymi postaciami. Odkrywca często przoduje i przeciera szlak. Jednakże, często zatrzymują się i badają to, co ich zaintrygowało po drodze. 

Rola społeczna: Nie wszyscy Odkrywcy przedzierają się przez dzicz lub badają stare ruiny. Czasami, Odkrywca to nauczyciel, naukowiec, detektyw lub reporter śledczy. W każdym wypadku, Odkrywca z odwagą zmaga się z nowymi wyzwaniami i zbiera wiedzę, którą może się dzielić z innymi.

Zaawansowani Odkrywcy: Wysokopoziomowi Odkrywcy zyskują więcej umiejętności, trochę zdolności bojowych i dużo zdolności, które pomagają im poradzić sobie z niebezpieczeństwem. W skrócie, stają się uniwersalni, zdolni dać sobie radę z każdym wyzwaniem. 

\subsubsection{Odkrywca - Wtrącenia Gracza}

Kiedy grasz Odkrywcą, możesz wydać 1 PD by skorzystać z poniższych \mytext{wtrąceń gracza}, jeśli sytuacja jest odpowiednia i MG się zgadza.

Szczęśliwa Awaria: Pułapka lub niebezpieczne urządzenie doświadcza awarii, zanim może Ciebie zranić.

Nieoczekiwana Wskazówka: W momencie, gdy myślisz, że kompletnie zgubiłeś drogę, element krajobrazu, drogowskaz, lub po prostu ułożenie terenu sprawia, że odkrywasz najlepszą drogę naprzód, przynajmniej w tym momencie.

Słaba Trucizna: Trucizna lub choroba okazuje się nie być tak poważna, jak na początku wyglądała, i zadaje tylko połowę obrażeń, które zadałaby normalnie. 

\begin{table*}[t]
 \centering
 \begin{tabularx}{\textwidth}{ | X | X  |}
  \hline
   \textbf{Statystyka} & \textbf{Początkowa Wartość Puli} \\ \hline
    Moc & 10  \\ \hline
    Szybkość & 9  \\ \hline
    Intelekt & 9  \\ \hline
 \end{tabularx}
  \caption {Pula Statystyk Odkrywcy}
  \label {Pula Statystyk Odkrywcy}
 \end{table*}
 
 Otrzymujesz dodatkowe 6 punktów, które możesz rozdzielić pomiędzy swoje Pule zgodnie ze swoim życzeniem.
 
 \subsubsection{Historia Odkrywcy}
 
Twój typ pomaga Ci określić połączenie Twojej postaci z settingiem. Rzuć k20 lub wybierz z poniższej listy, by określić konkretny fakt o Twojej historii, który łączy Cię zresztą świata. Możesz także stworzyć swój własny fakt.

 \begin{table*}[t]
 \centering
 \begin{tabularx}{\textwidth}{| p{0.10\textwidth} | X |}
  \hline
  \textbf{k20} & \textbf{Historia Odkrywcy}  \\ \hline
    1 & Byłeś gwiazdą sportu w swoim liceum. Dalej jesteś w dobrej kondycji, ale człowieku, co to było wtedy! \\ \hline
    2 & Twój brat jest głównym śpiewakiem w naprawdę popularnym zespole. \\ \hline
    3 & Dokonałeś szeregu odkryć podczas swoich podróży, ale nie wszystkie okoliczności, by na nich zarobić, jeszcze pojawiły się przed Tobą. \\ \hline
    4 & Byłeś policjantem, ale zrezygnowałeś z pracy po doświadczeniu korupcji w siłach porządkowych. \\ \hline
    5 & Twoi rodzice byli misjonarzami, więc spędziłeś dużą część swojego młodego życia, podróżując do egzotycznych miejsc.  \\ \hline
    6 & Służyłeś w armii z honorem. \\ \hline
    7 & Otrzymałeś pomoc od sekretnej organizacji, która opłaciła Twoją edukację. Teraz ona pragnie znacznie więcej od Ciebie.  \\ \hline
    8 & Uczęszczałeś na prestiżowy uniwersytet dzięki stypendium dla sportowców, ale lśniłeś zarówno na boisku, jak i podczas zajęć.  \\ \hline
    9 & Twój najlepszy przyjaciel z dzieciństwa jest teraz wplywowym członkiem rządu. \\ \hline
    10 & Byłeś nauczycielem. Twoi studenci wspominają Cię miło. \\ \hline
    11 & Przez krótki czas byłeś kryminalistą, który został złapany i poszedł do więzenia – potem próbowałeś wyjść na prostą. \\ \hline
    12 & Twoje największe jak dotąd odkrycie zostało ukradzione przez Twojego rywala.  \\ \hline
    13 & Należysz do ekskluzywnej organizacji Odkrywców, której istnienie nie jest szeroko znane. \\ \hline
    14 & Zostałeś porwany jako dziecko w tajemniczych okolicznościach, are wróciłeś do domu bezpieczny. Media dalej czasami wspominają ową sytuację.  \\ \hline
    15 & Kiedy byłeś młody, byłeś uzależniony od narkotyków, a teraz powoli wstajesz na nogi. \\ \hline
    16 & Kiedy badałeś odległą lokację, dostrzegłeś coś, czego nigdy nie byłeś w stanie wyjaśnić. \\ \hline
    17 & Posiadasz mały bar lub restaurację. \\ \hline
    18 & Opublikowałeś książkę o swoich odkryciach i poczynaniach, która zyskała pewne uznanie. \\ \hline
    19 & Twoja siostra posiada sklep i daje Tobie pokaźną zniżkę. \\ \hline
    20 & Twój ojciec to wysoki rangą oficer w armii i posiada wiele koneksji. \\ \hline
 \end{tabularx}
  \caption {Historia Odkrywcy}
  \label {Historia Odkrywcy}
 \end{table*}
 
 \subsubsection{Odkrywca Pierszego Poziomu}
 
Odkrywca pierwszego poziomu ma poniższe zdolności:

Wysiłek: Twój Wysiłek to 1.

Fizyczna Natura: Masz Skupienie w Mocy 1, w Szybkości i Intelekcie zaś – 0.

Korzystanie z Cypherów: Możesz mieć przy sobie 2 cyphery naraz.

Początkujący Ekwipunek: Odpowiednie ubranie i broń Twojego wyboru, plus 2 drogie przedmioty, 2 przedmioty średniej ceny i do 4 niedrogich przedmiotów.

Bronie: Możesz korzystać z lekkich i średnich broni bez kary. Posiadasz nieumiejętność z ciężkimi brońmi – Twoje ataki nimi są utrudnione.

Specjalne Zdolności: Wybierz cztery z poniższych zdolności. Nie możesz wybrać tej samej zdolności więcej niż raz, chyba, że jej opis stanowi inaczej. Pełen opis wszystkich zdolności znajduje się w rozdziale \mytext{Zdolności}, który także zawiera zdolności Posmaków i specjalizacji w pojedynczym, rozległym katalogu. 

\begin{itemize}
\item Blok
\item Broń Niepotrzebna
\item Deszyfracja
\item Mięśnie z Żelaza
\item Przypływ Pewności Siebie
\item Szybkostopy
\item Ulepszone Skupienie
\item Umiejętności Fizyczne
\item Umiejętności Wiedzy
\item Wyszkolony Bez Zbroi
\item Wyszkolony we Wszystkich Broniach
\item Wyszkolony w Zbroi
\item Wytrzymałość
\item Zmysł Niebezpieczeństwa
\item Znajdowanie Drogi
\end{itemize}

\subsubsection{Odkrywca Drugiego Poziomu}

Wybierz cztery z poniższych zdolności (lub z niższego poziomu) i dodaj do swoich zdolności. Dodatkowo, możesz zamienić jedną ze zdolności niższego poziomu na inną z niższego poziomu.

\begin{itemize}
\item Ciekawy
\item Instynkt Niebezpieczeństwa
\item Koordynacja Ręka-Oko
\item Na Straży
\item Negacja Zagrożenia
\item Oko do Szczegółów
\item Pomoc Bez Akcji
\item Szybkie Odzyskanie Zdrowia
\item Ucieczka
\item Umiejętna Obrona
\item Umiejętności Podróżnicze
\item Umiejętności Śledcze
\item Zwiększenie Zasięgu
\item Zniszczenie
\end{itemize}

\subsubsection{Odkrywca Trzeciego Poziomu}

Wybierz trzy zdolności z poniższej listy (lub niższego poziomu) i dodaj do swoich zdolności. Dodatkowo, możesz zamienić jedną ze zdolności niższego poziomu na inną zdolność z niższego poziomu.

\begin{itemize}
\item Bieg i Walka
\item Bieg Przez Przeszkody
\item Chwytaj Moment
\item Ekspercki Użytkownik Cypherów
\item Kontrolowany Upadek
\item Łamiący Kamienie
\item Odporność
\item Przemyślenie Problemów
\item Przywykły do Noszenia Zbroi
\item Ucieczka od Złego Losu
\item Umiejętny Atak
\item Zignorowanie Bólu
\item Znajdywacz Pułapek
\end{itemize}

\subsubsection{Odkrywca Czwartego Poziomu}

Wybierz dwie z poniższych zdolności (lub z niższego poziomu) i dodaj do swoich zdolności. Dodatkowo, możesz zamienić jedną ze zdolności niższego poziomu na inną niższego poziomu.

\begin{itemize}
\item Biegacz
\item Ciche Kroki
\item Czytając Znaki
\item Umiejętność Eksperta
\item Wyjątkowo Wytrzymały
\item Zręczny Wojownik
\item Zwiększony Efekt
\end{itemize}

\subsubsection{Odkrywca Piątego Poziomu}

Wybierz trzy z poniższych zdolności (lub z niższego poziomu) i dodaj do swoich zdolności. Dodatkowo, możesz zamienić jedną ze zdolności niższego poziomu na inną zdolność niższego poziomu.

\begin{itemize}
\item Atak z Wyskoku
\item Blokowanie
\item Czujny
\item Mistrzostwo Obrony
\item Mistrz Ruchu
\item Obdarzenie Fizyczne
\item Paczka Przyjaciół
\item Trudny do Zamordowania
\item Wydanie Rozkazu
\item Zaawansowany Użytkownik Cypherów
\end{itemize}

\subsubsection{Odkrywca Szóstego Poziomu}

Wybierz trzy z poniższych zdolności (lub z niższego poziomu) i dodaj do swoich zdolności. Dodatkowo, możesz zamienić jedną z zdolności niższego poziomu na inną z niższego poziomu.

\begin{itemize}
\item Dzielona Obrona
\item Dzikie Zdrowie
\item Mistrzostwo Ataków
\item Mistrzowska Biegłość w Pancerzach
\item Wielokrotny Atak
\item Większa Negacja Zagrożenia
\item Zainspirowanie Skoordynowanych Akcji
\item Znowu i Znowu
\end{itemize}

\subsubsection{Przykładowy Odkrywca}

Sam decyduje się na stworzenie Odkrywcy do kampanii science fiction. Ta postać będzie wytrzymałym odkrywcą badającym obce światy. Wydaje onu 3 wolne punkty na Pulę Mocy, 2 na Pulę Szybkości i 1 na Pulę Intelektu. Teraz jenu Pule to: Moc 13, Szybkość 11 i Intelekt 10. Jako postać pierwszego poziomu, ma on Wysiłek na 1, Skupienie w Mocy 1, a w Szybkości i Intelekcie 0. Póki co postać jest dosyć uniwersalna.

Sam zaczyna wybierać zdolności. Wybiera onu \mytext{Zmysł Niebezpieczeństwa} i \mytext{Przypływ Pewności Siebie}, uważając, że będą przydatne w wielu sytuacjach. Wybiera onu także \mytext{Wyszkolony w Zbroi}, uważając, że postać będzie nosić high-techową średnią zbroję w czasie eksploracji. Jako ostatnią zdolność, onu wybiera \mytext{Umiejętności Wiedzy}, wybierając biologię i geologię, by pomóc podczas misji planetarnych.

Odkrywca Sama może mieć przy sobie dwa cyphery, które w tym settingu mają naturę nanotechnologiczną. MG decyduje, że jeden z nich to zastrzyk nanitów, dający +1 do Skupienia w Mocy po użyciu, a drugi to urządzenie pozwalające na stworzenie jednego dowolnego prostego przedmiotu trzymanego w dłoni, zgodnie z życzeniem użytkownika. 

Odkrywca Sama nie jest naprawdę zorientowany na walkę, ale czasami Wszechświat to niebezpieczne miejsce, więc nosi on przy sobie średni blaster.

Sam dalej potrzebuje deskryptora i specjalizacji. Patrząc na rozdział o Deskryptorach, wybiera onu \mytext{Wytrzymały}, co zwiększa jej Pulę Mocy do 17. Postać ta będzie się też szybciej leczyć i lepiej działać, gdy jest zraniona. Jest ona wyszkolona w Obronie Mocy, ale ma \mytext{nieumiejętność} w inicjatywie – jednakże, anuluje to się ze Zmysłem Niebezpieczeństwa. Sam mogłoby się cofnąć i wybrać coś innego zamiast Zmysłu Niebezpieczeństwa, ale lubi onu to i tak to zostawia. Ogólnie rzecz ujmując, deskryptor uczynił postać bardzo wytrzymałą, nawet jeśli nieco powolną. 
Na swoją specjalizację, Sam wybiera \mytext{Bada Ciemnie Miejsca} (w tym przypadku ruiny obcych cywilizacji). Daje to postaci nieco dodatkowych umiejętności: szukanie, słuchanie, wspinaczkę, balansowanie i skakanie. Ten Badacz jest całkiem-całkiem.

Na swój motyw fabularny, Sam wybiera \mytext{Biznes}. Badanie obcych ruin czasem powoduje odkrycie dziwnych reliktów, i Sam uznału, że może je przetransportować do osób trzecich, zamiast pozwolić im wpaść w ręce piratów lub bogatych kolekcjonerów. Za małą opłatą, oczywiście. 

