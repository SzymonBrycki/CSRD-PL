\subsubsection{Dalsza Customizacja}\index{Dalsza Customizacja}

Zasady w tej sekcji są bardziej zaawansowane i zawsze zależą od MG. Mogą być użyte, by MG dostosował typ do konwencji lub settingu, lub przez gracza i MG, by dostosować koncept postaci.

\paragraph{Modyfikacja Aspektów Typu}

Poniższe aspekty czterech typów postaci mogą zostać zmodyfikowane podczas tworzenia postaci. Inne zdolności nie powinny być zmienione.

Pule Statystyk: Każda Pula postaci ma wartość startową. Gracz może zamieniać punkty w Pulach kosztem 1-na-1. Dla przykładu, może on przesunąć 2 punkty z Mocy na 2 punkty w Szybkości. Jednakże, żadna początkowa Statystyka nie może być wyższa niż 20.

Skupienie: Gracz może zacząć grę ze Skupieniem w dowolnej Statystyce na 1.

Korzystanie z Cypherów: Jeśli gracz odda zdolność noszenie jednego cyphera, uzyskuje on dodatkową umiejętność swojego wyboru.

Bronie: Pewne typy mają statyczne zdolności pierwszego poziomu które pozwalają im korzystać z lekkich, średnich i/lub ciężkich broni bez kary. Wojownicy mogą korzystać z wszystkich broni, Odkrywcy z lekkich i średnich, a Adepci i Mówcy mogą korzystać tylko z lekkich broni. Każda z tych zdolności może zostać poświęcona, by zyskać trening w odmiennej umiejętności, którą wybierze gracz.

\paragraph{Wady i Kary}

W dodatku do innych opcji customizacji, gracz może wybrać wzięcie kar lub wad, by zyskać dalsze bonusy.

Słabość: Słabość to, esencjalnie, przeciwieństwo Skupienia. Jeśli masz Słabość 1 w Szybkości, wszystkie akcje Szybkości wymagają od Ciebie dodatkowego 1 punktu z Twojej Puli. W każdym momencie, gracz może dać swojej postaci słabość w jednej statystycei otrzymać +1 do Skupienia w jednej z pozostałych dwóch. Tak więc gracz może wziąć słabość 1 w Szybkości i otrzymać +1 do swojego Skupienia w Mocy.

Normalnie, możesz mieć słabość tylko w statystyce, w której Twoje Skupienie wynosi 0. Co więcej, nie możesz mieć więcej niż jednej słabości, i niem ożesz mieć słabości większej niż 1, chyba, że dodatkowa słabość pochodzi z innego źródła (takiego jak zaraza lub niepełnosprawność wynikająca z akcji lub kondycji w grze).

Nieumiejętność: Nieumiejętności są jak negatywne umiejętności. Czynią jeden rodzaj akcji trudniejszym. Jeśli postać wybiera nieumiejętność, zyskuje ona umiejętność swojego wyboru. Normalnie, postać może mieć tylko jedną nieumiejętność, chyba, że pozostałe pochodzą z innego źródła (takiego jak deskryptor, choroba lub niepełnosprawność wynikająca z akcji lub kondycji w grze).

\paragraph{Posmaki}

Posmaki to grupa specjalnych zdolności które MG i gracze mogą wykorzystać, aby zmienić typ postaci – np.: w zgodzie z settingiem lub konwencją. Dla przykładu, jeśli gracz chce stworzyć czarodzieja, który jest też złodziejem, może zagrać Adeptem z Posmakiem w skradaniu się. W settingu science fiction, Wojownik może mieć także wiedzę o maszynach, więc postać może mieć posmak w technologii. 

Na danym poziomie, zdolności z standardowego typu są wymieniane na zdolności z posmaku. Tak więc, aby dodać Zmysł Niebezpieczeństwa z posmaku skradanie się do Wojownika, trzeba poświęcić coś innego – być może Ogłuszenie. Teraz postać może wybrać Zmysł Niebezpieczeństwa, tak jak każdą inną zdolność pierwszego poziomu, ale nigdy nie może wziąć Ogłuszenia.

MG zawsze powinien brać udział w modyfikacji typu za pośrednictwem posmaku. Dla przykładu, może on określić, że w grze science fiction chce stworzyć type zwany “Glam”, czyli Mówcę z pewnymi zdolnościami technologicznymi – konkretniej tymi, które czynią z niego ekstrawaganckiego pilota statków kosmicznych. Tak więc, zamienia on pierwszopoziomowe zdolności Fałszywa Tożsamość i Zainspirowanie Agresji na Implant Wizualnej Identyfikacji i Umiejętności Technologiczne, tak, że postać może połączyć siębezpośrednio ze statkiem i mieć umiejętności komputerowe i pilotażu.

Ostatecznie, posmak to głównie narzędzia dla MG do łatwego tworzenia typów zrobionych pod kampanie, poprzez parę lekkich zmian tu i ówdzie. Choć gracze mogą chcieć skorzystać z posmaków, by stworzyć postać, jakiej pragną, pamiętaj, że mogą oni także dookreślić swojego BG przy pomocy deskryptorów i specjalizacji. 

Dostępne posmaki to: skradanie się, technologia, magia, walka oraz umiejętności i wiedza.
Pełen opis wszystkich zdolności można znaleźć w rozdziale Zdolności, który zawiera także opisy zdolności typów i specjalności w jednym pokaźnym katalogu.

